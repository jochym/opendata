\documentclass[11pt,a4paper]{article}
\usepackage[utf8]{inputenc}
\usepackage{amsmath}
\usepackage{graphicx}
\usepackage{cite}

\title{Phonon-mediated superconductivity in novel hydrides under high pressure: A first-principles study}
\author{Paweł T. Jochym$^{1,*}$, Jan Kowalski$^{2}$}
\date{\today}

\begin{document}

\maketitle

\begin{abstract}
We investigate the structural and dynamical properties of the newly discovered $SH_3$ phase under pressures exceeding 200 GPa. Using density functional theory (DFT) as implemented in the VASP package, we calculate the phonon dispersion relations and the electron-phonon coupling constant $\alpha^2F(\omega)$. Our results show that the high superconducting transition temperature $T_c \approx 203$ K is primarily driven by high-frequency hydrogen vibrations. This study provides a comprehensive metadata set for reproducibility in computational materials science.
\end{abstract}

\section{Introduction}
The discovery of high-temperature superconductivity in hydrogen-rich materials has opened a new frontier in condensed matter physics \cite{Drozdov2015}. The $SH_3$ system, in particular, has shown remarkable properties under extreme pressure.

\section{Methods}
Calculations were performed using the Projector Augmented Wave (PAW) method within the Generalized Gradient Approximation (GGA-PBE). We used a plane-wave cutoff of 500 eV and a $12 \times 12 \times 12$ Monkhorst-Pack k-point grid. Phonon calculations were carried out using the finite displacement method as implemented in the Phonopy package \cite{Togo2015}.

\section{Results}
The calculated phonon dispersion relations are shown in Fig. \ref{fig:phonons}. The absence of imaginary frequencies confirms the dynamical stability of the $Im\bar{3}m$ phase at 200 GPa.

\begin{figure}[h]
    \centering
    \includegraphics[width=0.8\textwidth]{figures/phonon_dispersion.png}
    \caption{Phonon dispersion relations and density of states for $SH_3$ at 200 GPa.}
    \label{fig:phonons}
\end{figure}

\bibliographystyle{plain}
\begin{thebibliography}{9}
\bibitem{Drozdov2015} A. P. Drozdov, et al., Nature 525, 73-76 (2015).
\bibitem{Togo2015} A. Togo and I. Tanaka, Scr. Mater. 108, 1-5 (2015).
\end{thebibliography}

\end{document}
