\documentclass[%
%reprint,
superscriptaddress,
%groupedaddress,
longbibliography,
%unsortedaddress,
%runinaddress,
%frontmatterverbose, 
%preprint,
%preprintnumbers,
%nofootinbib,
nobibnotes,
%bibnotes,
amsmath,amssymb,
aps,
%pra,
prb,
%rmp,
%prstab,
%prstper,
%showkeys,
floatfix,
twocolumn
]{revtex4-2}

\usepackage{graphicx}% Include figure files
\usepackage{calc}% Calculate margins
\usepackage{dcolumn}% Align table columns on decimal point
\usepackage{bm}% bold math

\usepackage[urlcolor=blue,colorlinks=true,citecolor=blue,linkcolor=blue,pdfstartview={FitH},bookmarks=false]{hyperref} % add hypertext capabilities

%\usepackage[mathlines]{lineno} % Enable numbering of text and display math
% \linenumbers\relax % Commence numbering lines

% \usepackage[showframe,%Uncomment any one of the following lines to test 
% %scale=0.7, marginratio={1:1, 2:3}, ignoreall,% default settings
% %text={7in,10in},centering,
% %margin=1.5in,
% % total={6.5in,8.75in}, top=1.2in, left=0.9in, includefoot,
% % height=10in,a5paper,hmargin={3cm,0.8in},
% ]{geometry}

\usepackage{amsmath}
\usepackage{amssymb}
%\usepackage{orcidlink}
\usepackage{xcolor}
%\usepackage{datetime}
\usepackage[normalem]{ulem}

% Change tracking commands
\newcommand{\trackchange}[3]{\textcolor{#3}{\sout{#1}#2}}  % Full color strikeout, insert
%\renewcommand{\trackchange}[3]{\textcolor{#3}{#2}}        % Just color silent remove and insert
%\renewcommand{\trackchange}[3]{{#2}}                      % No indication, silent remove and insert

% Author marker definitions

\definecolor{myblue}{RGB}{0,127,85}
% \definecolor{violet}{RGB}{102,0,204}
% \definecolor{orange}{RGB}{255,128,0}
% \definecolor{green}{RGB}{0,128,0}
\newcommand{\DL}[1]{\trackchange{}{#1}{blue}}
\newcommand{\AP}[1]{\trackchange{}{#1}{red}}
\newcommand{\PJ}[2]{\trackchange{#1}{#2}{orange}}
\newcommand{\JL}[2]{\trackchange{#1}{#2}{myblue}}
\newcommand{\SP}[2]{\trackchange{#1}{#2}{blue}}
\newcommand{\PP}[2]{\trackchange{#1}{#2}{teal}}
\newcommand{\AI}[2]{\trackchange{#1}{#2}{olive}}
\newcommand{\MS}[1]{\trackchange{}{#1}{purple}}

\newcommand{\TODO}[1]{\textcolor{red}{TODO: #1}}

\sloppy

\begin{document}

\title{Ab initio study of the anharmonic properties and thermal conductivity in $\beta$-FeSi$_2$}

\author{Svitlana~Pastukh}
\email[e-mail: ]{svitlana.pastukh@ifj.edu.pl}
\affiliation{Institute of Nuclear Physics, Polish Academy of Sciences, ul. W. E. Radzikowskiego 152, 31-342 Krak\'{o}w, Poland}

\author{Ma\l{}gorzata~Sternik}
\affiliation{Institute of Nuclear Physics, Polish Academy of Sciences, ul. W. E. Radzikowskiego 152, 31-342 Krak\'{o}w, Poland}

\author{Pawe\l{}~T.~Jochym}
\affiliation{Institute of Nuclear Physics, Polish Academy of Sciences, ul. W. E. Radzikowskiego 152, 31-342 Krak\'{o}w, Poland}


\author{Jan~\L{}a\.{z}ewski}
\affiliation{Institute of Nuclear Physics, Polish Academy of Sciences, ul. W. E. Radzikowskiego 152, 31-342 Krak\'{o}w, Poland}

\author{Andrzej~Ptok}
\affiliation{Institute of Nuclear Physics, Polish Academy of Sciences, ul. W. E. Radzikowskiego 152, 31-342 Krak\'{o}w, Poland}

\author{Svetoslav~Stankov}
\affiliation{Institute for Photon Science and Synchrotron Radiation, Karlsruhe Institute of Technology, D-76131 Karlsruhe, Germany}
\affiliation{Laboratory for Applications of Synchrotron Radiation, Karlsruhe Institute of Technology, D-76131 Karlsruhe, Germany}

\author{Przemys\l{}aw~Piekarz}
\affiliation{Institute of Nuclear Physics, Polish Academy of Sciences, ul. W. E. Radzikowskiego 152, 31-342 Krak\'{o}w, Poland}

\date{\today}

\begin{abstract}

Iron silicides are good candidates for applications in  optoelectronic and thermoelectric devices.
Lattice dynamical properties and thermal conductivity in the $\beta$-FeSi$_2$ semiconductor
are investigated with the first-principles computational methods. 
Phonon dispersion relations are calculated via the
temperature-dependent effective potential method and self-consistent phonon theory. 
To properly model thermal transport, we explicitly consider
the impact of phonon-phonon interactions by analyzing
anharmonic contributions to the phonon self-energy. 
This yields temperature-dependent phonon frequencies and linewidths,
reflecting the finite lifetime of phonons due to scattering
processes. The calculated phonon frequencies and line profiles are used to obtain 
the Raman spectra, which shows good agreement with the experimental data. 
We revealed an enhanced anharmonic behaviour of the Raman modes with the highest frequencies.  
The lattice thermal conductivity is then obtained as a function of temperature and crystallite size within
the relaxation-time approximation.
Phonon transport shows a small anisotropy due to the orthorhombic structure and a very weak dependence
on the quartic anharmonic corrections. The results obtained for an infinite material and for several crystallite sizes
were analyzed and compared with the available experimental data.
\end{abstract}

\maketitle


\section{Introduction}

The comprehensive determination of important physical properties of
crystals, such as thermal expansion, lattice thermal
conductivity or structural phase transitions, requires a fundamental 
understanding of the anharmonic effects.
Although the investigation of anharmonic interactions in crystals has
attracted a considerable interest for decades~\cite{cowley_1968}, a substantial progress
has only recently been achieved thanks to advances in theoretical and
numerical methods and increased computational power.
Now, phonon frequencies, lifetimes, and heat transfer in a wide range of
materials
can be quantitatively predicted using the available computational resources
based on the density functional theory (DFT)~\cite{lindsay_2013,mcgaughey_2019,lindsay_2019}.
In the case of strongly anharmonic systems, the self-consistent phonon
(SCPH) theory~\cite{tadano_2015} as well as the perturbative approach~\cite{tadano_2018}, using higher
order interatomic force constants derived from the fitting to the displacement force data obtained
with DFT, have proven to be successful.

Transition-metal silicides are promising materials for fabrication of electronic components
designed for integration with silicon-based circuits~\cite{murarka_1995}.
At room temperature, iron disilicide ($\beta$-FeSi$_2$) is a
direct-bandgap semiconductor~\cite{bost_1985}, making this material a good
candidate for application in optoelectronic devices such as infrared
detectors or light emitters~\cite{bost_1988}. The development of light-emitting diodes utilizing FeSi$_2$/Si heterostructures has been successfully demonstrated~\cite{leong_1997,suemasu_2001}. Due to
a high thermal stability and strong light absorption, FeSi$_2$ is
also a suitable photovoltaic material~\cite{powalla_1993,liu_2006,okuhara_2017}.

$\beta$-FeSi$_2$ crystallizes in the base-centered orthorhombic
lattice~\cite{dusausoy_1971} transforming to the
tetragonal metallic $\alpha$-FeSi$_{2}$ phase around $1200$~K~\cite{starke_2002}.
Optical studies indicated a direct band gap of the values
$0.85$--$0.89$~eV~\cite{bost_1988,dimitriadis_1990,arushanov_1995,wan_2003},
however, the {\it ab initio} calculations predicted a smaller indirect gap close to 0.8 eV~\cite{christensen_1990}. 
The existence of such an indirect gap was then confirmed by the optical linear transmittance measurements at
low temperatures~\cite{giannini_1992}. As shown by first-principles studies,
the character of the band gap is very sensitive to the orientation 
of a crystal grown on silicon~\cite{clark_1998}.

$\beta$-FeSi$_2$ belongs also to good thermoelectric materials~\cite{ware_1964}, with potential applications resulting from its chemical stability up to high temperatures, nontoxicity, and low cost of preparation~\cite{yamada_2012,nozariasbmarz_2017}. 
It has already been implemented in cars~\cite{birkholz_1988} and portable power sources~\cite{uemura_1989}. 
Its thermoelectric performance can be improved by doping~\cite{ito_2001,tani_2001,kim_2003,chen_2005,pandey_2013,le_tonquesse_2019}, 
which enhances the electric transport and reduces the thermal conductivity~\cite{waldecker_1973,du_2019,du_2020}.  
The thermal conductivity can be also reduced by the modification of microstructure~\cite{ail_2015} or by
nanostructurization~\cite{watanabe_2017,taniguchi_2017,hsin_2017,abbassi_2021}.

The lattice thermal conductivity is directly connected with anharmonic effects and phonon scattering processes.
The vibrational properties of \mbox{$\beta$-FeSi$_2$} were studied by the infrared and Raman spectroscopy~\cite{lefki_1991,guizzetti_1997,maeda_2004,baleva_2008,liu_2011,maeda_2011}. The observed anisotropy in the phonon spectra results from the enhanced sensitivity of the infrared and Raman features to the local lattice distortions~\cite{guizzetti_1997}. The Fe phonon density of states was measured by nuclear inelastic scattering (NIS), showing a good agreement with the density functional theory (DFT) calculations~\cite{walterfang_2005}. Using the DFT approach, the phonon dispersion curves, phonon density of states, as well as various thermodynamic properties were obtained within the harmonic approximation~\cite{tani_2010,liang_2011}. The extended Klemens model was applied to study
the anharmonic effect on phonon frequencies and linewidths observed by the Raman spectroscopy~\cite{zhang_2023}.
The impact of nanostructurization on lattice dynamics was explored in the $\beta$-FeSi$_2$ nanorods grown on the Si(110) surface by the NIS and {\it ab initio} methods~\cite{kalt_2022}.

In this work, we investigate the lattice dynamical properties of $\beta$-FeSi$_2$ 
using the DFT calculations. We study the effect of anharmonic terms in the temperature-dependent potential on phonon frequencies and lifetimes. We focus on the Raman modes, comparing the theoretical results with the experimental data.
The thermal conductivity is derived in a broad temperature range and the effect of crystallite size is analyzed.



This study is structured as follows.
In Sec.~\ref{sec.com} we describe the details of computational methods.
Next, in Sec.~\ref{sec.result} we present and discuss our results.
In particular we present the crystal structure (Sec.~\ref{sec.crys}) and lattice dynamics (Sec.~\ref{sec.lattice}).
We investigate also the thermal conductivity comparing the obtained results with the available experimental data (Sec.~\ref{sec.thermal}).
Finally, Sec.~\ref{sec.summary} summarizes our key findings and conclusions.

\section{Calculation method}
\label{sec.com}


The calculations were performed using the projector augmented-wave potentials~\cite{blochl_1994} and the generalized gradient approximation~\cite{perdew_1996} implemented in the Vienna Ab initio Simulation Package (VASP)~\cite{kresse.hafner.94,kresse.furthmuller.96,kresse.joubert.99}. 
The lattice parameters and atomic positions were optimized in the ${\bm a} \times ({\bm b}-{\bm c}) \times ({\bm b}+{\bm c})$ supercell containing 32 formula units and four primitive cells.
The integration in the reciprocal space was conducted using the $2 \times 2 \times 2$ Monkhorst--Pack mesh~\cite{monkhorst_1976} and the cut-off energy was set to $500$~eV. For convergence conditions, we set the energy change below $10^{-5}$ and $10^{-8}$ for the ionic and electronic loops, respectively. 

The lattice dynamical properties were studied within the temperature-dependent effective potential (TDEP) approach~\cite{hellman_2013}. The atomic potential with the third and fourth order anharmonic terms was derived from interatomic forces induced by displacements of all atoms at finite temperatures.
The sets of atomic displacements were generated by the high efficiency configuration space sampling (HECSS)~\cite{jochym_2021} and forces were obtained by VASP. The interatomic force constants and phonon frequencies were calculated with the {\sc Alamode} software~\cite{tadano_2014}.

Furthermore, we have attempted to construct a {\emph{temperature independent}} 
anharmonic model. We have used combined data from all investigated temperatures 
(300, 600 and 1000~K) and fitted a large (over 15~000 free parameters), fourth-order 
interaction model to this dataset. 
Subsequently, we have used this model to calculate line profiles and positions of Raman-active modes
at multiple temperatures.

The changes in phonon frequencies induced by the anharmonic effects were investigated within two approaches.
First, the impact of the quartic anharmonic terms was included using the SCPH theory~\cite{tadano_2015}.
Second, the mode profiles (frequency shifts and line widths) were determined from the real and imaginary parts of the phonon self-energy resulting from the cubic and quartic anharmonic terms of the above mentioned large model~\cite{tadano_2018}. 
The longitudinal optic-transverse optic (LO-TO) splitting was also evaluated, using the static dielectric tensor and Born effective charges calculated within density functional perturbation theory~\cite{gajdos_2006}.

\begin{figure}[]
    \centering
    \includegraphics[width=\linewidth]{fig1_new.png}
\caption{%
(a) The conventional unit cell of $\beta$-FeSi$_2$ (with Cmca symmetry) and (b) the corresponding Brillouin zone with selected high-symmetry points.
}
  \label{fig.struct}
\end{figure}


To further characterize the vibrational properties, Raman scattering was investigated using the Phonopy-Spectroscopy package~\cite{skelton_2017}. This enabled the identification of Raman-active modes and the calculation of Raman tensors. Anharmonic force constants, derived from calculations using {\sc Alamode}, were then used to obtain theoretical line profiles for the Raman modes. The presented Raman scattering spectra combine these anharmonic line profiles with the Raman tensor amplitudes.
This analysis was also based on the large quartic model mentioned above.

Finally, the thermal conductivity was calculated as a function of temperature and crystallite size within the relaxation-time approximation (RTA) \SP{}{as implemented in {\sc Alamode}~\cite{tadano_2014}}. The phonon lifetimes were calculated from the phonon self-energy including the cubic and quartic anharmonic terms. \SP{}{The RTA provides a solution to the Boltzmann transport equation (BTE) under the assumption that scattering events are independent and can be treated through mode-resolved relaxation times.
To verify the validity of this approximation for $\beta$-FeSi$_2$, we additionally
solved the BTE iteratively.} \SP{}{These calculations were performed with
{\sc Phono3py}~\cite{togo_2023}. For cross-validation, the additional RTA calculations were performed on the same $\bm{q}$-grid as the BTE calculations, using the implementation provided by {\sc Phono3py}.}

\section{Results}
\label{sec.result}

\subsection{Crystal structure}
\label{sec.crys}

The $\beta$-FeSi$_2$ structure adopts a base-centered orthorhombic lattice with the space group Cmca (No.~64) as shown in Fig.~\ref{fig.struct}(a).
The unit cell consists of two primitive cells and contains 48 atoms.
Iron (silicon) atoms possess two nonequivalent positions: \mbox{Fe-I} and \mbox{Fe-II} (\mbox{Si-I} and \mbox{Si-II}), presented in Fig.~\ref{fig.struct}(a) as gray and purple (orange and yellow) spheres, respectively.
This crystal structure is derived from the fluorite-type lattice with strongly distorted Si cubes and Fe atoms occupying 
one-half of the central sites.
The Fe-I and Fe-II sites create different layers perpendicular to the $x$ direction, and they are separated by layers containing both Si sites.
Each Fe atom is coordinated by 8 Si atoms with slightly different Fe-Si distances.
The optimized lattice constants ($a = 9.874$~{\AA}, $b = 7.767$~{\AA}, and
$c = 7.811$~{\AA}) agree very well with the experimental data ($a = 9.863$~{\AA}, $b = 7.791$~{\AA}, and $c = 7.833$~{\AA})~\cite{dusausoy_1971}.

Iron atoms occupy the Wyckoff sites 8\textit{d} ($0.2166$, $0$, $0$) and  8\textit{f} ($0$, $0.3072$, $0.1879$), corresponding to \mbox{Fe-I} and \mbox{Fe-II}, respectively. 
Silicon atoms are located at two inequivalent 16\textit{g} positions: ($0.1282$, $0.2737$, $0.0495$) and \mbox{($0.3734$, $0.0445$, $0.2270$)}, assigned as \mbox{Si-I} and \mbox{Si-II}.
The optimized positions of atoms agree very well with the experimental data~\cite{dusausoy_1971} and the previous theoretical studies~\cite{tani_2010,liang_2011}.


\subsection{Lattice dynamics}
\label{sec.lattice}

\begin{figure}[]
    \centering
    \includegraphics[width=\linewidth]{Fig1c.pdf}
\caption{%
The phonon dispersion curves along high symmetry directions obtained within SCPH (for temperatures from $0$ to $1000$~K).
Dashed black lines indicate the phonon dispersions obtained from the harmonic approximation. The white dots indicate Raman-active modes with A$_g$ symmetry.
The vertical plot shows the phonon density of states (DOS) calculated at a reference temperature of $600$~K.
}
  \label{fig.ph_band}
\end{figure}


In Fig.~\ref{fig.ph_band} we present the phonon dispersion relations of $\beta$-FeSi$_2$ along high-symmetry directions in the Brillouin zone [Fig.~\ref{fig.struct}(b)].
Due to 24 atoms in the primitive cell, the phonon spectrum consists of 69 optical modes and three acoustic modes.
The phonon dispersions were calculated within the SCPH approach in the temperature range $0$--$1000$~K (presented by color lines in Fig.~\ref{fig.ph_band}), and they are compared with the results obtained from the harmonic part of the effective potential corresponding to temperature $T=300$~K (indicated by dashed black lines in Fig.~\ref{fig.ph_band}).
As we can see within the SCPH method, the anharmonic effects are rather weak and leads only to small renormalization of phonon frequencies.
Only the highest modes show more pronounced shifts of their frequencies to larger values.
The total and partial element-projected phonon density of states obtained within the harmonic approximation are presented in Fig.~\ref{fig.ph_band}.
Up to around $320$~cm$^{-1}$, the contributions from both elements are very similar, while for higher frequencies the spectrum is dominated by
the Si vibrations.


\begin{table}[!t]
\begin{ruledtabular}
\caption{Calculated and experimental Raman-active modes of $\beta$-FeSi$_2$ with their irreducible representations (IR). Present theoretical results are compared with the previous theoretical data from Ref.~\cite{tani_2010} and experimental results from \mbox{Refs.~\cite{lefki_1991,maeda_2004}.} The experimental frequencies with known symmetries (A$_g$) are shown in bold, while other experimental modes are assigned to the best fitting theoretical values.}
\begin{tabular}{c c c c c}
\textbf{IR} & \multicolumn{4}{c}{\textbf{Frequency (cm$^{-1}$)}} \\
 & Present  & Theor.~\cite{tani_2010} & Exp.~\cite{lefki_1991} & Exp.~\cite{maeda_2004} \\
\hline
B$_{2g}$ & 175.4 & 179 & 176 &  \\
B$_{1g}$ & 176.3 & 185 & 179  &  \\
B$_{1g}$ & 193.3 & 198 &  & 190.6 \\
A$_g$    & 196.6 & 208 & \textbf{195} & \textbf{194.0} \\
A$_g$    & 203.9 & 210 & \textbf{197} & 199.6 \\
B$_{3g}$ & 205.5 & 212 & 200 &  \\
B$_{3g}$ & 226.3 & 236 & 206 & 227.1 \\
B$_{1g}$ & 233.3 & 240 &  &  231.6 \\
B$_{2g}$ & 248.6 & 254 &  &  \\
A$_g$    & 250.1 & 257 & \textbf{247} & \textbf{247.3} \\
A$_g$    & 254.9 & 264 & \textbf{253} & 254.3 \\
B$_{1g}$ & 275.5 & 285 &  &  274.1 \\
B$_{2g}$ & 282.4 & 295 &  &  281.2 \\
B$_{3g}$ & 286.8 & 297 &  &  \\
B$_{1g}$ & 307.1 & 317 &  &  \\
B$_{2g}$ & 312.6 & 326 &  &  311.8 \\
B$_{1g}$ & 319.0 & 324 &  &  \\
B$_{3g}$ & 327.9 & 341 &  & 325.8 \\
B$_{2g}$ & 333.3 & 345 &  &  \\
A$_g$    & 339.3 & 352 & \textbf{346} & 339.5 \\
B$_{2g}$ & 343.0 & 350 &  &  \\
B$_{1g}$ & 353.5 & 366 &  &  \\
B$_{2g}$ & 372.8 & 383 &  & 370.7 \\
B$_{1g}$ & 375.1 & 385 &  &  \\
B$_{3g}$ & 375.2 & 386 &  &  \\
B$_{3g}$ & 385.3 & 401 &  &  \\
A$_g$    & 386.5 & 398 &  & 386.2 \\
B$_{2g}$ & 387.7 & 402 &  & 388.2 \\
A$_g$    & 404.6 & 415 &  & 400.4 \\
B$_{2g}$ & 405.1 & 420 &  &  \\
B$_{1g}$ & 412.7 & 428 &  &  \\
B$_{3g}$ & 418.4 & 431 &  &  \\
B$_{3g}$ & 441.4 & 458 &  & 442.6 \\
B$_{3g}$ & 447.3 & 466 &  & 446.3 \\
A$_g$    & 448.4 & 464 &  &  \\
A$_g$    & 499.1 & 517 &  &  \\
\end{tabular}
\label{T1Raman}
\end{ruledtabular}
\end{table}

%subsection{Raman}

\begin{figure*}
\centering
  \includegraphics[width=\textwidth]{spectrum_RTA+raman.pdf}
\caption{Raman spectrum of $\beta$-FeSi$_2$ calculated within the perturbative approach at three temperatures 
($300$, $600$, $900$~K -- blue, orange, and red lines, respectively). 
The calculated spectrum includes all Raman-active modes. The A$_g$ modes
are indicated by purple vertical lines at peak positions corresponding to T=300~K. 
The frequencies derived from the harmonic approximation at T=300~K are indicated by green lines.
Connecting arrows indicate the correspondence between harmonic and
anharmonic frequencies, demonstrating the frequency shifts due to 
phonon interactions. Experimental values for the A$_g$ modes
based on Ref.~\cite{lefki_1991} are marked with black dashed lines.
}
\label{raman}
\end{figure*}


The phonon spectrum at the $\Gamma$ point consists of 36 Raman modes classified according to the irreducible representations: $9A_\text{g}+9B_\text{1g}+9B_\text{2g}+9B_\text{3g}$. 
Fig.~\ref{raman} shows the Raman spectrum of A$_g$ symmetry calculated for $\beta$-FeSi$_2$ within the perturbative approach at three temperatures $300$, $600$, and $900$~K (solid blue, orange, and red curves, respectively), including third-order and fourth-order anharmonic corrections.
The calculated Raman spectrum includes all Raman-active modes.
The five experimentally observed A$_g$ modes are highlighted by black dashed lines, based on data from Ref.~\cite{lefki_1991}.
Additional peaks not marked with vertical lines correspond to phonon modes with symmetries other than A$_g$.
The frequencies of Raman modes obtained in anharmonic calculations are compared with the previous results calculated within the harmonic approximation and the experimental values in Tab.~\ref{T1Raman}. 
We have marked in bold the experimentally determined A$_g$ modes, which are compared with the calculations.
Since the experimental studies did not provide the accurate assignement of the Raman modes with the B$_{1g}$, B$_{2g}$, and B$_{3g}$ symmetry~\cite{lefki_1991,maeda_2004}, we cannot compare them directly with the theoretical results.
However, in Tab.~\ref{T1Raman} we have assigned the measured frequencies to the best fitting theoretical values without taking into account the symmetry of the modes, except for the known A$_{g}$ modes. 


The impact of anharmonicity on the phonon frequencies is well visible from the comparison of the results obtained within the harmonic approximation 
and from the anharmonic calculation (vertical green and purple lines in Fig.~\ref{raman}, respectively).
Here we show only the A$_g$ modes, which are compared with the experimental results (vertical black dashed lines).
Anharmonic frequencies calculated at $300$~K are indicated by purple lines, while frequencies derived from harmonic approximation are marked with green lines. The grey solid lines connect corresponding modes obtained in both approximations.
In most cases, the results obtained within the harmonic approximation do not agree with the experimental frequencies. 
%Only the modes close to $250$~cm$^{-1}$ and $400$~cm$^{-1}$ correspond well to the experimental values. MS
As we can see, the inclusion of the anharmonic correction leads to a significant modification of the phonon frequencies.
These anharmonic effects are stronger for higher-frequency modes mainly because of
the dominant contribution from Si atoms, which vibrate with larger amplitudes than heavier Fe atoms. 
When atoms move to larger distances the potential deviates more from the harmonic approximation,
and the anharmonic corrections become stronger.


The modification of phonon frequencies observed in Fig.~\ref{raman} is much larger than in the SCPH scheme presented in Fig.~\ref{fig.ph_band}. The SCPH approach includes only the leading-order contribution to 
the phonon self-energy obtained from the quartic anharmonic terms~\cite{tadano_2015}. Therefore, it does not describe fully the changes of phonon frequencies found within the perturbation theory (see Fig.~\ref{raman}).
Especially, it is well visible for two highest A$_g$ modes, which exhibit also the largest line broadening 
and the strongest dependence on temperature.
Therefore, a better agreement with experimentally observed frequencies is visible,
confirming the significant influence of the anharmonicity on the frequencies and line profiles of phonon modes.
In fact, the decrease of phonon frequencies should be even stronger due to thermal expansion, 
which is not included in our calculations.
Within SCPH the frequencies of the highest modes increase with increasing temperature as we see in Fig.~\ref{fig.ph_band}. The comparison of two different approaches applied to study anharmonic properties of $\beta$-FeSi$_2$ shows that the perturbation theory, which includes the cubic and quartic terms, better describes the changes of phonon frequencies with temperature than the SCPH method. \SP{}{This indicates that the leading-order contribution included in SCPH are not important in this material.}

Additionaly we should nottice that for other than A$_g$ modes we cannot make an unambiguous assignment of theoretical frequencies to experimental ones. Note that the spectrum in Fig.~\ref{fig.ph_band} contains all Raman-active modes. 
The limitation to A$_g$ modes concerns only the indicated positions of the peaks. 

\SP{}{The full Raman spectrum, with the frequencies of the B$_{1g}$, B$_{2g}$ and B$_{3g}$ modes marked, is shown in Appendix A, Fig.~\ref{raman1}. This represents our theoretical prediction of possible Raman-mode assignments, which can be verified in future experiments.}
%This is the theoretical prediction of possible assignment of Raman modes that can be verified in future experiments.  


\subsection{Thermal conductivity}
\label{sec.thermal}


\begin{figure*}[!t]
\centering
  \includegraphics[width=\linewidth]{time3.pdf}
\caption{Phonon lifetimes calculated for three temperatures as a function of phonon frequency. The colors correspond to the phonon branches.}
  \label{thermtime}
\end{figure*}


In this section, we analyze the thermal conductivity tensor of $\beta$-FeSi$_2$ obtained within the RTA approach~\cite{tadano_2018} as a function of temperature
%
\begin{equation}
\kappa_{\text{ph}}^{\mu\nu}(T) = \frac{1}{NV} \sum_{\bm{q},j} c_{\bm{q}j}(T) v_{\bm{q}j}^{\mu} v_{\bm{q}j}^{\nu}\tau_{\bm{q}j}(T),
\end{equation}
% 
where $c_{\bm{q}j}$ is the mode heat capacity and $v_{\bm{q}j}$ is the mode group velocity. 
The relaxation time is approximated by the phonon lifetime $\tau_{\bm{q}j}$
calculated for $j$-th branch at the wave vector $\bm{q}$.
$V$ is the unit cell volume and $N$ is the number of unit cells in the crystal.
The phonon lifetime is calculated using this formula
%
\begin{equation}
\tau_{\bm{q}j}(T)=\frac{1}{2\Gamma_{\bm{q}j}^{\text{anh}}(T)},
\end{equation}
%
where $\Gamma_{\bm{q}j}^{\text{anh}}$ is the anharmonic phonon linewidth obtained from 
the imaginary part of the phonon self-energy within the perturbation theory.

In Fig.~\ref{thermtime}, we present $\tau_{\bm{q}j}$ obtained for three temperatures 300, 600, and 1000~K as a function of frequency. As we see, the acoustic phonons close to the $\Gamma$ point have the longest lifetimes,
which are diminished with increasing frequency reaching local minima around $200$~cm$^{-1}$.
For higher frequencies, phonon lifetimes first increase to local maxima around $300$~cm$^{-1}$ and then decrease to
the lowest values in the range of highest optical modes. The shortest lifetimes correspond to the largest line broadening
observed for the Raman modes in Fig~\ref{raman}. The phonon group velocities 
$v_{\bm{q}j}=\partial\omega_{\bm{q}j}/\partial\bm{q}$, which are obtained by the central difference formula, are presented in Fig.~\ref{thermvelocity}. Their temperature dependence is negligible, therefore, we present only the results for $T=600$~K.
At low frequencies, there are clearly two ranges of group velocities of the acoustic phonons. 
The larger values correspond to the longitudinal modes, while the lower values are obtained from the
transverse acoustic branches. Group velocities of acoustic phonons decrease for larger frequencies
and reach the average values typical for optic branches.

\begin{figure}[!t]
\centering
  \includegraphics[width=\linewidth]{GV.pdf}
\caption{Mode group velocities calculated as a function of phonon frequency. The colors correspond to the phonon branches.
}
  \label{thermvelocity}
\end{figure}

In Fig.~\ref{anizotropy}(a), we present the three diagonal elements of $\kappa_{\text{ph}}^{\mu\nu}$ corresponding to the main directions of the crystal structure. 
They were obtained from the force constants calculated at the base temperature $T=600$~K and the crystallite size $0.1$~$\mu$m to account for boundary-limited phonon transport. 
Due to the orthorhombic symmetry, we observe a small anisotropy in phonon transport in the whole temperature range. 
At low temperatures, the three components of the heat conductivity increase in a very similar way with the $\kappa_{\text{ph}}^{yy}$ element slightly larger than two other components. 
After reaching the maximum, we observe a change in the largest component from $\kappa_{\text{ph}}^{yy}$ to 
$\kappa_{\text{ph}}^{xx}$.    
In Fig.~\ref{anizotropy}(b), the thermal conductivity is shown for three base temperatures, at which the interatomic potential was obtained ($300$~K, $600$~K, and $1000$~K), using the energy expansion up to third- and fourth-order anharmonic terms, and the same structure size of $0.1$~$\mu$m. At lowest temperatures, the thermal conductivity strongly increases, reaching the maximum around $T=180$~K, then it shows a slower decrease with temperature.
The differences between the two levels of approximation are minimal, suggesting that third-order calculations already capture the dominant phonon scattering mechanisms. The dependence on the base temperature is also very weak, showing the changes in the heat conductivity within a few percent. 
\SP{}{Further verification of the reliability of our thermal conductivity results is given in Appendix~\ref{ThermalAB}, where the full BTE calculations show good agreement with RTA and higher-resolution RTA results, demonstrating that the $8\times8\times8$ $\bm{q}$-mesh already provides converged values.}

\begin{figure}[!t]
\centering
  \includegraphics[width=\linewidth]{Anizotropy.pdf}
\caption{(a) The anisotropic thermal conductivity of $\beta$-FeSi$_2$ calculated along the lattice directions at 600~K. (b) The average temperature-dependent thermal conductivity taken at $300$~K, $600$~K, and $1000$~K, including anharmonic corrections up to cubic (A3) and quartic (A4) terms. In both cases the crystallite size is 0.1~$\mu$m.}
  \label{anizotropy}
\end{figure}


In Fig.~\ref{therm}, we fix the base temperature at $600$~K and examine the effect of crystallite size on thermal conductivity, varying it from $0.01$ to $0.5$~$\mu$m.
With decreasing the crystallite size, we observe a shift of the position of the maximum to larger temperatures and a decrease of the thermal conductivity in the entire temperature range.
Theoretical results are compared with several experimental data obtained above the room temperature. 
The measured thermal conductivity depends to a large extent on the sample quality, its purity and the size of the crystalline grains which depends on 
the production processes.
Many measurements were performed using crystallites of micrometric or unknown size ~\cite{waldecker_1973,ito_2002,kim_2003,du_2020}, however, 
numerous attempts to minimize $\kappa$ by reducing grain sizes to $56$~nm~\cite{dabrowski_2019, dabrowski_microstructure_2021}, $30$-$400$~nm~\cite{le_tonquesse_2019}, $50$ and $200$~nm~\cite{abbassi_2021}, or introducing pores into the material~\cite{sam2023improved} 
are also carried out. 
Another way to change the thermal conductivity is to dope $\beta$-FeSi$_2$ with different elements~\cite{ito_2002,kim_2003,du_2020,cheng_2024}, however, this effect is beyond our investigation.

\begin{figure}[!t]
\centering
  \includegraphics[width=\linewidth]{Thermal_conductivity3.pdf}
\caption{
The phonon thermal conductivity of $\beta$-FeSi$_2$: theoretical results for the infinite crystalline size and with boundary conditions, compared with experimental data for different structure sizes.
}
  \label{therm}
\end{figure}
%Fig.~\ref{therm}\cite{abbassi_2021}\cite{waldecker_1973}\cite{dabrowski_2019}\cite{sam2023improved}\cite{kim_2003}\cite{du_2020}\cite{ito_2002}\cite{le_tonquesse_2019}.

We observe a decrease in the thermal conductivity with reducing crystalline grain sizes in all analyzed experimental data. 
For instance, by decreasing the crystallite size to $50$~nm, the thermal conductivity at room temperature was reduced by a factor of $1.7$, what can be  compared to the annealed sample with 200 nm grains~\cite{abbassi_2021}. 
It is worth noting that the rate of decrease in value with increasing temperature in both cases, for grain sizes of $50$~nm and $200$~nm, is significantly different, which is consistent with our calculations. 
The same trend can be observed by comparing the thermal conductivity measured for a sample with bulk crystallite sizes with the thermal conductivity of a sample with grains smaller than 400 nm~\cite{le_tonquesse_2019}.
The theoretical results obtained for the same crystallite size show higher values due to factors not captured in the idealized model, such as crystal imperfection or mechanical strain. Usually, a decrease in the crystallite size is related to an increased concentration of grain boundaries, point defects, and stacking faults that influence the phonon scattering~\cite{le_tonquesse_2019,abbassi_2021}.   

We should note that the total thermal conductivity is a combination
of the lattice and electronic contributions to the heat transport.
In semiconductors, the electronic thermal conductivity is negligible at low temperatures and significantly increases 
only much above the room temperatures~\cite{gu_2020}.
For $\beta$-FeSi$_2$, the electronic thermal conductivity was obtained from the electric conductivity using the Wiedemann-Franz law~\cite{ito_2002,kim_2003,le_tonquesse_2019}.
In the undoped material, its value does not exceed $0.1$~W/mK in the measurement up to $T=950$~K~\cite{kim_2003}.
By doping, the electronic thermal conductivity can be enhanced, and it has a direct impact on the thermoelectric properties of $\beta$-FeSi$_2$ at high temperatures~\cite{ito_2002,kim_2003}.
In the present study, we consider only the phonon contribution to the thermal conductivity,
therefore, agreement with experimental data may deteriorate with increasing temperature.

\section{Summary}
\label{sec.summary}

We performed {\it ab initio} studies on lattice dynamical and thermal transport properties of $\beta$-FeSi$_2$. The effect of anharmonicity was analyzed within two approaches -- the SCPH method and the perturbation theory.
The phonon dispersion curves obtained within SCPH show small renormalization of frequencies comparing to the harmonic approximation. 
The Raman spectra were calculated within the procedure which takes into account the peak intensities obtained from the Raman tensors and the line profiles obtained from the phonon self energy derived within the perturbation theory based on the large, temperature-independent, quartic model fitted to the data from the wide range of temperatures (300-1000~K). 
The anharmonic corrections strongly affect the frequencies and line profiles of some modes and results in overall better agreement with the experimental data. 
We analyzed the phonon lifetimes and group velocities obtained as functions of the phonon frequency.
Then the lattice thermal conductivity was calculated for a broad range of temperatures and grain sizes.
We found a small anisotropy in the phonon thermal transport resulting from the orthorhombic structure and a weak effect of the quartic anharmonic terms. 
The thermal conductivity calculated for various crystalline grain sizes show a good qualitative agreement with the available measurements.

\begin{acknowledgments}
Some figures in this work were rendered using {\sc Vesta}~\cite{momma.izumi.11} software.
This work was partially supported by the Ministry of Education, Youth and Sports of the Czech Republic through the e-INFRA CZ (ID:90254).
\end{acknowledgments}

\appendix

\section{Raman spectrum}
\label{RamanAA}

Based on the polarized Raman measurements reported in Ref.~\cite{maeda_2004}, two Raman peaks were identified as belonging to the A$_g$ symmetry class, and several additional peaks were observed with similar or different polarization dependence. Although the authors of Ref.~\cite{maeda_2004} provided estimates of the relative Raman tensor components, they did not specify which of the remaining modes correspond to the B$_{1g}$, B$_{2g}$, or B$_{3g}$ symmetries. Because of this missing experimental information, a direct symmetry-resolved comparison between the measurement and theory is not currently possible for the non-A$_g$ modes. 
To provide a complete theoretical picture of the B$_g$-type modes, we show here the calculated Raman-active frequencies and intensities for the B$_{1g}$, B$_{2g}$, and B$_{3g}$ symmetries only. 
Fig.~\ref{raman1} shows the predicted Raman modes for the B$_g$ symmetries in $\beta$-FeSi$_2$. The results obtained within the harmonic approximation are compared with the anharmonic perturbation theory calculations which provides both, frequency shifts and predicted line profiles of the modes.
Although the experimentally measured peaks cannot be directly assigned to these symmetries due to the lack of polarization-resolved data, the theoretical predictions provide a reference for comparison. Matching the measured frequencies to the closest theoretical B$_g$ modes (Table~\ref{T1Raman}) allows for a tentative assignment, which can guide future polarization-resolved Raman experiments aimed at determining the precise symmetry of the unresolved peaks.

\begin{figure*}[t]
\centering
  \includegraphics[width=\linewidth]{spectrum_RTA+raman_Bi_modes.pdf}
\caption{Raman spectrum of $\beta$-FeSi$_2$ calculated at three temperatures 
($300$, $600$, $900$~K -- blue, orange, and red lines, respectively). 
The calculated spectrum includes all Raman-active modes. The B$_{ig}$ modes
are indicated by purple vertical lines at peak positions corresponding to T=300~K. 
The frequencies derived from the harmonic approximation at T=300~K are indicated 
by green, yellow and pink lines.
Connecting arrows indicate the correspondence between harmonic and anharmonic 
frequencies, demonstrating the frequency shifts due to phonon interactions.}
\label{raman1}
\end{figure*}
\section{Thermal conductivity obtained from BTE and RTA}
\label{ThermalAB}

The thermal conductivity was computed by solving the full BTE on the largest feasible $\bm{q}$-point grid, $8\times8\times8$, and compared with the corresponding RTA results obtained on the same grid. As shown in Fig.~\ref{bte}, the difference between the components of the thermal conductivity tensor obtained within BTE and RTA at this resolution is very small, indicating a good agreement between these two approaches.
Moreover, we performed an additional calculation using RTA on a denser $20\times20\times20$ grid. As seen in the Fig.~\ref{bte}, the higher-resolution data remain in a good agreement with both the BTE and RTA results for the $8\times8\times8$ grid.
It shows that the $8\times8\times8$ mesh already provides good results for this structure and confirms reliability of the calculations.

\begin{figure}[!h]
\centering
  \includegraphics[width=\linewidth]{BTEvsRTAvsANP.pdf}
\caption{Thermal conductivity of $\beta$-FeSi$_2$ obtained within the BTE and RTA methods using the Phono3py software on the $8\times8\times8$ q-point grid, compared with the RTA results computed with ALAMODE on a denser $20\times20\times20$ grid.}
\label{bte}
\end{figure}

%\section*{Data availability}
%The data that support the findings of this article are openly available~\footnote{give me DOI}.
% see https://journals.aps.org/authors/data-availability-statements#citation

\bibliography{refs.bib}
%\bibliographystyle{ieeetr}


\end{document}

\documentclass[%
%reprint,
superscriptaddress,
%groupedaddress,
longbibliography,
%unsortedaddress,
%runinaddress,
%frontmatterverbose, 
%preprint,
%preprintnumbers,
%nofootinbib,
nobibnotes,
%bibnotes,
amsmath,amssymb,
aps,
%pra,
prb,
%rmp,
%prstab,
%prstper,
%showkeys,
floatfix,
twocolumn
]{revtex4-2}

\usepackage{graphicx}% Include figure files
\usepackage{calc}% Calculate margins
\usepackage{dcolumn}% Align table columns on decimal point
\usepackage{bm}% bold math

\usepackage[urlcolor=blue,colorlinks=true,citecolor=blue,linkcolor=blue,pdfstartview={FitH},bookmarks=false]{hyperref} % add hypertext capabilities

%\usepackage[mathlines]{lineno} % Enable numbering of text and display math
% \linenumbers\relax % Commence numbering lines

% \usepackage[showframe,%Uncomment any one of the following lines to test 
% %scale=0.7, marginratio={1:1, 2:3}, ignoreall,% default settings
% %text={7in,10in},centering,
% %margin=1.5in,
% % total={6.5in,8.75in}, top=1.2in, left=0.9in, includefoot,
% % height=10in,a5paper,hmargin={3cm,0.8in},
% ]{geometry}

\usepackage{amsmath}
\usepackage{amssymb}
%\usepackage{orcidlink}
\usepackage{xcolor}
%\usepackage{datetime}
\usepackage[normalem]{ulem}

% Change tracking commands
\newcommand{\trackchange}[3]{\textcolor{#3}{\sout{#1}#2}}  % Full color strikeout, insert
%\renewcommand{\trackchange}[3]{\textcolor{#3}{#2}}        % Just color silent remove and insert
%\renewcommand{\trackchange}[3]{{#2}}                      % No indication, silent remove and insert

% Author marker definitions

\definecolor{myblue}{RGB}{0,127,85}
% \definecolor{violet}{RGB}{102,0,204}
% \definecolor{orange}{RGB}{255,128,0}
% \definecolor{green}{RGB}{0,128,0}
\newcommand{\DL}[1]{\trackchange{}{#1}{blue}}
\newcommand{\AP}[1]{\trackchange{}{#1}{red}}
\newcommand{\PJ}[2]{\trackchange{#1}{#2}{orange}}
\newcommand{\JL}[2]{\trackchange{#1}{#2}{myblue}}
\newcommand{\SP}[2]{\trackchange{#1}{#2}{blue}}
\newcommand{\PP}[2]{\trackchange{#1}{#2}{teal}}
\newcommand{\AI}[2]{\trackchange{#1}{#2}{olive}}
\newcommand{\MS}[1]{\trackchange{}{#1}{purple}}

\newcommand{\TODO}[1]{\textcolor{red}{TODO: #1}}

\sloppy

\begin{document}

\title{Ab initio study of the anharmonic properties and thermal conductivity in $\beta$-FeSi$_2$}

\author{Svitlana~Pastukh}
\email[e-mail: ]{svitlana.pastukh@ifj.edu.pl}
\affiliation{Institute of Nuclear Physics, Polish Academy of Sciences, ul. W. E. Radzikowskiego 152, 31-342 Krak\'{o}w, Poland}

\author{Ma\l{}gorzata~Sternik}
\affiliation{Institute of Nuclear Physics, Polish Academy of Sciences, ul. W. E. Radzikowskiego 152, 31-342 Krak\'{o}w, Poland}

\author{Pawe\l{}~T.~Jochym}
\affiliation{Institute of Nuclear Physics, Polish Academy of Sciences, ul. W. E. Radzikowskiego 152, 31-342 Krak\'{o}w, Poland}


\author{Jan~\L{}a\.{z}ewski}
\affiliation{Institute of Nuclear Physics, Polish Academy of Sciences, ul. W. E. Radzikowskiego 152, 31-342 Krak\'{o}w, Poland}

\author{Andrzej~Ptok}
\affiliation{Institute of Nuclear Physics, Polish Academy of Sciences, ul. W. E. Radzikowskiego 152, 31-342 Krak\'{o}w, Poland}

\author{Svetoslav~Stankov}
\affiliation{Institute for Photon Science and Synchrotron Radiation, Karlsruhe Institute of Technology, D-76131 Karlsruhe, Germany}
\affiliation{Laboratory for Applications of Synchrotron Radiation, Karlsruhe Institute of Technology, D-76131 Karlsruhe, Germany}

\author{Przemys\l{}aw~Piekarz}
\affiliation{Institute of Nuclear Physics, Polish Academy of Sciences, ul. W. E. Radzikowskiego 152, 31-342 Krak\'{o}w, Poland}

\date{\today}

\begin{abstract}

Iron silicides are good candidates for applications in  optoelectronic and thermoelectric devices.
Lattice dynamical properties and thermal conductivity in the $\beta$-FeSi$_2$ semiconductor
are investigated with the first-principles computational methods. 
Phonon dispersion relations are calculated via the
temperature-dependent effective potential method and self-consistent phonon theory. 
To properly model thermal transport, we explicitly consider
the impact of phonon-phonon interactions by analyzing
anharmonic contributions to the phonon self-energy. 
This yields temperature-dependent phonon frequencies and linewidths,
reflecting the finite lifetime of phonons due to scattering
processes. The calculated phonon frequencies and line profiles are used to obtain 
the Raman spectra, which shows good agreement with the experimental data. 
We revealed an enhanced anharmonic behaviour of the Raman modes with the highest frequencies.  
The lattice thermal conductivity is then obtained as a function of temperature and crystallite size within
the relaxation-time approximation.
Phonon transport shows a small anisotropy due to the orthorhombic structure and a very weak dependence
on the quartic anharmonic corrections. The results obtained for an infinite material and for several crystallite sizes
were analyzed and compared with the available experimental data.
\end{abstract}

\maketitle


\section{Introduction}

The comprehensive determination of important physical properties of
crystals, such as thermal expansion, lattice thermal
conductivity or structural phase transitions, requires a fundamental 
understanding of the anharmonic effects.
Although the investigation of anharmonic interactions in crystals has
attracted a considerable interest for decades~\cite{cowley_1968}, a substantial progress
has only recently been achieved thanks to advances in theoretical and
numerical methods and increased computational power.
Now, phonon frequencies, lifetimes, and heat transfer in a wide range of
materials
can be quantitatively predicted using the available computational resources
based on the density functional theory (DFT)~\cite{lindsay_2013,mcgaughey_2019,lindsay_2019}.
In the case of strongly anharmonic systems, the self-consistent phonon
(SCPH) theory~\cite{tadano_2015} as well as the perturbative approach~\cite{tadano_2018}, using higher
order interatomic force constants derived from the fitting to the displacement force data obtained
with DFT, have proven to be successful.

Transition-metal silicides are promising materials for fabrication of electronic components
designed for integration with silicon-based circuits~\cite{murarka_1995}.
At room temperature, iron disilicide ($\beta$-FeSi$_2$) is a
direct-bandgap semiconductor~\cite{bost_1985}, making this material a good
candidate for application in optoelectronic devices such as infrared
detectors or light emitters~\cite{bost_1988}. The development of light-emitting diodes utilizing FeSi$_2$/Si heterostructures has been successfully demonstrated~\cite{leong_1997,suemasu_2001}. Due to
a high thermal stability and strong light absorption, FeSi$_2$ is
also a suitable photovoltaic material~\cite{powalla_1993,liu_2006,okuhara_2017}.

$\beta$-FeSi$_2$ crystallizes in the base-centered orthorhombic
lattice~\cite{dusausoy_1971} transforming to the
tetragonal metallic $\alpha$-FeSi$_{2}$ phase around $1200$~K~\cite{starke_2002}.
Optical studies indicated a direct band gap of the values
$0.85$--$0.89$~eV~\cite{bost_1988,dimitriadis_1990,arushanov_1995,wan_2003},
however, the {\it ab initio} calculations predicted a smaller indirect gap close to 0.8 eV~\cite{christensen_1990}. 
The existence of such an indirect gap was then confirmed by the optical linear transmittance measurements at
low temperatures~\cite{giannini_1992}. As shown by first-principles studies,
the character of the band gap is very sensitive to the orientation 
of a crystal grown on silicon~\cite{clark_1998}.

$\beta$-FeSi$_2$ belongs also to good thermoelectric materials~\cite{ware_1964}, with potential applications resulting from its chemical stability up to high temperatures, nontoxicity, and low cost of preparation~\cite{yamada_2012,nozariasbmarz_2017}. 
It has already been implemented in cars~\cite{birkholz_1988} and portable power sources~\cite{uemura_1989}. 
Its thermoelectric performance can be improved by doping~\cite{ito_2001,tani_2001,kim_2003,chen_2005,pandey_2013,le_tonquesse_2019}, 
which enhances the electric transport and reduces the thermal conductivity~\cite{waldecker_1973,du_2019,du_2020}.  
The thermal conductivity can be also reduced by the modification of microstructure~\cite{ail_2015} or by
nanostructurization~\cite{watanabe_2017,taniguchi_2017,hsin_2017,abbassi_2021}.

The lattice thermal conductivity is directly connected with anharmonic effects and phonon scattering processes.
The vibrational properties of \mbox{$\beta$-FeSi$_2$} were studied by the infrared and Raman spectroscopy~\cite{lefki_1991,guizzetti_1997,maeda_2004,baleva_2008,liu_2011,maeda_2011}. The observed anisotropy in the phonon spectra results from the enhanced sensitivity of the infrared and Raman features to the local lattice distortions~\cite{guizzetti_1997}. The Fe phonon density of states was measured by nuclear inelastic scattering (NIS), showing a good agreement with the density functional theory (DFT) calculations~\cite{walterfang_2005}. Using the DFT approach, the phonon dispersion curves, phonon density of states, as well as various thermodynamic properties were obtained within the harmonic approximation~\cite{tani_2010,liang_2011}. The extended Klemens model was applied to study
the anharmonic effect on phonon frequencies and linewidths observed by the Raman spectroscopy~\cite{zhang_2023}.
The impact of nanostructurization on lattice dynamics was explored in the $\beta$-FeSi$_2$ nanorods grown on the Si(110) surface by the NIS and {\it ab initio} methods~\cite{kalt_2022}.

In this work, we investigate the lattice dynamical properties of $\beta$-FeSi$_2$ 
using the DFT calculations. We study the effect of anharmonic terms in the temperature-dependent potential on phonon frequencies and lifetimes. We focus on the Raman modes, comparing the theoretical results with the experimental data.
The thermal conductivity is derived in a broad temperature range and the effect of crystallite size is analyzed.



This study is structured as follows.
In Sec.~\ref{sec.com} we describe the details of computational methods.
Next, in Sec.~\ref{sec.result} we present and discuss our results.
In particular we present the crystal structure (Sec.~\ref{sec.crys}) and lattice dynamics (Sec.~\ref{sec.lattice}).
We investigate also the thermal conductivity comparing the obtained results with the available experimental data (Sec.~\ref{sec.thermal}).
Finally, Sec.~\ref{sec.summary} summarizes our key findings and conclusions.

\section{Calculation method}
\label{sec.com}


The calculations were performed using the projector augmented-wave potentials~\cite{blochl_1994} and the generalized gradient approximation~\cite{perdew_1996} implemented in the Vienna Ab initio Simulation Package (VASP)~\cite{kresse.hafner.94,kresse.furthmuller.96,kresse.joubert.99}. 
The lattice parameters and atomic positions were optimized in the ${\bm a} \times ({\bm b}-{\bm c}) \times ({\bm b}+{\bm c})$ supercell containing 32 formula units and four primitive cells.
The integration in the reciprocal space was conducted using the $2 \times 2 \times 2$ Monkhorst--Pack mesh~\cite{monkhorst_1976} and the cut-off energy was set to $500$~eV. For convergence conditions, we set the energy change below $10^{-5}$ and $10^{-8}$ for the ionic and electronic loops, respectively. 

The lattice dynamical properties were studied within the temperature-dependent effective potential (TDEP) approach~\cite{hellman_2013}. The atomic potential with the third and fourth order anharmonic terms was derived from interatomic forces induced by displacements of all atoms at finite temperatures.
The sets of atomic displacements were generated by the high efficiency configuration space sampling (HECSS)~\cite{jochym_2021} and forces were obtained by VASP. The interatomic force constants and phonon frequencies were calculated with the {\sc Alamode} software~\cite{tadano_2014}.

Furthermore, we have attempted to construct a {\emph{temperature independent}} 
anharmonic model. We have used combined data from all investigated temperatures 
(300, 600 and 1000~K) and fitted a large (over 15~000 free parameters), fourth-order 
interaction model to this dataset. 
Subsequently, we have used this model to calculate line profiles and positions of Raman-active modes
at multiple temperatures.

The changes in phonon frequencies induced by the anharmonic effects were investigated within two approaches.
First, the impact of the quartic anharmonic terms was included using the SCPH theory~\cite{tadano_2015}.
Second, the mode profiles (frequency shifts and line widths) were determined from the real and imaginary parts of the phonon self-energy resulting from the cubic and quartic anharmonic terms of the above mentioned large model~\cite{tadano_2018}. 
The longitudinal optic-transverse optic (LO-TO) splitting was also evaluated, using the static dielectric tensor and Born effective charges calculated within density functional perturbation theory~\cite{gajdos_2006}.

\begin{figure}[]
    \centering
    \includegraphics[width=\linewidth]{fig1_new.png}
\caption{%
(a) The conventional unit cell of $\beta$-FeSi$_2$ (with Cmca symmetry) and (b) the corresponding Brillouin zone with selected high-symmetry points.
}
  \label{fig.struct}
\end{figure}

% To further characterize the vibrational properties, Raman-active scattering was investigated using the Phonopy-Spectroscopy package~\cite{skelton_2017}. This enabled the identification of Raman-active modes and the calculation of Raman tensors. Anharmonic force constants, obtained from calculations using {\sc Alamode}, were then used to obtain theoretical line profiles for the Raman modes. The presented Raman scattering spectra combine these anharmonic line profiles with the Raman tensor amplitudes.
% This analysis was also based on the large quartic model mentioned above.

% Finally, the thermal conductivity was obtained as a function of temperature and crystallite size within the relaxation-time approximation (RTA) \PJ{}{as implemented in {\sc Alamode}\cite{tadano_2014}}. The phonon lifetimes were calculated from the phonon self-energy including the cubic and quartic anharmonic terms. \SP{}{The RTA provides a solution of the Boltzmann transport equation (BTE) under the assumption that scattering events are independent and can be treated through mode-resolved relaxation times.
% To verify the validity of this approximation for $\beta$-FeSi$_2$, we have additionally
% executed an iterative solution of the BTE.}\PJ{}{These calculations were performed with
% {\sc Phono3py}\cite{togo_2023}. For cross-validation, the additional RTA calculations were performed on the same $\bm{q}$-grid as BTE with implementation provided by {\sc Phono3py}.}.

To further characterize the vibrational properties, Raman scattering was investigated using the Phonopy-Spectroscopy package~\cite{skelton_2017}. This enabled the identification of Raman-active modes and the calculation of Raman tensors. Anharmonic force constants, derived from calculations using {\sc Alamode}, were then used to obtain theoretical line profiles for the Raman modes. The presented Raman scattering spectra combine these anharmonic line profiles with the Raman tensor amplitudes.
This analysis was also based on the large quartic model described above.

Finally, the thermal conductivity was calculated as a function of temperature and crystallite size within the relaxation-time approximation (RTA) \PJ{}{as implemented in {\sc Alamode}~\cite{tadano_2014}}. The phonon lifetimes were calculated from the phonon self-energy including the cubic and quartic anharmonic terms. \SP{}{The RTA provides a solution to the Boltzmann transport equation (BTE) under the assumption that scattering events are independent and can be treated through mode-resolved relaxation times.
To verify the validity of this approximation for $\beta$-FeSi$_2$, we additionally
solved the BTE iteratively.} \PJ{}{These calculations were performed with
{\sc Phono3py}~\cite{togo_2023}. For cross-validation, the additional RTA calculations were performed on the same $\bm{q}$-grid as the BTE calculations, using the implementation provided by {\sc Phono3py}.}

% (see. Appendix~\ref{ThermalAB} for the comparison)\cite{togo_2023}.}
%As shown in Appendix~\ref{ThermalAB}, the iterative BTE results exhibit good agreement with the RTA values, confirming that RTA is sufficiently accurate for this material and for the considered temperature range.}


\section{Results}
\label{sec.result}

\subsection{Crystal structure}
\label{sec.crys}

The $\beta$-FeSi$_2$ structure adopts a base-centered orthorhombic lattice with the space group Cmca (No.~64) as shown in Fig.~\ref{fig.struct}(a).
The unit cell consists of two primitive cells and contains 48 atoms.
Iron (silicon) atoms possess two nonequivalent positions: \mbox{Fe-I} and \mbox{Fe-II} (\mbox{Si-I} and \mbox{Si-II}), presented in Fig.~\ref{fig.struct}(a) as gray and purple (orange and yellow) spheres, respectively.
This crystal structure is derived from the fluorite-type lattice with strongly distorted Si cubes and Fe atoms occupying 
one-half of the central sites.
The Fe-I and Fe-II sites create different layers perpendicular to the $x$ direction, and they are separated by layers containing both Si sites.
Each Fe atom is coordinated by 8 Si atoms with slightly different Fe-Si distances.
The optimized lattice constants ($a = 9.874$~{\AA}, $b = 7.767$~{\AA}, and
$c = 7.811$~{\AA}) agree very well with the experimental data ($a = 9.863$~{\AA}, $b = 7.791$~{\AA}, and $c = 7.833$~{\AA})~\cite{dusausoy_1971}.

Iron atoms occupy the Wyckoff sites 8\textit{d} ($0.2166$, $0$, $0$) and  8\textit{f} ($0$, $0.3072$, $0.1879$), corresponding to \mbox{Fe-I} and \mbox{Fe-II}, respectively. 
Silicon atoms are located at two inequivalent 16\textit{g} positions: ($0.1282$, $0.2737$, $0.0495$) and \mbox{($0.3734$, $0.0445$, $0.2270$)}, assigned as \mbox{Si-I} and \mbox{Si-II}.
The optimized positions of atoms agree very well with the experimental data~\cite{dusausoy_1971} and the previous theoretical studies~\cite{tani_2010,liang_2011}.


\subsection{Lattice dynamics}
\label{sec.lattice}

\begin{figure}[]
    \centering
    \includegraphics[width=\linewidth]{Fig1c.pdf}
\caption{%
The phonon dispersion curves along high symmetry directions obtained within SCPH (for temperatures from $0$ to $1000$~K).
Dashed black lines indicate the phonon dispersions obtained from the harmonic approximation. The white dots indicate Raman-active modes with A$_g$ symmetry.
The vertical plot shows the phonon density of states (DOS) calculated at a reference temperature of $600$~K.
}
  \label{fig.ph_band}
\end{figure}


In Fig.~\ref{fig.ph_band} we present the phonon dispersion relations of $\beta$-FeSi$_2$ along high-symmetry directions in the Brillouin zone [Fig.~\ref{fig.struct}(b)].
Due to 24 atoms in the primitive cell, the phonon spectrum consists of 69 optical modes and three acoustic modes.
The phonon dispersions were calculated within the SCPH approach in the temperature range $0$--$1000$~K (presented by color lines in Fig.~\ref{fig.ph_band}), and they are compared with the results obtained from the harmonic part of the effective potential corresponding to temperature $T=300$~K (indicated by dashed black lines in Fig.~\ref{fig.ph_band}).
As we can see within the SCPH method, the anharmonic effects are rather weak and leads only to small renormalization of phonon frequencies.
Only the highest modes show more pronounced shifts of their frequencies to larger values.
The total and partial element-projected phonon density of states obtained within the harmonic approximation are presented in Fig.~\ref{fig.ph_band}.
Up to around $320$~cm$^{-1}$, the contributions from both elements are very similar, while for higher frequencies the spectrum is dominated by
the Si vibrations.


\begin{table}[!t]
\begin{ruledtabular}
\caption{Calculated and experimental Raman-active modes of $\beta$-FeSi$_2$ with their irreducible representations (IR). Present theoretical results are compared with the previous theoretical data from Ref.~\cite{tani_2010} and experimental results from \mbox{Refs.~\cite{lefki_1991,maeda_2004}.} The experimental frequencies with known symmetries (A$_g$) are shown in bold, while other experimental modes are assigned to the best fitting theoretical values.}
\begin{tabular}{c c c c c}
\textbf{IR} & \multicolumn{4}{c}{\textbf{Frequency (cm$^{-1}$)}} \\
 & Present  & Theor.~\cite{tani_2010} & Exp.~\cite{lefki_1991} & Exp.~\cite{maeda_2004} \\
\hline
B$_{2g}$ & 175.4 & 179 & 176 &  \\
B$_{1g}$ & 176.3 & 185 & 179  &  \\
B$_{1g}$ & 193.3 & 198 &  & 190.6 \\
A$_g$    & 196.6 & 208 & \textbf{195} & \textbf{194.0} \\
A$_g$    & 203.9 & 210 & \textbf{197} & 199.6 \\
B$_{3g}$ & 205.5 & 212 & 200 &  \\
B$_{3g}$ & 226.3 & 236 & 206 & 227.1 \\
B$_{1g}$ & 233.3 & 240 &  &  231.6 \\
B$_{2g}$ & 248.6 & 254 &  &  \\
A$_g$    & 250.1 & 257 & \textbf{247} & \textbf{247.3} \\
A$_g$    & 254.9 & 264 & \textbf{253} & 254.3 \\
B$_{1g}$ & 275.5 & 285 &  &  274.1 \\
B$_{2g}$ & 282.4 & 295 &  &  281.2 \\
B$_{3g}$ & 286.8 & 297 &  &  \\
B$_{1g}$ & 307.1 & 317 &  &  \\
B$_{2g}$ & 312.6 & 326 &  &  311.8 \\
B$_{1g}$ & 319.0 & 324 &  &  \\
B$_{3g}$ & 327.9 & 341 &  & 325.8 \\
B$_{2g}$ & 333.3 & 345 &  &  \\
A$_g$    & 339.3 & 352 & \textbf{346} & 339.5 \\
B$_{2g}$ & 343.0 & 350 &  &  \\
B$_{1g}$ & 353.5 & 366 &  &  \\
B$_{2g}$ & 372.8 & 383 &  & 370.7 \\
B$_{1g}$ & 375.1 & 385 &  &  \\
B$_{3g}$ & 375.2 & 386 &  &  \\
B$_{3g}$ & 385.3 & 401 &  &  \\
A$_g$    & 386.5 & 398 &  & 386.2 \\
B$_{2g}$ & 387.7 & 402 &  & 388.2 \\
A$_g$    & 404.6 & 415 &  & 400.4 \\
B$_{2g}$ & 405.1 & 420 &  &  \\
B$_{1g}$ & 412.7 & 428 &  &  \\
B$_{3g}$ & 418.4 & 431 &  &  \\
B$_{3g}$ & 441.4 & 458 &  & 442.6 \\
B$_{3g}$ & 447.3 & 466 &  & 446.3 \\
A$_g$    & 448.4 & 464 &  &  \\
A$_g$    & 499.1 & 517 &  &  \\
\end{tabular}
\label{T1Raman}
\end{ruledtabular}
\end{table}

%subsection{Raman}

\begin{figure*}
\centering
  \includegraphics[width=\textwidth]{spectrum_RTA+raman.pdf}
\caption{Raman spectrum of $\beta$-FeSi$_2$ calculated within the perturbative approach at three temperatures 
($300$, $600$, $900$~K -- blue, orange, and red lines, respectively). 
The calculated spectrum includes all Raman-active modes. The A$_g$ modes
are indicated by purple vertical lines at peak positions corresponding to T=300~K. 
The frequencies derived from the harmonic approximation at T=300~K are indicated by green lines.
Connecting arrows indicate the correspondence between harmonic and
anharmonic frequencies, demonstrating the frequency shifts due to 
phonon interactions. Experimental values for the A$_g$ modes
based on Ref.~\cite{lefki_1991} are marked with black dashed lines.
}
\label{raman}
\end{figure*}


The phonon spectrum at the $\Gamma$ point consists of 36 Raman modes classified according to the irreducible representations: $9A_\text{g}+9B_\text{1g}+9B_\text{2g}+9B_\text{3g}$. 
Fig.~\ref{raman} shows the Raman spectrum of A$_g$ symmetry calculated for $\beta$-FeSi$_2$ within the perturbative approach at three temperatures $300$, $600$, and $900$~K (solid blue, orange, and red curves, respectively), including third-order and fourth-order anharmonic corrections.
The calculated Raman spectrum includes all Raman-active modes.
The five experimentally observed A$_g$ modes are highlighted by black dashed lines, based on data from Ref.~\cite{lefki_1991}.
Additional peaks not marked with vertical lines correspond to phonon modes with symmetries other than A$_g$.
The frequencies of Raman modes obtained in anharmonic calculations are compared with the previous results calculated within the harmonic approximation and the experimental values in Tab.~\ref{T1Raman}. 
We have marked in bold the experimentally determined A$_g$ modes, which are compared with the calculations.
Since the experimental studies did not provide the accurate assignement of the Raman modes with the B$_{1g}$, B$_{2g}$, and B$_{3g}$ symmetry~\cite{lefki_1991,maeda_2004}, we cannot compare them directly with the theoretical results.
However, in Tab.~\ref{T1Raman} we have assigned the measured frequencies to the best fitting theoretical values without taking into account the symmetry of the modes, except for the known A$_{g}$ modes. 


The impact of anharmonicity on the phonon frequencies is well visible from the comparison of the results obtained within the harmonic approximation 
and from the anharmonic calculation (vertical green and purple lines in Fig.~\ref{raman}, respectively).
Here we show only the A$_g$ modes, which are compared with the experimental results (vertical black dashed lines).
Anharmonic frequencies calculated at $300$~K are indicated by purple lines, while frequencies derived from harmonic approximation are marked with green lines. The grey solid lines connect corresponding modes obtained in both approximations.
In most cases, the results obtained within the harmonic approximation do not agree with the experimental frequencies. 
%Only the modes close to $250$~cm$^{-1}$ and $400$~cm$^{-1}$ correspond well to the experimental values. MS
As we can see, the inclusion of the anharmonic correction leads to a significant modification of the phonon frequencies.
These anharmonic effects are stronger for higher-frequency modes mainly because of
the dominant contribution from Si atoms, which vibrate with larger amplitudes than heavier Fe atoms. 
When atoms move to larger distances the potential deviates more from the harmonic approximation,
and the anharmonic corrections become stronger.


The modification of phonon frequencies observed in Fig.~\ref{raman} is much larger than in the SCPH scheme presented in Fig.~\ref{fig.ph_band}. The SCPH approach includes only the leading-order contribution to 
the phonon self-energy obtained from the quartic anharmonic terms~\cite{tadano_2015}. Therefore, it does not describe fully the changes of phonon frequencies found within the perturbation theory (see Fig.~\ref{raman}).
Especially, it is well visible for two highest A$_g$ modes, which exhibit also the largest line broadening 
and the strongest dependence on temperature.
Therefore, a better agreement with experimentally observed frequencies is visible,
confirming the significant influence of the anharmonicity on the frequencies and line profiles of phonon modes.
In fact, the decrease of phonon frequencies should be even stronger due to thermal expansion, 
which is not included in our calculations.
Within SCPH the frequencies of the highest modes increase with increasing temperature as we see in Fig.~\ref{fig.ph_band}. The comparison of two different approaches applied to study anharmonic properties of $\beta$-FeSi$_2$ shows that the perturbation theory, which includes the cubic and quartic terms, better describes the changes of phonon frequencies with temperature than the SCPH method. \PJ{}{This indicates that the leading-order contribution included in SCPH are not important in this material.}

Additionaly we should nottice that for other than A$_g$ modes we cannot make an unambiguous assignment of theoretical frequencies to experimental ones. Note that the spectrum in Fig.~\ref{fig.ph_band} contains all Raman-active modes. 
The limitation to A$_g$ modes concerns only the indicated positions of the peaks. 
%\SP{}{Furthermore, we present the full spectrum for all Raman active modes with the comparison of the harmonic and anharmonic calculations in the Appendix~\ref{RamanAA}.}
\textcolor{red}{The full Raman spectrum, with the frequencies of the B$_{1g}$, B$_{2g}$ and B$_{3g}$ modes marked, is shown in Appendix A, Fig.~\ref{raman1}. This represents our theoretical prediction of possible Raman-mode assignments, which can be verified in future experiments.}
%This is the theoretical prediction of possible assignment of Raman modes that can be verified in future experiments.  


\subsection{Thermal conductivity}
\label{sec.thermal}


\begin{figure*}[!t]
\centering
  \includegraphics[width=\linewidth]{time3.pdf}
\caption{Phonon lifetimes calculated for three temperatures as a function of phonon frequency. The colors correspond to the phonon branches.}
  \label{thermtime}
\end{figure*}


In this section, we analyze the thermal conductivity tensor of $\beta$-FeSi$_2$ obtained within the RTA approach~\cite{tadano_2018} as a function of temperature
%
\begin{equation}
\kappa_{\text{ph}}^{\mu\nu}(T) = \frac{1}{NV} \sum_{\bm{q},j} c_{\bm{q}j}(T) v_{\bm{q}j}^{\mu} v_{\bm{q}j}^{\nu}\tau_{\bm{q}j}(T),
\end{equation}
% 
where $c_{\bm{q}j}$ is the mode heat capacity and $v_{\bm{q}j}$ is the mode group velocity. 
The relaxation time is approximated by the phonon lifetime $\tau_{\bm{q}j}$
calculated for $j$-th branch at the wave vector $\bm{q}$.
$V$ is the unit cell volume and $N$ is the number of unit cells in the crystal.
The phonon lifetime is calculated using this formula
%
\begin{equation}
\tau_{\bm{q}j}(T)=\frac{1}{2\Gamma_{\bm{q}j}^{\text{anh}}(T)},
\end{equation}
%
where $\Gamma_{\bm{q}j}^{\text{anh}}$ is the anharmonic phonon linewidth obtained from 
the imaginary part of the phonon self-energy within the perturbation theory.

In Fig.~\ref{thermtime}, we present $\tau_{\bm{q}j}$ obtained for three temperatures 300, 600, and 1000~K as a function of frequency. As we see, the acoustic phonons close to the $\Gamma$ point have the longest lifetimes,
which are diminished with increasing frequency reaching local minima around $200$~cm$^{-1}$.
For higher frequencies, phonon lifetimes first increase to local maxima around $300$~cm$^{-1}$ and then decrease to
the lowest values in the range of highest optical modes. The shortest lifetimes correspond to the largest line broadening
observed for the Raman modes in Fig~\ref{raman}. The phonon group velocities 
$v_{\bm{q}j}=\partial\omega_{\bm{q}j}/\partial\bm{q}$, which are obtained by the central difference formula, are presented in Fig.~\ref{thermvelocity}. Their temperature dependence is negligible, therefore, we present only the results for $T=600$~K.
At low frequencies, there are clearly two ranges of group velocities of the acoustic phonons. 
The larger values correspond to the longitudinal modes, while the lower values are obtained from the
transverse acoustic branches. Group velocities of acoustic phonons decrease for larger frequencies
and reach the average values typical for optic branches.

\begin{figure}[!t]
\centering
  \includegraphics[width=\linewidth]{GV.pdf}
\caption{Mode group velocities calculated as a function of phonon frequency. The colors correspond to the phonon branches.
}
  \label{thermvelocity}
\end{figure}

In Fig.~\ref{anizotropy}(a), we present the three diagonal elements of $\kappa_{\text{ph}}^{\mu\nu}$ corresponding to the main directions of the crystal structure. 
They were obtained from the force constants calculated at the base temperature $T=600$~K and the crystallite size $0.1$~$\mu$m to account for boundary-limited phonon transport. 
Due to the orthorhombic symmetry, we observe a small anisotropy in phonon transport in the whole temperature range. 
At low temperatures, the three components of the heat conductivity increase in a very similar way with the $\kappa_{\text{ph}}^{yy}$ element slightly larger than two other components. 
After reaching the maximum, we observe a change in the largest component from $\kappa_{\text{ph}}^{yy}$ to 
$\kappa_{\text{ph}}^{xx}$.    
In Fig.~\ref{anizotropy}(b), the thermal conductivity is shown for three base temperatures, at which the interatomic potential was obtained ($300$~K, $600$~K, and $1000$~K), using the energy expansion up to third- and fourth-order anharmonic terms, and the same structure size of $0.1$~$\mu$m. At lowest temperatures, the thermal conductivity strongly increases, reaching the maximum around $T=180$~K, then it shows a slower decrease with temperature.
The differences between the two levels of approximation are minimal, suggesting that third-order calculations already capture the dominant phonon scattering mechanisms. The dependence on the base temperature is also very weak, showing the changes in the heat conductivity within a few percent. 
\SP{}{Further verification of the reliability of our thermal conductivity results is given in Appendix~\ref{ThermalAB}, where the full BTE calculations show good agreement with RTA and higher-resolution RTA results, demonstrating that the $8\times8\times8$ $\bm{q}$-mesh already provides converged values.}

\begin{figure}[!t]
\centering
  \includegraphics[width=\linewidth]{Anizotropy.pdf}
\caption{(a) The anisotropic thermal conductivity of $\beta$-FeSi$_2$ calculated along the lattice directions at 600~K. (b) The average temperature-dependent thermal conductivity taken at $300$~K, $600$~K, and $1000$~K, including anharmonic corrections up to cubic (A3) and quartic (A4) terms. In both cases the crystallite size is 0.1~$\mu$m.}
  \label{anizotropy}
\end{figure}


In Fig.~\ref{therm}, we fix the base temperature at $600$~K and examine the effect of crystallite size on thermal conductivity, varying it from $0.01$ to $0.5$~$\mu$m.
With decreasing the crystallite size, we observe a shift of the position of the maximum to larger temperatures and a decrease of the thermal conductivity in the entire temperature range.
Theoretical results are compared with several experimental data obtained above the room temperature. 
The measured thermal conductivity depends to a large extent on the sample quality, its purity and the size of the crystalline grains which depends on 
the production processes.
Many measurements were performed using crystallites of micrometric or unknown size ~\cite{waldecker_1973,ito_2002,kim_2003,du_2020}, however, 
numerous attempts to minimize $\kappa$ by reducing grain sizes to $56$~nm~\cite{dabrowski_2019, dabrowski_microstructure_2021}, $30$-$400$~nm~\cite{le_tonquesse_2019}, $50$ and $200$~nm~\cite{abbassi_2021}, or introducing pores into the material~\cite{sam2023improved} 
are also carried out. 
Another way to change the thermal conductivity is to dope $\beta$-FeSi$_2$ with different elements~\cite{ito_2002,kim_2003,du_2020,cheng_2024}, however, this effect is beyond our investigation.

\begin{figure}[!t]
\centering
  \includegraphics[width=\linewidth]{Thermal_conductivity3.pdf}
\caption{
The phonon thermal conductivity of $\beta$-FeSi$_2$: theoretical results for the infinite crystalline size and with boundary conditions, compared with experimental data for different structure sizes.
}
  \label{therm}
\end{figure}
%Fig.~\ref{therm}\cite{abbassi_2021}\cite{waldecker_1973}\cite{dabrowski_2019}\cite{sam2023improved}\cite{kim_2003}\cite{du_2020}\cite{ito_2002}\cite{le_tonquesse_2019}.

We observe a decrease in the thermal conductivity with reducing crystalline grain sizes in all analyzed experimental data. 
For instance, by decreasing the crystallite size to $50$~nm, the thermal conductivity at room temperature was reduced by a factor of $1.7$, what can be  compared to the annealed sample with 200 nm grains~\cite{abbassi_2021}. 
It is worth noting that the rate of decrease in value with increasing temperature in both cases, for grain sizes of $50$~nm and $200$~nm, is significantly different, which is consistent with our calculations. 
The same trend can be observed by comparing the thermal conductivity measured for a sample with bulk crystallite sizes with the thermal conductivity of a sample with grains smaller than 400 nm~\cite{le_tonquesse_2019}.
The theoretical results obtained for the same crystallite size show higher values due to factors not captured in the idealized model, such as crystal imperfection or mechanical strain. Usually, a decrease in the crystallite size is related to an increased concentration of grain boundaries, point defects, and stacking faults that influence the phonon scattering~\cite{le_tonquesse_2019,abbassi_2021}.   

We should note that the total thermal conductivity is a combination
of the lattice and electronic contributions to the heat transport.
In semiconductors, the electronic thermal conductivity is negligible at low temperatures and significantly increases 
only much above the room temperatures~\cite{gu_2020}.
For $\beta$-FeSi$_2$, the electronic thermal conductivity was obtained from the electric conductivity using the Wiedemann-Franz law~\cite{ito_2002,kim_2003,le_tonquesse_2019}.
In the undoped material, its value does not exceed $0.1$~W/mK in the measurement up to $T=950$~K~\cite{kim_2003}.
By doping, the electronic thermal conductivity can be enhanced, and it has a direct impact on the thermoelectric properties of $\beta$-FeSi$_2$ at high temperatures~\cite{ito_2002,kim_2003}.
In the present study, we consider only the phonon contribution to the thermal conductivity,
therefore, agreement with experimental data may deteriorate with increasing temperature.

\section{Summary}
\label{sec.summary}

We performed {\it ab initio} studies on lattice dynamical and thermal transport properties of $\beta$-FeSi$_2$. The effect of anharmonicity was analyzed within two approaches -- the SCPH method and the perturbation theory.
The phonon dispersion curves obtained within SCPH show small renormalization of frequencies comparing to the harmonic approximation. 
The Raman spectra were calculated within the procedure which takes into account the peak intensities obtained from the Raman tensors and the line profiles obtained from the phonon self energy derived within the perturbation theory based on the large, temperature-independent, quartic model fitted to the data from the wide range of temperatures (300-1000~K). 
The anharmonic corrections strongly affect the frequencies and line profiles of some modes and results in overall better agreement with the experimental data. 
We analyzed the phonon lifetimes and group velocities obtained as functions of the phonon frequency.
Then the lattice thermal conductivity was calculated for a broad range of temperatures and grain sizes.
We found a small anisotropy in the phonon thermal transport resulting from the orthorhombic structure and a weak effect of the quartic anharmonic terms. 
The thermal conductivity calculated for various crystalline grain sizes show a good qualitative agreement with the available measurements.

\begin{acknowledgments}
Some figures in this work were rendered using {\sc Vesta}~\cite{momma.izumi.11} software.
This work was partially supported by the Ministry of Education, Youth and Sports of the Czech Republic through the e-INFRA CZ (ID:90254).
\end{acknowledgments}

\appendix

\section{Raman spectrum}
\label{RamanAA}

Based on the polarized Raman measurements reported in Ref.~\cite{maeda_2004}, two Raman peaks were identified as belonging to the A$_g$ symmetry class, and several additional peaks were observed with similar or different polarization dependence. Although the authors of Ref.~\cite{maeda_2004} provided estimates of the relative Raman tensor components, they did not specify which of the remaining modes correspond to the B$_{1g}$, B$_{2g}$, or B$_{3g}$ symmetries. Because of this missing experimental information, a direct symmetry-resolved comparison between the measurement and theory is not currently possible for the non-A$_g$ modes. 
To provide a complete theoretical picture of the B$_g$-type modes, we show here the calculated Raman-active frequencies and intensities for the B$_{1g}$, B$_{2g}$, and B$_{3g}$ symmetries only. 
%These results represent the predicted Raman modes for the B$_g$ symmetries in $\beta$-FeSi$_2$. 
\textcolor{red} {Fig.~\ref{raman1} shows the predicted Raman modes for the B$_g$ symmetries in $\beta$-FeSi$_2$. The results obtained within the harmonic approximation are compared with the anharmonic perturbation theory calculations which provides both, frequency shifts and predicted line profiles of the modes.}
Although the experimentally measured peaks cannot be directly assigned to these symmetries due to the lack of polarization-resolved data, the theoretical predictions provide a reference for comparison. Matching the measured frequencies to the closest theoretical B$_g$ modes (Table~\ref{T1Raman}) allows for a tentative assignment, which can guide future polarization-resolved Raman experiments aimed at determining the precise symmetry of the unresolved peaks.

\begin{figure*}[t]
\centering
  \includegraphics[width=\linewidth]{spectrum_RTA+raman_Bi_modes.pdf}
\caption{Raman spectrum of $\beta$-FeSi$_2$ calculated at three temperatures 
($300$, $600$, $900$~K -- blue, orange, and red lines, respectively). 
The calculated spectrum includes all Raman-active modes. The B$_{ig}$ modes
are indicated by purple vertical lines at peak positions corresponding to T=300~K. 
The frequencies derived from the harmonic approximation at T=300~K are indicated 
by green, yellow and pink lines.
Connecting arrows indicate the correspondence between harmonic and anharmonic 
frequencies, demonstrating the frequency shifts due to phonon interactions.}
\label{raman1}
\end{figure*}
\section{Thermal conductivity obtained from BTE and RTA}
\label{ThermalAB}

The thermal conductivity was computed by solving the full BTE on the largest feasible $\bm{q}$-point grid, $8\times8\times8$, and compared with the corresponding RTA results obtained on the same grid. As shown in Fig.~\ref{bte}, the difference between the components of the thermal conductivity tensor obtained within BTE and RTA at this resolution is very small, indicating a good agreement between these two approaches.
%THIS PART SHOULD GO RATHER TO THE RESPONSE
%Extending the full BTE calculation to larger grids is computationally prohibitive: the computational cost of BTE is roughly two orders of %magnitude higher than that of RTA, and the required memory and runtime exceed our available resources. 
Moreover, we performed an additional calculation using RTA on a denser $20\times20\times20$ grid. As seen in the Fig.~\ref{bte}, the higher-resolution data remain in a good agreement with both the BTE and RTA results for the $8\times8\times8$ grid.
It shows that the $8\times8\times8$ mesh already provides good results for this structure and confirms reliability of the calculations.

\begin{figure}[!h]
\centering
  \includegraphics[width=\linewidth]{BTEvsRTAvsANP.pdf}
\caption{Thermal conductivity of $\beta$-FeSi$_2$ obtained within the BTE and RTA methods using the Phono3py software on the $8\times8\times8$ q-point grid, compared with the RTA results computed with ALAMODE on a denser $20\times20\times20$ grid.}
\label{bte}
\end{figure}

%\section*{Data availability}
%The data that support the findings of this article are openly available~\footnote{give me DOI}.
% see https://journals.aps.org/authors/data-availability-statements#citation

\bibliography{refs.bib}
%\bibliographystyle{ieeetr}


\end{document}

\documentclass[%
%reprint,
superscriptaddress,
%groupedaddress,
longbibliography,
%unsortedaddress,
%runinaddress,
%frontmatterverbose, 
%preprint,
%preprintnumbers,
%nofootinbib,
nobibnotes,
%bibnotes,
amsmath,amssymb,
aps,
%pra,
prb,
%rmp,
%prstab,
%prstper,
%showkeys,
floatfix,
twocolumn
]{revtex4-2}

\usepackage{graphicx}% Include figure files
\usepackage{calc}% Calculate margins
\usepackage{dcolumn}% Align table columns on decimal point
\usepackage{bm}% bold math

\usepackage[urlcolor=blue,colorlinks=true,citecolor=blue,linkcolor=blue,pdfstartview={FitH},bookmarks=false]{hyperref} % add hypertext capabilities

%\usepackage[mathlines]{lineno} % Enable numbering of text and display math
% \linenumbers\relax % Commence numbering lines

% \usepackage[showframe,%Uncomment any one of the following lines to test 
% %scale=0.7, marginratio={1:1, 2:3}, ignoreall,% default settings
% %text={7in,10in},centering,
% %margin=1.5in,
% % total={6.5in,8.75in}, top=1.2in, left=0.9in, includefoot,
% % height=10in,a5paper,hmargin={3cm,0.8in},
% ]{geometry}

\usepackage{amsmath}
\usepackage{amssymb}
%\usepackage{orcidlink}
\usepackage{xcolor}
%\usepackage{datetime}
\usepackage[normalem]{ulem}

% Change tracking commands
\newcommand{\trackchange}[3]{\textcolor{#3}{\sout{#1}#2}}  % Full color strikeout, insert
%\renewcommand{\trackchange}[3]{\textcolor{#3}{#2}}        % Just color silent remove and insert
%\renewcommand{\trackchange}[3]{{#2}}                      % No indication, silent remove and insert

% Author marker definitions

\definecolor{myblue}{RGB}{0,127,85}
% \definecolor{violet}{RGB}{102,0,204}
% \definecolor{orange}{RGB}{255,128,0}
% \definecolor{green}{RGB}{0,128,0}
\newcommand{\DL}[1]{\trackchange{}{#1}{blue}}
\newcommand{\AP}[1]{\trackchange{}{#1}{red}}
\newcommand{\PJ}[2]{\trackchange{#1}{#2}{orange}}
\newcommand{\JL}[2]{\trackchange{#1}{#2}{myblue}}
\newcommand{\SP}[2]{\trackchange{#1}{#2}{blue}}
\newcommand{\PP}[2]{\trackchange{#1}{#2}{teal}}
\newcommand{\AI}[2]{\trackchange{#1}{#2}{olive}}
\newcommand{\MS}[1]{\trackchange{}{#1}{purple}}

\newcommand{\TODO}[1]{\textcolor{red}{TODO: #1}}

\sloppy

\begin{document}

\title{Ab initio study of the anharmonic properties and thermal conductivity in $\beta$-FeSi$_2$}

\author{Svitlana~Pastukh}
\email[e-mail: ]{svitlana.pastukh@ifj.edu.pl}
\affiliation{Institute of Nuclear Physics, Polish Academy of Sciences, ul. W. E. Radzikowskiego 152, 31-342 Krak\'{o}w, Poland}

\author{Ma\l{}gorzata~Sternik}
\affiliation{Institute of Nuclear Physics, Polish Academy of Sciences, ul. W. E. Radzikowskiego 152, 31-342 Krak\'{o}w, Poland}

\author{Pawe\l{}~T.~Jochym}
\affiliation{Institute of Nuclear Physics, Polish Academy of Sciences, ul. W. E. Radzikowskiego 152, 31-342 Krak\'{o}w, Poland}


\author{Jan~\L{}a\.{z}ewski}
\affiliation{Institute of Nuclear Physics, Polish Academy of Sciences, ul. W. E. Radzikowskiego 152, 31-342 Krak\'{o}w, Poland}

\author{Andrzej~Ptok}
\affiliation{Institute of Nuclear Physics, Polish Academy of Sciences, ul. W. E. Radzikowskiego 152, 31-342 Krak\'{o}w, Poland}

\author{Svetoslav~Stankov}
\affiliation{Institute for Photon Science and Synchrotron Radiation, Karlsruhe Institute of Technology, D-76131 Karlsruhe, Germany}
\affiliation{Laboratory for Applications of Synchrotron Radiation, Karlsruhe Institute of Technology, D-76131 Karlsruhe, Germany}

\author{Przemys\l{}aw~Piekarz}
\affiliation{Institute of Nuclear Physics, Polish Academy of Sciences, ul. W. E. Radzikowskiego 152, 31-342 Krak\'{o}w, Poland}

\date{\today}

\begin{abstract}

Iron silicides are good candidates for applications in  optoelectronic and thermoelectric devices.
Lattice dynamical properties and thermal conductivity in the $\beta$-FeSi$_2$ semiconductor
are investigated with the first-principles computational methods. 
Phonon dispersion relations are calculated via the
temperature-dependent effective potential method and self-consistent phonon theory. 
To properly model thermal transport, we explicitly consider
the impact of phonon-phonon interactions by analyzing
anharmonic contributions to the phonon self-energy. 
This yields temperature-dependent phonon frequencies and linewidths,
reflecting the finite lifetime of phonons due to scattering
processes. The calculated phonon frequencies and line profiles are used to obtain 
the Raman spectra, which shows good agreement with the experimental data. 
We revealed an enhanced anharmonic behaviour of the Raman modes with the highest frequencies.  
The lattice thermal conductivity is then obtained as a function of temperature and crystallite size within
the relaxation-time approximation.
Phonon transport shows a small anisotropy due to the orthorhombic structure and a very weak dependence
on the quartic anharmonic corrections. The results obtained for an infinite material and for several crystallite sizes
were analyzed and compared with the available experimental data.
\end{abstract}

\maketitle


\section{Introduction}

The comprehensive determination of important physical properties of
crystals, such as thermal expansion, lattice thermal
conductivity or structural phase transitions, requires a fundamental 
understanding of the anharmonic effects.
Although the investigation of anharmonic interactions in crystals has
attracted a considerable interest for decades~\cite{cowley_1968}, a substantial progress
has only recently been achieved thanks to advances in theoretical and
numerical methods and increased computational power.
Now, phonon frequencies, lifetimes, and heat transfer in a wide range of
materials
can be quantitatively predicted using the available computational resources
based on the density functional theory (DFT)~\cite{lindsay_2013,mcgaughey_2019,lindsay_2019}.
In the case of strongly anharmonic systems, the self-consistent phonon
(SCPH) theory~\cite{tadano_2015} as well as the perturbative approach~\cite{tadano_2018}, using higher
order interatomic force constants derived from the fitting to the displacement force data obtained
with DFT, have proven to be successful.

Transition-metal silicides are promising materials for fabrication of electronic components
designed for integration with silicon-based circuits~\cite{murarka_1995}.
At room temperature, iron disilicide ($\beta$-FeSi$_2$) is a
direct-bandgap semiconductor~\cite{bost_1985}, making this material a good
candidate for application in optoelectronic devices such as infrared
detectors or light emitters~\cite{bost_1988}. The development of light-emitting diodes utilizing FeSi$_2$/Si heterostructures has been successfully demonstrated~\cite{leong_1997,suemasu_2001}. Due to
a high thermal stability and strong light absorption, FeSi$_2$ is
also a suitable photovoltaic material~\cite{powalla_1993,liu_2006,okuhara_2017}.

$\beta$-FeSi$_2$ crystallizes in the base-centered orthorhombic
lattice~\cite{dusausoy_1971} transforming to the
tetragonal metallic $\alpha$-FeSi$_{2}$ phase around $1200$~K~\cite{starke_2002}.
Optical studies indicated a direct band gap of the values
$0.85$--$0.89$~eV~\cite{bost_1988,dimitriadis_1990,arushanov_1995,wan_2003},
however, the {\it ab initio} calculations predicted a smaller indirect gap close to 0.8 eV~\cite{christensen_1990}. 
The existence of such an indirect gap was then confirmed by the optical linear transmittance measurements at
low temperatures~\cite{giannini_1992}. As shown by first-principles studies,
the character of the band gap is very sensitive to the orientation 
of a crystal grown on silicon~\cite{clark_1998}.

$\beta$-FeSi$_2$ belongs also to good thermoelectric materials~\cite{ware_1964}, with potential applications resulting from its chemical stability up to high temperatures, nontoxicity, and low cost of preparation~\cite{yamada_2012,nozariasbmarz_2017}. 
It has already been implemented in cars~\cite{birkholz_1988} and portable power sources~\cite{uemura_1989}. 
Its thermoelectric performance can be improved by doping~\cite{ito_2001,tani_2001,kim_2003,chen_2005,pandey_2013,le_tonquesse_2019}, 
which enhances the electric transport and reduces the thermal conductivity~\cite{waldecker_1973,du_2019,du_2020}.  
The thermal conductivity can be also reduced by the modification of microstructure~\cite{ail_2015} or by
nanostructurization~\cite{watanabe_2017,taniguchi_2017,hsin_2017,abbassi_2021}.

The lattice thermal conductivity is directly connected with anharmonic effects and phonon scattering processes.
The vibrational properties of \mbox{$\beta$-FeSi$_2$} were studied by the infrared and Raman spectroscopy~\cite{lefki_1991,guizzetti_1997,maeda_2004,baleva_2008,liu_2011,maeda_2011}. The observed anisotropy in the phonon spectra results from the enhanced sensitivity of the infrared and Raman features to the local lattice distortions~\cite{guizzetti_1997}. The Fe phonon density of states was measured by nuclear inelastic scattering (NIS), showing a good agreement with the density functional theory (DFT) calculations~\cite{walterfang_2005}. Using the DFT approach, the phonon dispersion curves, phonon density of states, as well as various thermodynamic properties were obtained within the harmonic approximation~\cite{tani_2010,liang_2011}. The extended Klemens model was applied to study
the anharmonic effect on phonon frequencies and linewidths observed by the Raman spectroscopy~\cite{zhang_2023}.
The impact of nanostructurization on lattice dynamics was explored in the $\beta$-FeSi$_2$ nanorods grown on the Si(110) surface by the NIS and {\it ab initio} methods~\cite{kalt_2022}.

In this work, we investigate the lattice dynamical properties of $\beta$-FeSi$_2$ 
using the DFT calculations. We study the effect of anharmonic terms in the temperature-dependent potential on phonon frequencies and lifetimes. We focus on the Raman modes, comparing the theoretical results with the experimental data.
The thermal conductivity is derived in a broad temperature range and the effect of crystallite size is analyzed.



This study is structured as follows.
In Sec.~\ref{sec.com} we describe the details of computational methods.
Next, in Sec.~\ref{sec.result} we present and discuss our results.
In particular we present the crystal structure (Sec.~\ref{sec.crys}) and lattice dynamics (Sec.~\ref{sec.lattice}).
We investigate also the thermal conductivity comparing the obtained results with the available experimental data (Sec.~\ref{sec.thermal}).
Finally, Sec.~\ref{sec.summary} summarizes our key findings and conclusions.

\section{Calculation method}
\label{sec.com}


The calculations were performed using the projector augmented-wave potentials~\cite{blochl_1994} and the generalized gradient approximation~\cite{perdew_1996} implemented in the Vienna Ab initio Simulation Package (VASP)~\cite{kresse.hafner.94,kresse.furthmuller.96,kresse.joubert.99}. 
The lattice parameters and atomic positions were optimized in the ${\bm a} \times ({\bm b}-{\bm c}) \times ({\bm b}+{\bm c})$ supercell containing 32 formula units and four primitive cells.
The integration in the reciprocal space was conducted using the $2 \times 2 \times 2$ Monkhorst--Pack mesh~\cite{monkhorst_1976} and the cut-off energy was set to $500$~eV. For convergence conditions, we set the energy change below $10^{-5}$ and $10^{-8}$ for the ionic and electronic loops, respectively. 

The lattice dynamical properties were studied within the temperature-dependent effective potential (TDEP) approach~\cite{hellman_2013}. The atomic potential with the third and fourth order anharmonic terms was derived from interatomic forces induced by displacements of all atoms at finite temperatures.
The sets of atomic displacements were generated by the high efficiency configuration space sampling (HECSS)~\cite{jochym_2021} and forces were obtained by VASP. The interatomic force constants and phonon frequencies were calculated with the {\sc Alamode} software~\cite{tadano_2014}.

Furthermore, we have attempted to construct a {\emph{temperature independent}} 
anharmonic model. We have used combined data from all investigated temperatures 
(300, 600 and 1000~K) and fitted a large (over 15~000 free parameters), fourth-order 
interaction model to this dataset. 
Subsequently, we have used this model to calculate line profiles and positions of Raman-active modes
at multiple temperatures.

The changes in phonon frequencies induced by the anharmonic effects were investigated within two approaches.
First, the impact of the quartic anharmonic terms was included using the SCPH theory~\cite{tadano_2015}.
Second, the mode profiles (frequency shifts and line widths) were determined from the real and imaginary parts of the phonon self-energy resulting from the cubic and quartic anharmonic terms of the above mentioned large model~\cite{tadano_2018}. 
The longitudinal optic-transverse optic (LO-TO) splitting was also evaluated, using the static dielectric tensor and Born effective charges calculated within density functional perturbation theory~\cite{gajdos_2006}.

\begin{figure}[]
    \centering
    \includegraphics[width=\linewidth]{fig1_new.png}
\caption{%
(a) The conventional unit cell of $\beta$-FeSi$_2$ (with Cmca symmetry) and (b) the corresponding Brillouin zone with selected high-symmetry points.
}
  \label{fig.struct}
\end{figure}

% To further characterize the vibrational properties, Raman-active scattering was investigated using the Phonopy-Spectroscopy package~\cite{skelton_2017}. This enabled the identification of Raman-active modes and the calculation of Raman tensors. Anharmonic force constants, obtained from calculations using {\sc Alamode}, were then used to obtain theoretical line profiles for the Raman modes. The presented Raman scattering spectra combine these anharmonic line profiles with the Raman tensor amplitudes.
% This analysis was also based on the large quartic model mentioned above.

% Finally, the thermal conductivity was obtained as a function of temperature and crystallite size within the relaxation-time approximation (RTA) \PJ{}{as implemented in {\sc Alamode}\cite{tadano_2014}}. The phonon lifetimes were calculated from the phonon self-energy including the cubic and quartic anharmonic terms. \SP{}{The RTA provides a solution of the Boltzmann transport equation (BTE) under the assumption that scattering events are independent and can be treated through mode-resolved relaxation times.
% To verify the validity of this approximation for $\beta$-FeSi$_2$, we have additionally
% executed an iterative solution of the BTE.}\PJ{}{These calculations were performed with
% {\sc Phono3py}\cite{togo_2023}. For cross-validation, the additional RTA calculations were performed on the same $\bm{q}$-grid as BTE with implementation provided by {\sc Phono3py}.}.

To further characterize the vibrational properties, Raman scattering was investigated using the Phonopy-Spectroscopy package~\cite{skelton_2017}. This enabled the identification of Raman-active modes and the calculation of Raman tensors. Anharmonic force constants, derived from calculations using {\sc Alamode}, were then used to obtain theoretical line profiles for the Raman modes. The presented Raman scattering spectra combine these anharmonic line profiles with the Raman tensor amplitudes.
This analysis was also based on the large quartic model described above.

Finally, the thermal conductivity was calculated as a function of temperature and crystallite size within the relaxation-time approximation (RTA) \PJ{}{as implemented in {\sc Alamode}~\cite{tadano_2014}}. The phonon lifetimes were calculated from the phonon self-energy including the cubic and quartic anharmonic terms. \SP{}{The RTA provides a solution to the Boltzmann transport equation (BTE) under the assumption that scattering events are independent and can be treated through mode-resolved relaxation times.
To verify the validity of this approximation for $\beta$-FeSi$_2$, we additionally
solved the BTE iteratively.} \PJ{}{These calculations were performed with
{\sc Phono3py}~\cite{togo_2023}. For cross-validation, the additional RTA calculations were performed on the same $\bm{q}$-grid as the BTE calculations, using the implementation provided by {\sc Phono3py}.}

% (see. Appendix~\ref{ThermalAB} for the comparison)\cite{togo_2023}.}
%As shown in Appendix~\ref{ThermalAB}, the iterative BTE results exhibit good agreement with the RTA values, confirming that RTA is sufficiently accurate for this material and for the considered temperature range.}


\section{Results}
\label{sec.result}

\subsection{Crystal structure}
\label{sec.crys}

The $\beta$-FeSi$_2$ structure adopts a base-centered orthorhombic lattice with the space group Cmca (No.~64) as shown in Fig.~\ref{fig.struct}(a).
The unit cell consists of two primitive cells and contains 48 atoms.
Iron (silicon) atoms possess two nonequivalent positions: \mbox{Fe-I} and \mbox{Fe-II} (\mbox{Si-I} and \mbox{Si-II}), presented in Fig.~\ref{fig.struct}(a) as gray and purple (orange and yellow) spheres, respectively.
This crystal structure is derived from the fluorite-type lattice with strongly distorted Si cubes and Fe atoms occupying 
one-half of the central sites.
The Fe-I and Fe-II sites create different layers perpendicular to the $x$ direction, and they are separated by layers containing both Si sites.
Each Fe atom is coordinated by 8 Si atoms with slightly different Fe-Si distances.
The optimized lattice constants ($a = 9.874$~{\AA}, $b = 7.767$~{\AA}, and
$c = 7.811$~{\AA}) agree very well with the experimental data ($a = 9.863$~{\AA}, $b = 7.791$~{\AA}, and $c = 7.833$~{\AA})~\cite{dusausoy_1971}.

Iron atoms occupy the Wyckoff sites 8\textit{d} ($0.2166$, $0$, $0$) and  8\textit{f} ($0$, $0.3072$, $0.1879$), corresponding to \mbox{Fe-I} and \mbox{Fe-II}, respectively. 
Silicon atoms are located at two inequivalent 16\textit{g} positions: ($0.1282$, $0.2737$, $0.0495$) and \mbox{($0.3734$, $0.0445$, $0.2270$)}, assigned as \mbox{Si-I} and \mbox{Si-II}.
The optimized positions of atoms agree very well with the experimental data~\cite{dusausoy_1971} and the previous theoretical studies~\cite{tani_2010,liang_2011}.


\subsection{Lattice dynamics}
\label{sec.lattice}

\begin{figure}[]
    \centering
    \includegraphics[width=\linewidth]{Fig1c.pdf}
\caption{%
The phonon dispersion curves along high symmetry directions obtained within SCPH (for temperatures from $0$ to $1000$~K).
Dashed black lines indicate the phonon dispersions obtained from the harmonic approximation. The white dots indicate Raman-active modes with A$_g$ symmetry.
The vertical plot shows the phonon density of states (DOS) calculated at a reference temperature of $600$~K.
}
  \label{fig.ph_band}
\end{figure}


In Fig.~\ref{fig.ph_band} we present the phonon dispersion relations of $\beta$-FeSi$_2$ along high-symmetry directions in the Brillouin zone [Fig.~\ref{fig.struct}(b)].
Due to 24 atoms in the primitive cell, the phonon spectrum consists of 69 optical modes and three acoustic modes.
The phonon dispersions were calculated within the SCPH approach in the temperature range $0$--$1000$~K (presented by color lines in Fig.~\ref{fig.ph_band}), and they are compared with the results obtained from the harmonic part of the effective potential corresponding to temperature $T=300$~K (indicated by dashed black lines in Fig.~\ref{fig.ph_band}).
As we can see within the SCPH method, the anharmonic effects are rather weak and leads only to small renormalization of phonon frequencies.
Only the highest modes show more pronounced shifts of their frequencies to larger values.
The total and partial element-projected phonon density of states obtained within the harmonic approximation are presented in Fig.~\ref{fig.ph_band}.
Up to around $320$~cm$^{-1}$, the contributions from both elements are very similar, while for higher frequencies the spectrum is dominated by
the Si vibrations.


\begin{table}[!t]
\begin{ruledtabular}
\caption{Calculated and experimental Raman-active modes of $\beta$-FeSi$_2$ with their irreducible representations (IR). Present theoretical results are compared with the previous theoretical data from Ref.~\cite{tani_2010} and experimental results from \mbox{Refs.~\cite{lefki_1991,maeda_2004}.} The experimental frequencies with known symmetries (A$_g$) are shown in bold, while other experimental modes are assigned to the best fitting theoretical values.}
\begin{tabular}{c c c c c}
\textbf{IR} & \multicolumn{4}{c}{\textbf{Frequency (cm$^{-1}$)}} \\
 & Present  & Theor.~\cite{tani_2010} & Exp.~\cite{lefki_1991} & Exp.~\cite{maeda_2004} \\
\hline
B$_{2g}$ & 175.4 & 179 & 176 &  \\
B$_{1g}$ & 176.3 & 185 & 179  &  \\
B$_{1g}$ & 193.3 & 198 &  & 190.6 \\
A$_g$    & 196.6 & 208 & \textbf{195} & \textbf{194.0} \\
A$_g$    & 203.9 & 210 & \textbf{197} & 199.6 \\
B$_{3g}$ & 205.5 & 212 & 200 &  \\
B$_{3g}$ & 226.3 & 236 & 206 & 227.1 \\
B$_{1g}$ & 233.3 & 240 &  &  231.6 \\
B$_{2g}$ & 248.6 & 254 &  &  \\
A$_g$    & 250.1 & 257 & \textbf{247} & \textbf{247.3} \\
A$_g$    & 254.9 & 264 & \textbf{253} & 254.3 \\
B$_{1g}$ & 275.5 & 285 &  &  274.1 \\
B$_{2g}$ & 282.4 & 295 &  &  281.2 \\
B$_{3g}$ & 286.8 & 297 &  &  \\
B$_{1g}$ & 307.1 & 317 &  &  \\
B$_{2g}$ & 312.6 & 326 &  &  311.8 \\
B$_{1g}$ & 319.0 & 324 &  &  \\
B$_{3g}$ & 327.9 & 341 &  & 325.8 \\
B$_{2g}$ & 333.3 & 345 &  &  \\
A$_g$    & 339.3 & 352 & \textbf{346} & 339.5 \\
B$_{2g}$ & 343.0 & 350 &  &  \\
B$_{1g}$ & 353.5 & 366 &  &  \\
B$_{2g}$ & 372.8 & 383 &  & 370.7 \\
B$_{1g}$ & 375.1 & 385 &  &  \\
B$_{3g}$ & 375.2 & 386 &  &  \\
B$_{3g}$ & 385.3 & 401 &  &  \\
A$_g$    & 386.5 & 398 &  & 386.2 \\
B$_{2g}$ & 387.7 & 402 &  & 388.2 \\
A$_g$    & 404.6 & 415 &  & 400.4 \\
B$_{2g}$ & 405.1 & 420 &  &  \\
B$_{1g}$ & 412.7 & 428 &  &  \\
B$_{3g}$ & 418.4 & 431 &  &  \\
B$_{3g}$ & 441.4 & 458 &  & 442.6 \\
B$_{3g}$ & 447.3 & 466 &  & 446.3 \\
A$_g$    & 448.4 & 464 &  &  \\
A$_g$    & 499.1 & 517 &  &  \\
\end{tabular}
\label{T1Raman}
\end{ruledtabular}
\end{table}

%subsection{Raman}

\begin{figure*}
\centering
  \includegraphics[width=\textwidth]{spectrum_RTA+raman.pdf}
\caption{Raman spectrum of $\beta$-FeSi$_2$ calculated within the perturbative approach at three temperatures 
($300$, $600$, $900$~K -- blue, orange, and red lines, respectively). 
The calculated spectrum includes all Raman-active modes. The A$_g$ modes
are indicated by purple vertical lines at peak positions corresponding to T=300~K. 
The frequencies derived from the harmonic approximation at T=300~K are indicated by green lines.
Connecting arrows indicate the correspondence between harmonic and
anharmonic frequencies, demonstrating the frequency shifts due to 
phonon interactions. Experimental values for the A$_g$ modes
based on Ref.~\cite{lefki_1991} are marked with black dashed lines.
}
\label{raman}
\end{figure*}


The phonon spectrum at the $\Gamma$ point consists of 36 Raman modes classified according to the irreducible representations: $9A_\text{g}+9B_\text{1g}+9B_\text{2g}+9B_\text{3g}$. 
Fig.~\ref{raman} shows the Raman spectrum of A$_g$ symmetry calculated for $\beta$-FeSi$_2$ within the perturbative approach at three temperatures $300$, $600$, and $900$~K (solid blue, orange, and red curves, respectively), including third-order and fourth-order anharmonic corrections.
The calculated Raman spectrum includes all Raman-active modes.
The five experimentally observed A$_g$ modes are highlighted by black dashed lines, based on data from Ref.~\cite{lefki_1991}.
Additional peaks not marked with vertical lines correspond to phonon modes with symmetries other than A$_g$.
The frequencies of Raman modes obtained in anharmonic calculations are compared with the previous results calculated within the harmonic approximation and the experimental values in Tab.~\ref{T1Raman}. 
We have marked in bold the experimentally determined A$_g$ modes, which are compared with the calculations.
Since the experimental studies did not provide the accurate assignement of the Raman modes with the B$_{1g}$, B$_{2g}$, and B$_{3g}$ symmetry~\cite{lefki_1991,maeda_2004}, we cannot compare them directly with the theoretical results.
However, in Tab.~\ref{T1Raman} we have assigned the measured frequencies to the best fitting theoretical values without taking into account the symmetry of the modes, except for the known A$_{g}$ modes. 


The impact of anharmonicity on the phonon frequencies is well visible from the comparison of the results obtained within the harmonic approximation 
and from the anharmonic calculation (vertical green and purple lines in Fig.~\ref{raman}, respectively).
Here we show only the A$_g$ modes, which are compared with the experimental results (vertical black dashed lines).
Anharmonic frequencies calculated at $300$~K are indicated by purple lines, while frequencies derived from harmonic approximation are marked with green lines. The grey solid lines connect corresponding modes obtained in both approximations.
In most cases, the results obtained within the harmonic approximation do not agree with the experimental frequencies. 
%Only the modes close to $250$~cm$^{-1}$ and $400$~cm$^{-1}$ correspond well to the experimental values. MS
As we can see, the inclusion of the anharmonic correction leads to a significant modification of the phonon frequencies.
These anharmonic effects are stronger for higher-frequency modes mainly because of
the dominant contribution from Si atoms, which vibrate with larger amplitudes than heavier Fe atoms. 
When atoms move to larger distances the potential deviates more from the harmonic approximation,
and the anharmonic corrections become stronger.


The modification of phonon frequencies observed in Fig.~\ref{raman} is much larger than in the SCPH scheme presented in Fig.~\ref{fig.ph_band}. The SCPH approach includes only the leading-order contribution to 
the phonon self-energy obtained from the quartic anharmonic terms~\cite{tadano_2015}. Therefore, it does not describe fully the changes of phonon frequencies found within the perturbation theory (see Fig.~\ref{raman}).
Especially, it is well visible for two highest A$_g$ modes, which exhibit also the largest line broadening 
and the strongest dependence on temperature.
Therefore, a better agreement with experimentally observed frequencies is visible,
confirming the significant influence of the anharmonicity on the frequencies and line profiles of phonon modes.
In fact, the decrease of phonon frequencies should be even stronger due to thermal expansion, 
which is not included in our calculations.
Within SCPH the frequencies of the highest modes increase with increasing temperature as we see in Fig.~\ref{fig.ph_band}. The comparison of two different approaches applied to study anharmonic properties of $\beta$-FeSi$_2$ shows that the perturbation theory, which includes the cubic and quartic terms, better describes the changes of phonon frequencies with temperature than the SCPH method. \PJ{}{This indicates that the leading-order contribution included in SCPH are not important in this material.}

Additionaly we should nottice that for other than A$_g$ modes we cannot make an unambiguous assignment of theoretical frequencies to experimental ones. Note that the spectrum in Fig.~\ref{fig.ph_band} contains all Raman-active modes. 
The limitation to A$_g$ modes concerns only the indicated positions of the peaks. 
%\SP{}{Furthermore, we present the full spectrum for all Raman active modes with the comparison of the harmonic and anharmonic calculations in the Appendix~\ref{RamanAA}.}
\textcolor{red}{The full Raman spectrum, with the frequencies of the B$_{1g}$, B$_{2g}$ and B$_{3g}$ modes marked, is shown in Appendix A, Fig.~\ref{raman1}. This represents our theoretical prediction of possible Raman-mode assignments, which can be verified in future experiments.}
%This is the theoretical prediction of possible assignment of Raman modes that can be verified in future experiments.  


\subsection{Thermal conductivity}
\label{sec.thermal}


\begin{figure*}[!t]
\centering
  \includegraphics[width=\linewidth]{time3.pdf}
\caption{Phonon lifetimes calculated for three temperatures as a function of phonon frequency. The colors correspond to the phonon branches.}
  \label{thermtime}
\end{figure*}


In this section, we analyze the thermal conductivity tensor of $\beta$-FeSi$_2$ obtained within the RTA approach~\cite{tadano_2018} as a function of temperature
%
\begin{equation}
\kappa_{\text{ph}}^{\mu\nu}(T) = \frac{1}{NV} \sum_{\bm{q},j} c_{\bm{q}j}(T) v_{\bm{q}j}^{\mu} v_{\bm{q}j}^{\nu}\tau_{\bm{q}j}(T),
\end{equation}
% 
where $c_{\bm{q}j}$ is the mode heat capacity and $v_{\bm{q}j}$ is the mode group velocity. 
The relaxation time is approximated by the phonon lifetime $\tau_{\bm{q}j}$
calculated for $j$-th branch at the wave vector $\bm{q}$.
$V$ is the unit cell volume and $N$ is the number of unit cells in the crystal.
The phonon lifetime is calculated using this formula
%
\begin{equation}
\tau_{\bm{q}j}(T)=\frac{1}{2\Gamma_{\bm{q}j}^{\text{anh}}(T)},
\end{equation}
%
where $\Gamma_{\bm{q}j}^{\text{anh}}$ is the anharmonic phonon linewidth obtained from 
the imaginary part of the phonon self-energy within the perturbation theory.

In Fig.~\ref{thermtime}, we present $\tau_{\bm{q}j}$ obtained for three temperatures 300, 600, and 1000~K as a function of frequency. As we see, the acoustic phonons close to the $\Gamma$ point have the longest lifetimes,
which are diminished with increasing frequency reaching local minima around $200$~cm$^{-1}$.
For higher frequencies, phonon lifetimes first increase to local maxima around $300$~cm$^{-1}$ and then decrease to
the lowest values in the range of highest optical modes. The shortest lifetimes correspond to the largest line broadening
observed for the Raman modes in Fig~\ref{raman}. The phonon group velocities 
$v_{\bm{q}j}=\partial\omega_{\bm{q}j}/\partial\bm{q}$, which are obtained by the central difference formula, are presented in Fig.~\ref{thermvelocity}. Their temperature dependence is negligible, therefore, we present only the results for $T=600$~K.
At low frequencies, there are clearly two ranges of group velocities of the acoustic phonons. 
The larger values correspond to the longitudinal modes, while the lower values are obtained from the
transverse acoustic branches. Group velocities of acoustic phonons decrease for larger frequencies
and reach the average values typical for optic branches.

\begin{figure}[!t]
\centering
  \includegraphics[width=\linewidth]{GV.pdf}
\caption{Mode group velocities calculated as a function of phonon frequency. The colors correspond to the phonon branches.
}
  \label{thermvelocity}
\end{figure}

In Fig.~\ref{anizotropy}(a), we present the three diagonal elements of $\kappa_{\text{ph}}^{\mu\nu}$ corresponding to the main directions of the crystal structure. 
They were obtained from the force constants calculated at the base temperature $T=600$~K and the crystallite size $0.1$~$\mu$m to account for boundary-limited phonon transport. 
Due to the orthorhombic symmetry, we observe a small anisotropy in phonon transport in the whole temperature range. 
At low temperatures, the three components of the heat conductivity increase in a very similar way with the $\kappa_{\text{ph}}^{yy}$ element slightly larger than two other components. 
After reaching the maximum, we observe a change in the largest component from $\kappa_{\text{ph}}^{yy}$ to 
$\kappa_{\text{ph}}^{xx}$.    
In Fig.~\ref{anizotropy}(b), the thermal conductivity is shown for three base temperatures, at which the interatomic potential was obtained ($300$~K, $600$~K, and $1000$~K), using the energy expansion up to third- and fourth-order anharmonic terms, and the same structure size of $0.1$~$\mu$m. At lowest temperatures, the thermal conductivity strongly increases, reaching the maximum around $T=180$~K, then it shows a slower decrease with temperature.
The differences between the two levels of approximation are minimal, suggesting that third-order calculations already capture the dominant phonon scattering mechanisms. The dependence on the base temperature is also very weak, showing the changes in the heat conductivity within a few percent. 
\SP{}{Further verification of the reliability of our thermal conductivity results is given in Appendix~\ref{ThermalAB}, where the full BTE calculations show good agreement with RTA and higher-resolution RTA results, demonstrating that the $8\times8\times8$ $\bm{q}$-mesh already provides converged values.}

\begin{figure}[!t]
\centering
  \includegraphics[width=\linewidth]{Anizotropy.pdf}
\caption{(a) The anisotropic thermal conductivity of $\beta$-FeSi$_2$ calculated along the lattice directions at 600~K. (b) The average temperature-dependent thermal conductivity taken at $300$~K, $600$~K, and $1000$~K, including anharmonic corrections up to cubic (A3) and quartic (A4) terms. In both cases the crystallite size is 0.1~$\mu$m.}
  \label{anizotropy}
\end{figure}


In Fig.~\ref{therm}, we fix the base temperature at $600$~K and examine the effect of crystallite size on thermal conductivity, varying it from $0.01$ to $0.5$~$\mu$m.
With decreasing the crystallite size, we observe a shift of the position of the maximum to larger temperatures and a decrease of the thermal conductivity in the entire temperature range.
Theoretical results are compared with several experimental data obtained above the room temperature. 
The measured thermal conductivity depends to a large extent on the sample quality, its purity and the size of the crystalline grains which depends on 
the production processes.
Many measurements were performed using crystallites of micrometric or unknown size ~\cite{waldecker_1973,ito_2002,kim_2003,du_2020}, however, 
numerous attempts to minimize $\kappa$ by reducing grain sizes to $56$~nm~\cite{dabrowski_2019, dabrowski_microstructure_2021}, $30$-$400$~nm~\cite{le_tonquesse_2019}, $50$ and $200$~nm~\cite{abbassi_2021}, or introducing pores into the material~\cite{sam2023improved} 
are also carried out. 
Another way to change the thermal conductivity is to dope $\beta$-FeSi$_2$ with different elements~\cite{ito_2002,kim_2003,du_2020,cheng_2024}, however, this effect is beyond our investigation.

\begin{figure}[!t]
\centering
  \includegraphics[width=\linewidth]{Thermal_conductivity3.pdf}
\caption{
The phonon thermal conductivity of $\beta$-FeSi$_2$: theoretical results for the infinite crystalline size and with boundary conditions, compared with experimental data for different structure sizes.
}
  \label{therm}
\end{figure}
%Fig.~\ref{therm}\cite{abbassi_2021}\cite{waldecker_1973}\cite{dabrowski_2019}\cite{sam2023improved}\cite{kim_2003}\cite{du_2020}\cite{ito_2002}\cite{le_tonquesse_2019}.

We observe a decrease in the thermal conductivity with reducing crystalline grain sizes in all analyzed experimental data. 
For instance, by decreasing the crystallite size to $50$~nm, the thermal conductivity at room temperature was reduced by a factor of $1.7$, what can be  compared to the annealed sample with 200 nm grains~\cite{abbassi_2021}. 
It is worth noting that the rate of decrease in value with increasing temperature in both cases, for grain sizes of $50$~nm and $200$~nm, is significantly different, which is consistent with our calculations. 
The same trend can be observed by comparing the thermal conductivity measured for a sample with bulk crystallite sizes with the thermal conductivity of a sample with grains smaller than 400 nm~\cite{le_tonquesse_2019}.
The theoretical results obtained for the same crystallite size show higher values due to factors not captured in the idealized model, such as crystal imperfection or mechanical strain. Usually, a decrease in the crystallite size is related to an increased concentration of grain boundaries, point defects, and stacking faults that influence the phonon scattering~\cite{le_tonquesse_2019,abbassi_2021}.   

We should note that the total thermal conductivity is a combination
of the lattice and electronic contributions to the heat transport.
In semiconductors, the electronic thermal conductivity is negligible at low temperatures and significantly increases 
only much above the room temperatures~\cite{gu_2020}.
For $\beta$-FeSi$_2$, the electronic thermal conductivity was obtained from the electric conductivity using the Wiedemann-Franz law~\cite{ito_2002,kim_2003,le_tonquesse_2019}.
In the undoped material, its value does not exceed $0.1$~W/mK in the measurement up to $T=950$~K~\cite{kim_2003}.
By doping, the electronic thermal conductivity can be enhanced, and it has a direct impact on the thermoelectric properties of $\beta$-FeSi$_2$ at high temperatures~\cite{ito_2002,kim_2003}.
In the present study, we consider only the phonon contribution to the thermal conductivity,
therefore, agreement with experimental data may deteriorate with increasing temperature.

\section{Summary}
\label{sec.summary}

We performed {\it ab initio} studies on lattice dynamical and thermal transport properties of $\beta$-FeSi$_2$. The effect of anharmonicity was analyzed within two approaches -- the SCPH method and the perturbation theory.
The phonon dispersion curves obtained within SCPH show small renormalization of frequencies comparing to the harmonic approximation. 
The Raman spectra were calculated within the procedure which takes into account the peak intensities obtained from the Raman tensors and the line profiles obtained from the phonon self energy derived within the perturbation theory based on the large, temperature-independent, quartic model fitted to the data from the wide range of temperatures (300-1000~K). 
The anharmonic corrections strongly affect the frequencies and line profiles of some modes and results in overall better agreement with the experimental data. 
We analyzed the phonon lifetimes and group velocities obtained as functions of the phonon frequency.
Then the lattice thermal conductivity was calculated for a broad range of temperatures and grain sizes.
We found a small anisotropy in the phonon thermal transport resulting from the orthorhombic structure and a weak effect of the quartic anharmonic terms. 
The thermal conductivity calculated for various crystalline grain sizes show a good qualitative agreement with the available measurements.

\begin{acknowledgments}
Some figures in this work were rendered using {\sc Vesta}~\cite{momma.izumi.11} software.
This work was partially supported by the Ministry of Education, Youth and Sports of the Czech Republic through the e-INFRA CZ (ID:90254).
\end{acknowledgments}

\appendix

\section{Raman spectrum}
\label{RamanAA}

Based on the polarized Raman measurements reported in Ref.~\cite{maeda_2004}, two Raman peaks were identified as belonging to the A$_g$ symmetry class, and several additional peaks were observed with similar or different polarization dependence. Although the authors of Ref.~\cite{maeda_2004} provided estimates of the relative Raman tensor components, they did not specify which of the remaining modes correspond to the B$_{1g}$, B$_{2g}$, or B$_{3g}$ symmetries. Because of this missing experimental information, a direct symmetry-resolved comparison between the measurement and theory is not currently possible for the non-A$_g$ modes. 
To provide a complete theoretical picture of the B$_g$-type modes, we show here the calculated Raman-active frequencies and intensities for the B$_{1g}$, B$_{2g}$, and B$_{3g}$ symmetries only. 
%These results represent the predicted Raman modes for the B$_g$ symmetries in $\beta$-FeSi$_2$. 
\textcolor{red} {Fig.~\ref{raman1} shows the predicted Raman modes for the B$_g$ symmetries in $\beta$-FeSi$_2$. The results obtained within the harmonic approximation are compared with the anharmonic perturbation theory calculations which provides both, frequency shifts and predicted line profiles of the modes.}
Although the experimentally measured peaks cannot be directly assigned to these symmetries due to the lack of polarization-resolved data, the theoretical predictions provide a reference for comparison. Matching the measured frequencies to the closest theoretical B$_g$ modes (Table~\ref{T1Raman}) allows for a tentative assignment, which can guide future polarization-resolved Raman experiments aimed at determining the precise symmetry of the unresolved peaks.

\begin{figure*}[t]
\centering
  \includegraphics[width=\linewidth]{spectrum_RTA+raman_Bi_modes.pdf}
\caption{Raman spectrum of $\beta$-FeSi$_2$ calculated at three temperatures 
($300$, $600$, $900$~K -- blue, orange, and red lines, respectively). 
The calculated spectrum includes all Raman-active modes. The B$_{ig}$ modes
are indicated by purple vertical lines at peak positions corresponding to T=300~K. 
The frequencies derived from the harmonic approximation at T=300~K are indicated 
by green, yellow and pink lines.
Connecting arrows indicate the correspondence between harmonic and anharmonic 
frequencies, demonstrating the frequency shifts due to phonon interactions.}
\label{raman1}
\end{figure*}
\section{Thermal conductivity obtained from BTE and RTA}
\label{ThermalAB}

The thermal conductivity was computed by solving the full BTE on the largest feasible $\bm{q}$-point grid, $8\times8\times8$, and compared with the corresponding RTA results obtained on the same grid. As shown in Fig.~\ref{bte}, the difference between the components of the thermal conductivity tensor obtained within BTE and RTA at this resolution is very small, indicating a good agreement between these two approaches.
%THIS PART SHOULD GO RATHER TO THE RESPONSE
%Extending the full BTE calculation to larger grids is computationally prohibitive: the computational cost of BTE is roughly two orders of %magnitude higher than that of RTA, and the required memory and runtime exceed our available resources. 
Moreover, we performed an additional calculation using RTA on a denser $20\times20\times20$ grid. As seen in the Fig.~\ref{bte}, the higher-resolution data remain in a good agreement with both the BTE and RTA results for the $8\times8\times8$ grid.
It shows that the $8\times8\times8$ mesh already provides good results for this structure and confirms reliability of the calculations.

\begin{figure}[!h]
\centering
  \includegraphics[width=\linewidth]{BTEvsRTAvsANP.pdf}
\caption{Thermal conductivity of $\beta$-FeSi$_2$ obtained within the BTE and RTA methods using the Phono3py software on the $8\times8\times8$ q-point grid, compared with the RTA results computed with ALAMODE on a denser $20\times20\times20$ grid.}
\label{bte}
\end{figure}

%\section*{Data availability}
%The data that support the findings of this article are openly available~\footnote{give me DOI}.
% see https://journals.aps.org/authors/data-availability-statements#citation

\bibliography{refs.bib}
%\bibliographystyle{ieeetr}


\end{document}

\documentclass[%
%reprint,
superscriptaddress,
%groupedaddress,
longbibliography,
%unsortedaddress,
%runinaddress,
%frontmatterverbose, 
%preprint,
%preprintnumbers,
%nofootinbib,
nobibnotes,
%bibnotes,
amsmath,amssymb,
aps,
%pra,
prb,
%rmp,
%prstab,
%prstper,
%showkeys,
floatfix,
twocolumn
]{revtex4-2}

\usepackage{graphicx}% Include figure files
\usepackage{calc}% Calculate margins
\usepackage{dcolumn}% Align table columns on decimal point
\usepackage{bm}% bold math

\usepackage[urlcolor=blue,colorlinks=true,citecolor=blue,linkcolor=blue,pdfstartview={FitH},bookmarks=false]{hyperref} % add hypertext capabilities

%\usepackage[mathlines]{lineno} % Enable numbering of text and display math
% \linenumbers\relax % Commence numbering lines

% \usepackage[showframe,%Uncomment any one of the following lines to test 
% %scale=0.7, marginratio={1:1, 2:3}, ignoreall,% default settings
% %text={7in,10in},centering,
% %margin=1.5in,
% % total={6.5in,8.75in}, top=1.2in, left=0.9in, includefoot,
% % height=10in,a5paper,hmargin={3cm,0.8in},
% ]{geometry}

\usepackage{amsmath}
\usepackage{amssymb}
%\usepackage{orcidlink}
\usepackage{xcolor}
%\usepackage{datetime}
\usepackage[normalem]{ulem}

% Change tracking commands
\newcommand{\trackchange}[3]{\textcolor{#3}{\sout{#1}#2}}  % Full color strikeout, insert
%\renewcommand{\trackchange}[3]{\textcolor{#3}{#2}}        % Just color silent remove and insert
%\renewcommand{\trackchange}[3]{{#2}}                      % No indication, silent remove and insert

% Author marker definitions

\definecolor{myblue}{RGB}{0,127,85}
% \definecolor{violet}{RGB}{102,0,204}
% \definecolor{orange}{RGB}{255,128,0}
% \definecolor{green}{RGB}{0,128,0}
\newcommand{\DL}[1]{\trackchange{}{#1}{blue}}
\newcommand{\AP}[1]{\trackchange{}{#1}{red}}
\newcommand{\PJ}[2]{\trackchange{#1}{#2}{orange}}
\newcommand{\JL}[2]{\trackchange{#1}{#2}{myblue}}
\newcommand{\SP}[2]{\trackchange{#1}{#2}{blue}}
\newcommand{\PP}[2]{\trackchange{#1}{#2}{teal}}
\newcommand{\AI}[2]{\trackchange{#1}{#2}{olive}}
\newcommand{\MS}[1]{\trackchange{}{#1}{purple}}

\newcommand{\TODO}[1]{\textcolor{red}{TODO: #1}}

\sloppy

\begin{document}

\title{Ab initio study of the anharmonic properties and thermal conductivity in $\beta$-FeSi$_2$}

\author{Svitlana~Pastukh}
\email[e-mail: ]{svitlana.pastukh@ifj.edu.pl}
\affiliation{Institute of Nuclear Physics, Polish Academy of Sciences, ul. W. E. Radzikowskiego 152, 31-342 Krak\'{o}w, Poland}

\author{Ma\l{}gorzata~Sternik}
\affiliation{Institute of Nuclear Physics, Polish Academy of Sciences, ul. W. E. Radzikowskiego 152, 31-342 Krak\'{o}w, Poland}

\author{Pawe\l{}~T.~Jochym}
\affiliation{Institute of Nuclear Physics, Polish Academy of Sciences, ul. W. E. Radzikowskiego 152, 31-342 Krak\'{o}w, Poland}


\author{Jan~\L{}a\.{z}ewski}
\affiliation{Institute of Nuclear Physics, Polish Academy of Sciences, ul. W. E. Radzikowskiego 152, 31-342 Krak\'{o}w, Poland}

\author{Andrzej~Ptok}
\affiliation{Institute of Nuclear Physics, Polish Academy of Sciences, ul. W. E. Radzikowskiego 152, 31-342 Krak\'{o}w, Poland}

\author{Svetoslav~Stankov}
\affiliation{Institute for Photon Science and Synchrotron Radiation, Karlsruhe Institute of Technology, D-76131 Karlsruhe, Germany}
\affiliation{Laboratory for Applications of Synchrotron Radiation, Karlsruhe Institute of Technology, D-76131 Karlsruhe, Germany}

\author{Przemys\l{}aw~Piekarz}
\affiliation{Institute of Nuclear Physics, Polish Academy of Sciences, ul. W. E. Radzikowskiego 152, 31-342 Krak\'{o}w, Poland}

\date{\today}

\begin{abstract}

Iron silicides are good candidates for applications in  optoelectronic and thermoelectric devices.
Lattice dynamical properties and thermal conductivity in the $\beta$-FeSi$_2$ semiconductor
are investigated with the first-principles computational methods. 
Phonon dispersion relations are calculated via the
temperature-dependent effective potential method and self-consistent phonon theory. 
To properly model thermal transport, we explicitly consider
the impact of phonon-phonon interactions by analyzing
anharmonic contributions to the phonon self-energy. 
This yields temperature-dependent phonon frequencies and linewidths,
reflecting the finite lifetime of phonons due to scattering
processes. The calculated phonon frequencies and line profiles are used to obtain 
the Raman spectra, which shows good agreement with the experimental data. 
We revealed an enhanced anharmonic behaviour of the Raman modes with the highest frequencies.  
The lattice thermal conductivity is then obtained as a function of temperature and crystallite size within
the relaxation-time approximation.
Phonon transport shows a small anisotropy due to the orthorhombic structure and a very weak dependence
on the quartic anharmonic corrections. The results obtained for an infinite material and for several crystallite sizes
were analyzed and compared with the available experimental data.
\end{abstract}

\maketitle


\section{Introduction}

The comprehensive determination of important physical properties of
crystals, such as thermal expansion, lattice thermal
conductivity or structural phase transitions, requires a fundamental 
understanding of the anharmonic effects.
Although the investigation of anharmonic interactions in crystals has
attracted a considerable interest for decades~\cite{cowley_1968}, a substantial progress
has only recently been achieved thanks to advances in theoretical and
numerical methods and increased computational power.
Now, phonon frequencies, lifetimes, and heat transfer in a wide range of
materials
can be quantitatively predicted using the available computational resources
based on the density functional theory (DFT)~\cite{lindsay_2013,mcgaughey_2019,lindsay_2019}.
In the case of strongly anharmonic systems, the self-consistent phonon
(SCPH) theory~\cite{tadano_2015} as well as the perturbative approach~\cite{tadano_2018}, using higher
order interatomic force constants derived from the fitting to the displacement force data obtained
with DFT, have proven to be successful.

Transition-metal silicides are promising materials for fabrication of electronic components
designed for integration with silicon-based circuits~\cite{murarka_1995}.
At room temperature, iron disilicide ($\beta$-FeSi$_2$) is a
direct-bandgap semiconductor~\cite{bost_1985}, making this material a good
candidate for application in optoelectronic devices such as infrared
detectors or light emitters~\cite{bost_1988}. The development of light-emitting diodes utilizing FeSi$_2$/Si heterostructures has been successfully demonstrated~\cite{leong_1997,suemasu_2001}. Due to
a high thermal stability and strong light absorption, FeSi$_2$ is
also a suitable photovoltaic material~\cite{powalla_1993,liu_2006,okuhara_2017}.

$\beta$-FeSi$_2$ crystallizes in the base-centered orthorhombic
lattice~\cite{dusausoy_1971} transforming to the
tetragonal metallic $\alpha$-FeSi$_{2}$ phase around $1200$~K~\cite{starke_2002}.
Optical studies indicated a direct band gap of the values
$0.85$--$0.89$~eV~\cite{bost_1988,dimitriadis_1990,arushanov_1995,wan_2003},
however, the {\it ab initio} calculations predicted a smaller indirect gap close to 0.8 eV~\cite{christensen_1990}. 
The existence of such an indirect gap was then confirmed by the optical linear transmittance measurements at
low temperatures~\cite{giannini_1992}. As shown by first-principles studies,
the character of the band gap is very sensitive to the orientation 
of a crystal grown on silicon~\cite{clark_1998}.

$\beta$-FeSi$_2$ belongs also to good thermoelectric materials~\cite{ware_1964}, with potential applications resulting from its chemical stability up to high temperatures, nontoxicity, and low cost of preparation~\cite{yamada_2012,nozariasbmarz_2017}. 
It has already been implemented in cars~\cite{birkholz_1988} and portable power sources~\cite{uemura_1989}. 
Its thermoelectric performance can be improved by doping~\cite{ito_2001,tani_2001,kim_2003,chen_2005,pandey_2013,le_tonquesse_2019}, 
which enhances the electric transport and reduces the thermal conductivity~\cite{waldecker_1973,du_2019,du_2020}.  
The thermal conductivity can be also reduced by the modification of microstructure~\cite{ail_2015} or by
nanostructurization~\cite{watanabe_2017,taniguchi_2017,hsin_2017,abbassi_2021}.

The lattice thermal conductivity is directly connected with anharmonic effects and phonon scattering processes.
The vibrational properties of \mbox{$\beta$-FeSi$_2$} were studied by the infrared and Raman spectroscopy~\cite{lefki_1991,guizzetti_1997,maeda_2004,baleva_2008,liu_2011,maeda_2011}. The observed anisotropy in the phonon spectra results from the enhanced sensitivity of the infrared and Raman features to the local lattice distortions~\cite{guizzetti_1997}. The Fe phonon density of states was measured by nuclear inelastic scattering (NIS), showing a good agreement with the density functional theory (DFT) calculations~\cite{walterfang_2005}. Using the DFT approach, the phonon dispersion curves, phonon density of states, as well as various thermodynamic properties were obtained within the harmonic approximation~\cite{tani_2010,liang_2011}. The extended Klemens model was applied to study
the anharmonic effect on phonon frequencies and linewidths observed by the Raman spectroscopy~\cite{zhang_2023}.
The impact of nanostructurization on lattice dynamics was explored in the $\beta$-FeSi$_2$ nanorods grown on the Si(110) surface by the NIS and {\it ab initio} methods~\cite{kalt_2022}.

In this work, we investigate the lattice dynamical properties of $\beta$-FeSi$_2$ 
using the DFT calculations. We study the effect of anharmonic terms in the temperature-dependent potential on phonon frequencies and lifetimes. We focus on the Raman modes, comparing the theoretical results with the experimental data.
The thermal conductivity is derived in a broad temperature range and the effect of crystallite size is analyzed.



This study is structured as follows.
In Sec.~\ref{sec.com} we describe the details of computational methods.
Next, in Sec.~\ref{sec.result} we present and discuss our results.
In particular we present the crystal structure (Sec.~\ref{sec.crys}) and lattice dynamics (Sec.~\ref{sec.lattice}).
We investigate also the thermal conductivity comparing the obtained results with the available experimental data (Sec.~\ref{sec.thermal}).
Finally, Sec.~\ref{sec.summary} summarizes our key findings and conclusions.

\section{Calculation method}
\label{sec.com}


The calculations were performed using the projector augmented-wave potentials~\cite{blochl_1994} and the generalized gradient approximation~\cite{perdew_1996} implemented in the Vienna Ab initio Simulation Package (VASP)~\cite{kresse.hafner.94,kresse.furthmuller.96,kresse.joubert.99}. 
The lattice parameters and atomic positions were optimized in the ${\bm a} \times ({\bm b}-{\bm c}) \times ({\bm b}+{\bm c})$ supercell containing 32 formula units and four primitive cells.
The integration in the reciprocal space was conducted using the $2 \times 2 \times 2$ Monkhorst--Pack mesh~\cite{monkhorst_1976} and the cut-off energy was set to $500$~eV. For convergence conditions, we set the energy change below $10^{-5}$ and $10^{-8}$ for the ionic and electronic loops, respectively. 

The lattice dynamical properties were studied within the temperature-dependent effective potential (TDEP) approach~\cite{hellman_2013}. The atomic potential with the third and fourth order anharmonic terms was derived from interatomic forces induced by displacements of all atoms at finite temperatures.
The sets of atomic displacements were generated by the high efficiency configuration space sampling (HECSS)~\cite{jochym_2021} and forces were obtained by VASP. The interatomic force constants and phonon frequencies were calculated with the {\sc Alamode} software~\cite{tadano_2014}.

Furthermore, we have attempted to construct a {\emph{temperature independent}} 
anharmonic model. We have used combined data from all investigated temperatures 
(300, 600 and 1000~K) and fitted a large (over 15~000 free parameters), fourth-order 
interaction model to this dataset. 
Subsequently, we have used this model to calculate line profiles and positions of Raman-active modes
at multiple temperatures.

The changes in phonon frequencies induced by the anharmonic effects were investigated within two approaches.
First, the impact of the quartic anharmonic terms was included using the SCPH theory~\cite{tadano_2015}.
Second, the mode profiles (frequency shifts and line widths) were determined from the real and imaginary parts of the phonon self-energy resulting from the cubic and quartic anharmonic terms of the above mentioned large model~\cite{tadano_2018}. 
The longitudinal optic-transverse optic (LO-TO) splitting was also evaluated, using the static dielectric tensor and Born effective charges calculated within density functional perturbation theory~\cite{gajdos_2006}.

\begin{figure}[]
    \centering
    \includegraphics[width=\linewidth]{fig1_new.png}
\caption{%
(a) The conventional unit cell of $\beta$-FeSi$_2$ (with Cmca symmetry) and (b) the corresponding Brillouin zone with selected high-symmetry points.
}
  \label{fig.struct}
\end{figure}

% To further characterize the vibrational properties, Raman-active scattering was investigated using the Phonopy-Spectroscopy package~\cite{skelton_2017}. This enabled the identification of Raman-active modes and the calculation of Raman tensors. Anharmonic force constants, obtained from calculations using {\sc Alamode}, were then used to obtain theoretical line profiles for the Raman modes. The presented Raman scattering spectra combine these anharmonic line profiles with the Raman tensor amplitudes.
% This analysis was also based on the large quartic model mentioned above.

% Finally, the thermal conductivity was obtained as a function of temperature and crystallite size within the relaxation-time approximation (RTA) \PJ{}{as implemented in {\sc Alamode}\cite{tadano_2014}}. The phonon lifetimes were calculated from the phonon self-energy including the cubic and quartic anharmonic terms. \SP{}{The RTA provides a solution of the Boltzmann transport equation (BTE) under the assumption that scattering events are independent and can be treated through mode-resolved relaxation times.
% To verify the validity of this approximation for $\beta$-FeSi$_2$, we have additionally
% executed an iterative solution of the BTE.}\PJ{}{These calculations were performed with
% {\sc Phono3py}\cite{togo_2023}. For cross-validation, the additional RTA calculations were performed on the same $\bm{q}$-grid as BTE with implementation provided by {\sc Phono3py}.}.

To further characterize the vibrational properties, Raman scattering was investigated using the Phonopy-Spectroscopy package~\cite{skelton_2017}. This enabled the identification of Raman-active modes and the calculation of Raman tensors. Anharmonic force constants, derived from calculations using {\sc Alamode}, were then used to obtain theoretical line profiles for the Raman modes. The presented Raman scattering spectra combine these anharmonic line profiles with the Raman tensor amplitudes.
This analysis was also based on the large quartic model described above.

Finally, the thermal conductivity was calculated as a function of temperature and crystallite size within the relaxation-time approximation (RTA) \PJ{}{as implemented in {\sc Alamode}~\cite{tadano_2014}}. The phonon lifetimes were calculated from the phonon self-energy including the cubic and quartic anharmonic terms. \SP{}{The RTA provides a solution to the Boltzmann transport equation (BTE) under the assumption that scattering events are independent and can be treated through mode-resolved relaxation times.
To verify the validity of this approximation for $\beta$-FeSi$_2$, we additionally
solved the BTE iteratively.} \PJ{}{These calculations were performed with
{\sc Phono3py}~\cite{togo_2023}. For cross-validation, the additional RTA calculations were performed on the same $\bm{q}$-grid as the BTE calculations, using the implementation provided by {\sc Phono3py}.}

% (see. Appendix~\ref{ThermalAB} for the comparison)\cite{togo_2023}.}
%As shown in Appendix~\ref{ThermalAB}, the iterative BTE results exhibit good agreement with the RTA values, confirming that RTA is sufficiently accurate for this material and for the considered temperature range.}


\section{Results}
\label{sec.result}

\subsection{Crystal structure}
\label{sec.crys}

The $\beta$-FeSi$_2$ structure adopts a base-centered orthorhombic lattice with the space group Cmca (No.~64) as shown in Fig.~\ref{fig.struct}(a).
The unit cell consists of two primitive cells and contains 48 atoms.
Iron (silicon) atoms possess two nonequivalent positions: \mbox{Fe-I} and \mbox{Fe-II} (\mbox{Si-I} and \mbox{Si-II}), presented in Fig.~\ref{fig.struct}(a) as gray and purple (orange and yellow) spheres, respectively.
This crystal structure is derived from the fluorite-type lattice with strongly distorted Si cubes and Fe atoms occupying 
one-half of the central sites.
The Fe-I and Fe-II sites create different layers perpendicular to the $x$ direction, and they are separated by layers containing both Si sites.
Each Fe atom is coordinated by 8 Si atoms with slightly different Fe-Si distances.
The optimized lattice constants ($a = 9.874$~{\AA}, $b = 7.767$~{\AA}, and
$c = 7.811$~{\AA}) agree very well with the experimental data ($a = 9.863$~{\AA}, $b = 7.791$~{\AA}, and $c = 7.833$~{\AA})~\cite{dusausoy_1971}.

Iron atoms occupy the Wyckoff sites 8\textit{d} ($0.2166$, $0$, $0$) and  8\textit{f} ($0$, $0.3072$, $0.1879$), corresponding to \mbox{Fe-I} and \mbox{Fe-II}, respectively. 
Silicon atoms are located at two inequivalent 16\textit{g} positions: ($0.1282$, $0.2737$, $0.0495$) and \mbox{($0.3734$, $0.0445$, $0.2270$)}, assigned as \mbox{Si-I} and \mbox{Si-II}.
The optimized positions of atoms agree very well with the experimental data~\cite{dusausoy_1971} and the previous theoretical studies~\cite{tani_2010,liang_2011}.


\subsection{Lattice dynamics}
\label{sec.lattice}

\begin{figure}[]
    \centering
    \includegraphics[width=\linewidth]{Fig1c.pdf}
\caption{%
The phonon dispersion curves along high symmetry directions obtained within SCPH (for temperatures from $0$ to $1000$~K).
Dashed black lines indicate the phonon dispersions obtained from the harmonic approximation. The white dots indicate Raman-active modes with A$_g$ symmetry.
The vertical plot shows the phonon density of states (DOS) calculated at a reference temperature of $600$~K.
}
  \label{fig.ph_band}
\end{figure}


In Fig.~\ref{fig.ph_band} we present the phonon dispersion relations of $\beta$-FeSi$_2$ along high-symmetry directions in the Brillouin zone [Fig.~\ref{fig.struct}(b)].
Due to 24 atoms in the primitive cell, the phonon spectrum consists of 69 optical modes and three acoustic modes.
The phonon dispersions were calculated within the SCPH approach in the temperature range $0$--$1000$~K (presented by color lines in Fig.~\ref{fig.ph_band}), and they are compared with the results obtained from the harmonic part of the effective potential corresponding to temperature $T=300$~K (indicated by dashed black lines in Fig.~\ref{fig.ph_band}).
As we can see within the SCPH method, the anharmonic effects are rather weak and leads only to small renormalization of phonon frequencies.
Only the highest modes show more pronounced shifts of their frequencies to larger values.
The total and partial element-projected phonon density of states obtained within the harmonic approximation are presented in Fig.~\ref{fig.ph_band}.
Up to around $320$~cm$^{-1}$, the contributions from both elements are very similar, while for higher frequencies the spectrum is dominated by
the Si vibrations.


\begin{table}[!t]
\begin{ruledtabular}
\caption{Calculated and experimental Raman-active modes of $\beta$-FeSi$_2$ with their irreducible representations (IR). Present theoretical results are compared with the previous theoretical data from Ref.~\cite{tani_2010} and experimental results from \mbox{Refs.~\cite{lefki_1991,maeda_2004}.} The experimental frequencies with known symmetries (A$_g$) are shown in bold, while other experimental modes are assigned to the best fitting theoretical values.}
\begin{tabular}{c c c c c}
\textbf{IR} & \multicolumn{4}{c}{\textbf{Frequency (cm$^{-1}$)}} \\
 & Present  & Theor.~\cite{tani_2010} & Exp.~\cite{lefki_1991} & Exp.~\cite{maeda_2004} \\
\hline
B$_{2g}$ & 175.4 & 179 & 176 &  \\
B$_{1g}$ & 176.3 & 185 & 179  &  \\
B$_{1g}$ & 193.3 & 198 &  & 190.6 \\
A$_g$    & 196.6 & 208 & \textbf{195} & \textbf{194.0} \\
A$_g$    & 203.9 & 210 & \textbf{197} & 199.6 \\
B$_{3g}$ & 205.5 & 212 & 200 &  \\
B$_{3g}$ & 226.3 & 236 & 206 & 227.1 \\
B$_{1g}$ & 233.3 & 240 &  &  231.6 \\
B$_{2g}$ & 248.6 & 254 &  &  \\
A$_g$    & 250.1 & 257 & \textbf{247} & \textbf{247.3} \\
A$_g$    & 254.9 & 264 & \textbf{253} & 254.3 \\
B$_{1g}$ & 275.5 & 285 &  &  274.1 \\
B$_{2g}$ & 282.4 & 295 &  &  281.2 \\
B$_{3g}$ & 286.8 & 297 &  &  \\
B$_{1g}$ & 307.1 & 317 &  &  \\
B$_{2g}$ & 312.6 & 326 &  &  311.8 \\
B$_{1g}$ & 319.0 & 324 &  &  \\
B$_{3g}$ & 327.9 & 341 &  & 325.8 \\
B$_{2g}$ & 333.3 & 345 &  &  \\
A$_g$    & 339.3 & 352 & \textbf{346} & 339.5 \\
B$_{2g}$ & 343.0 & 350 &  &  \\
B$_{1g}$ & 353.5 & 366 &  &  \\
B$_{2g}$ & 372.8 & 383 &  & 370.7 \\
B$_{1g}$ & 375.1 & 385 &  &  \\
B$_{3g}$ & 375.2 & 386 &  &  \\
B$_{3g}$ & 385.3 & 401 &  &  \\
A$_g$    & 386.5 & 398 &  & 386.2 \\
B$_{2g}$ & 387.7 & 402 &  & 388.2 \\
A$_g$    & 404.6 & 415 &  & 400.4 \\
B$_{2g}$ & 405.1 & 420 &  &  \\
B$_{1g}$ & 412.7 & 428 &  &  \\
B$_{3g}$ & 418.4 & 431 &  &  \\
B$_{3g}$ & 441.4 & 458 &  & 442.6 \\
B$_{3g}$ & 447.3 & 466 &  & 446.3 \\
A$_g$    & 448.4 & 464 &  &  \\
A$_g$    & 499.1 & 517 &  &  \\
\end{tabular}
\label{T1Raman}
\end{ruledtabular}
\end{table}

%subsection{Raman}

\begin{figure*}
\centering
  \includegraphics[width=\textwidth]{spectrum_RTA+raman.pdf}
\caption{Raman spectrum of $\beta$-FeSi$_2$ calculated within the perturbative approach at three temperatures 
($300$, $600$, $900$~K -- blue, orange, and red lines, respectively). 
The calculated spectrum includes all Raman-active modes. The A$_g$ modes
are indicated by purple vertical lines at peak positions corresponding to T=300~K. 
The frequencies derived from the harmonic approximation at T=300~K are indicated by green lines.
Connecting arrows indicate the correspondence between harmonic and
anharmonic frequencies, demonstrating the frequency shifts due to 
phonon interactions. Experimental values for the A$_g$ modes
based on Ref.~\cite{lefki_1991} are marked with black dashed lines.
}
\label{raman}
\end{figure*}


The phonon spectrum at the $\Gamma$ point consists of 36 Raman modes classified according to the irreducible representations: $9A_\text{g}+9B_\text{1g}+9B_\text{2g}+9B_\text{3g}$. 
Fig.~\ref{raman} shows the Raman spectrum of A$_g$ symmetry calculated for $\beta$-FeSi$_2$ within the perturbative approach at three temperatures $300$, $600$, and $900$~K (solid blue, orange, and red curves, respectively), including third-order and fourth-order anharmonic corrections.
The calculated Raman spectrum includes all Raman-active modes.
The five experimentally observed A$_g$ modes are highlighted by black dashed lines, based on data from Ref.~\cite{lefki_1991}.
Additional peaks not marked with vertical lines correspond to phonon modes with symmetries other than A$_g$.
The frequencies of Raman modes obtained in anharmonic calculations are compared with the previous results calculated within the harmonic approximation and the experimental values in Tab.~\ref{T1Raman}. 
We have marked in bold the experimentally determined A$_g$ modes, which are compared with the calculations.
Since the experimental studies did not provide the accurate assignement of the Raman modes with the B$_{1g}$, B$_{2g}$, and B$_{3g}$ symmetry~\cite{lefki_1991,maeda_2004}, we cannot compare them directly with the theoretical results.
However, in Tab.~\ref{T1Raman} we have assigned the measured frequencies to the best fitting theoretical values without taking into account the symmetry of the modes, except for the known A$_{g}$ modes. 


The impact of anharmonicity on the phonon frequencies is well visible from the comparison of the results obtained within the harmonic approximation 
and from the anharmonic calculation (vertical green and purple lines in Fig.~\ref{raman}, respectively).
Here we show only the A$_g$ modes, which are compared with the experimental results (vertical black dashed lines).
Anharmonic frequencies calculated at $300$~K are indicated by purple lines, while frequencies derived from harmonic approximation are marked with green lines. The grey solid lines connect corresponding modes obtained in both approximations.
In most cases, the results obtained within the harmonic approximation do not agree with the experimental frequencies. 
%Only the modes close to $250$~cm$^{-1}$ and $400$~cm$^{-1}$ correspond well to the experimental values. MS
As we can see, the inclusion of the anharmonic correction leads to a significant modification of the phonon frequencies.
These anharmonic effects are stronger for higher-frequency modes mainly because of
the dominant contribution from Si atoms, which vibrate with larger amplitudes than heavier Fe atoms. 
When atoms move to larger distances the potential deviates more from the harmonic approximation,
and the anharmonic corrections become stronger.


The modification of phonon frequencies observed in Fig.~\ref{raman} is much larger than in the SCPH scheme presented in Fig.~\ref{fig.ph_band}. The SCPH approach includes only the leading-order contribution to 
the phonon self-energy obtained from the quartic anharmonic terms~\cite{tadano_2015}. Therefore, it does not describe fully the changes of phonon frequencies found within the perturbation theory (see Fig.~\ref{raman}).
Especially, it is well visible for two highest A$_g$ modes, which exhibit also the largest line broadening 
and the strongest dependence on temperature.
Therefore, a better agreement with experimentally observed frequencies is visible,
confirming the significant influence of the anharmonicity on the frequencies and line profiles of phonon modes.
In fact, the decrease of phonon frequencies should be even stronger due to thermal expansion, 
which is not included in our calculations.
Within SCPH the frequencies of the highest modes increase with increasing temperature as we see in Fig.~\ref{fig.ph_band}. The comparison of two different approaches applied to study anharmonic properties of $\beta$-FeSi$_2$ shows that the perturbation theory, which includes the cubic and quartic terms, better describes the changes of phonon frequencies with temperature than the SCPH method. \PJ{}{This indicates that the leading-order contribution included in SCPH are not important in this material.}

Additionaly we should nottice that for other than A$_g$ modes we cannot make an unambiguous assignment of theoretical frequencies to experimental ones. Note that the spectrum in Fig.~\ref{fig.ph_band} contains all Raman-active modes. 
The limitation to A$_g$ modes concerns only the indicated positions of the peaks. 
%\SP{}{Furthermore, we present the full spectrum for all Raman active modes with the comparison of the harmonic and anharmonic calculations in the Appendix~\ref{RamanAA}.}
\textcolor{red}{The full Raman spectrum, with the frequencies of the B$_{1g}$, B$_{2g}$ and B$_{3g}$ modes marked, is shown in Appendix A, Fig.~\ref{raman1}. This represents our theoretical prediction of possible Raman-mode assignments, which can be verified in future experiments.}
%This is the theoretical prediction of possible assignment of Raman modes that can be verified in future experiments.  


\subsection{Thermal conductivity}
\label{sec.thermal}


\begin{figure*}[!t]
\centering
  \includegraphics[width=\linewidth]{time3.pdf}
\caption{Phonon lifetimes calculated for three temperatures as a function of phonon frequency. The colors correspond to the phonon branches.}
  \label{thermtime}
\end{figure*}


In this section, we analyze the thermal conductivity tensor of $\beta$-FeSi$_2$ obtained within the RTA approach~\cite{tadano_2018} as a function of temperature
%
\begin{equation}
\kappa_{\text{ph}}^{\mu\nu}(T) = \frac{1}{NV} \sum_{\bm{q},j} c_{\bm{q}j}(T) v_{\bm{q}j}^{\mu} v_{\bm{q}j}^{\nu}\tau_{\bm{q}j}(T),
\end{equation}
% 
where $c_{\bm{q}j}$ is the mode heat capacity and $v_{\bm{q}j}$ is the mode group velocity. 
The relaxation time is approximated by the phonon lifetime $\tau_{\bm{q}j}$
calculated for $j$-th branch at the wave vector $\bm{q}$.
$V$ is the unit cell volume and $N$ is the number of unit cells in the crystal.
The phonon lifetime is calculated using this formula
%
\begin{equation}
\tau_{\bm{q}j}(T)=\frac{1}{2\Gamma_{\bm{q}j}^{\text{anh}}(T)},
\end{equation}
%
where $\Gamma_{\bm{q}j}^{\text{anh}}$ is the anharmonic phonon linewidth obtained from 
the imaginary part of the phonon self-energy within the perturbation theory.

In Fig.~\ref{thermtime}, we present $\tau_{\bm{q}j}$ obtained for three temperatures 300, 600, and 1000~K as a function of frequency. As we see, the acoustic phonons close to the $\Gamma$ point have the longest lifetimes,
which are diminished with increasing frequency reaching local minima around $200$~cm$^{-1}$.
For higher frequencies, phonon lifetimes first increase to local maxima around $300$~cm$^{-1}$ and then decrease to
the lowest values in the range of highest optical modes. The shortest lifetimes correspond to the largest line broadening
observed for the Raman modes in Fig~\ref{raman}. The phonon group velocities 
$v_{\bm{q}j}=\partial\omega_{\bm{q}j}/\partial\bm{q}$, which are obtained by the central difference formula, are presented in Fig.~\ref{thermvelocity}. Their temperature dependence is negligible, therefore, we present only the results for $T=600$~K.
At low frequencies, there are clearly two ranges of group velocities of the acoustic phonons. 
The larger values correspond to the longitudinal modes, while the lower values are obtained from the
transverse acoustic branches. Group velocities of acoustic phonons decrease for larger frequencies
and reach the average values typical for optic branches.

\begin{figure}[!t]
\centering
  \includegraphics[width=\linewidth]{GV.pdf}
\caption{Mode group velocities calculated as a function of phonon frequency. The colors correspond to the phonon branches.
}
  \label{thermvelocity}
\end{figure}

In Fig.~\ref{anizotropy}(a), we present the three diagonal elements of $\kappa_{\text{ph}}^{\mu\nu}$ corresponding to the main directions of the crystal structure. 
They were obtained from the force constants calculated at the base temperature $T=600$~K and the crystallite size $0.1$~$\mu$m to account for boundary-limited phonon transport. 
Due to the orthorhombic symmetry, we observe a small anisotropy in phonon transport in the whole temperature range. 
At low temperatures, the three components of the heat conductivity increase in a very similar way with the $\kappa_{\text{ph}}^{yy}$ element slightly larger than two other components. 
After reaching the maximum, we observe a change in the largest component from $\kappa_{\text{ph}}^{yy}$ to 
$\kappa_{\text{ph}}^{xx}$.    
In Fig.~\ref{anizotropy}(b), the thermal conductivity is shown for three base temperatures, at which the interatomic potential was obtained ($300$~K, $600$~K, and $1000$~K), using the energy expansion up to third- and fourth-order anharmonic terms, and the same structure size of $0.1$~$\mu$m. At lowest temperatures, the thermal conductivity strongly increases, reaching the maximum around $T=180$~K, then it shows a slower decrease with temperature.
The differences between the two levels of approximation are minimal, suggesting that third-order calculations already capture the dominant phonon scattering mechanisms. The dependence on the base temperature is also very weak, showing the changes in the heat conductivity within a few percent. 
\SP{}{Further verification of the reliability of our thermal conductivity results is given in Appendix~\ref{ThermalAB}, where the full BTE calculations show good agreement with RTA and higher-resolution RTA results, demonstrating that the $8\times8\times8$ $\bm{q}$-mesh already provides converged values.}

\begin{figure}[!t]
\centering
  \includegraphics[width=\linewidth]{Anizotropy.pdf}
\caption{(a) The anisotropic thermal conductivity of $\beta$-FeSi$_2$ calculated along the lattice directions at 600~K. (b) The average temperature-dependent thermal conductivity taken at $300$~K, $600$~K, and $1000$~K, including anharmonic corrections up to cubic (A3) and quartic (A4) terms. In both cases the crystallite size is 0.1~$\mu$m.}
  \label{anizotropy}
\end{figure}


In Fig.~\ref{therm}, we fix the base temperature at $600$~K and examine the effect of crystallite size on thermal conductivity, varying it from $0.01$ to $0.5$~$\mu$m.
With decreasing the crystallite size, we observe a shift of the position of the maximum to larger temperatures and a decrease of the thermal conductivity in the entire temperature range.
Theoretical results are compared with several experimental data obtained above the room temperature. 
The measured thermal conductivity depends to a large extent on the sample quality, its purity and the size of the crystalline grains which depends on 
the production processes.
Many measurements were performed using crystallites of micrometric or unknown size ~\cite{waldecker_1973,ito_2002,kim_2003,du_2020}, however, 
numerous attempts to minimize $\kappa$ by reducing grain sizes to $56$~nm~\cite{dabrowski_2019, dabrowski_microstructure_2021}, $30$-$400$~nm~\cite{le_tonquesse_2019}, $50$ and $200$~nm~\cite{abbassi_2021}, or introducing pores into the material~\cite{sam2023improved} 
are also carried out. 
Another way to change the thermal conductivity is to dope $\beta$-FeSi$_2$ with different elements~\cite{ito_2002,kim_2003,du_2020,cheng_2024}, however, this effect is beyond our investigation.

\begin{figure}[!t]
\centering
  \includegraphics[width=\linewidth]{Thermal_conductivity3.pdf}
\caption{
The phonon thermal conductivity of $\beta$-FeSi$_2$: theoretical results for the infinite crystalline size and with boundary conditions, compared with experimental data for different structure sizes.
}
  \label{therm}
\end{figure}
%Fig.~\ref{therm}\cite{abbassi_2021}\cite{waldecker_1973}\cite{dabrowski_2019}\cite{sam2023improved}\cite{kim_2003}\cite{du_2020}\cite{ito_2002}\cite{le_tonquesse_2019}.

We observe a decrease in the thermal conductivity with reducing crystalline grain sizes in all analyzed experimental data. 
For instance, by decreasing the crystallite size to $50$~nm, the thermal conductivity at room temperature was reduced by a factor of $1.7$, what can be  compared to the annealed sample with 200 nm grains~\cite{abbassi_2021}. 
It is worth noting that the rate of decrease in value with increasing temperature in both cases, for grain sizes of $50$~nm and $200$~nm, is significantly different, which is consistent with our calculations. 
The same trend can be observed by comparing the thermal conductivity measured for a sample with bulk crystallite sizes with the thermal conductivity of a sample with grains smaller than 400 nm~\cite{le_tonquesse_2019}.
The theoretical results obtained for the same crystallite size show higher values due to factors not captured in the idealized model, such as crystal imperfection or mechanical strain. Usually, a decrease in the crystallite size is related to an increased concentration of grain boundaries, point defects, and stacking faults that influence the phonon scattering~\cite{le_tonquesse_2019,abbassi_2021}.   

We should note that the total thermal conductivity is a combination
of the lattice and electronic contributions to the heat transport.
In semiconductors, the electronic thermal conductivity is negligible at low temperatures and significantly increases 
only much above the room temperatures~\cite{gu_2020}.
For $\beta$-FeSi$_2$, the electronic thermal conductivity was obtained from the electric conductivity using the Wiedemann-Franz law~\cite{ito_2002,kim_2003,le_tonquesse_2019}.
In the undoped material, its value does not exceed $0.1$~W/mK in the measurement up to $T=950$~K~\cite{kim_2003}.
By doping, the electronic thermal conductivity can be enhanced, and it has a direct impact on the thermoelectric properties of $\beta$-FeSi$_2$ at high temperatures~\cite{ito_2002,kim_2003}.
In the present study, we consider only the phonon contribution to the thermal conductivity,
therefore, agreement with experimental data may deteriorate with increasing temperature.

\section{Summary}
\label{sec.summary}

We performed {\it ab initio} studies on lattice dynamical and thermal transport properties of $\beta$-FeSi$_2$. The effect of anharmonicity was analyzed within two approaches -- the SCPH method and the perturbation theory.
The phonon dispersion curves obtained within SCPH show small renormalization of frequencies comparing to the harmonic approximation. 
The Raman spectra were calculated within the procedure which takes into account the peak intensities obtained from the Raman tensors and the line profiles obtained from the phonon self energy derived within the perturbation theory based on the large, temperature-independent, quartic model fitted to the data from the wide range of temperatures (300-1000~K). 
The anharmonic corrections strongly affect the frequencies and line profiles of some modes and results in overall better agreement with the experimental data. 
We analyzed the phonon lifetimes and group velocities obtained as functions of the phonon frequency.
Then the lattice thermal conductivity was calculated for a broad range of temperatures and grain sizes.
We found a small anisotropy in the phonon thermal transport resulting from the orthorhombic structure and a weak effect of the quartic anharmonic terms. 
The thermal conductivity calculated for various crystalline grain sizes show a good qualitative agreement with the available measurements.

\begin{acknowledgments}
Some figures in this work were rendered using {\sc Vesta}~\cite{momma.izumi.11} software.
This work was partially supported by the Ministry of Education, Youth and Sports of the Czech Republic through the e-INFRA CZ (ID:90254).
\end{acknowledgments}

\appendix

\section{Raman spectrum}
\label{RamanAA}

Based on the polarized Raman measurements reported in Ref.~\cite{maeda_2004}, two Raman peaks were identified as belonging to the A$_g$ symmetry class, and several additional peaks were observed with similar or different polarization dependence. Although the authors of Ref.~\cite{maeda_2004} provided estimates of the relative Raman tensor components, they did not specify which of the remaining modes correspond to the B$_{1g}$, B$_{2g}$, or B$_{3g}$ symmetries. Because of this missing experimental information, a direct symmetry-resolved comparison between the measurement and theory is not currently possible for the non-A$_g$ modes. 
To provide a complete theoretical picture of the B$_g$-type modes, we show here the calculated Raman-active frequencies and intensities for the B$_{1g}$, B$_{2g}$, and B$_{3g}$ symmetries only. 
%These results represent the predicted Raman modes for the B$_g$ symmetries in $\beta$-FeSi$_2$. 
\textcolor{red} {Fig.~\ref{raman1} shows the predicted Raman modes for the B$_g$ symmetries in $\beta$-FeSi$_2$. The results obtained within the harmonic approximation are compared with the anharmonic perturbation theory calculations which provides both, frequency shifts and predicted line profiles of the modes.}
Although the experimentally measured peaks cannot be directly assigned to these symmetries due to the lack of polarization-resolved data, the theoretical predictions provide a reference for comparison. Matching the measured frequencies to the closest theoretical B$_g$ modes (Table~\ref{T1Raman}) allows for a tentative assignment, which can guide future polarization-resolved Raman experiments aimed at determining the precise symmetry of the unresolved peaks.

\begin{figure*}[t]
\centering
  \includegraphics[width=\linewidth]{spectrum_RTA+raman_Bi_modes.pdf}
\caption{Raman spectrum of $\beta$-FeSi$_2$ calculated at three temperatures 
($300$, $600$, $900$~K -- blue, orange, and red lines, respectively). 
The calculated spectrum includes all Raman-active modes. The B$_{ig}$ modes
are indicated by purple vertical lines at peak positions corresponding to T=300~K. 
The frequencies derived from the harmonic approximation at T=300~K are indicated 
by green, yellow and pink lines.
Connecting arrows indicate the correspondence between harmonic and anharmonic 
frequencies, demonstrating the frequency shifts due to phonon interactions.}
\label{raman1}
\end{figure*}
\section{Thermal conductivity obtained from BTE and RTA}
\label{ThermalAB}

The thermal conductivity was computed by solving the full BTE on the largest feasible $\bm{q}$-point grid, $8\times8\times8$, and compared with the corresponding RTA results obtained on the same grid. As shown in Fig.~\ref{bte}, the difference between the components of the thermal conductivity tensor obtained within BTE and RTA at this resolution is very small, indicating a good agreement between these two approaches.
%THIS PART SHOULD GO RATHER TO THE RESPONSE
%Extending the full BTE calculation to larger grids is computationally prohibitive: the computational cost of BTE is roughly two orders of %magnitude higher than that of RTA, and the required memory and runtime exceed our available resources. 
Moreover, we performed an additional calculation using RTA on a denser $20\times20\times20$ grid. As seen in the Fig.~\ref{bte}, the higher-resolution data remain in a good agreement with both the BTE and RTA results for the $8\times8\times8$ grid.
It shows that the $8\times8\times8$ mesh already provides good results for this structure and confirms reliability of the calculations.

\begin{figure}[!h]
\centering
  \includegraphics[width=\linewidth]{BTEvsRTAvsANP.pdf}
\caption{Thermal conductivity of $\beta$-FeSi$_2$ obtained within the BTE and RTA methods using the Phono3py software on the $8\times8\times8$ q-point grid, compared with the RTA results computed with ALAMODE on a denser $20\times20\times20$ grid.}
\label{bte}
\end{figure}

%\section*{Data availability}
%The data that support the findings of this article are openly available~\footnote{give me DOI}.
% see https://journals.aps.org/authors/data-availability-statements#citation

\bibliography{refs.bib}
%\bibliographystyle{ieeetr}


\end{document}

\documentclass[%
%reprint,
superscriptaddress,
%groupedaddress,
longbibliography,
%unsortedaddress,
%runinaddress,
%frontmatterverbose, 
%preprint,
%preprintnumbers,
%nofootinbib,
nobibnotes,
%bibnotes,
amsmath,amssymb,
aps,
%pra,
prb,
%rmp,
%prstab,
%prstper,
%showkeys,
floatfix,
twocolumn
]{revtex4-2}

\usepackage{graphicx}% Include figure files
\usepackage{calc}% Calculate margins
\usepackage{dcolumn}% Align table columns on decimal point
\usepackage{bm}% bold math

\usepackage[urlcolor=blue,colorlinks=true,citecolor=blue,linkcolor=blue,pdfstartview={FitH},bookmarks=false]{hyperref} % add hypertext capabilities

%\usepackage[mathlines]{lineno} % Enable numbering of text and display math
% \linenumbers\relax % Commence numbering lines

% \usepackage[showframe,%Uncomment any one of the following lines to test 
% %scale=0.7, marginratio={1:1, 2:3}, ignoreall,% default settings
% %text={7in,10in},centering,
% %margin=1.5in,
% % total={6.5in,8.75in}, top=1.2in, left=0.9in, includefoot,
% % height=10in,a5paper,hmargin={3cm,0.8in},
% ]{geometry}

\usepackage{amsmath}
\usepackage{amssymb}
%\usepackage{orcidlink}
\usepackage{xcolor}
%\usepackage{datetime}
\usepackage[normalem]{ulem}

% Change tracking commands
\newcommand{\trackchange}[3]{\textcolor{#3}{\sout{#1}#2}}  % Full color strikeout, insert
%\renewcommand{\trackchange}[3]{\textcolor{#3}{#2}}        % Just color silent remove and insert
%\renewcommand{\trackchange}[3]{{#2}}                      % No indication, silent remove and insert

% Author marker definitions

\definecolor{myblue}{RGB}{0,127,85}
% \definecolor{violet}{RGB}{102,0,204}
% \definecolor{orange}{RGB}{255,128,0}
% \definecolor{green}{RGB}{0,128,0}
\newcommand{\DL}[1]{\trackchange{}{#1}{blue}}
\newcommand{\AP}[1]{\trackchange{}{#1}{red}}
\newcommand{\PJ}[2]{\trackchange{#1}{#2}{orange}}
\newcommand{\JL}[2]{\trackchange{#1}{#2}{myblue}}
\newcommand{\SP}[2]{\trackchange{#1}{#2}{blue}}
\newcommand{\PP}[2]{\trackchange{#1}{#2}{teal}}
\newcommand{\AI}[2]{\trackchange{#1}{#2}{olive}}
\newcommand{\MS}[1]{\trackchange{}{#1}{purple}}

\newcommand{\TODO}[1]{\textcolor{red}{TODO: #1}}

\sloppy

\begin{document}

\title{Ab initio study of the anharmonic properties and thermal conductivity in $\beta$-FeSi$_2$}

\author{Svitlana~Pastukh}
\email[e-mail: ]{svitlana.pastukh@ifj.edu.pl}
\affiliation{Institute of Nuclear Physics, Polish Academy of Sciences, ul. W. E. Radzikowskiego 152, 31-342 Krak\'{o}w, Poland}

\author{Ma\l{}gorzata~Sternik}
\affiliation{Institute of Nuclear Physics, Polish Academy of Sciences, ul. W. E. Radzikowskiego 152, 31-342 Krak\'{o}w, Poland}

\author{Pawe\l{}~T.~Jochym}
\affiliation{Institute of Nuclear Physics, Polish Academy of Sciences, ul. W. E. Radzikowskiego 152, 31-342 Krak\'{o}w, Poland}


\author{Jan~\L{}a\.{z}ewski}
\affiliation{Institute of Nuclear Physics, Polish Academy of Sciences, ul. W. E. Radzikowskiego 152, 31-342 Krak\'{o}w, Poland}

\author{Andrzej~Ptok}
\affiliation{Institute of Nuclear Physics, Polish Academy of Sciences, ul. W. E. Radzikowskiego 152, 31-342 Krak\'{o}w, Poland}

\author{Svetoslav~Stankov}
\affiliation{Institute for Photon Science and Synchrotron Radiation, Karlsruhe Institute of Technology, D-76131 Karlsruhe, Germany}
\affiliation{Laboratory for Applications of Synchrotron Radiation, Karlsruhe Institute of Technology, D-76131 Karlsruhe, Germany}

\author{Przemys\l{}aw~Piekarz}
\affiliation{Institute of Nuclear Physics, Polish Academy of Sciences, ul. W. E. Radzikowskiego 152, 31-342 Krak\'{o}w, Poland}

\date{\today}

\begin{abstract}

Iron silicides are good candidates for applications in  optoelectronic and thermoelectric devices.
Lattice dynamical properties and thermal conductivity in the $\beta$-FeSi$_2$ semiconductor
are investigated with the first-principles computational methods. 
Phonon dispersion relations are calculated via the
temperature-dependent effective potential method and self-consistent phonon theory. 
To properly model thermal transport, we explicitly consider
the impact of phonon-phonon interactions by analyzing
anharmonic contributions to the phonon self-energy. 
This yields temperature-dependent phonon frequencies and linewidths,
reflecting the finite lifetime of phonons due to scattering
processes. The calculated phonon frequencies and line profiles are used to obtain 
the Raman spectra, which shows good agreement with the experimental data. 
We revealed an enhanced anharmonic behaviour of the Raman modes with the highest frequencies.  
The lattice thermal conductivity is then obtained as a function of temperature and crystallite size within
the relaxation-time approximation.
Phonon transport shows a small anisotropy due to the orthorhombic structure and a very weak dependence
on the quartic anharmonic corrections. The results obtained for an infinite material and for several crystallite sizes
were analyzed and compared with the available experimental data.
\end{abstract}

\maketitle


\section{Introduction}

The comprehensive determination of important physical properties of
crystals, such as thermal expansion, lattice thermal
conductivity or structural phase transitions, requires a fundamental 
understanding of the anharmonic effects.
Although the investigation of anharmonic interactions in crystals has
attracted a considerable interest for decades~\cite{cowley_1968}, a substantial progress
has only recently been achieved thanks to advances in theoretical and
numerical methods and increased computational power.
Now, phonon frequencies, lifetimes, and heat transfer in a wide range of
materials
can be quantitatively predicted using the available computational resources
based on the density functional theory (DFT)~\cite{lindsay_2013,mcgaughey_2019,lindsay_2019}.
In the case of strongly anharmonic systems, the self-consistent phonon
(SCPH) theory~\cite{tadano_2015} as well as the perturbative approach~\cite{tadano_2018}, using higher
order interatomic force constants derived from the fitting to the displacement force data obtained
with DFT, have proven to be successful.

Transition-metal silicides are promising materials for fabrication of electronic components
designed for integration with silicon-based circuits~\cite{murarka_1995}.
At room temperature, iron disilicide ($\beta$-FeSi$_2$) is a
direct-bandgap semiconductor~\cite{bost_1985}, making this material a good
candidate for application in optoelectronic devices such as infrared
detectors or light emitters~\cite{bost_1988}. The development of light-emitting diodes utilizing FeSi$_2$/Si heterostructures has been successfully demonstrated~\cite{leong_1997,suemasu_2001}. Due to
a high thermal stability and strong light absorption, FeSi$_2$ is
also a suitable photovoltaic material~\cite{powalla_1993,liu_2006,okuhara_2017}.

$\beta$-FeSi$_2$ crystallizes in the base-centered orthorhombic
lattice~\cite{dusausoy_1971} transforming to the
tetragonal metallic $\alpha$-FeSi$_{2}$ phase around $1200$~K~\cite{starke_2002}.
Optical studies indicated a direct band gap of the values
$0.85$--$0.89$~eV~\cite{bost_1988,dimitriadis_1990,arushanov_1995,wan_2003},
however, the {\it ab initio} calculations predicted a smaller indirect gap close to 0.8 eV~\cite{christensen_1990}. 
The existence of such an indirect gap was then confirmed by the optical linear transmittance measurements at
low temperatures~\cite{giannini_1992}. As shown by first-principles studies,
the character of the band gap is very sensitive to the orientation 
of a crystal grown on silicon~\cite{clark_1998}.

$\beta$-FeSi$_2$ belongs also to good thermoelectric materials~\cite{ware_1964}, with potential applications resulting from its chemical stability up to high temperatures, nontoxicity, and low cost of preparation~\cite{yamada_2012,nozariasbmarz_2017}. 
It has already been implemented in cars~\cite{birkholz_1988} and portable power sources~\cite{uemura_1989}. 
Its thermoelectric performance can be improved by doping~\cite{ito_2001,tani_2001,kim_2003,chen_2005,pandey_2013,le_tonquesse_2019}, 
which enhances the electric transport and reduces the thermal conductivity~\cite{waldecker_1973,du_2019,du_2020}.  
The thermal conductivity can be also reduced by the modification of microstructure~\cite{ail_2015} or by
nanostructurization~\cite{watanabe_2017,taniguchi_2017,hsin_2017,abbassi_2021}.

The lattice thermal conductivity is directly connected with anharmonic effects and phonon scattering processes.
The vibrational properties of \mbox{$\beta$-FeSi$_2$} were studied by the infrared and Raman spectroscopy~\cite{lefki_1991,guizzetti_1997,maeda_2004,baleva_2008,liu_2011,maeda_2011}. The observed anisotropy in the phonon spectra results from the enhanced sensitivity of the infrared and Raman features to the local lattice distortions~\cite{guizzetti_1997}. The Fe phonon density of states was measured by nuclear inelastic scattering (NIS), showing a good agreement with the density functional theory (DFT) calculations~\cite{walterfang_2005}. Using the DFT approach, the phonon dispersion curves, phonon density of states, as well as various thermodynamic properties were obtained within the harmonic approximation~\cite{tani_2010,liang_2011}. The extended Klemens model was applied to study
the anharmonic effect on phonon frequencies and linewidths observed by the Raman spectroscopy~\cite{zhang_2023}.
The impact of nanostructurization on lattice dynamics was explored in the $\beta$-FeSi$_2$ nanorods grown on the Si(110) surface by the NIS and {\it ab initio} methods~\cite{kalt_2022}.

In this work, we investigate the lattice dynamical properties of $\beta$-FeSi$_2$ 
using the DFT calculations. We study the effect of anharmonic terms in the temperature-dependent potential on phonon frequencies and lifetimes. We focus on the Raman modes, comparing the theoretical results with the experimental data.
The thermal conductivity is derived in a broad temperature range and the effect of crystallite size is analyzed.



This study is structured as follows.
In Sec.~\ref{sec.com} we describe the details of computational methods.
Next, in Sec.~\ref{sec.result} we present and discuss our results.
In particular we present the crystal structure (Sec.~\ref{sec.crys}) and lattice dynamics (Sec.~\ref{sec.lattice}).
We investigate also the thermal conductivity comparing the obtained results with the available experimental data (Sec.~\ref{sec.thermal}).
Finally, Sec.~\ref{sec.summary} summarizes our key findings and conclusions.

\section{Calculation method}
\label{sec.com}


The calculations were performed using the projector augmented-wave potentials~\cite{blochl_1994} and the generalized gradient approximation~\cite{perdew_1996} implemented in the Vienna Ab initio Simulation Package (VASP)~\cite{kresse.hafner.94,kresse.furthmuller.96,kresse.joubert.99}. 
The lattice parameters and atomic positions were optimized in the ${\bm a} \times ({\bm b}-{\bm c}) \times ({\bm b}+{\bm c})$ supercell containing 32 formula units and four primitive cells.
The integration in the reciprocal space was conducted using the $2 \times 2 \times 2$ Monkhorst--Pack mesh~\cite{monkhorst_1976} and the cut-off energy was set to $500$~eV. For convergence conditions, we set the energy change below $10^{-5}$ and $10^{-8}$ for the ionic and electronic loops, respectively. 

The lattice dynamical properties were studied within the temperature-dependent effective potential (TDEP) approach~\cite{hellman_2013}. The atomic potential with the third and fourth order anharmonic terms was derived from interatomic forces induced by displacements of all atoms at finite temperatures.
The sets of atomic displacements were generated by the high efficiency configuration space sampling (HECSS)~\cite{jochym_2021} and forces were obtained by VASP. The interatomic force constants and phonon frequencies were calculated with the {\sc Alamode} software~\cite{tadano_2014}.

Furthermore, we have attempted to construct a {\emph{temperature independent}} 
anharmonic model. We have used combined data from all investigated temperatures 
(300, 600 and 1000~K) and fitted a large (over 15~000 free parameters), fourth-order 
interaction model to this dataset. 
Subsequently, we have used this model to calculate line profiles and positions of Raman-active modes
at multiple temperatures.

The changes in phonon frequencies induced by the anharmonic effects were investigated within two approaches.
First, the impact of the quartic anharmonic terms was included using the SCPH theory~\cite{tadano_2015}.
Second, the mode profiles (frequency shifts and line widths) were determined from the real and imaginary parts of the phonon self-energy resulting from the cubic and quartic anharmonic terms of the above mentioned large model~\cite{tadano_2018}. 
The longitudinal optic-transverse optic (LO-TO) splitting was also evaluated, using the static dielectric tensor and Born effective charges calculated within density functional perturbation theory~\cite{gajdos_2006}.

\begin{figure}[]
    \centering
    \includegraphics[width=\linewidth]{fig1_new.png}
\caption{%
(a) The conventional unit cell of $\beta$-FeSi$_2$ (with Cmca symmetry) and (b) the corresponding Brillouin zone with selected high-symmetry points.
}
  \label{fig.struct}
\end{figure}


To further characterize the vibrational properties, Raman scattering was investigated using the Phonopy-Spectroscopy package~\cite{skelton_2017}. This enabled the identification of Raman-active modes and the calculation of Raman tensors. Anharmonic force constants, derived from calculations using {\sc Alamode}, were then used to obtain theoretical line profiles for the Raman modes. The presented Raman scattering spectra combine these anharmonic line profiles with the Raman tensor amplitudes.
This analysis was also based on the large quartic model mentioned above.

Finally, the thermal conductivity was calculated as a function of temperature and crystallite size within the relaxation-time approximation (RTA) \SP{}{as implemented in {\sc Alamode}~\cite{tadano_2014}}. The phonon lifetimes were calculated from the phonon self-energy including the cubic and quartic anharmonic terms. \SP{}{The RTA provides a solution to the Boltzmann transport equation (BTE) under the assumption that scattering events are independent and can be treated through mode-resolved relaxation times.
To verify the validity of this approximation for $\beta$-FeSi$_2$, we additionally
solved the BTE iteratively.} \SP{}{These calculations were performed with
{\sc Phono3py}~\cite{togo_2023}. For cross-validation, the additional RTA calculations were performed on the same $\bm{q}$-grid as the BTE calculations, using the implementation provided by {\sc Phono3py}.}

\section{Results}
\label{sec.result}

\subsection{Crystal structure}
\label{sec.crys}

The $\beta$-FeSi$_2$ structure adopts a base-centered orthorhombic lattice with the space group Cmca (No.~64) as shown in Fig.~\ref{fig.struct}(a).
The unit cell consists of two primitive cells and contains 48 atoms.
Iron (silicon) atoms possess two nonequivalent positions: \mbox{Fe-I} and \mbox{Fe-II} (\mbox{Si-I} and \mbox{Si-II}), presented in Fig.~\ref{fig.struct}(a) as gray and purple (orange and yellow) spheres, respectively.
This crystal structure is derived from the fluorite-type lattice with strongly distorted Si cubes and Fe atoms occupying 
one-half of the central sites.
The Fe-I and Fe-II sites create different layers perpendicular to the $x$ direction, and they are separated by layers containing both Si sites.
Each Fe atom is coordinated by 8 Si atoms with slightly different Fe-Si distances.
The optimized lattice constants ($a = 9.874$~{\AA}, $b = 7.767$~{\AA}, and
$c = 7.811$~{\AA}) agree very well with the experimental data ($a = 9.863$~{\AA}, $b = 7.791$~{\AA}, and $c = 7.833$~{\AA})~\cite{dusausoy_1971}.

Iron atoms occupy the Wyckoff sites 8\textit{d} ($0.2166$, $0$, $0$) and  8\textit{f} ($0$, $0.3072$, $0.1879$), corresponding to \mbox{Fe-I} and \mbox{Fe-II}, respectively. 
Silicon atoms are located at two inequivalent 16\textit{g} positions: ($0.1282$, $0.2737$, $0.0495$) and \mbox{($0.3734$, $0.0445$, $0.2270$)}, assigned as \mbox{Si-I} and \mbox{Si-II}.
The optimized positions of atoms agree very well with the experimental data~\cite{dusausoy_1971} and the previous theoretical studies~\cite{tani_2010,liang_2011}.


\subsection{Lattice dynamics}
\label{sec.lattice}

\begin{figure}[]
    \centering
    \includegraphics[width=\linewidth]{Fig1c.pdf}
\caption{%
The phonon dispersion curves along high symmetry directions obtained within SCPH (for temperatures from $0$ to $1000$~K).
Dashed black lines indicate the phonon dispersions obtained from the harmonic approximation. The white dots indicate Raman-active modes with A$_g$ symmetry.
The vertical plot shows the phonon density of states (DOS) calculated at a reference temperature of $600$~K.
}
  \label{fig.ph_band}
\end{figure}


In Fig.~\ref{fig.ph_band} we present the phonon dispersion relations of $\beta$-FeSi$_2$ along high-symmetry directions in the Brillouin zone [Fig.~\ref{fig.struct}(b)].
Due to 24 atoms in the primitive cell, the phonon spectrum consists of 69 optical modes and three acoustic modes.
The phonon dispersions were calculated within the SCPH approach in the temperature range $0$--$1000$~K (presented by color lines in Fig.~\ref{fig.ph_band}), and they are compared with the results obtained from the harmonic part of the effective potential corresponding to temperature $T=300$~K (indicated by dashed black lines in Fig.~\ref{fig.ph_band}).
As we can see within the SCPH method, the anharmonic effects are rather weak and leads only to small renormalization of phonon frequencies.
Only the highest modes show more pronounced shifts of their frequencies to larger values.
The total and partial element-projected phonon density of states obtained within the harmonic approximation are presented in Fig.~\ref{fig.ph_band}.
Up to around $320$~cm$^{-1}$, the contributions from both elements are very similar, while for higher frequencies the spectrum is dominated by
the Si vibrations.


\begin{table}[!t]
\begin{ruledtabular}
\caption{Calculated and experimental Raman-active modes of $\beta$-FeSi$_2$ with their irreducible representations (IR). Present theoretical results are compared with the previous theoretical data from Ref.~\cite{tani_2010} and experimental results from \mbox{Refs.~\cite{lefki_1991,maeda_2004}.} The experimental frequencies with known symmetries (A$_g$) are shown in bold, while other experimental modes are assigned to the best fitting theoretical values.}
\begin{tabular}{c c c c c}
\textbf{IR} & \multicolumn{4}{c}{\textbf{Frequency (cm$^{-1}$)}} \\
 & Present  & Theor.~\cite{tani_2010} & Exp.~\cite{lefki_1991} & Exp.~\cite{maeda_2004} \\
\hline
B$_{2g}$ & 175.4 & 179 & 176 &  \\
B$_{1g}$ & 176.3 & 185 & 179  &  \\
B$_{1g}$ & 193.3 & 198 &  & 190.6 \\
A$_g$    & 196.6 & 208 & \textbf{195} & \textbf{194.0} \\
A$_g$    & 203.9 & 210 & \textbf{197} & 199.6 \\
B$_{3g}$ & 205.5 & 212 & 200 &  \\
B$_{3g}$ & 226.3 & 236 & 206 & 227.1 \\
B$_{1g}$ & 233.3 & 240 &  &  231.6 \\
B$_{2g}$ & 248.6 & 254 &  &  \\
A$_g$    & 250.1 & 257 & \textbf{247} & \textbf{247.3} \\
A$_g$    & 254.9 & 264 & \textbf{253} & 254.3 \\
B$_{1g}$ & 275.5 & 285 &  &  274.1 \\
B$_{2g}$ & 282.4 & 295 &  &  281.2 \\
B$_{3g}$ & 286.8 & 297 &  &  \\
B$_{1g}$ & 307.1 & 317 &  &  \\
B$_{2g}$ & 312.6 & 326 &  &  311.8 \\
B$_{1g}$ & 319.0 & 324 &  &  \\
B$_{3g}$ & 327.9 & 341 &  & 325.8 \\
B$_{2g}$ & 333.3 & 345 &  &  \\
A$_g$    & 339.3 & 352 & \textbf{346} & 339.5 \\
B$_{2g}$ & 343.0 & 350 &  &  \\
B$_{1g}$ & 353.5 & 366 &  &  \\
B$_{2g}$ & 372.8 & 383 &  & 370.7 \\
B$_{1g}$ & 375.1 & 385 &  &  \\
B$_{3g}$ & 375.2 & 386 &  &  \\
B$_{3g}$ & 385.3 & 401 &  &  \\
A$_g$    & 386.5 & 398 &  & 386.2 \\
B$_{2g}$ & 387.7 & 402 &  & 388.2 \\
A$_g$    & 404.6 & 415 &  & 400.4 \\
B$_{2g}$ & 405.1 & 420 &  &  \\
B$_{1g}$ & 412.7 & 428 &  &  \\
B$_{3g}$ & 418.4 & 431 &  &  \\
B$_{3g}$ & 441.4 & 458 &  & 442.6 \\
B$_{3g}$ & 447.3 & 466 &  & 446.3 \\
A$_g$    & 448.4 & 464 &  &  \\
A$_g$    & 499.1 & 517 &  &  \\
\end{tabular}
\label{T1Raman}
\end{ruledtabular}
\end{table}

%subsection{Raman}

\begin{figure*}
\centering
  \includegraphics[width=\textwidth]{spectrum_RTA+raman.pdf}
\caption{Raman spectrum of $\beta$-FeSi$_2$ calculated within the perturbative approach at three temperatures 
($300$, $600$, $900$~K -- blue, orange, and red lines, respectively). 
The calculated spectrum includes all Raman-active modes. The A$_g$ modes
are indicated by purple vertical lines at peak positions corresponding to T=300~K. 
The frequencies derived from the harmonic approximation at T=300~K are indicated by green lines.
Connecting arrows indicate the correspondence between harmonic and
anharmonic frequencies, demonstrating the frequency shifts due to 
phonon interactions. Experimental values for the A$_g$ modes
based on Ref.~\cite{lefki_1991} are marked with black dashed lines.
}
\label{raman}
\end{figure*}


The phonon spectrum at the $\Gamma$ point consists of 36 Raman modes classified according to the irreducible representations: $9A_\text{g}+9B_\text{1g}+9B_\text{2g}+9B_\text{3g}$. 
Fig.~\ref{raman} shows the Raman spectrum of A$_g$ symmetry calculated for $\beta$-FeSi$_2$ within the perturbative approach at three temperatures $300$, $600$, and $900$~K (solid blue, orange, and red curves, respectively), including third-order and fourth-order anharmonic corrections.
The calculated Raman spectrum includes all Raman-active modes.
The five experimentally observed A$_g$ modes are highlighted by black dashed lines, based on data from Ref.~\cite{lefki_1991}.
Additional peaks not marked with vertical lines correspond to phonon modes with symmetries other than A$_g$.
The frequencies of Raman modes obtained in anharmonic calculations are compared with the previous results calculated within the harmonic approximation and the experimental values in Tab.~\ref{T1Raman}. 
We have marked in bold the experimentally determined A$_g$ modes, which are compared with the calculations.
Since the experimental studies did not provide the accurate assignement of the Raman modes with the B$_{1g}$, B$_{2g}$, and B$_{3g}$ symmetry~\cite{lefki_1991,maeda_2004}, we cannot compare them directly with the theoretical results.
However, in Tab.~\ref{T1Raman} we have assigned the measured frequencies to the best fitting theoretical values without taking into account the symmetry of the modes, except for the known A$_{g}$ modes. 


The impact of anharmonicity on the phonon frequencies is well visible from the comparison of the results obtained within the harmonic approximation 
and from the anharmonic calculation (vertical green and purple lines in Fig.~\ref{raman}, respectively).
Here we show only the A$_g$ modes, which are compared with the experimental results (vertical black dashed lines).
Anharmonic frequencies calculated at $300$~K are indicated by purple lines, while frequencies derived from harmonic approximation are marked with green lines. The grey solid lines connect corresponding modes obtained in both approximations.
In most cases, the results obtained within the harmonic approximation do not agree with the experimental frequencies. 
%Only the modes close to $250$~cm$^{-1}$ and $400$~cm$^{-1}$ correspond well to the experimental values. MS
As we can see, the inclusion of the anharmonic correction leads to a significant modification of the phonon frequencies.
These anharmonic effects are stronger for higher-frequency modes mainly because of
the dominant contribution from Si atoms, which vibrate with larger amplitudes than heavier Fe atoms. 
When atoms move to larger distances the potential deviates more from the harmonic approximation,
and the anharmonic corrections become stronger.


The modification of phonon frequencies observed in Fig.~\ref{raman} is much larger than in the SCPH scheme presented in Fig.~\ref{fig.ph_band}. The SCPH approach includes only the leading-order contribution to 
the phonon self-energy obtained from the quartic anharmonic terms~\cite{tadano_2015}. Therefore, it does not describe fully the changes of phonon frequencies found within the perturbation theory (see Fig.~\ref{raman}).
Especially, it is well visible for two highest A$_g$ modes, which exhibit also the largest line broadening 
and the strongest dependence on temperature.
Therefore, a better agreement with experimentally observed frequencies is visible,
confirming the significant influence of the anharmonicity on the frequencies and line profiles of phonon modes.
In fact, the decrease of phonon frequencies should be even stronger due to thermal expansion, 
which is not included in our calculations.
Within SCPH the frequencies of the highest modes increase with increasing temperature as we see in Fig.~\ref{fig.ph_band}. The comparison of two different approaches applied to study anharmonic properties of $\beta$-FeSi$_2$ shows that the perturbation theory, which includes the cubic and quartic terms, better describes the changes of phonon frequencies with temperature than the SCPH method. \SP{}{This indicates that the leading-order contribution included in SCPH are not important in this material.}

Additionaly we should nottice that for other than A$_g$ modes we cannot make an unambiguous assignment of theoretical frequencies to experimental ones. Note that the spectrum in Fig.~\ref{fig.ph_band} contains all Raman-active modes. 
The limitation to A$_g$ modes concerns only the indicated positions of the peaks. 

\SP{}{The full Raman spectrum, with the frequencies of the B$_{1g}$, B$_{2g}$ and B$_{3g}$ modes marked, is shown in Appendix A, Fig.~\ref{raman1}. This represents our theoretical prediction of possible Raman-mode assignments, which can be verified in future experiments.}
%This is the theoretical prediction of possible assignment of Raman modes that can be verified in future experiments.  


\subsection{Thermal conductivity}
\label{sec.thermal}


\begin{figure*}[!t]
\centering
  \includegraphics[width=\linewidth]{time3.pdf}
\caption{Phonon lifetimes calculated for three temperatures as a function of phonon frequency. The colors correspond to the phonon branches.}
  \label{thermtime}
\end{figure*}


In this section, we analyze the thermal conductivity tensor of $\beta$-FeSi$_2$ obtained within the RTA approach~\cite{tadano_2018} as a function of temperature
%
\begin{equation}
\kappa_{\text{ph}}^{\mu\nu}(T) = \frac{1}{NV} \sum_{\bm{q},j} c_{\bm{q}j}(T) v_{\bm{q}j}^{\mu} v_{\bm{q}j}^{\nu}\tau_{\bm{q}j}(T),
\end{equation}
% 
where $c_{\bm{q}j}$ is the mode heat capacity and $v_{\bm{q}j}$ is the mode group velocity. 
The relaxation time is approximated by the phonon lifetime $\tau_{\bm{q}j}$
calculated for $j$-th branch at the wave vector $\bm{q}$.
$V$ is the unit cell volume and $N$ is the number of unit cells in the crystal.
The phonon lifetime is calculated using this formula
%
\begin{equation}
\tau_{\bm{q}j}(T)=\frac{1}{2\Gamma_{\bm{q}j}^{\text{anh}}(T)},
\end{equation}
%
where $\Gamma_{\bm{q}j}^{\text{anh}}$ is the anharmonic phonon linewidth obtained from 
the imaginary part of the phonon self-energy within the perturbation theory.

In Fig.~\ref{thermtime}, we present $\tau_{\bm{q}j}$ obtained for three temperatures 300, 600, and 1000~K as a function of frequency. As we see, the acoustic phonons close to the $\Gamma$ point have the longest lifetimes,
which are diminished with increasing frequency reaching local minima around $200$~cm$^{-1}$.
For higher frequencies, phonon lifetimes first increase to local maxima around $300$~cm$^{-1}$ and then decrease to
the lowest values in the range of highest optical modes. The shortest lifetimes correspond to the largest line broadening
observed for the Raman modes in Fig~\ref{raman}. The phonon group velocities 
$v_{\bm{q}j}=\partial\omega_{\bm{q}j}/\partial\bm{q}$, which are obtained by the central difference formula, are presented in Fig.~\ref{thermvelocity}. Their temperature dependence is negligible, therefore, we present only the results for $T=600$~K.
At low frequencies, there are clearly two ranges of group velocities of the acoustic phonons. 
The larger values correspond to the longitudinal modes, while the lower values are obtained from the
transverse acoustic branches. Group velocities of acoustic phonons decrease for larger frequencies
and reach the average values typical for optic branches.

\begin{figure}[!t]
\centering
  \includegraphics[width=\linewidth]{GV.pdf}
\caption{Mode group velocities calculated as a function of phonon frequency. The colors correspond to the phonon branches.
}
  \label{thermvelocity}
\end{figure}

In Fig.~\ref{anizotropy}(a), we present the three diagonal elements of $\kappa_{\text{ph}}^{\mu\nu}$ corresponding to the main directions of the crystal structure. 
They were obtained from the force constants calculated at the base temperature $T=600$~K and the crystallite size $0.1$~$\mu$m to account for boundary-limited phonon transport. 
Due to the orthorhombic symmetry, we observe a small anisotropy in phonon transport in the whole temperature range. 
At low temperatures, the three components of the heat conductivity increase in a very similar way with the $\kappa_{\text{ph}}^{yy}$ element slightly larger than two other components. 
After reaching the maximum, we observe a change in the largest component from $\kappa_{\text{ph}}^{yy}$ to 
$\kappa_{\text{ph}}^{xx}$.    
In Fig.~\ref{anizotropy}(b), the thermal conductivity is shown for three base temperatures, at which the interatomic potential was obtained ($300$~K, $600$~K, and $1000$~K), using the energy expansion up to third- and fourth-order anharmonic terms, and the same structure size of $0.1$~$\mu$m. At lowest temperatures, the thermal conductivity strongly increases, reaching the maximum around $T=180$~K, then it shows a slower decrease with temperature.
The differences between the two levels of approximation are minimal, suggesting that third-order calculations already capture the dominant phonon scattering mechanisms. The dependence on the base temperature is also very weak, showing the changes in the heat conductivity within a few percent. 
\SP{}{Further verification of the reliability of our thermal conductivity results is given in Appendix~\ref{ThermalAB}, where the full BTE calculations show good agreement with RTA and higher-resolution RTA results, demonstrating that the $8\times8\times8$ $\bm{q}$-mesh already provides converged values.}

\begin{figure}[!t]
\centering
  \includegraphics[width=\linewidth]{Anizotropy.pdf}
\caption{(a) The anisotropic thermal conductivity of $\beta$-FeSi$_2$ calculated along the lattice directions at 600~K. (b) The average temperature-dependent thermal conductivity taken at $300$~K, $600$~K, and $1000$~K, including anharmonic corrections up to cubic (A3) and quartic (A4) terms. In both cases the crystallite size is 0.1~$\mu$m.}
  \label{anizotropy}
\end{figure}


In Fig.~\ref{therm}, we fix the base temperature at $600$~K and examine the effect of crystallite size on thermal conductivity, varying it from $0.01$ to $0.5$~$\mu$m.
With decreasing the crystallite size, we observe a shift of the position of the maximum to larger temperatures and a decrease of the thermal conductivity in the entire temperature range.
Theoretical results are compared with several experimental data obtained above the room temperature. 
The measured thermal conductivity depends to a large extent on the sample quality, its purity and the size of the crystalline grains which depends on 
the production processes.
Many measurements were performed using crystallites of micrometric or unknown size ~\cite{waldecker_1973,ito_2002,kim_2003,du_2020}, however, 
numerous attempts to minimize $\kappa$ by reducing grain sizes to $56$~nm~\cite{dabrowski_2019, dabrowski_microstructure_2021}, $30$-$400$~nm~\cite{le_tonquesse_2019}, $50$ and $200$~nm~\cite{abbassi_2021}, or introducing pores into the material~\cite{sam2023improved} 
are also carried out. 
Another way to change the thermal conductivity is to dope $\beta$-FeSi$_2$ with different elements~\cite{ito_2002,kim_2003,du_2020,cheng_2024}, however, this effect is beyond our investigation.

\begin{figure}[!t]
\centering
  \includegraphics[width=\linewidth]{Thermal_conductivity3.pdf}
\caption{
The phonon thermal conductivity of $\beta$-FeSi$_2$: theoretical results for the infinite crystalline size and with boundary conditions, compared with experimental data for different structure sizes.
}
  \label{therm}
\end{figure}
%Fig.~\ref{therm}\cite{abbassi_2021}\cite{waldecker_1973}\cite{dabrowski_2019}\cite{sam2023improved}\cite{kim_2003}\cite{du_2020}\cite{ito_2002}\cite{le_tonquesse_2019}.

We observe a decrease in the thermal conductivity with reducing crystalline grain sizes in all analyzed experimental data. 
For instance, by decreasing the crystallite size to $50$~nm, the thermal conductivity at room temperature was reduced by a factor of $1.7$, what can be  compared to the annealed sample with 200 nm grains~\cite{abbassi_2021}. 
It is worth noting that the rate of decrease in value with increasing temperature in both cases, for grain sizes of $50$~nm and $200$~nm, is significantly different, which is consistent with our calculations. 
The same trend can be observed by comparing the thermal conductivity measured for a sample with bulk crystallite sizes with the thermal conductivity of a sample with grains smaller than 400 nm~\cite{le_tonquesse_2019}.
The theoretical results obtained for the same crystallite size show higher values due to factors not captured in the idealized model, such as crystal imperfection or mechanical strain. Usually, a decrease in the crystallite size is related to an increased concentration of grain boundaries, point defects, and stacking faults that influence the phonon scattering~\cite{le_tonquesse_2019,abbassi_2021}.   

We should note that the total thermal conductivity is a combination
of the lattice and electronic contributions to the heat transport.
In semiconductors, the electronic thermal conductivity is negligible at low temperatures and significantly increases 
only much above the room temperatures~\cite{gu_2020}.
For $\beta$-FeSi$_2$, the electronic thermal conductivity was obtained from the electric conductivity using the Wiedemann-Franz law~\cite{ito_2002,kim_2003,le_tonquesse_2019}.
In the undoped material, its value does not exceed $0.1$~W/mK in the measurement up to $T=950$~K~\cite{kim_2003}.
By doping, the electronic thermal conductivity can be enhanced, and it has a direct impact on the thermoelectric properties of $\beta$-FeSi$_2$ at high temperatures~\cite{ito_2002,kim_2003}.
In the present study, we consider only the phonon contribution to the thermal conductivity,
therefore, agreement with experimental data may deteriorate with increasing temperature.

\section{Summary}
\label{sec.summary}

We performed {\it ab initio} studies on lattice dynamical and thermal transport properties of $\beta$-FeSi$_2$. The effect of anharmonicity was analyzed within two approaches -- the SCPH method and the perturbation theory.
The phonon dispersion curves obtained within SCPH show small renormalization of frequencies comparing to the harmonic approximation. 
The Raman spectra were calculated within the procedure which takes into account the peak intensities obtained from the Raman tensors and the line profiles obtained from the phonon self energy derived within the perturbation theory based on the large, temperature-independent, quartic model fitted to the data from the wide range of temperatures (300-1000~K). 
The anharmonic corrections strongly affect the frequencies and line profiles of some modes and results in overall better agreement with the experimental data. 
We analyzed the phonon lifetimes and group velocities obtained as functions of the phonon frequency.
Then the lattice thermal conductivity was calculated for a broad range of temperatures and grain sizes.
We found a small anisotropy in the phonon thermal transport resulting from the orthorhombic structure and a weak effect of the quartic anharmonic terms. 
The thermal conductivity calculated for various crystalline grain sizes show a good qualitative agreement with the available measurements.

\begin{acknowledgments}
Some figures in this work were rendered using {\sc Vesta}~\cite{momma.izumi.11} software.
This work was partially supported by the Ministry of Education, Youth and Sports of the Czech Republic through the e-INFRA CZ (ID:90254).
\end{acknowledgments}

\appendix

\section{Raman spectrum}
\label{RamanAA}

Based on the polarized Raman measurements reported in Ref.~\cite{maeda_2004}, two Raman peaks were identified as belonging to the A$_g$ symmetry class, and several additional peaks were observed with similar or different polarization dependence. Although the authors of Ref.~\cite{maeda_2004} provided estimates of the relative Raman tensor components, they did not specify which of the remaining modes correspond to the B$_{1g}$, B$_{2g}$, or B$_{3g}$ symmetries. Because of this missing experimental information, a direct symmetry-resolved comparison between the measurement and theory is not currently possible for the non-A$_g$ modes. 
To provide a complete theoretical picture of the B$_g$-type modes, we show here the calculated Raman-active frequencies and intensities for the B$_{1g}$, B$_{2g}$, and B$_{3g}$ symmetries only. 
Fig.~\ref{raman1} shows the predicted Raman modes for the B$_g$ symmetries in $\beta$-FeSi$_2$. The results obtained within the harmonic approximation are compared with the anharmonic perturbation theory calculations which provides both, frequency shifts and predicted line profiles of the modes.
Although the experimentally measured peaks cannot be directly assigned to these symmetries due to the lack of polarization-resolved data, the theoretical predictions provide a reference for comparison. Matching the measured frequencies to the closest theoretical B$_g$ modes (Table~\ref{T1Raman}) allows for a tentative assignment, which can guide future polarization-resolved Raman experiments aimed at determining the precise symmetry of the unresolved peaks.

\begin{figure*}[t]
\centering
  \includegraphics[width=\linewidth]{spectrum_RTA+raman_Bi_modes.pdf}
\caption{Raman spectrum of $\beta$-FeSi$_2$ calculated at three temperatures 
($300$, $600$, $900$~K -- blue, orange, and red lines, respectively). 
The calculated spectrum includes all Raman-active modes. The B$_{ig}$ modes
are indicated by purple vertical lines at peak positions corresponding to T=300~K. 
The frequencies derived from the harmonic approximation at T=300~K are indicated 
by green, yellow and pink lines.
Connecting arrows indicate the correspondence between harmonic and anharmonic 
frequencies, demonstrating the frequency shifts due to phonon interactions.}
\label{raman1}
\end{figure*}
\section{Thermal conductivity obtained from BTE and RTA}
\label{ThermalAB}

The thermal conductivity was computed by solving the full BTE on the largest feasible $\bm{q}$-point grid, $8\times8\times8$, and compared with the corresponding RTA results obtained on the same grid. As shown in Fig.~\ref{bte}, the difference between the components of the thermal conductivity tensor obtained within BTE and RTA at this resolution is very small, indicating a good agreement between these two approaches.
Moreover, we performed an additional calculation using RTA on a denser $20\times20\times20$ grid. As seen in the Fig.~\ref{bte}, the higher-resolution data remain in a good agreement with both the BTE and RTA results for the $8\times8\times8$ grid.
It shows that the $8\times8\times8$ mesh already provides good results for this structure and confirms reliability of the calculations.

\begin{figure}[!h]
\centering
  \includegraphics[width=\linewidth]{BTEvsRTAvsANP.pdf}
\caption{Thermal conductivity of $\beta$-FeSi$_2$ obtained within the BTE and RTA methods using the Phono3py software on the $8\times8\times8$ q-point grid, compared with the RTA results computed with ALAMODE on a denser $20\times20\times20$ grid.}
\label{bte}
\end{figure}

%\section*{Data availability}
%The data that support the findings of this article are openly available~\footnote{give me DOI}.
% see https://journals.aps.org/authors/data-availability-statements#citation

\bibliography{refs.bib}
%\bibliographystyle{ieeetr}


\end{document}

\documentclass[%
%reprint,
superscriptaddress,
%groupedaddress,
longbibliography,
%unsortedaddress,
%runinaddress,
%frontmatterverbose, 
%preprint,
%preprintnumbers,
%nofootinbib,
nobibnotes,
%bibnotes,
amsmath,amssymb,
aps,
%pra,
prb,
%rmp,
%prstab,
%prstper,
%showkeys,
floatfix,
twocolumn
]{revtex4-2}

\usepackage{graphicx}% Include figure files
\usepackage{calc}% Calculate margins
\usepackage{dcolumn}% Align table columns on decimal point
\usepackage{bm}% bold math

\usepackage[urlcolor=blue,colorlinks=true,citecolor=blue,linkcolor=blue,pdfstartview={FitH},bookmarks=false]{hyperref} % add hypertext capabilities

%\usepackage[mathlines]{lineno} % Enable numbering of text and display math
% \linenumbers\relax % Commence numbering lines

% \usepackage[showframe,%Uncomment any one of the following lines to test 
% %scale=0.7, marginratio={1:1, 2:3}, ignoreall,% default settings
% %text={7in,10in},centering,
% %margin=1.5in,
% % total={6.5in,8.75in}, top=1.2in, left=0.9in, includefoot,
% % height=10in,a5paper,hmargin={3cm,0.8in},
% ]{geometry}

\usepackage{amsmath}
\usepackage{amssymb}
%\usepackage{orcidlink}
\usepackage{xcolor}
%\usepackage{datetime}
\usepackage[normalem]{ulem}

% Change tracking commands
\newcommand{\trackchange}[3]{\textcolor{#3}{\sout{#1}#2}}  % Full color strikeout, insert
%\renewcommand{\trackchange}[3]{\textcolor{#3}{#2}}        % Just color silent remove and insert
%\renewcommand{\trackchange}[3]{{#2}}                      % No indication, silent remove and insert

% Author marker definitions

\definecolor{myblue}{RGB}{0,127,85}
% \definecolor{violet}{RGB}{102,0,204}
% \definecolor{orange}{RGB}{255,128,0}
% \definecolor{green}{RGB}{0,128,0}
\newcommand{\DL}[1]{\trackchange{}{#1}{blue}}
\newcommand{\AP}[1]{\trackchange{}{#1}{red}}
\newcommand{\PJ}[2]{\trackchange{#1}{#2}{orange}}
\newcommand{\JL}[2]{\trackchange{#1}{#2}{myblue}}
\newcommand{\SP}[2]{\trackchange{#1}{#2}{blue}}
\newcommand{\PP}[2]{\trackchange{#1}{#2}{teal}}
\newcommand{\AI}[2]{\trackchange{#1}{#2}{olive}}
\newcommand{\MS}[1]{\trackchange{}{#1}{purple}}

\newcommand{\TODO}[1]{\textcolor{red}{TODO: #1}}

\sloppy

\begin{document}

\title{Ab initio study of the anharmonic properties and thermal conductivity in $\beta$-FeSi$_2$}

\author{Svitlana~Pastukh}
\email[e-mail: ]{svitlana.pastukh@ifj.edu.pl}
\affiliation{Institute of Nuclear Physics, Polish Academy of Sciences, ul. W. E. Radzikowskiego 152, 31-342 Krak\'{o}w, Poland}

\author{Ma\l{}gorzata~Sternik}
\affiliation{Institute of Nuclear Physics, Polish Academy of Sciences, ul. W. E. Radzikowskiego 152, 31-342 Krak\'{o}w, Poland}

\author{Pawe\l{}~T.~Jochym}
\affiliation{Institute of Nuclear Physics, Polish Academy of Sciences, ul. W. E. Radzikowskiego 152, 31-342 Krak\'{o}w, Poland}


\author{Jan~\L{}a\.{z}ewski}
\affiliation{Institute of Nuclear Physics, Polish Academy of Sciences, ul. W. E. Radzikowskiego 152, 31-342 Krak\'{o}w, Poland}

\author{Andrzej~Ptok}
\affiliation{Institute of Nuclear Physics, Polish Academy of Sciences, ul. W. E. Radzikowskiego 152, 31-342 Krak\'{o}w, Poland}

\author{Svetoslav~Stankov}
\affiliation{Institute for Photon Science and Synchrotron Radiation, Karlsruhe Institute of Technology, D-76131 Karlsruhe, Germany}
\affiliation{Laboratory for Applications of Synchrotron Radiation, Karlsruhe Institute of Technology, D-76131 Karlsruhe, Germany}

\author{Przemys\l{}aw~Piekarz}
\affiliation{Institute of Nuclear Physics, Polish Academy of Sciences, ul. W. E. Radzikowskiego 152, 31-342 Krak\'{o}w, Poland}

\date{\today}

\begin{abstract}

Iron silicides are good candidates for applications in  optoelectronic and thermoelectric devices.
Lattice dynamical properties and thermal conductivity in the $\beta$-FeSi$_2$ semiconductor
are investigated with the first-principles computational methods. 
Phonon dispersion relations are calculated via the
temperature-dependent effective potential method and self-consistent phonon theory. 
To properly model thermal transport, we explicitly consider
the impact of phonon-phonon interactions by analyzing
anharmonic contributions to the phonon self-energy. 
This yields temperature-dependent phonon frequencies and linewidths,
reflecting the finite lifetime of phonons due to scattering
processes. The calculated phonon frequencies and line profiles are used to obtain 
the Raman spectra, which shows good agreement with the experimental data. 
We revealed an enhanced anharmonic behaviour of the Raman modes with the highest frequencies.  
The lattice thermal conductivity is then obtained as a function of temperature and crystallite size within
the relaxation-time approximation.
Phonon transport shows a small anisotropy due to the orthorhombic structure and a very weak dependence
on the quartic anharmonic corrections. The results obtained for an infinite material and for several crystallite sizes
were analyzed and compared with the available experimental data.
\end{abstract}

\maketitle


\section{Introduction}

The comprehensive determination of important physical properties of
crystals, such as thermal expansion, lattice thermal
conductivity or structural phase transitions, requires a fundamental 
understanding of the anharmonic effects.
Although the investigation of anharmonic interactions in crystals has
attracted a considerable interest for decades~\cite{cowley_1968}, a substantial progress
has only recently been achieved thanks to advances in theoretical and
numerical methods and increased computational power.
Now, phonon frequencies, lifetimes, and heat transfer in a wide range of
materials
can be quantitatively predicted using the available computational resources
based on the density functional theory (DFT)~\cite{lindsay_2013,mcgaughey_2019,lindsay_2019}.
In the case of strongly anharmonic systems, the self-consistent phonon
(SCPH) theory~\cite{tadano_2015} as well as the perturbative approach~\cite{tadano_2018}, using higher
order interatomic force constants derived from the fitting to the displacement force data obtained
with DFT, have proven to be successful.

Transition-metal silicides are promising materials for fabrication of electronic components
designed for integration with silicon-based circuits~\cite{murarka_1995}.
At room temperature, iron disilicide ($\beta$-FeSi$_2$) is a
direct-bandgap semiconductor~\cite{bost_1985}, making this material a good
candidate for application in optoelectronic devices such as infrared
detectors or light emitters~\cite{bost_1988}. The development of light-emitting diodes utilizing FeSi$_2$/Si heterostructures has been successfully demonstrated~\cite{leong_1997,suemasu_2001}. Due to
a high thermal stability and strong light absorption, FeSi$_2$ is
also a suitable photovoltaic material~\cite{powalla_1993,liu_2006,okuhara_2017}.

$\beta$-FeSi$_2$ crystallizes in the base-centered orthorhombic
lattice~\cite{dusausoy_1971} transforming to the
tetragonal metallic $\alpha$-FeSi$_{2}$ phase around $1200$~K~\cite{starke_2002}.
Optical studies indicated a direct band gap of the values
$0.85$--$0.89$~eV~\cite{bost_1988,dimitriadis_1990,arushanov_1995,wan_2003},
however, the {\it ab initio} calculations predicted a smaller indirect gap close to 0.8 eV~\cite{christensen_1990}. 
The existence of such an indirect gap was then confirmed by the optical linear transmittance measurements at
low temperatures~\cite{giannini_1992}. As shown by first-principles studies,
the character of the band gap is very sensitive to the orientation 
of a crystal grown on silicon~\cite{clark_1998}.

$\beta$-FeSi$_2$ belongs also to good thermoelectric materials~\cite{ware_1964}, with potential applications resulting from its chemical stability up to high temperatures, nontoxicity, and low cost of preparation~\cite{yamada_2012,nozariasbmarz_2017}. 
It has already been implemented in cars~\cite{birkholz_1988} and portable power sources~\cite{uemura_1989}. 
Its thermoelectric performance can be improved by doping~\cite{ito_2001,tani_2001,kim_2003,chen_2005,pandey_2013,le_tonquesse_2019}, 
which enhances the electric transport and reduces the thermal conductivity~\cite{waldecker_1973,du_2019,du_2020}.  
The thermal conductivity can be also reduced by the modification of microstructure~\cite{ail_2015} or by
nanostructurization~\cite{watanabe_2017,taniguchi_2017,hsin_2017,abbassi_2021}.

The lattice thermal conductivity is directly connected with anharmonic effects and phonon scattering processes.
The vibrational properties of \mbox{$\beta$-FeSi$_2$} were studied by the infrared and Raman spectroscopy~\cite{lefki_1991,guizzetti_1997,maeda_2004,baleva_2008,liu_2011,maeda_2011}. The observed anisotropy in the phonon spectra results from the enhanced sensitivity of the infrared and Raman features to the local lattice distortions~\cite{guizzetti_1997}. The Fe phonon density of states was measured by nuclear inelastic scattering (NIS), showing a good agreement with the density functional theory (DFT) calculations~\cite{walterfang_2005}. Using the DFT approach, the phonon dispersion curves, phonon density of states, as well as various thermodynamic properties were obtained within the harmonic approximation~\cite{tani_2010,liang_2011}. The extended Klemens model was applied to study
the anharmonic effect on phonon frequencies and linewidths observed by the Raman spectroscopy~\cite{zhang_2023}.
The impact of nanostructurization on lattice dynamics was explored in the $\beta$-FeSi$_2$ nanorods grown on the Si(110) surface by the NIS and {\it ab initio} methods~\cite{kalt_2022}.

In this work, we investigate the lattice dynamical properties of $\beta$-FeSi$_2$ 
using the DFT calculations. We study the effect of anharmonic terms in the temperature-dependent potential on phonon frequencies and lifetimes. We focus on the Raman modes, comparing the theoretical results with the experimental data.
The thermal conductivity is derived in a broad temperature range and the effect of crystallite size is analyzed.



This study is structured as follows.
In Sec.~\ref{sec.com} we describe the details of computational methods.
Next, in Sec.~\ref{sec.result} we present and discuss our results.
In particular we present the crystal structure (Sec.~\ref{sec.crys}) and lattice dynamics (Sec.~\ref{sec.lattice}).
We investigate also the thermal conductivity comparing the obtained results with the available experimental data (Sec.~\ref{sec.thermal}).
Finally, Sec.~\ref{sec.summary} summarizes our key findings and conclusions.

\section{Calculation method}
\label{sec.com}


The calculations were performed using the projector augmented-wave potentials~\cite{blochl_1994} and the generalized gradient approximation~\cite{perdew_1996} implemented in the Vienna Ab initio Simulation Package (VASP)~\cite{kresse.hafner.94,kresse.furthmuller.96,kresse.joubert.99}. 
The lattice parameters and atomic positions were optimized in the ${\bm a} \times ({\bm b}-{\bm c}) \times ({\bm b}+{\bm c})$ supercell containing 32 formula units and four primitive cells.
The integration in the reciprocal space was conducted using the $2 \times 2 \times 2$ Monkhorst--Pack mesh~\cite{monkhorst_1976} and the cut-off energy was set to $500$~eV. For convergence conditions, we set the energy change below $10^{-5}$ and $10^{-8}$ for the ionic and electronic loops, respectively. 

The lattice dynamical properties were studied within the temperature-dependent effective potential (TDEP) approach~\cite{hellman_2013}. The atomic potential with the third and fourth order anharmonic terms was derived from interatomic forces induced by displacements of all atoms at finite temperatures.
The sets of atomic displacements were generated by the high efficiency configuration space sampling (HECSS)~\cite{jochym_2021} and forces were obtained by VASP. The interatomic force constants and phonon frequencies were calculated with the {\sc Alamode} software~\cite{tadano_2014}.

Furthermore, we have attempted to construct a {\emph{temperature independent}} 
anharmonic model. We have used combined data from all investigated temperatures 
(300, 600 and 1000~K) and fitted a large (over 15~000 free parameters), fourth-order 
interaction model to this dataset. 
Subsequently, we have used this model to calculate line profiles and positions of Raman-active modes
at multiple temperatures.

The changes in phonon frequencies induced by the anharmonic effects were investigated within two approaches.
First, the impact of the quartic anharmonic terms was included using the SCPH theory~\cite{tadano_2015}.
Second, the mode profiles (frequency shifts and line widths) were determined from the real and imaginary parts of the phonon self-energy resulting from the cubic and quartic anharmonic terms of the above mentioned large model~\cite{tadano_2018}. 
The longitudinal optic-transverse optic (LO-TO) splitting was also evaluated, using the static dielectric tensor and Born effective charges calculated within density functional perturbation theory~\cite{gajdos_2006}.

\begin{figure}[]
    \centering
    \includegraphics[width=\linewidth]{fig1_new.png}
\caption{%
(a) The conventional unit cell of $\beta$-FeSi$_2$ (with Cmca symmetry) and (b) the corresponding Brillouin zone with selected high-symmetry points.
}
  \label{fig.struct}
\end{figure}

% To further characterize the vibrational properties, Raman-active scattering was investigated using the Phonopy-Spectroscopy package~\cite{skelton_2017}. This enabled the identification of Raman-active modes and the calculation of Raman tensors. Anharmonic force constants, obtained from calculations using {\sc Alamode}, were then used to obtain theoretical line profiles for the Raman modes. The presented Raman scattering spectra combine these anharmonic line profiles with the Raman tensor amplitudes.
% This analysis was also based on the large quartic model mentioned above.

% Finally, the thermal conductivity was obtained as a function of temperature and crystallite size within the relaxation-time approximation (RTA) \PJ{}{as implemented in {\sc Alamode}\cite{tadano_2014}}. The phonon lifetimes were calculated from the phonon self-energy including the cubic and quartic anharmonic terms. \SP{}{The RTA provides a solution of the Boltzmann transport equation (BTE) under the assumption that scattering events are independent and can be treated through mode-resolved relaxation times.
% To verify the validity of this approximation for $\beta$-FeSi$_2$, we have additionally
% executed an iterative solution of the BTE.}\PJ{}{These calculations were performed with
% {\sc Phono3py}\cite{togo_2023}. For cross-validation, the additional RTA calculations were performed on the same $\bm{q}$-grid as BTE with implementation provided by {\sc Phono3py}.}.

To further characterize the vibrational properties, Raman scattering was investigated using the Phonopy-Spectroscopy package~\cite{skelton_2017}. This enabled the identification of Raman-active modes and the calculation of Raman tensors. Anharmonic force constants, derived from calculations using {\sc Alamode}, were then used to obtain theoretical line profiles for the Raman modes. The presented Raman scattering spectra combine these anharmonic line profiles with the Raman tensor amplitudes.
This analysis was also based on the large quartic model described above.

Finally, the thermal conductivity was calculated as a function of temperature and crystallite size within the relaxation-time approximation (RTA) \PJ{}{as implemented in {\sc Alamode}~\cite{tadano_2014}}. The phonon lifetimes were calculated from the phonon self-energy including the cubic and quartic anharmonic terms. \SP{}{The RTA provides a solution to the Boltzmann transport equation (BTE) under the assumption that scattering events are independent and can be treated through mode-resolved relaxation times.
To verify the validity of this approximation for $\beta$-FeSi$_2$, we additionally
solved the BTE iteratively.} \PJ{}{These calculations were performed with
{\sc Phono3py}~\cite{togo_2023}. For cross-validation, the additional RTA calculations were performed on the same $\bm{q}$-grid as the BTE calculations, using the implementation provided by {\sc Phono3py}.}

% (see. Appendix~\ref{ThermalAB} for the comparison)\cite{togo_2023}.}
%As shown in Appendix~\ref{ThermalAB}, the iterative BTE results exhibit good agreement with the RTA values, confirming that RTA is sufficiently accurate for this material and for the considered temperature range.}


\section{Results}
\label{sec.result}

\subsection{Crystal structure}
\label{sec.crys}

The $\beta$-FeSi$_2$ structure adopts a base-centered orthorhombic lattice with the space group Cmca (No.~64) as shown in Fig.~\ref{fig.struct}(a).
The unit cell consists of two primitive cells and contains 48 atoms.
Iron (silicon) atoms possess two nonequivalent positions: \mbox{Fe-I} and \mbox{Fe-II} (\mbox{Si-I} and \mbox{Si-II}), presented in Fig.~\ref{fig.struct}(a) as gray and purple (orange and yellow) spheres, respectively.
This crystal structure is derived from the fluorite-type lattice with strongly distorted Si cubes and Fe atoms occupying 
one-half of the central sites.
The Fe-I and Fe-II sites create different layers perpendicular to the $x$ direction, and they are separated by layers containing both Si sites.
Each Fe atom is coordinated by 8 Si atoms with slightly different Fe-Si distances.
The optimized lattice constants ($a = 9.874$~{\AA}, $b = 7.767$~{\AA}, and
$c = 7.811$~{\AA}) agree very well with the experimental data ($a = 9.863$~{\AA}, $b = 7.791$~{\AA}, and $c = 7.833$~{\AA})~\cite{dusausoy_1971}.

Iron atoms occupy the Wyckoff sites 8\textit{d} ($0.2166$, $0$, $0$) and  8\textit{f} ($0$, $0.3072$, $0.1879$), corresponding to \mbox{Fe-I} and \mbox{Fe-II}, respectively. 
Silicon atoms are located at two inequivalent 16\textit{g} positions: ($0.1282$, $0.2737$, $0.0495$) and \mbox{($0.3734$, $0.0445$, $0.2270$)}, assigned as \mbox{Si-I} and \mbox{Si-II}.
The optimized positions of atoms agree very well with the experimental data~\cite{dusausoy_1971} and the previous theoretical studies~\cite{tani_2010,liang_2011}.


\subsection{Lattice dynamics}
\label{sec.lattice}

\begin{figure}[]
    \centering
    \includegraphics[width=\linewidth]{Fig1c.pdf}
\caption{%
The phonon dispersion curves along high symmetry directions obtained within SCPH (for temperatures from $0$ to $1000$~K).
Dashed black lines indicate the phonon dispersions obtained from the harmonic approximation. The white dots indicate Raman-active modes with A$_g$ symmetry.
The vertical plot shows the phonon density of states (DOS) calculated at a reference temperature of $600$~K.
}
  \label{fig.ph_band}
\end{figure}


In Fig.~\ref{fig.ph_band} we present the phonon dispersion relations of $\beta$-FeSi$_2$ along high-symmetry directions in the Brillouin zone [Fig.~\ref{fig.struct}(b)].
Due to 24 atoms in the primitive cell, the phonon spectrum consists of 69 optical modes and three acoustic modes.
The phonon dispersions were calculated within the SCPH approach in the temperature range $0$--$1000$~K (presented by color lines in Fig.~\ref{fig.ph_band}), and they are compared with the results obtained from the harmonic part of the effective potential corresponding to temperature $T=300$~K (indicated by dashed black lines in Fig.~\ref{fig.ph_band}).
As we can see within the SCPH method, the anharmonic effects are rather weak and leads only to small renormalization of phonon frequencies.
Only the highest modes show more pronounced shifts of their frequencies to larger values.
The total and partial element-projected phonon density of states obtained within the harmonic approximation are presented in Fig.~\ref{fig.ph_band}.
Up to around $320$~cm$^{-1}$, the contributions from both elements are very similar, while for higher frequencies the spectrum is dominated by
the Si vibrations.


\begin{table}[!t]
\begin{ruledtabular}
\caption{Calculated and experimental Raman-active modes of $\beta$-FeSi$_2$ with their irreducible representations (IR). Present theoretical results are compared with the previous theoretical data from Ref.~\cite{tani_2010} and experimental results from \mbox{Refs.~\cite{lefki_1991,maeda_2004}.} The experimental frequencies with known symmetries (A$_g$) are shown in bold, while other experimental modes are assigned to the best fitting theoretical values.}
\begin{tabular}{c c c c c}
\textbf{IR} & \multicolumn{4}{c}{\textbf{Frequency (cm$^{-1}$)}} \\
 & Present  & Theor.~\cite{tani_2010} & Exp.~\cite{lefki_1991} & Exp.~\cite{maeda_2004} \\
\hline
B$_{2g}$ & 175.4 & 179 & 176 &  \\
B$_{1g}$ & 176.3 & 185 & 179  &  \\
B$_{1g}$ & 193.3 & 198 &  & 190.6 \\
A$_g$    & 196.6 & 208 & \textbf{195} & \textbf{194.0} \\
A$_g$    & 203.9 & 210 & \textbf{197} & 199.6 \\
B$_{3g}$ & 205.5 & 212 & 200 &  \\
B$_{3g}$ & 226.3 & 236 & 206 & 227.1 \\
B$_{1g}$ & 233.3 & 240 &  &  231.6 \\
B$_{2g}$ & 248.6 & 254 &  &  \\
A$_g$    & 250.1 & 257 & \textbf{247} & \textbf{247.3} \\
A$_g$    & 254.9 & 264 & \textbf{253} & 254.3 \\
B$_{1g}$ & 275.5 & 285 &  &  274.1 \\
B$_{2g}$ & 282.4 & 295 &  &  281.2 \\
B$_{3g}$ & 286.8 & 297 &  &  \\
B$_{1g}$ & 307.1 & 317 &  &  \\
B$_{2g}$ & 312.6 & 326 &  &  311.8 \\
B$_{1g}$ & 319.0 & 324 &  &  \\
B$_{3g}$ & 327.9 & 341 &  & 325.8 \\
B$_{2g}$ & 333.3 & 345 &  &  \\
A$_g$    & 339.3 & 352 & \textbf{346} & 339.5 \\
B$_{2g}$ & 343.0 & 350 &  &  \\
B$_{1g}$ & 353.5 & 366 &  &  \\
B$_{2g}$ & 372.8 & 383 &  & 370.7 \\
B$_{1g}$ & 375.1 & 385 &  &  \\
B$_{3g}$ & 375.2 & 386 &  &  \\
B$_{3g}$ & 385.3 & 401 &  &  \\
A$_g$    & 386.5 & 398 &  & 386.2 \\
B$_{2g}$ & 387.7 & 402 &  & 388.2 \\
A$_g$    & 404.6 & 415 &  & 400.4 \\
B$_{2g}$ & 405.1 & 420 &  &  \\
B$_{1g}$ & 412.7 & 428 &  &  \\
B$_{3g}$ & 418.4 & 431 &  &  \\
B$_{3g}$ & 441.4 & 458 &  & 442.6 \\
B$_{3g}$ & 447.3 & 466 &  & 446.3 \\
A$_g$    & 448.4 & 464 &  &  \\
A$_g$    & 499.1 & 517 &  &  \\
\end{tabular}
\label{T1Raman}
\end{ruledtabular}
\end{table}

%subsection{Raman}

\begin{figure*}
\centering
  \includegraphics[width=\textwidth]{spectrum_RTA+raman.pdf}
\caption{Raman spectrum of $\beta$-FeSi$_2$ calculated within the perturbative approach at three temperatures 
($300$, $600$, $900$~K -- blue, orange, and red lines, respectively). 
The calculated spectrum includes all Raman-active modes. The A$_g$ modes
are indicated by purple vertical lines at peak positions corresponding to T=300~K. 
The frequencies derived from the harmonic approximation at T=300~K are indicated by green lines.
Connecting arrows indicate the correspondence between harmonic and
anharmonic frequencies, demonstrating the frequency shifts due to 
phonon interactions. Experimental values for the A$_g$ modes
based on Ref.~\cite{lefki_1991} are marked with black dashed lines.
}
\label{raman}
\end{figure*}


The phonon spectrum at the $\Gamma$ point consists of 36 Raman modes classified according to the irreducible representations: $9A_\text{g}+9B_\text{1g}+9B_\text{2g}+9B_\text{3g}$. 
Fig.~\ref{raman} shows the Raman spectrum of A$_g$ symmetry calculated for $\beta$-FeSi$_2$ within the perturbative approach at three temperatures $300$, $600$, and $900$~K (solid blue, orange, and red curves, respectively), including third-order and fourth-order anharmonic corrections.
The calculated Raman spectrum includes all Raman-active modes.
The five experimentally observed A$_g$ modes are highlighted by black dashed lines, based on data from Ref.~\cite{lefki_1991}.
Additional peaks not marked with vertical lines correspond to phonon modes with symmetries other than A$_g$.
The frequencies of Raman modes obtained in anharmonic calculations are compared with the previous results calculated within the harmonic approximation and the experimental values in Tab.~\ref{T1Raman}. 
We have marked in bold the experimentally determined A$_g$ modes, which are compared with the calculations.
Since the experimental studies did not provide the accurate assignement of the Raman modes with the B$_{1g}$, B$_{2g}$, and B$_{3g}$ symmetry~\cite{lefki_1991,maeda_2004}, we cannot compare them directly with the theoretical results.
However, in Tab.~\ref{T1Raman} we have assigned the measured frequencies to the best fitting theoretical values without taking into account the symmetry of the modes, except for the known A$_{g}$ modes. 


The impact of anharmonicity on the phonon frequencies is well visible from the comparison of the results obtained within the harmonic approximation 
and from the anharmonic calculation (vertical green and purple lines in Fig.~\ref{raman}, respectively).
Here we show only the A$_g$ modes, which are compared with the experimental results (vertical black dashed lines).
Anharmonic frequencies calculated at $300$~K are indicated by purple lines, while frequencies derived from harmonic approximation are marked with green lines. The grey solid lines connect corresponding modes obtained in both approximations.
In most cases, the results obtained within the harmonic approximation do not agree with the experimental frequencies. 
%Only the modes close to $250$~cm$^{-1}$ and $400$~cm$^{-1}$ correspond well to the experimental values. MS
As we can see, the inclusion of the anharmonic correction leads to a significant modification of the phonon frequencies.
These anharmonic effects are stronger for higher-frequency modes mainly because of
the dominant contribution from Si atoms, which vibrate with larger amplitudes than heavier Fe atoms. 
When atoms move to larger distances the potential deviates more from the harmonic approximation,
and the anharmonic corrections become stronger.


The modification of phonon frequencies observed in Fig.~\ref{raman} is much larger than in the SCPH scheme presented in Fig.~\ref{fig.ph_band}. The SCPH approach includes only the leading-order contribution to 
the phonon self-energy obtained from the quartic anharmonic terms~\cite{tadano_2015}. Therefore, it does not describe fully the changes of phonon frequencies found within the perturbation theory (see Fig.~\ref{raman}).
Especially, it is well visible for two highest A$_g$ modes, which exhibit also the largest line broadening 
and the strongest dependence on temperature.
Therefore, a better agreement with experimentally observed frequencies is visible,
confirming the significant influence of the anharmonicity on the frequencies and line profiles of phonon modes.
In fact, the decrease of phonon frequencies should be even stronger due to thermal expansion, 
which is not included in our calculations.
Within SCPH the frequencies of the highest modes increase with increasing temperature as we see in Fig.~\ref{fig.ph_band}. The comparison of two different approaches applied to study anharmonic properties of $\beta$-FeSi$_2$ shows that the perturbation theory, which includes the cubic and quartic terms, better describes the changes of phonon frequencies with temperature than the SCPH method. \PJ{}{This indicates that the leading-order contribution included in SCPH are not important in this material.}

Additionaly we should nottice that for other than A$_g$ modes we cannot make an unambiguous assignment of theoretical frequencies to experimental ones. Note that the spectrum in Fig.~\ref{fig.ph_band} contains all Raman-active modes. 
The limitation to A$_g$ modes concerns only the indicated positions of the peaks. 
%\SP{}{Furthermore, we present the full spectrum for all Raman active modes with the comparison of the harmonic and anharmonic calculations in the Appendix~\ref{RamanAA}.}
\textcolor{red}{The full Raman spectrum, with the frequencies of the B$_{1g}$, B$_{2g}$ and B$_{3g}$ modes marked, is shown in Appendix A, Fig.~\ref{raman1}. This represents our theoretical prediction of possible Raman-mode assignments, which can be verified in future experiments.}
%This is the theoretical prediction of possible assignment of Raman modes that can be verified in future experiments.  


\subsection{Thermal conductivity}
\label{sec.thermal}


\begin{figure*}[!t]
\centering
  \includegraphics[width=\linewidth]{time3.pdf}
\caption{Phonon lifetimes calculated for three temperatures as a function of phonon frequency. The colors correspond to the phonon branches.}
  \label{thermtime}
\end{figure*}


In this section, we analyze the thermal conductivity tensor of $\beta$-FeSi$_2$ obtained within the RTA approach~\cite{tadano_2018} as a function of temperature
%
\begin{equation}
\kappa_{\text{ph}}^{\mu\nu}(T) = \frac{1}{NV} \sum_{\bm{q},j} c_{\bm{q}j}(T) v_{\bm{q}j}^{\mu} v_{\bm{q}j}^{\nu}\tau_{\bm{q}j}(T),
\end{equation}
% 
where $c_{\bm{q}j}$ is the mode heat capacity and $v_{\bm{q}j}$ is the mode group velocity. 
The relaxation time is approximated by the phonon lifetime $\tau_{\bm{q}j}$
calculated for $j$-th branch at the wave vector $\bm{q}$.
$V$ is the unit cell volume and $N$ is the number of unit cells in the crystal.
The phonon lifetime is calculated using this formula
%
\begin{equation}
\tau_{\bm{q}j}(T)=\frac{1}{2\Gamma_{\bm{q}j}^{\text{anh}}(T)},
\end{equation}
%
where $\Gamma_{\bm{q}j}^{\text{anh}}$ is the anharmonic phonon linewidth obtained from 
the imaginary part of the phonon self-energy within the perturbation theory.

In Fig.~\ref{thermtime}, we present $\tau_{\bm{q}j}$ obtained for three temperatures 300, 600, and 1000~K as a function of frequency. As we see, the acoustic phonons close to the $\Gamma$ point have the longest lifetimes,
which are diminished with increasing frequency reaching local minima around $200$~cm$^{-1}$.
For higher frequencies, phonon lifetimes first increase to local maxima around $300$~cm$^{-1}$ and then decrease to
the lowest values in the range of highest optical modes. The shortest lifetimes correspond to the largest line broadening
observed for the Raman modes in Fig~\ref{raman}. The phonon group velocities 
$v_{\bm{q}j}=\partial\omega_{\bm{q}j}/\partial\bm{q}$, which are obtained by the central difference formula, are presented in Fig.~\ref{thermvelocity}. Their temperature dependence is negligible, therefore, we present only the results for $T=600$~K.
At low frequencies, there are clearly two ranges of group velocities of the acoustic phonons. 
The larger values correspond to the longitudinal modes, while the lower values are obtained from the
transverse acoustic branches. Group velocities of acoustic phonons decrease for larger frequencies
and reach the average values typical for optic branches.

\begin{figure}[!t]
\centering
  \includegraphics[width=\linewidth]{GV.pdf}
\caption{Mode group velocities calculated as a function of phonon frequency. The colors correspond to the phonon branches.
}
  \label{thermvelocity}
\end{figure}

In Fig.~\ref{anizotropy}(a), we present the three diagonal elements of $\kappa_{\text{ph}}^{\mu\nu}$ corresponding to the main directions of the crystal structure. 
They were obtained from the force constants calculated at the base temperature $T=600$~K and the crystallite size $0.1$~$\mu$m to account for boundary-limited phonon transport. 
Due to the orthorhombic symmetry, we observe a small anisotropy in phonon transport in the whole temperature range. 
At low temperatures, the three components of the heat conductivity increase in a very similar way with the $\kappa_{\text{ph}}^{yy}$ element slightly larger than two other components. 
After reaching the maximum, we observe a change in the largest component from $\kappa_{\text{ph}}^{yy}$ to 
$\kappa_{\text{ph}}^{xx}$.    
In Fig.~\ref{anizotropy}(b), the thermal conductivity is shown for three base temperatures, at which the interatomic potential was obtained ($300$~K, $600$~K, and $1000$~K), using the energy expansion up to third- and fourth-order anharmonic terms, and the same structure size of $0.1$~$\mu$m. At lowest temperatures, the thermal conductivity strongly increases, reaching the maximum around $T=180$~K, then it shows a slower decrease with temperature.
The differences between the two levels of approximation are minimal, suggesting that third-order calculations already capture the dominant phonon scattering mechanisms. The dependence on the base temperature is also very weak, showing the changes in the heat conductivity within a few percent. 
\SP{}{Further verification of the reliability of our thermal conductivity results is given in Appendix~\ref{ThermalAB}, where the full BTE calculations show good agreement with RTA and higher-resolution RTA results, demonstrating that the $8\times8\times8$ $\bm{q}$-mesh already provides converged values.}

\begin{figure}[!t]
\centering
  \includegraphics[width=\linewidth]{Anizotropy.pdf}
\caption{(a) The anisotropic thermal conductivity of $\beta$-FeSi$_2$ calculated along the lattice directions at 600~K. (b) The average temperature-dependent thermal conductivity taken at $300$~K, $600$~K, and $1000$~K, including anharmonic corrections up to cubic (A3) and quartic (A4) terms. In both cases the crystallite size is 0.1~$\mu$m.}
  \label{anizotropy}
\end{figure}


In Fig.~\ref{therm}, we fix the base temperature at $600$~K and examine the effect of crystallite size on thermal conductivity, varying it from $0.01$ to $0.5$~$\mu$m.
With decreasing the crystallite size, we observe a shift of the position of the maximum to larger temperatures and a decrease of the thermal conductivity in the entire temperature range.
Theoretical results are compared with several experimental data obtained above the room temperature. 
The measured thermal conductivity depends to a large extent on the sample quality, its purity and the size of the crystalline grains which depends on 
the production processes.
Many measurements were performed using crystallites of micrometric or unknown size ~\cite{waldecker_1973,ito_2002,kim_2003,du_2020}, however, 
numerous attempts to minimize $\kappa$ by reducing grain sizes to $56$~nm~\cite{dabrowski_2019, dabrowski_microstructure_2021}, $30$-$400$~nm~\cite{le_tonquesse_2019}, $50$ and $200$~nm~\cite{abbassi_2021}, or introducing pores into the material~\cite{sam2023improved} 
are also carried out. 
Another way to change the thermal conductivity is to dope $\beta$-FeSi$_2$ with different elements~\cite{ito_2002,kim_2003,du_2020,cheng_2024}, however, this effect is beyond our investigation.

\begin{figure}[!t]
\centering
  \includegraphics[width=\linewidth]{Thermal_conductivity3.pdf}
\caption{
The phonon thermal conductivity of $\beta$-FeSi$_2$: theoretical results for the infinite crystalline size and with boundary conditions, compared with experimental data for different structure sizes.
}
  \label{therm}
\end{figure}
%Fig.~\ref{therm}\cite{abbassi_2021}\cite{waldecker_1973}\cite{dabrowski_2019}\cite{sam2023improved}\cite{kim_2003}\cite{du_2020}\cite{ito_2002}\cite{le_tonquesse_2019}.

We observe a decrease in the thermal conductivity with reducing crystalline grain sizes in all analyzed experimental data. 
For instance, by decreasing the crystallite size to $50$~nm, the thermal conductivity at room temperature was reduced by a factor of $1.7$, what can be  compared to the annealed sample with 200 nm grains~\cite{abbassi_2021}. 
It is worth noting that the rate of decrease in value with increasing temperature in both cases, for grain sizes of $50$~nm and $200$~nm, is significantly different, which is consistent with our calculations. 
The same trend can be observed by comparing the thermal conductivity measured for a sample with bulk crystallite sizes with the thermal conductivity of a sample with grains smaller than 400 nm~\cite{le_tonquesse_2019}.
The theoretical results obtained for the same crystallite size show higher values due to factors not captured in the idealized model, such as crystal imperfection or mechanical strain. Usually, a decrease in the crystallite size is related to an increased concentration of grain boundaries, point defects, and stacking faults that influence the phonon scattering~\cite{le_tonquesse_2019,abbassi_2021}.   

We should note that the total thermal conductivity is a combination
of the lattice and electronic contributions to the heat transport.
In semiconductors, the electronic thermal conductivity is negligible at low temperatures and significantly increases 
only much above the room temperatures~\cite{gu_2020}.
For $\beta$-FeSi$_2$, the electronic thermal conductivity was obtained from the electric conductivity using the Wiedemann-Franz law~\cite{ito_2002,kim_2003,le_tonquesse_2019}.
In the undoped material, its value does not exceed $0.1$~W/mK in the measurement up to $T=950$~K~\cite{kim_2003}.
By doping, the electronic thermal conductivity can be enhanced, and it has a direct impact on the thermoelectric properties of $\beta$-FeSi$_2$ at high temperatures~\cite{ito_2002,kim_2003}.
In the present study, we consider only the phonon contribution to the thermal conductivity,
therefore, agreement with experimental data may deteriorate with increasing temperature.

\section{Summary}
\label{sec.summary}

We performed {\it ab initio} studies on lattice dynamical and thermal transport properties of $\beta$-FeSi$_2$. The effect of anharmonicity was analyzed within two approaches -- the SCPH method and the perturbation theory.
The phonon dispersion curves obtained within SCPH show small renormalization of frequencies comparing to the harmonic approximation. 
The Raman spectra were calculated within the procedure which takes into account the peak intensities obtained from the Raman tensors and the line profiles obtained from the phonon self energy derived within the perturbation theory based on the large, temperature-independent, quartic model fitted to the data from the wide range of temperatures (300-1000~K). 
The anharmonic corrections strongly affect the frequencies and line profiles of some modes and results in overall better agreement with the experimental data. 
We analyzed the phonon lifetimes and group velocities obtained as functions of the phonon frequency.
Then the lattice thermal conductivity was calculated for a broad range of temperatures and grain sizes.
We found a small anisotropy in the phonon thermal transport resulting from the orthorhombic structure and a weak effect of the quartic anharmonic terms. 
The thermal conductivity calculated for various crystalline grain sizes show a good qualitative agreement with the available measurements.

\begin{acknowledgments}
Some figures in this work were rendered using {\sc Vesta}~\cite{momma.izumi.11} software.
This work was partially supported by the Ministry of Education, Youth and Sports of the Czech Republic through the e-INFRA CZ (ID:90254).
\end{acknowledgments}

\appendix

\section{Raman spectrum}
\label{RamanAA}

Based on the polarized Raman measurements reported in Ref.~\cite{maeda_2004}, two Raman peaks were identified as belonging to the A$_g$ symmetry class, and several additional peaks were observed with similar or different polarization dependence. Although the authors of Ref.~\cite{maeda_2004} provided estimates of the relative Raman tensor components, they did not specify which of the remaining modes correspond to the B$_{1g}$, B$_{2g}$, or B$_{3g}$ symmetries. Because of this missing experimental information, a direct symmetry-resolved comparison between the measurement and theory is not currently possible for the non-A$_g$ modes. 
To provide a complete theoretical picture of the B$_g$-type modes, we show here the calculated Raman-active frequencies and intensities for the B$_{1g}$, B$_{2g}$, and B$_{3g}$ symmetries only. 
%These results represent the predicted Raman modes for the B$_g$ symmetries in $\beta$-FeSi$_2$. 
\textcolor{red} {Fig.~\ref{raman1} shows the predicted Raman modes for the B$_g$ symmetries in $\beta$-FeSi$_2$. The results obtained within the harmonic approximation are compared with the anharmonic perturbation theory calculations which provides both, frequency shifts and predicted line profiles of the modes.}
Although the experimentally measured peaks cannot be directly assigned to these symmetries due to the lack of polarization-resolved data, the theoretical predictions provide a reference for comparison. Matching the measured frequencies to the closest theoretical B$_g$ modes (Table~\ref{T1Raman}) allows for a tentative assignment, which can guide future polarization-resolved Raman experiments aimed at determining the precise symmetry of the unresolved peaks.

\begin{figure*}[t]
\centering
  \includegraphics[width=\linewidth]{spectrum_RTA+raman_Bi_modes.pdf}
\caption{Raman spectrum of $\beta$-FeSi$_2$ calculated at three temperatures 
($300$, $600$, $900$~K -- blue, orange, and red lines, respectively). 
The calculated spectrum includes all Raman-active modes. The B$_{ig}$ modes
are indicated by purple vertical lines at peak positions corresponding to T=300~K. 
The frequencies derived from the harmonic approximation at T=300~K are indicated 
by green, yellow and pink lines.
Connecting arrows indicate the correspondence between harmonic and anharmonic 
frequencies, demonstrating the frequency shifts due to phonon interactions.}
\label{raman1}
\end{figure*}
\section{Thermal conductivity obtained from BTE and RTA}
\label{ThermalAB}

The thermal conductivity was computed by solving the full BTE on the largest feasible $\bm{q}$-point grid, $8\times8\times8$, and compared with the corresponding RTA results obtained on the same grid. As shown in Fig.~\ref{bte}, the difference between the components of the thermal conductivity tensor obtained within BTE and RTA at this resolution is very small, indicating a good agreement between these two approaches.
%THIS PART SHOULD GO RATHER TO THE RESPONSE
%Extending the full BTE calculation to larger grids is computationally prohibitive: the computational cost of BTE is roughly two orders of %magnitude higher than that of RTA, and the required memory and runtime exceed our available resources. 
Moreover, we performed an additional calculation using RTA on a denser $20\times20\times20$ grid. As seen in the Fig.~\ref{bte}, the higher-resolution data remain in a good agreement with both the BTE and RTA results for the $8\times8\times8$ grid.
It shows that the $8\times8\times8$ mesh already provides good results for this structure and confirms reliability of the calculations.

\begin{figure}[!h]
\centering
  \includegraphics[width=\linewidth]{BTEvsRTAvsANP.pdf}
\caption{Thermal conductivity of $\beta$-FeSi$_2$ obtained within the BTE and RTA methods using the Phono3py software on the $8\times8\times8$ q-point grid, compared with the RTA results computed with ALAMODE on a denser $20\times20\times20$ grid.}
\label{bte}
\end{figure}

%\section*{Data availability}
%The data that support the findings of this article are openly available~\footnote{give me DOI}.
% see https://journals.aps.org/authors/data-availability-statements#citation

\bibliography{refs.bib}
%\bibliographystyle{ieeetr}


\end{document}

\documentclass[%
%reprint,
superscriptaddress,
%groupedaddress,
longbibliography,
%unsortedaddress,
%runinaddress,
%frontmatterverbose, 
%preprint,
%preprintnumbers,
%nofootinbib,
nobibnotes,
%bibnotes,
amsmath,amssymb,
aps,
%pra,
prb,
%rmp,
%prstab,
%prstper,
%showkeys,
floatfix,
twocolumn
]{revtex4-2}

\usepackage{graphicx}% Include figure files
\usepackage{calc}% Calculate margins
\usepackage{dcolumn}% Align table columns on decimal point
\usepackage{bm}% bold math

\usepackage[urlcolor=blue,colorlinks=true,citecolor=blue,linkcolor=blue,pdfstartview={FitH},bookmarks=false]{hyperref} % add hypertext capabilities

%\usepackage[mathlines]{lineno} % Enable numbering of text and display math
% \linenumbers\relax % Commence numbering lines

% \usepackage[showframe,%Uncomment any one of the following lines to test 
% %scale=0.7, marginratio={1:1, 2:3}, ignoreall,% default settings
% %text={7in,10in},centering,
% %margin=1.5in,
% % total={6.5in,8.75in}, top=1.2in, left=0.9in, includefoot,
% % height=10in,a5paper,hmargin={3cm,0.8in},
% ]{geometry}

\usepackage{amsmath}
\usepackage{amssymb}
%\usepackage{orcidlink}
\usepackage{xcolor}
%\usepackage{datetime}
\usepackage[normalem]{ulem}

% Change tracking commands
\newcommand{\trackchange}[3]{\textcolor{#3}{\sout{#1}#2}}  % Full color strikeout, insert
%\renewcommand{\trackchange}[3]{\textcolor{#3}{#2}}        % Just color silent remove and insert
%\renewcommand{\trackchange}[3]{{#2}}                      % No indication, silent remove and insert

% Author marker definitions

\definecolor{myblue}{RGB}{0,127,85}
% \definecolor{violet}{RGB}{102,0,204}
% \definecolor{orange}{RGB}{255,128,0}
% \definecolor{green}{RGB}{0,128,0}
\newcommand{\DL}[1]{\trackchange{}{#1}{blue}}
\newcommand{\AP}[1]{\trackchange{}{#1}{red}}
\newcommand{\PJ}[2]{\trackchange{#1}{#2}{orange}}
\newcommand{\JL}[2]{\trackchange{#1}{#2}{myblue}}
\newcommand{\SP}[2]{\trackchange{#1}{#2}{blue}}
\newcommand{\PP}[2]{\trackchange{#1}{#2}{teal}}
\newcommand{\AI}[2]{\trackchange{#1}{#2}{olive}}
\newcommand{\MS}[1]{\trackchange{}{#1}{purple}}

\newcommand{\TODO}[1]{\textcolor{red}{TODO: #1}}

\sloppy

\begin{document}

\title{Ab initio study of the anharmonic properties and thermal conductivity in $\beta$-FeSi$_2$}

\author{Svitlana~Pastukh}
\email[e-mail: ]{svitlana.pastukh@ifj.edu.pl}
\affiliation{Institute of Nuclear Physics, Polish Academy of Sciences, ul. W. E. Radzikowskiego 152, 31-342 Krak\'{o}w, Poland}

\author{Ma\l{}gorzata~Sternik}
\affiliation{Institute of Nuclear Physics, Polish Academy of Sciences, ul. W. E. Radzikowskiego 152, 31-342 Krak\'{o}w, Poland}

\author{Pawe\l{}~T.~Jochym}
\affiliation{Institute of Nuclear Physics, Polish Academy of Sciences, ul. W. E. Radzikowskiego 152, 31-342 Krak\'{o}w, Poland}


\author{Jan~\L{}a\.{z}ewski}
\affiliation{Institute of Nuclear Physics, Polish Academy of Sciences, ul. W. E. Radzikowskiego 152, 31-342 Krak\'{o}w, Poland}

\author{Andrzej~Ptok}
\affiliation{Institute of Nuclear Physics, Polish Academy of Sciences, ul. W. E. Radzikowskiego 152, 31-342 Krak\'{o}w, Poland}

\author{Svetoslav~Stankov}
\affiliation{Institute for Photon Science and Synchrotron Radiation, Karlsruhe Institute of Technology, D-76131 Karlsruhe, Germany}
\affiliation{Laboratory for Applications of Synchrotron Radiation, Karlsruhe Institute of Technology, D-76131 Karlsruhe, Germany}

\author{Przemys\l{}aw~Piekarz}
\affiliation{Institute of Nuclear Physics, Polish Academy of Sciences, ul. W. E. Radzikowskiego 152, 31-342 Krak\'{o}w, Poland}

\date{\today}

\begin{abstract}

Iron silicides are good candidates for applications in  optoelectronic and thermoelectric devices.
Lattice dynamical properties and thermal conductivity in the $\beta$-FeSi$_2$ semiconductor
are investigated with the first-principles computational methods. 
Phonon dispersion relations are calculated via the
temperature-dependent effective potential method and self-consistent phonon theory. 
To properly model thermal transport, we explicitly consider
the impact of phonon-phonon interactions by analyzing
anharmonic contributions to the phonon self-energy. 
This yields temperature-dependent phonon frequencies and linewidths,
reflecting the finite lifetime of phonons due to scattering
processes. The calculated phonon frequencies and line profiles are used to obtain 
the Raman spectra, which shows good agreement with the experimental data. 
We revealed an enhanced anharmonic behaviour of the Raman modes with the highest frequencies.  
The lattice thermal conductivity is then obtained as a function of temperature and crystallite size within
the relaxation-time approximation.
Phonon transport shows a small anisotropy due to the orthorhombic structure and a very weak dependence
on the quartic anharmonic corrections. The results obtained for an infinite material and for several crystallite sizes
were analyzed and compared with the available experimental data.
\end{abstract}

\maketitle


\section{Introduction}

The comprehensive determination of important physical properties of
crystals, such as thermal expansion, lattice thermal
conductivity or structural phase transitions, requires a fundamental 
understanding of the anharmonic effects.
Although the investigation of anharmonic interactions in crystals has
attracted a considerable interest for decades~\cite{cowley_1968}, a substantial progress
has only recently been achieved thanks to advances in theoretical and
numerical methods and increased computational power.
Now, phonon frequencies, lifetimes, and heat transfer in a wide range of
materials
can be quantitatively predicted using the available computational resources
based on the density functional theory (DFT)~\cite{lindsay_2013,mcgaughey_2019,lindsay_2019}.
In the case of strongly anharmonic systems, the self-consistent phonon
(SCPH) theory~\cite{tadano_2015} as well as the perturbative approach~\cite{tadano_2018}, using higher
order interatomic force constants derived from the fitting to the displacement force data obtained
with DFT, have proven to be successful.

Transition-metal silicides are promising materials for fabrication of electronic components
designed for integration with silicon-based circuits~\cite{murarka_1995}.
At room temperature, iron disilicide ($\beta$-FeSi$_2$) is a
direct-bandgap semiconductor~\cite{bost_1985}, making this material a good
candidate for application in optoelectronic devices such as infrared
detectors or light emitters~\cite{bost_1988}. The development of light-emitting diodes utilizing FeSi$_2$/Si heterostructures has been successfully demonstrated~\cite{leong_1997,suemasu_2001}. Due to
a high thermal stability and strong light absorption, FeSi$_2$ is
also a suitable photovoltaic material~\cite{powalla_1993,liu_2006,okuhara_2017}.

$\beta$-FeSi$_2$ crystallizes in the base-centered orthorhombic
lattice~\cite{dusausoy_1971} transforming to the
tetragonal metallic $\alpha$-FeSi$_{2}$ phase around $1200$~K~\cite{starke_2002}.
Optical studies indicated a direct band gap of the values
$0.85$--$0.89$~eV~\cite{bost_1988,dimitriadis_1990,arushanov_1995,wan_2003},
however, the {\it ab initio} calculations predicted a smaller indirect gap close to 0.8 eV~\cite{christensen_1990}. 
The existence of such an indirect gap was then confirmed by the optical linear transmittance measurements at
low temperatures~\cite{giannini_1992}. As shown by first-principles studies,
the character of the band gap is very sensitive to the orientation 
of a crystal grown on silicon~\cite{clark_1998}.

$\beta$-FeSi$_2$ belongs also to good thermoelectric materials~\cite{ware_1964}, with potential applications resulting from its chemical stability up to high temperatures, nontoxicity, and low cost of preparation~\cite{yamada_2012,nozariasbmarz_2017}. 
It has already been implemented in cars~\cite{birkholz_1988} and portable power sources~\cite{uemura_1989}. 
Its thermoelectric performance can be improved by doping~\cite{ito_2001,tani_2001,kim_2003,chen_2005,pandey_2013,le_tonquesse_2019}, 
which enhances the electric transport and reduces the thermal conductivity~\cite{waldecker_1973,du_2019,du_2020}.  
The thermal conductivity can be also reduced by the modification of microstructure~\cite{ail_2015} or by
nanostructurization~\cite{watanabe_2017,taniguchi_2017,hsin_2017,abbassi_2021}.

The lattice thermal conductivity is directly connected with anharmonic effects and phonon scattering processes.
The vibrational properties of \mbox{$\beta$-FeSi$_2$} were studied by the infrared and Raman spectroscopy~\cite{lefki_1991,guizzetti_1997,maeda_2004,baleva_2008,liu_2011,maeda_2011}. The observed anisotropy in the phonon spectra results from the enhanced sensitivity of the infrared and Raman features to the local lattice distortions~\cite{guizzetti_1997}. The Fe phonon density of states was measured by nuclear inelastic scattering (NIS), showing a good agreement with the density functional theory (DFT) calculations~\cite{walterfang_2005}. Using the DFT approach, the phonon dispersion curves, phonon density of states, as well as various thermodynamic properties were obtained within the harmonic approximation~\cite{tani_2010,liang_2011}. The extended Klemens model was applied to study
the anharmonic effect on phonon frequencies and linewidths observed by the Raman spectroscopy~\cite{zhang_2023}.
The impact of nanostructurization on lattice dynamics was explored in the $\beta$-FeSi$_2$ nanorods grown on the Si(110) surface by the NIS and {\it ab initio} methods~\cite{kalt_2022}.

In this work, we investigate the lattice dynamical properties of $\beta$-FeSi$_2$ 
using the DFT calculations. We study the effect of anharmonic terms in the temperature-dependent potential on phonon frequencies and lifetimes. We focus on the Raman modes, comparing the theoretical results with the experimental data.
The thermal conductivity is derived in a broad temperature range and the effect of crystallite size is analyzed.



This study is structured as follows.
In Sec.~\ref{sec.com} we describe the details of computational methods.
Next, in Sec.~\ref{sec.result} we present and discuss our results.
In particular we present the crystal structure (Sec.~\ref{sec.crys}) and lattice dynamics (Sec.~\ref{sec.lattice}).
We investigate also the thermal conductivity comparing the obtained results with the available experimental data (Sec.~\ref{sec.thermal}).
Finally, Sec.~\ref{sec.summary} summarizes our key findings and conclusions.

\section{Calculation method}
\label{sec.com}


The calculations were performed using the projector augmented-wave potentials~\cite{blochl_1994} and the generalized gradient approximation~\cite{perdew_1996} implemented in the Vienna Ab initio Simulation Package (VASP)~\cite{kresse.hafner.94,kresse.furthmuller.96,kresse.joubert.99}. 
The lattice parameters and atomic positions were optimized in the ${\bm a} \times ({\bm b}-{\bm c}) \times ({\bm b}+{\bm c})$ supercell containing 32 formula units and four primitive cells.
The integration in the reciprocal space was conducted using the $2 \times 2 \times 2$ Monkhorst--Pack mesh~\cite{monkhorst_1976} and the cut-off energy was set to $500$~eV. For convergence conditions, we set the energy change below $10^{-5}$ and $10^{-8}$ for the ionic and electronic loops, respectively. 

The lattice dynamical properties were studied within the temperature-dependent effective potential (TDEP) approach~\cite{hellman_2013}. The atomic potential with the third and fourth order anharmonic terms was derived from interatomic forces induced by displacements of all atoms at finite temperatures.
The sets of atomic displacements were generated by the high efficiency configuration space sampling (HECSS)~\cite{jochym_2021} and forces were obtained by VASP. The interatomic force constants and phonon frequencies were calculated with the {\sc Alamode} software~\cite{tadano_2014}.

Furthermore, we have attempted to construct a {\emph{temperature independent}} 
anharmonic model. We have used combined data from all investigated temperatures 
(300, 600 and 1000~K) and fitted a large (over 15~000 free parameters), fourth-order 
interaction model to this dataset. 
Subsequently, we have used this model to calculate line profiles and positions of Raman-active modes
at multiple temperatures.

The changes in phonon frequencies induced by the anharmonic effects were investigated within two approaches.
First, the impact of the quartic anharmonic terms was included using the SCPH theory~\cite{tadano_2015}.
Second, the mode profiles (frequency shifts and line widths) were determined from the real and imaginary parts of the phonon self-energy resulting from the cubic and quartic anharmonic terms of the above mentioned large model~\cite{tadano_2018}. 
The longitudinal optic-transverse optic (LO-TO) splitting was also evaluated, using the static dielectric tensor and Born effective charges calculated within density functional perturbation theory~\cite{gajdos_2006}.

\begin{figure}[]
    \centering
    \includegraphics[width=\linewidth]{fig1_new.png}
\caption{%
(a) The conventional unit cell of $\beta$-FeSi$_2$ (with Cmca symmetry) and (b) the corresponding Brillouin zone with selected high-symmetry points.
}
  \label{fig.struct}
\end{figure}

% To further characterize the vibrational properties, Raman-active scattering was investigated using the Phonopy-Spectroscopy package~\cite{skelton_2017}. This enabled the identification of Raman-active modes and the calculation of Raman tensors. Anharmonic force constants, obtained from calculations using {\sc Alamode}, were then used to obtain theoretical line profiles for the Raman modes. The presented Raman scattering spectra combine these anharmonic line profiles with the Raman tensor amplitudes.
% This analysis was also based on the large quartic model mentioned above.

% Finally, the thermal conductivity was obtained as a function of temperature and crystallite size within the relaxation-time approximation (RTA) \PJ{}{as implemented in {\sc Alamode}\cite{tadano_2014}}. The phonon lifetimes were calculated from the phonon self-energy including the cubic and quartic anharmonic terms. \SP{}{The RTA provides a solution of the Boltzmann transport equation (BTE) under the assumption that scattering events are independent and can be treated through mode-resolved relaxation times.
% To verify the validity of this approximation for $\beta$-FeSi$_2$, we have additionally
% executed an iterative solution of the BTE.}\PJ{}{These calculations were performed with
% {\sc Phono3py}\cite{togo_2023}. For cross-validation, the additional RTA calculations were performed on the same $\bm{q}$-grid as BTE with implementation provided by {\sc Phono3py}.}.

To further characterize the vibrational properties, Raman scattering was investigated using the Phonopy-Spectroscopy package~\cite{skelton_2017}. This enabled the identification of Raman-active modes and the calculation of Raman tensors. Anharmonic force constants, derived from calculations using {\sc Alamode}, were then used to obtain theoretical line profiles for the Raman modes. The presented Raman scattering spectra combine these anharmonic line profiles with the Raman tensor amplitudes.
This analysis was also based on the large quartic model described above.

Finally, the thermal conductivity was calculated as a function of temperature and crystallite size within the relaxation-time approximation (RTA) \PJ{}{as implemented in {\sc Alamode}~\cite{tadano_2014}}. The phonon lifetimes were calculated from the phonon self-energy including the cubic and quartic anharmonic terms. \SP{}{The RTA provides a solution to the Boltzmann transport equation (BTE) under the assumption that scattering events are independent and can be treated through mode-resolved relaxation times.
To verify the validity of this approximation for $\beta$-FeSi$_2$, we additionally
solved the BTE iteratively.} \PJ{}{These calculations were performed with
{\sc Phono3py}~\cite{togo_2023}. For cross-validation, the additional RTA calculations were performed on the same $\bm{q}$-grid as the BTE calculations, using the implementation provided by {\sc Phono3py}.}

% (see. Appendix~\ref{ThermalAB} for the comparison)\cite{togo_2023}.}
%As shown in Appendix~\ref{ThermalAB}, the iterative BTE results exhibit good agreement with the RTA values, confirming that RTA is sufficiently accurate for this material and for the considered temperature range.}


\section{Results}
\label{sec.result}

\subsection{Crystal structure}
\label{sec.crys}

The $\beta$-FeSi$_2$ structure adopts a base-centered orthorhombic lattice with the space group Cmca (No.~64) as shown in Fig.~\ref{fig.struct}(a).
The unit cell consists of two primitive cells and contains 48 atoms.
Iron (silicon) atoms possess two nonequivalent positions: \mbox{Fe-I} and \mbox{Fe-II} (\mbox{Si-I} and \mbox{Si-II}), presented in Fig.~\ref{fig.struct}(a) as gray and purple (orange and yellow) spheres, respectively.
This crystal structure is derived from the fluorite-type lattice with strongly distorted Si cubes and Fe atoms occupying 
one-half of the central sites.
The Fe-I and Fe-II sites create different layers perpendicular to the $x$ direction, and they are separated by layers containing both Si sites.
Each Fe atom is coordinated by 8 Si atoms with slightly different Fe-Si distances.
The optimized lattice constants ($a = 9.874$~{\AA}, $b = 7.767$~{\AA}, and
$c = 7.811$~{\AA}) agree very well with the experimental data ($a = 9.863$~{\AA}, $b = 7.791$~{\AA}, and $c = 7.833$~{\AA})~\cite{dusausoy_1971}.

Iron atoms occupy the Wyckoff sites 8\textit{d} ($0.2166$, $0$, $0$) and  8\textit{f} ($0$, $0.3072$, $0.1879$), corresponding to \mbox{Fe-I} and \mbox{Fe-II}, respectively. 
Silicon atoms are located at two inequivalent 16\textit{g} positions: ($0.1282$, $0.2737$, $0.0495$) and \mbox{($0.3734$, $0.0445$, $0.2270$)}, assigned as \mbox{Si-I} and \mbox{Si-II}.
The optimized positions of atoms agree very well with the experimental data~\cite{dusausoy_1971} and the previous theoretical studies~\cite{tani_2010,liang_2011}.


\subsection{Lattice dynamics}
\label{sec.lattice}

\begin{figure}[]
    \centering
    \includegraphics[width=\linewidth]{Fig1c.pdf}
\caption{%
The phonon dispersion curves along high symmetry directions obtained within SCPH (for temperatures from $0$ to $1000$~K).
Dashed black lines indicate the phonon dispersions obtained from the harmonic approximation. The white dots indicate Raman-active modes with A$_g$ symmetry.
The vertical plot shows the phonon density of states (DOS) calculated at a reference temperature of $600$~K.
}
  \label{fig.ph_band}
\end{figure}


In Fig.~\ref{fig.ph_band} we present the phonon dispersion relations of $\beta$-FeSi$_2$ along high-symmetry directions in the Brillouin zone [Fig.~\ref{fig.struct}(b)].
Due to 24 atoms in the primitive cell, the phonon spectrum consists of 69 optical modes and three acoustic modes.
The phonon dispersions were calculated within the SCPH approach in the temperature range $0$--$1000$~K (presented by color lines in Fig.~\ref{fig.ph_band}), and they are compared with the results obtained from the harmonic part of the effective potential corresponding to temperature $T=300$~K (indicated by dashed black lines in Fig.~\ref{fig.ph_band}).
As we can see within the SCPH method, the anharmonic effects are rather weak and leads only to small renormalization of phonon frequencies.
Only the highest modes show more pronounced shifts of their frequencies to larger values.
The total and partial element-projected phonon density of states obtained within the harmonic approximation are presented in Fig.~\ref{fig.ph_band}.
Up to around $320$~cm$^{-1}$, the contributions from both elements are very similar, while for higher frequencies the spectrum is dominated by
the Si vibrations.


\begin{table}[!t]
\begin{ruledtabular}
\caption{Calculated and experimental Raman-active modes of $\beta$-FeSi$_2$ with their irreducible representations (IR). Present theoretical results are compared with the previous theoretical data from Ref.~\cite{tani_2010} and experimental results from \mbox{Refs.~\cite{lefki_1991,maeda_2004}.} The experimental frequencies with known symmetries (A$_g$) are shown in bold, while other experimental modes are assigned to the best fitting theoretical values.}
\begin{tabular}{c c c c c}
\textbf{IR} & \multicolumn{4}{c}{\textbf{Frequency (cm$^{-1}$)}} \\
 & Present  & Theor.~\cite{tani_2010} & Exp.~\cite{lefki_1991} & Exp.~\cite{maeda_2004} \\
\hline
B$_{2g}$ & 175.4 & 179 & 176 &  \\
B$_{1g}$ & 176.3 & 185 & 179  &  \\
B$_{1g}$ & 193.3 & 198 &  & 190.6 \\
A$_g$    & 196.6 & 208 & \textbf{195} & \textbf{194.0} \\
A$_g$    & 203.9 & 210 & \textbf{197} & 199.6 \\
B$_{3g}$ & 205.5 & 212 & 200 &  \\
B$_{3g}$ & 226.3 & 236 & 206 & 227.1 \\
B$_{1g}$ & 233.3 & 240 &  &  231.6 \\
B$_{2g}$ & 248.6 & 254 &  &  \\
A$_g$    & 250.1 & 257 & \textbf{247} & \textbf{247.3} \\
A$_g$    & 254.9 & 264 & \textbf{253} & 254.3 \\
B$_{1g}$ & 275.5 & 285 &  &  274.1 \\
B$_{2g}$ & 282.4 & 295 &  &  281.2 \\
B$_{3g}$ & 286.8 & 297 &  &  \\
B$_{1g}$ & 307.1 & 317 &  &  \\
B$_{2g}$ & 312.6 & 326 &  &  311.8 \\
B$_{1g}$ & 319.0 & 324 &  &  \\
B$_{3g}$ & 327.9 & 341 &  & 325.8 \\
B$_{2g}$ & 333.3 & 345 &  &  \\
A$_g$    & 339.3 & 352 & \textbf{346} & 339.5 \\
B$_{2g}$ & 343.0 & 350 &  &  \\
B$_{1g}$ & 353.5 & 366 &  &  \\
B$_{2g}$ & 372.8 & 383 &  & 370.7 \\
B$_{1g}$ & 375.1 & 385 &  &  \\
B$_{3g}$ & 375.2 & 386 &  &  \\
B$_{3g}$ & 385.3 & 401 &  &  \\
A$_g$    & 386.5 & 398 &  & 386.2 \\
B$_{2g}$ & 387.7 & 402 &  & 388.2 \\
A$_g$    & 404.6 & 415 &  & 400.4 \\
B$_{2g}$ & 405.1 & 420 &  &  \\
B$_{1g}$ & 412.7 & 428 &  &  \\
B$_{3g}$ & 418.4 & 431 &  &  \\
B$_{3g}$ & 441.4 & 458 &  & 442.6 \\
B$_{3g}$ & 447.3 & 466 &  & 446.3 \\
A$_g$    & 448.4 & 464 &  &  \\
A$_g$    & 499.1 & 517 &  &  \\
\end{tabular}
\label{T1Raman}
\end{ruledtabular}
\end{table}

%subsection{Raman}

\begin{figure*}
\centering
  \includegraphics[width=\textwidth]{spectrum_RTA+raman.pdf}
\caption{Raman spectrum of $\beta$-FeSi$_2$ calculated within the perturbative approach at three temperatures 
($300$, $600$, $900$~K -- blue, orange, and red lines, respectively). 
The calculated spectrum includes all Raman-active modes. The A$_g$ modes
are indicated by purple vertical lines at peak positions corresponding to T=300~K. 
The frequencies derived from the harmonic approximation at T=300~K are indicated by green lines.
Connecting arrows indicate the correspondence between harmonic and
anharmonic frequencies, demonstrating the frequency shifts due to 
phonon interactions. Experimental values for the A$_g$ modes
based on Ref.~\cite{lefki_1991} are marked with black dashed lines.
}
\label{raman}
\end{figure*}


The phonon spectrum at the $\Gamma$ point consists of 36 Raman modes classified according to the irreducible representations: $9A_\text{g}+9B_\text{1g}+9B_\text{2g}+9B_\text{3g}$. 
Fig.~\ref{raman} shows the Raman spectrum of A$_g$ symmetry calculated for $\beta$-FeSi$_2$ within the perturbative approach at three temperatures $300$, $600$, and $900$~K (solid blue, orange, and red curves, respectively), including third-order and fourth-order anharmonic corrections.
The calculated Raman spectrum includes all Raman-active modes.
The five experimentally observed A$_g$ modes are highlighted by black dashed lines, based on data from Ref.~\cite{lefki_1991}.
Additional peaks not marked with vertical lines correspond to phonon modes with symmetries other than A$_g$.
The frequencies of Raman modes obtained in anharmonic calculations are compared with the previous results calculated within the harmonic approximation and the experimental values in Tab.~\ref{T1Raman}. 
We have marked in bold the experimentally determined A$_g$ modes, which are compared with the calculations.
Since the experimental studies did not provide the accurate assignement of the Raman modes with the B$_{1g}$, B$_{2g}$, and B$_{3g}$ symmetry~\cite{lefki_1991,maeda_2004}, we cannot compare them directly with the theoretical results.
However, in Tab.~\ref{T1Raman} we have assigned the measured frequencies to the best fitting theoretical values without taking into account the symmetry of the modes, except for the known A$_{g}$ modes. 


The impact of anharmonicity on the phonon frequencies is well visible from the comparison of the results obtained within the harmonic approximation 
and from the anharmonic calculation (vertical green and purple lines in Fig.~\ref{raman}, respectively).
Here we show only the A$_g$ modes, which are compared with the experimental results (vertical black dashed lines).
Anharmonic frequencies calculated at $300$~K are indicated by purple lines, while frequencies derived from harmonic approximation are marked with green lines. The grey solid lines connect corresponding modes obtained in both approximations.
In most cases, the results obtained within the harmonic approximation do not agree with the experimental frequencies. 
%Only the modes close to $250$~cm$^{-1}$ and $400$~cm$^{-1}$ correspond well to the experimental values. MS
As we can see, the inclusion of the anharmonic correction leads to a significant modification of the phonon frequencies.
These anharmonic effects are stronger for higher-frequency modes mainly because of
the dominant contribution from Si atoms, which vibrate with larger amplitudes than heavier Fe atoms. 
When atoms move to larger distances the potential deviates more from the harmonic approximation,
and the anharmonic corrections become stronger.


The modification of phonon frequencies observed in Fig.~\ref{raman} is much larger than in the SCPH scheme presented in Fig.~\ref{fig.ph_band}. The SCPH approach includes only the leading-order contribution to 
the phonon self-energy obtained from the quartic anharmonic terms~\cite{tadano_2015}. Therefore, it does not describe fully the changes of phonon frequencies found within the perturbation theory (see Fig.~\ref{raman}).
Especially, it is well visible for two highest A$_g$ modes, which exhibit also the largest line broadening 
and the strongest dependence on temperature.
Therefore, a better agreement with experimentally observed frequencies is visible,
confirming the significant influence of the anharmonicity on the frequencies and line profiles of phonon modes.
In fact, the decrease of phonon frequencies should be even stronger due to thermal expansion, 
which is not included in our calculations.
Within SCPH the frequencies of the highest modes increase with increasing temperature as we see in Fig.~\ref{fig.ph_band}. The comparison of two different approaches applied to study anharmonic properties of $\beta$-FeSi$_2$ shows that the perturbation theory, which includes the cubic and quartic terms, better describes the changes of phonon frequencies with temperature than the SCPH method. \PJ{}{This indicates that the leading-order contribution included in SCPH are not important in this material.}

Additionaly we should nottice that for other than A$_g$ modes we cannot make an unambiguous assignment of theoretical frequencies to experimental ones. Note that the spectrum in Fig.~\ref{fig.ph_band} contains all Raman-active modes. 
The limitation to A$_g$ modes concerns only the indicated positions of the peaks. 
%\SP{}{Furthermore, we present the full spectrum for all Raman active modes with the comparison of the harmonic and anharmonic calculations in the Appendix~\ref{RamanAA}.}
\textcolor{red}{The full Raman spectrum, with the frequencies of the B$_{1g}$, B$_{2g}$ and B$_{3g}$ modes marked, is shown in Appendix A, Fig.~\ref{raman1}. This represents our theoretical prediction of possible Raman-mode assignments, which can be verified in future experiments.}
%This is the theoretical prediction of possible assignment of Raman modes that can be verified in future experiments.  


\subsection{Thermal conductivity}
\label{sec.thermal}


\begin{figure*}[!t]
\centering
  \includegraphics[width=\linewidth]{time3.pdf}
\caption{Phonon lifetimes calculated for three temperatures as a function of phonon frequency. The colors correspond to the phonon branches.}
  \label{thermtime}
\end{figure*}


In this section, we analyze the thermal conductivity tensor of $\beta$-FeSi$_2$ obtained within the RTA approach~\cite{tadano_2018} as a function of temperature
%
\begin{equation}
\kappa_{\text{ph}}^{\mu\nu}(T) = \frac{1}{NV} \sum_{\bm{q},j} c_{\bm{q}j}(T) v_{\bm{q}j}^{\mu} v_{\bm{q}j}^{\nu}\tau_{\bm{q}j}(T),
\end{equation}
% 
where $c_{\bm{q}j}$ is the mode heat capacity and $v_{\bm{q}j}$ is the mode group velocity. 
The relaxation time is approximated by the phonon lifetime $\tau_{\bm{q}j}$
calculated for $j$-th branch at the wave vector $\bm{q}$.
$V$ is the unit cell volume and $N$ is the number of unit cells in the crystal.
The phonon lifetime is calculated using this formula
%
\begin{equation}
\tau_{\bm{q}j}(T)=\frac{1}{2\Gamma_{\bm{q}j}^{\text{anh}}(T)},
\end{equation}
%
where $\Gamma_{\bm{q}j}^{\text{anh}}$ is the anharmonic phonon linewidth obtained from 
the imaginary part of the phonon self-energy within the perturbation theory.

In Fig.~\ref{thermtime}, we present $\tau_{\bm{q}j}$ obtained for three temperatures 300, 600, and 1000~K as a function of frequency. As we see, the acoustic phonons close to the $\Gamma$ point have the longest lifetimes,
which are diminished with increasing frequency reaching local minima around $200$~cm$^{-1}$.
For higher frequencies, phonon lifetimes first increase to local maxima around $300$~cm$^{-1}$ and then decrease to
the lowest values in the range of highest optical modes. The shortest lifetimes correspond to the largest line broadening
observed for the Raman modes in Fig~\ref{raman}. The phonon group velocities 
$v_{\bm{q}j}=\partial\omega_{\bm{q}j}/\partial\bm{q}$, which are obtained by the central difference formula, are presented in Fig.~\ref{thermvelocity}. Their temperature dependence is negligible, therefore, we present only the results for $T=600$~K.
At low frequencies, there are clearly two ranges of group velocities of the acoustic phonons. 
The larger values correspond to the longitudinal modes, while the lower values are obtained from the
transverse acoustic branches. Group velocities of acoustic phonons decrease for larger frequencies
and reach the average values typical for optic branches.

\begin{figure}[!t]
\centering
  \includegraphics[width=\linewidth]{GV.pdf}
\caption{Mode group velocities calculated as a function of phonon frequency. The colors correspond to the phonon branches.
}
  \label{thermvelocity}
\end{figure}

In Fig.~\ref{anizotropy}(a), we present the three diagonal elements of $\kappa_{\text{ph}}^{\mu\nu}$ corresponding to the main directions of the crystal structure. 
They were obtained from the force constants calculated at the base temperature $T=600$~K and the crystallite size $0.1$~$\mu$m to account for boundary-limited phonon transport. 
Due to the orthorhombic symmetry, we observe a small anisotropy in phonon transport in the whole temperature range. 
At low temperatures, the three components of the heat conductivity increase in a very similar way with the $\kappa_{\text{ph}}^{yy}$ element slightly larger than two other components. 
After reaching the maximum, we observe a change in the largest component from $\kappa_{\text{ph}}^{yy}$ to 
$\kappa_{\text{ph}}^{xx}$.    
In Fig.~\ref{anizotropy}(b), the thermal conductivity is shown for three base temperatures, at which the interatomic potential was obtained ($300$~K, $600$~K, and $1000$~K), using the energy expansion up to third- and fourth-order anharmonic terms, and the same structure size of $0.1$~$\mu$m. At lowest temperatures, the thermal conductivity strongly increases, reaching the maximum around $T=180$~K, then it shows a slower decrease with temperature.
The differences between the two levels of approximation are minimal, suggesting that third-order calculations already capture the dominant phonon scattering mechanisms. The dependence on the base temperature is also very weak, showing the changes in the heat conductivity within a few percent. 
\SP{}{Further verification of the reliability of our thermal conductivity results is given in Appendix~\ref{ThermalAB}, where the full BTE calculations show good agreement with RTA and higher-resolution RTA results, demonstrating that the $8\times8\times8$ $\bm{q}$-mesh already provides converged values.}

\begin{figure}[!t]
\centering
  \includegraphics[width=\linewidth]{Anizotropy.pdf}
\caption{(a) The anisotropic thermal conductivity of $\beta$-FeSi$_2$ calculated along the lattice directions at 600~K. (b) The average temperature-dependent thermal conductivity taken at $300$~K, $600$~K, and $1000$~K, including anharmonic corrections up to cubic (A3) and quartic (A4) terms. In both cases the crystallite size is 0.1~$\mu$m.}
  \label{anizotropy}
\end{figure}


In Fig.~\ref{therm}, we fix the base temperature at $600$~K and examine the effect of crystallite size on thermal conductivity, varying it from $0.01$ to $0.5$~$\mu$m.
With decreasing the crystallite size, we observe a shift of the position of the maximum to larger temperatures and a decrease of the thermal conductivity in the entire temperature range.
Theoretical results are compared with several experimental data obtained above the room temperature. 
The measured thermal conductivity depends to a large extent on the sample quality, its purity and the size of the crystalline grains which depends on 
the production processes.
Many measurements were performed using crystallites of micrometric or unknown size ~\cite{waldecker_1973,ito_2002,kim_2003,du_2020}, however, 
numerous attempts to minimize $\kappa$ by reducing grain sizes to $56$~nm~\cite{dabrowski_2019, dabrowski_microstructure_2021}, $30$-$400$~nm~\cite{le_tonquesse_2019}, $50$ and $200$~nm~\cite{abbassi_2021}, or introducing pores into the material~\cite{sam2023improved} 
are also carried out. 
Another way to change the thermal conductivity is to dope $\beta$-FeSi$_2$ with different elements~\cite{ito_2002,kim_2003,du_2020,cheng_2024}, however, this effect is beyond our investigation.

\begin{figure}[!t]
\centering
  \includegraphics[width=\linewidth]{Thermal_conductivity3.pdf}
\caption{
The phonon thermal conductivity of $\beta$-FeSi$_2$: theoretical results for the infinite crystalline size and with boundary conditions, compared with experimental data for different structure sizes.
}
  \label{therm}
\end{figure}
%Fig.~\ref{therm}\cite{abbassi_2021}\cite{waldecker_1973}\cite{dabrowski_2019}\cite{sam2023improved}\cite{kim_2003}\cite{du_2020}\cite{ito_2002}\cite{le_tonquesse_2019}.

We observe a decrease in the thermal conductivity with reducing crystalline grain sizes in all analyzed experimental data. 
For instance, by decreasing the crystallite size to $50$~nm, the thermal conductivity at room temperature was reduced by a factor of $1.7$, what can be  compared to the annealed sample with 200 nm grains~\cite{abbassi_2021}. 
It is worth noting that the rate of decrease in value with increasing temperature in both cases, for grain sizes of $50$~nm and $200$~nm, is significantly different, which is consistent with our calculations. 
The same trend can be observed by comparing the thermal conductivity measured for a sample with bulk crystallite sizes with the thermal conductivity of a sample with grains smaller than 400 nm~\cite{le_tonquesse_2019}.
The theoretical results obtained for the same crystallite size show higher values due to factors not captured in the idealized model, such as crystal imperfection or mechanical strain. Usually, a decrease in the crystallite size is related to an increased concentration of grain boundaries, point defects, and stacking faults that influence the phonon scattering~\cite{le_tonquesse_2019,abbassi_2021}.   

We should note that the total thermal conductivity is a combination
of the lattice and electronic contributions to the heat transport.
In semiconductors, the electronic thermal conductivity is negligible at low temperatures and significantly increases 
only much above the room temperatures~\cite{gu_2020}.
For $\beta$-FeSi$_2$, the electronic thermal conductivity was obtained from the electric conductivity using the Wiedemann-Franz law~\cite{ito_2002,kim_2003,le_tonquesse_2019}.
In the undoped material, its value does not exceed $0.1$~W/mK in the measurement up to $T=950$~K~\cite{kim_2003}.
By doping, the electronic thermal conductivity can be enhanced, and it has a direct impact on the thermoelectric properties of $\beta$-FeSi$_2$ at high temperatures~\cite{ito_2002,kim_2003}.
In the present study, we consider only the phonon contribution to the thermal conductivity,
therefore, agreement with experimental data may deteriorate with increasing temperature.

\section{Summary}
\label{sec.summary}

We performed {\it ab initio} studies on lattice dynamical and thermal transport properties of $\beta$-FeSi$_2$. The effect of anharmonicity was analyzed within two approaches -- the SCPH method and the perturbation theory.
The phonon dispersion curves obtained within SCPH show small renormalization of frequencies comparing to the harmonic approximation. 
The Raman spectra were calculated within the procedure which takes into account the peak intensities obtained from the Raman tensors and the line profiles obtained from the phonon self energy derived within the perturbation theory based on the large, temperature-independent, quartic model fitted to the data from the wide range of temperatures (300-1000~K). 
The anharmonic corrections strongly affect the frequencies and line profiles of some modes and results in overall better agreement with the experimental data. 
We analyzed the phonon lifetimes and group velocities obtained as functions of the phonon frequency.
Then the lattice thermal conductivity was calculated for a broad range of temperatures and grain sizes.
We found a small anisotropy in the phonon thermal transport resulting from the orthorhombic structure and a weak effect of the quartic anharmonic terms. 
The thermal conductivity calculated for various crystalline grain sizes show a good qualitative agreement with the available measurements.

\begin{acknowledgments}
Some figures in this work were rendered using {\sc Vesta}~\cite{momma.izumi.11} software.
This work was partially supported by the Ministry of Education, Youth and Sports of the Czech Republic through the e-INFRA CZ (ID:90254).
\end{acknowledgments}

\appendix

\section{Raman spectrum}
\label{RamanAA}

Based on the polarized Raman measurements reported in Ref.~\cite{maeda_2004}, two Raman peaks were identified as belonging to the A$_g$ symmetry class, and several additional peaks were observed with similar or different polarization dependence. Although the authors of Ref.~\cite{maeda_2004} provided estimates of the relative Raman tensor components, they did not specify which of the remaining modes correspond to the B$_{1g}$, B$_{2g}$, or B$_{3g}$ symmetries. Because of this missing experimental information, a direct symmetry-resolved comparison between the measurement and theory is not currently possible for the non-A$_g$ modes. 
To provide a complete theoretical picture of the B$_g$-type modes, we show here the calculated Raman-active frequencies and intensities for the B$_{1g}$, B$_{2g}$, and B$_{3g}$ symmetries only. 
%These results represent the predicted Raman modes for the B$_g$ symmetries in $\beta$-FeSi$_2$. 
\textcolor{red} {Fig.~\ref{raman1} shows the predicted Raman modes for the B$_g$ symmetries in $\beta$-FeSi$_2$. The results obtained within the harmonic approximation are compared with the anharmonic perturbation theory calculations which provides both, frequency shifts and predicted line profiles of the modes.}
Although the experimentally measured peaks cannot be directly assigned to these symmetries due to the lack of polarization-resolved data, the theoretical predictions provide a reference for comparison. Matching the measured frequencies to the closest theoretical B$_g$ modes (Table~\ref{T1Raman}) allows for a tentative assignment, which can guide future polarization-resolved Raman experiments aimed at determining the precise symmetry of the unresolved peaks.

\begin{figure*}[t]
\centering
  \includegraphics[width=\linewidth]{spectrum_RTA+raman_Bi_modes.pdf}
\caption{Raman spectrum of $\beta$-FeSi$_2$ calculated at three temperatures 
($300$, $600$, $900$~K -- blue, orange, and red lines, respectively). 
The calculated spectrum includes all Raman-active modes. The B$_{ig}$ modes
are indicated by purple vertical lines at peak positions corresponding to T=300~K. 
The frequencies derived from the harmonic approximation at T=300~K are indicated 
by green, yellow and pink lines.
Connecting arrows indicate the correspondence between harmonic and anharmonic 
frequencies, demonstrating the frequency shifts due to phonon interactions.}
\label{raman1}
\end{figure*}
\section{Thermal conductivity obtained from BTE and RTA}
\label{ThermalAB}

The thermal conductivity was computed by solving the full BTE on the largest feasible $\bm{q}$-point grid, $8\times8\times8$, and compared with the corresponding RTA results obtained on the same grid. As shown in Fig.~\ref{bte}, the difference between the components of the thermal conductivity tensor obtained within BTE and RTA at this resolution is very small, indicating a good agreement between these two approaches.
%THIS PART SHOULD GO RATHER TO THE RESPONSE
%Extending the full BTE calculation to larger grids is computationally prohibitive: the computational cost of BTE is roughly two orders of %magnitude higher than that of RTA, and the required memory and runtime exceed our available resources. 
Moreover, we performed an additional calculation using RTA on a denser $20\times20\times20$ grid. As seen in the Fig.~\ref{bte}, the higher-resolution data remain in a good agreement with both the BTE and RTA results for the $8\times8\times8$ grid.
It shows that the $8\times8\times8$ mesh already provides good results for this structure and confirms reliability of the calculations.

\begin{figure}[!h]
\centering
  \includegraphics[width=\linewidth]{BTEvsRTAvsANP.pdf}
\caption{Thermal conductivity of $\beta$-FeSi$_2$ obtained within the BTE and RTA methods using the Phono3py software on the $8\times8\times8$ q-point grid, compared with the RTA results computed with ALAMODE on a denser $20\times20\times20$ grid.}
\label{bte}
\end{figure}

%\section*{Data availability}
%The data that support the findings of this article are openly available~\footnote{give me DOI}.
% see https://journals.aps.org/authors/data-availability-statements#citation

\bibliography{refs.bib}
%\bibliographystyle{ieeetr}


\end{document}

\documentclass[%
%reprint,
superscriptaddress,
%groupedaddress,
longbibliography,
%unsortedaddress,
%runinaddress,
%frontmatterverbose, 
%preprint,
%preprintnumbers,
%nofootinbib,
nobibnotes,
%bibnotes,
amsmath,amssymb,
aps,
%pra,
prb,
%rmp,
%prstab,
%prstper,
%showkeys,
floatfix,
twocolumn
]{revtex4-2}

\usepackage{graphicx}% Include figure files
\usepackage{calc}% Calculate margins
\usepackage{dcolumn}% Align table columns on decimal point
\usepackage{bm}% bold math

\usepackage[urlcolor=blue,colorlinks=true,citecolor=blue,linkcolor=blue,pdfstartview={FitH},bookmarks=false]{hyperref} % add hypertext capabilities

%\usepackage[mathlines]{lineno} % Enable numbering of text and display math
% \linenumbers\relax % Commence numbering lines

% \usepackage[showframe,%Uncomment any one of the following lines to test 
% %scale=0.7, marginratio={1:1, 2:3}, ignoreall,% default settings
% %text={7in,10in},centering,
% %margin=1.5in,
% % total={6.5in,8.75in}, top=1.2in, left=0.9in, includefoot,
% % height=10in,a5paper,hmargin={3cm,0.8in},
% ]{geometry}

\usepackage{amsmath}
\usepackage{amssymb}
%\usepackage{orcidlink}
\usepackage{xcolor}
%\usepackage{datetime}
\usepackage[normalem]{ulem}

% Change tracking commands
\newcommand{\trackchange}[3]{\textcolor{#3}{\sout{#1}#2}}  % Full color strikeout, insert
%\renewcommand{\trackchange}[3]{\textcolor{#3}{#2}}        % Just color silent remove and insert
%\renewcommand{\trackchange}[3]{{#2}}                      % No indication, silent remove and insert

% Author marker definitions

\definecolor{myblue}{RGB}{0,127,85}
% \definecolor{violet}{RGB}{102,0,204}
% \definecolor{orange}{RGB}{255,128,0}
% \definecolor{green}{RGB}{0,128,0}
\newcommand{\DL}[1]{\trackchange{}{#1}{blue}}
\newcommand{\AP}[1]{\trackchange{}{#1}{red}}
\newcommand{\PJ}[2]{\trackchange{#1}{#2}{orange}}
\newcommand{\JL}[2]{\trackchange{#1}{#2}{myblue}}
\newcommand{\SP}[2]{\trackchange{#1}{#2}{blue}}
\newcommand{\PP}[2]{\trackchange{#1}{#2}{teal}}
\newcommand{\AI}[2]{\trackchange{#1}{#2}{olive}}
\newcommand{\MS}[1]{\trackchange{}{#1}{purple}}

\newcommand{\TODO}[1]{\textcolor{red}{TODO: #1}}

\sloppy

\begin{document}

\title{Ab initio study of the anharmonic properties and thermal conductivity in $\beta$-FeSi$_2$}

\author{Svitlana~Pastukh}
\email[e-mail: ]{svitlana.pastukh@ifj.edu.pl}
\affiliation{Institute of Nuclear Physics, Polish Academy of Sciences, ul. W. E. Radzikowskiego 152, 31-342 Krak\'{o}w, Poland}

\author{Ma\l{}gorzata~Sternik}
\affiliation{Institute of Nuclear Physics, Polish Academy of Sciences, ul. W. E. Radzikowskiego 152, 31-342 Krak\'{o}w, Poland}

\author{Pawe\l{}~T.~Jochym}
\affiliation{Institute of Nuclear Physics, Polish Academy of Sciences, ul. W. E. Radzikowskiego 152, 31-342 Krak\'{o}w, Poland}


\author{Jan~\L{}a\.{z}ewski}
\affiliation{Institute of Nuclear Physics, Polish Academy of Sciences, ul. W. E. Radzikowskiego 152, 31-342 Krak\'{o}w, Poland}

\author{Andrzej~Ptok}
\affiliation{Institute of Nuclear Physics, Polish Academy of Sciences, ul. W. E. Radzikowskiego 152, 31-342 Krak\'{o}w, Poland}

\author{Svetoslav~Stankov}
\affiliation{Institute for Photon Science and Synchrotron Radiation, Karlsruhe Institute of Technology, D-76131 Karlsruhe, Germany}
\affiliation{Laboratory for Applications of Synchrotron Radiation, Karlsruhe Institute of Technology, D-76131 Karlsruhe, Germany}

\author{Przemys\l{}aw~Piekarz}
\affiliation{Institute of Nuclear Physics, Polish Academy of Sciences, ul. W. E. Radzikowskiego 152, 31-342 Krak\'{o}w, Poland}

\date{\today}

\begin{abstract}

Iron silicides are good candidates for applications in  optoelectronic and thermoelectric devices.
Lattice dynamical properties and thermal conductivity in the $\beta$-FeSi$_2$ semiconductor
are investigated with the first-principles computational methods. 
Phonon dispersion relations are calculated via the
temperature-dependent effective potential method and self-consistent phonon theory. 
To properly model thermal transport, we explicitly consider
the impact of phonon-phonon interactions by analyzing
anharmonic contributions to the phonon self-energy. 
This yields temperature-dependent phonon frequencies and linewidths,
reflecting the finite lifetime of phonons due to scattering
processes. The calculated phonon frequencies and line profiles are used to obtain 
the Raman spectra, which shows good agreement with the experimental data. 
We revealed an enhanced anharmonic behaviour of the Raman modes with the highest frequencies.  
The lattice thermal conductivity is then obtained as a function of temperature and crystallite size within
the relaxation-time approximation.
Phonon transport shows a small anisotropy due to the orthorhombic structure and a very weak dependence
on the quartic anharmonic corrections. The results obtained for an infinite material and for several crystallite sizes
were analyzed and compared with the available experimental data.
\end{abstract}

\maketitle


\section{Introduction}

The comprehensive determination of important physical properties of
crystals, such as thermal expansion, lattice thermal
conductivity or structural phase transitions, requires a fundamental 
understanding of the anharmonic effects.
Although the investigation of anharmonic interactions in crystals has
attracted a considerable interest for decades~\cite{cowley_1968}, a substantial progress
has only recently been achieved thanks to advances in theoretical and
numerical methods and increased computational power.
Now, phonon frequencies, lifetimes, and heat transfer in a wide range of
materials
can be quantitatively predicted using the available computational resources
based on the density functional theory (DFT)~\cite{lindsay_2013,mcgaughey_2019,lindsay_2019}.
In the case of strongly anharmonic systems, the self-consistent phonon
(SCPH) theory~\cite{tadano_2015} as well as the perturbative approach~\cite{tadano_2018}, using higher
order interatomic force constants derived from the fitting to the displacement force data obtained
with DFT, have proven to be successful.

Transition-metal silicides are promising materials for fabrication of electronic components
designed for integration with silicon-based circuits~\cite{murarka_1995}.
At room temperature, iron disilicide ($\beta$-FeSi$_2$) is a
direct-bandgap semiconductor~\cite{bost_1985}, making this material a good
candidate for application in optoelectronic devices such as infrared
detectors or light emitters~\cite{bost_1988}. The development of light-emitting diodes utilizing FeSi$_2$/Si heterostructures has been successfully demonstrated~\cite{leong_1997,suemasu_2001}. Due to
a high thermal stability and strong light absorption, FeSi$_2$ is
also a suitable photovoltaic material~\cite{powalla_1993,liu_2006,okuhara_2017}.

$\beta$-FeSi$_2$ crystallizes in the base-centered orthorhombic
lattice~\cite{dusausoy_1971} transforming to the
tetragonal metallic $\alpha$-FeSi$_{2}$ phase around $1200$~K~\cite{starke_2002}.
Optical studies indicated a direct band gap of the values
$0.85$--$0.89$~eV~\cite{bost_1988,dimitriadis_1990,arushanov_1995,wan_2003},
however, the {\it ab initio} calculations predicted a smaller indirect gap close to 0.8 eV~\cite{christensen_1990}. 
The existence of such an indirect gap was then confirmed by the optical linear transmittance measurements at
low temperatures~\cite{giannini_1992}. As shown by first-principles studies,
the character of the band gap is very sensitive to the orientation 
of a crystal grown on silicon~\cite{clark_1998}.

$\beta$-FeSi$_2$ belongs also to good thermoelectric materials~\cite{ware_1964}, with potential applications resulting from its chemical stability up to high temperatures, nontoxicity, and low cost of preparation~\cite{yamada_2012,nozariasbmarz_2017}. 
It has already been implemented in cars~\cite{birkholz_1988} and portable power sources~\cite{uemura_1989}. 
Its thermoelectric performance can be improved by doping~\cite{ito_2001,tani_2001,kim_2003,chen_2005,pandey_2013,le_tonquesse_2019}, 
which enhances the electric transport and reduces the thermal conductivity~\cite{waldecker_1973,du_2019,du_2020}.  
The thermal conductivity can be also reduced by the modification of microstructure~\cite{ail_2015} or by
nanostructurization~\cite{watanabe_2017,taniguchi_2017,hsin_2017,abbassi_2021}.

The lattice thermal conductivity is directly connected with anharmonic effects and phonon scattering processes.
The vibrational properties of \mbox{$\beta$-FeSi$_2$} were studied by the infrared and Raman spectroscopy~\cite{lefki_1991,guizzetti_1997,maeda_2004,baleva_2008,liu_2011,maeda_2011}. The observed anisotropy in the phonon spectra results from the enhanced sensitivity of the infrared and Raman features to the local lattice distortions~\cite{guizzetti_1997}. The Fe phonon density of states was measured by nuclear inelastic scattering (NIS), showing a good agreement with the density functional theory (DFT) calculations~\cite{walterfang_2005}. Using the DFT approach, the phonon dispersion curves, phonon density of states, as well as various thermodynamic properties were obtained within the harmonic approximation~\cite{tani_2010,liang_2011}. The extended Klemens model was applied to study
the anharmonic effect on phonon frequencies and linewidths observed by the Raman spectroscopy~\cite{zhang_2023}.
The impact of nanostructurization on lattice dynamics was explored in the $\beta$-FeSi$_2$ nanorods grown on the Si(110) surface by the NIS and {\it ab initio} methods~\cite{kalt_2022}.

In this work, we investigate the lattice dynamical properties of $\beta$-FeSi$_2$ 
using the DFT calculations. We study the effect of anharmonic terms in the temperature-dependent potential on phonon frequencies and lifetimes. We focus on the Raman modes, comparing the theoretical results with the experimental data.
The thermal conductivity is derived in a broad temperature range and the effect of crystallite size is analyzed.



This study is structured as follows.
In Sec.~\ref{sec.com} we describe the details of computational methods.
Next, in Sec.~\ref{sec.result} we present and discuss our results.
In particular we present the crystal structure (Sec.~\ref{sec.crys}) and lattice dynamics (Sec.~\ref{sec.lattice}).
We investigate also the thermal conductivity comparing the obtained results with the available experimental data (Sec.~\ref{sec.thermal}).
Finally, Sec.~\ref{sec.summary} summarizes our key findings and conclusions.

\section{Calculation method}
\label{sec.com}


The calculations were performed using the projector augmented-wave potentials~\cite{blochl_1994} and the generalized gradient approximation~\cite{perdew_1996} implemented in the Vienna Ab initio Simulation Package (VASP)~\cite{kresse.hafner.94,kresse.furthmuller.96,kresse.joubert.99}. 
The lattice parameters and atomic positions were optimized in the ${\bm a} \times ({\bm b}-{\bm c}) \times ({\bm b}+{\bm c})$ supercell containing 32 formula units and four primitive cells.
The integration in the reciprocal space was conducted using the $2 \times 2 \times 2$ Monkhorst--Pack mesh~\cite{monkhorst_1976} and the cut-off energy was set to $500$~eV. For convergence conditions, we set the energy change below $10^{-5}$ and $10^{-8}$ for the ionic and electronic loops, respectively. 

The lattice dynamical properties were studied within the temperature-dependent effective potential (TDEP) approach~\cite{hellman_2013}. The atomic potential with the third and fourth order anharmonic terms was derived from interatomic forces induced by displacements of all atoms at finite temperatures.
The sets of atomic displacements were generated by the high efficiency configuration space sampling (HECSS)~\cite{jochym_2021} and forces were obtained by VASP. The interatomic force constants and phonon frequencies were calculated with the {\sc Alamode} software~\cite{tadano_2014}.

Furthermore, we have attempted to construct a {\emph{temperature independent}} 
anharmonic model. We have used combined data from all investigated temperatures 
(300, 600 and 1000~K) and fitted a large (over 15~000 free parameters), fourth-order 
interaction model to this dataset. 
Subsequently, we have used this model to calculate line profiles and positions of Raman-active modes
at multiple temperatures.

The changes in phonon frequencies induced by the anharmonic effects were investigated within two approaches.
First, the impact of the quartic anharmonic terms was included using the SCPH theory~\cite{tadano_2015}.
Second, the mode profiles (frequency shifts and line widths) were determined from the real and imaginary parts of the phonon self-energy resulting from the cubic and quartic anharmonic terms of the above mentioned large model~\cite{tadano_2018}. 
The longitudinal optic-transverse optic (LO-TO) splitting was also evaluated, using the static dielectric tensor and Born effective charges calculated within density functional perturbation theory~\cite{gajdos_2006}.

\begin{figure}[]
    \centering
    \includegraphics[width=\linewidth]{fig1_new.png}
\caption{%
(a) The conventional unit cell of $\beta$-FeSi$_2$ (with Cmca symmetry) and (b) the corresponding Brillouin zone with selected high-symmetry points.
}
  \label{fig.struct}
\end{figure}

% To further characterize the vibrational properties, Raman-active scattering was investigated using the Phonopy-Spectroscopy package~\cite{skelton_2017}. This enabled the identification of Raman-active modes and the calculation of Raman tensors. Anharmonic force constants, obtained from calculations using {\sc Alamode}, were then used to obtain theoretical line profiles for the Raman modes. The presented Raman scattering spectra combine these anharmonic line profiles with the Raman tensor amplitudes.
% This analysis was also based on the large quartic model mentioned above.

% Finally, the thermal conductivity was obtained as a function of temperature and crystallite size within the relaxation-time approximation (RTA) \PJ{}{as implemented in {\sc Alamode}\cite{tadano_2014}}. The phonon lifetimes were calculated from the phonon self-energy including the cubic and quartic anharmonic terms. \SP{}{The RTA provides a solution of the Boltzmann transport equation (BTE) under the assumption that scattering events are independent and can be treated through mode-resolved relaxation times.
% To verify the validity of this approximation for $\beta$-FeSi$_2$, we have additionally
% executed an iterative solution of the BTE.}\PJ{}{These calculations were performed with
% {\sc Phono3py}\cite{togo_2023}. For cross-validation, the additional RTA calculations were performed on the same $\bm{q}$-grid as BTE with implementation provided by {\sc Phono3py}.}.

To further characterize the vibrational properties, Raman scattering was investigated using the Phonopy-Spectroscopy package~\cite{skelton_2017}. This enabled the identification of Raman-active modes and the calculation of Raman tensors. Anharmonic force constants, derived from calculations using {\sc Alamode}, were then used to obtain theoretical line profiles for the Raman modes. The presented Raman scattering spectra combine these anharmonic line profiles with the Raman tensor amplitudes.
This analysis was also based on the large quartic model described above.

Finally, the thermal conductivity was calculated as a function of temperature and crystallite size within the relaxation-time approximation (RTA) \PJ{}{as implemented in {\sc Alamode}~\cite{tadano_2014}}. The phonon lifetimes were calculated from the phonon self-energy including the cubic and quartic anharmonic terms. \SP{}{The RTA provides a solution to the Boltzmann transport equation (BTE) under the assumption that scattering events are independent and can be treated through mode-resolved relaxation times.
To verify the validity of this approximation for $\beta$-FeSi$_2$, we additionally
solved the BTE iteratively.} \PJ{}{These calculations were performed with
{\sc Phono3py}~\cite{togo_2023}. For cross-validation, the additional RTA calculations were performed on the same $\bm{q}$-grid as the BTE calculations, using the implementation provided by {\sc Phono3py}.}

% (see. Appendix~\ref{ThermalAB} for the comparison)\cite{togo_2023}.}
%As shown in Appendix~\ref{ThermalAB}, the iterative BTE results exhibit good agreement with the RTA values, confirming that RTA is sufficiently accurate for this material and for the considered temperature range.}


\section{Results}
\label{sec.result}

\subsection{Crystal structure}
\label{sec.crys}

The $\beta$-FeSi$_2$ structure adopts a base-centered orthorhombic lattice with the space group Cmca (No.~64) as shown in Fig.~\ref{fig.struct}(a).
The unit cell consists of two primitive cells and contains 48 atoms.
Iron (silicon) atoms possess two nonequivalent positions: \mbox{Fe-I} and \mbox{Fe-II} (\mbox{Si-I} and \mbox{Si-II}), presented in Fig.~\ref{fig.struct}(a) as gray and purple (orange and yellow) spheres, respectively.
This crystal structure is derived from the fluorite-type lattice with strongly distorted Si cubes and Fe atoms occupying 
one-half of the central sites.
The Fe-I and Fe-II sites create different layers perpendicular to the $x$ direction, and they are separated by layers containing both Si sites.
Each Fe atom is coordinated by 8 Si atoms with slightly different Fe-Si distances.
The optimized lattice constants ($a = 9.874$~{\AA}, $b = 7.767$~{\AA}, and
$c = 7.811$~{\AA}) agree very well with the experimental data ($a = 9.863$~{\AA}, $b = 7.791$~{\AA}, and $c = 7.833$~{\AA})~\cite{dusausoy_1971}.

Iron atoms occupy the Wyckoff sites 8\textit{d} ($0.2166$, $0$, $0$) and  8\textit{f} ($0$, $0.3072$, $0.1879$), corresponding to \mbox{Fe-I} and \mbox{Fe-II}, respectively. 
Silicon atoms are located at two inequivalent 16\textit{g} positions: ($0.1282$, $0.2737$, $0.0495$) and \mbox{($0.3734$, $0.0445$, $0.2270$)}, assigned as \mbox{Si-I} and \mbox{Si-II}.
The optimized positions of atoms agree very well with the experimental data~\cite{dusausoy_1971} and the previous theoretical studies~\cite{tani_2010,liang_2011}.


\subsection{Lattice dynamics}
\label{sec.lattice}

\begin{figure}[]
    \centering
    \includegraphics[width=\linewidth]{Fig1c.pdf}
\caption{%
The phonon dispersion curves along high symmetry directions obtained within SCPH (for temperatures from $0$ to $1000$~K).
Dashed black lines indicate the phonon dispersions obtained from the harmonic approximation. The white dots indicate Raman-active modes with A$_g$ symmetry.
The vertical plot shows the phonon density of states (DOS) calculated at a reference temperature of $600$~K.
}
  \label{fig.ph_band}
\end{figure}


In Fig.~\ref{fig.ph_band} we present the phonon dispersion relations of $\beta$-FeSi$_2$ along high-symmetry directions in the Brillouin zone [Fig.~\ref{fig.struct}(b)].
Due to 24 atoms in the primitive cell, the phonon spectrum consists of 69 optical modes and three acoustic modes.
The phonon dispersions were calculated within the SCPH approach in the temperature range $0$--$1000$~K (presented by color lines in Fig.~\ref{fig.ph_band}), and they are compared with the results obtained from the harmonic part of the effective potential corresponding to temperature $T=300$~K (indicated by dashed black lines in Fig.~\ref{fig.ph_band}).
As we can see within the SCPH method, the anharmonic effects are rather weak and leads only to small renormalization of phonon frequencies.
Only the highest modes show more pronounced shifts of their frequencies to larger values.
The total and partial element-projected phonon density of states obtained within the harmonic approximation are presented in Fig.~\ref{fig.ph_band}.
Up to around $320$~cm$^{-1}$, the contributions from both elements are very similar, while for higher frequencies the spectrum is dominated by
the Si vibrations.


\begin{table}[!t]
\begin{ruledtabular}
\caption{Calculated and experimental Raman-active modes of $\beta$-FeSi$_2$ with their irreducible representations (IR). Present theoretical results are compared with the previous theoretical data from Ref.~\cite{tani_2010} and experimental results from \mbox{Refs.~\cite{lefki_1991,maeda_2004}.} The experimental frequencies with known symmetries (A$_g$) are shown in bold, while other experimental modes are assigned to the best fitting theoretical values.}
\begin{tabular}{c c c c c}
\textbf{IR} & \multicolumn{4}{c}{\textbf{Frequency (cm$^{-1}$)}} \\
 & Present  & Theor.~\cite{tani_2010} & Exp.~\cite{lefki_1991} & Exp.~\cite{maeda_2004} \\
\hline
B$_{2g}$ & 175.4 & 179 & 176 &  \\
B$_{1g}$ & 176.3 & 185 & 179  &  \\
B$_{1g}$ & 193.3 & 198 &  & 190.6 \\
A$_g$    & 196.6 & 208 & \textbf{195} & \textbf{194.0} \\
A$_g$    & 203.9 & 210 & \textbf{197} & 199.6 \\
B$_{3g}$ & 205.5 & 212 & 200 &  \\
B$_{3g}$ & 226.3 & 236 & 206 & 227.1 \\
B$_{1g}$ & 233.3 & 240 &  &  231.6 \\
B$_{2g}$ & 248.6 & 254 &  &  \\
A$_g$    & 250.1 & 257 & \textbf{247} & \textbf{247.3} \\
A$_g$    & 254.9 & 264 & \textbf{253} & 254.3 \\
B$_{1g}$ & 275.5 & 285 &  &  274.1 \\
B$_{2g}$ & 282.4 & 295 &  &  281.2 \\
B$_{3g}$ & 286.8 & 297 &  &  \\
B$_{1g}$ & 307.1 & 317 &  &  \\
B$_{2g}$ & 312.6 & 326 &  &  311.8 \\
B$_{1g}$ & 319.0 & 324 &  &  \\
B$_{3g}$ & 327.9 & 341 &  & 325.8 \\
B$_{2g}$ & 333.3 & 345 &  &  \\
A$_g$    & 339.3 & 352 & \textbf{346} & 339.5 \\
B$_{2g}$ & 343.0 & 350 &  &  \\
B$_{1g}$ & 353.5 & 366 &  &  \\
B$_{2g}$ & 372.8 & 383 &  & 370.7 \\
B$_{1g}$ & 375.1 & 385 &  &  \\
B$_{3g}$ & 375.2 & 386 &  &  \\
B$_{3g}$ & 385.3 & 401 &  &  \\
A$_g$    & 386.5 & 398 &  & 386.2 \\
B$_{2g}$ & 387.7 & 402 &  & 388.2 \\
A$_g$    & 404.6 & 415 &  & 400.4 \\
B$_{2g}$ & 405.1 & 420 &  &  \\
B$_{1g}$ & 412.7 & 428 &  &  \\
B$_{3g}$ & 418.4 & 431 &  &  \\
B$_{3g}$ & 441.4 & 458 &  & 442.6 \\
B$_{3g}$ & 447.3 & 466 &  & 446.3 \\
A$_g$    & 448.4 & 464 &  &  \\
A$_g$    & 499.1 & 517 &  &  \\
\end{tabular}
\label{T1Raman}
\end{ruledtabular}
\end{table}

%subsection{Raman}

\begin{figure*}
\centering
  \includegraphics[width=\textwidth]{spectrum_RTA+raman.pdf}
\caption{Raman spectrum of $\beta$-FeSi$_2$ calculated within the perturbative approach at three temperatures 
($300$, $600$, $900$~K -- blue, orange, and red lines, respectively). 
The calculated spectrum includes all Raman-active modes. The A$_g$ modes
are indicated by purple vertical lines at peak positions corresponding to T=300~K. 
The frequencies derived from the harmonic approximation at T=300~K are indicated by green lines.
Connecting arrows indicate the correspondence between harmonic and
anharmonic frequencies, demonstrating the frequency shifts due to 
phonon interactions. Experimental values for the A$_g$ modes
based on Ref.~\cite{lefki_1991} are marked with black dashed lines.
}
\label{raman}
\end{figure*}


The phonon spectrum at the $\Gamma$ point consists of 36 Raman modes classified according to the irreducible representations: $9A_\text{g}+9B_\text{1g}+9B_\text{2g}+9B_\text{3g}$. 
Fig.~\ref{raman} shows the Raman spectrum of A$_g$ symmetry calculated for $\beta$-FeSi$_2$ within the perturbative approach at three temperatures $300$, $600$, and $900$~K (solid blue, orange, and red curves, respectively), including third-order and fourth-order anharmonic corrections.
The calculated Raman spectrum includes all Raman-active modes.
The five experimentally observed A$_g$ modes are highlighted by black dashed lines, based on data from Ref.~\cite{lefki_1991}.
Additional peaks not marked with vertical lines correspond to phonon modes with symmetries other than A$_g$.
The frequencies of Raman modes obtained in anharmonic calculations are compared with the previous results calculated within the harmonic approximation and the experimental values in Tab.~\ref{T1Raman}. 
We have marked in bold the experimentally determined A$_g$ modes, which are compared with the calculations.
Since the experimental studies did not provide the accurate assignement of the Raman modes with the B$_{1g}$, B$_{2g}$, and B$_{3g}$ symmetry~\cite{lefki_1991,maeda_2004}, we cannot compare them directly with the theoretical results.
However, in Tab.~\ref{T1Raman} we have assigned the measured frequencies to the best fitting theoretical values without taking into account the symmetry of the modes, except for the known A$_{g}$ modes. 


The impact of anharmonicity on the phonon frequencies is well visible from the comparison of the results obtained within the harmonic approximation 
and from the anharmonic calculation (vertical green and purple lines in Fig.~\ref{raman}, respectively).
Here we show only the A$_g$ modes, which are compared with the experimental results (vertical black dashed lines).
Anharmonic frequencies calculated at $300$~K are indicated by purple lines, while frequencies derived from harmonic approximation are marked with green lines. The grey solid lines connect corresponding modes obtained in both approximations.
In most cases, the results obtained within the harmonic approximation do not agree with the experimental frequencies. 
%Only the modes close to $250$~cm$^{-1}$ and $400$~cm$^{-1}$ correspond well to the experimental values. MS
As we can see, the inclusion of the anharmonic correction leads to a significant modification of the phonon frequencies.
These anharmonic effects are stronger for higher-frequency modes mainly because of
the dominant contribution from Si atoms, which vibrate with larger amplitudes than heavier Fe atoms. 
When atoms move to larger distances the potential deviates more from the harmonic approximation,
and the anharmonic corrections become stronger.


The modification of phonon frequencies observed in Fig.~\ref{raman} is much larger than in the SCPH scheme presented in Fig.~\ref{fig.ph_band}. The SCPH approach includes only the leading-order contribution to 
the phonon self-energy obtained from the quartic anharmonic terms~\cite{tadano_2015}. Therefore, it does not describe fully the changes of phonon frequencies found within the perturbation theory (see Fig.~\ref{raman}).
Especially, it is well visible for two highest A$_g$ modes, which exhibit also the largest line broadening 
and the strongest dependence on temperature.
Therefore, a better agreement with experimentally observed frequencies is visible,
confirming the significant influence of the anharmonicity on the frequencies and line profiles of phonon modes.
In fact, the decrease of phonon frequencies should be even stronger due to thermal expansion, 
which is not included in our calculations.
Within SCPH the frequencies of the highest modes increase with increasing temperature as we see in Fig.~\ref{fig.ph_band}. The comparison of two different approaches applied to study anharmonic properties of $\beta$-FeSi$_2$ shows that the perturbation theory, which includes the cubic and quartic terms, better describes the changes of phonon frequencies with temperature than the SCPH method. \PJ{}{This indicates that the leading-order contribution included in SCPH are not important in this material.}

Additionaly we should nottice that for other than A$_g$ modes we cannot make an unambiguous assignment of theoretical frequencies to experimental ones. Note that the spectrum in Fig.~\ref{fig.ph_band} contains all Raman-active modes. 
The limitation to A$_g$ modes concerns only the indicated positions of the peaks. 
%\SP{}{Furthermore, we present the full spectrum for all Raman active modes with the comparison of the harmonic and anharmonic calculations in the Appendix~\ref{RamanAA}.}
\textcolor{red}{The full Raman spectrum, with the frequencies of the B$_{1g}$, B$_{2g}$ and B$_{3g}$ modes marked, is shown in Appendix A, Fig.~\ref{raman1}. This represents our theoretical prediction of possible Raman-mode assignments, which can be verified in future experiments.}
%This is the theoretical prediction of possible assignment of Raman modes that can be verified in future experiments.  


\subsection{Thermal conductivity}
\label{sec.thermal}


\begin{figure*}[!t]
\centering
  \includegraphics[width=\linewidth]{time3.pdf}
\caption{Phonon lifetimes calculated for three temperatures as a function of phonon frequency. The colors correspond to the phonon branches.}
  \label{thermtime}
\end{figure*}


In this section, we analyze the thermal conductivity tensor of $\beta$-FeSi$_2$ obtained within the RTA approach~\cite{tadano_2018} as a function of temperature
%
\begin{equation}
\kappa_{\text{ph}}^{\mu\nu}(T) = \frac{1}{NV} \sum_{\bm{q},j} c_{\bm{q}j}(T) v_{\bm{q}j}^{\mu} v_{\bm{q}j}^{\nu}\tau_{\bm{q}j}(T),
\end{equation}
% 
where $c_{\bm{q}j}$ is the mode heat capacity and $v_{\bm{q}j}$ is the mode group velocity. 
The relaxation time is approximated by the phonon lifetime $\tau_{\bm{q}j}$
calculated for $j$-th branch at the wave vector $\bm{q}$.
$V$ is the unit cell volume and $N$ is the number of unit cells in the crystal.
The phonon lifetime is calculated using this formula
%
\begin{equation}
\tau_{\bm{q}j}(T)=\frac{1}{2\Gamma_{\bm{q}j}^{\text{anh}}(T)},
\end{equation}
%
where $\Gamma_{\bm{q}j}^{\text{anh}}$ is the anharmonic phonon linewidth obtained from 
the imaginary part of the phonon self-energy within the perturbation theory.

In Fig.~\ref{thermtime}, we present $\tau_{\bm{q}j}$ obtained for three temperatures 300, 600, and 1000~K as a function of frequency. As we see, the acoustic phonons close to the $\Gamma$ point have the longest lifetimes,
which are diminished with increasing frequency reaching local minima around $200$~cm$^{-1}$.
For higher frequencies, phonon lifetimes first increase to local maxima around $300$~cm$^{-1}$ and then decrease to
the lowest values in the range of highest optical modes. The shortest lifetimes correspond to the largest line broadening
observed for the Raman modes in Fig~\ref{raman}. The phonon group velocities 
$v_{\bm{q}j}=\partial\omega_{\bm{q}j}/\partial\bm{q}$, which are obtained by the central difference formula, are presented in Fig.~\ref{thermvelocity}. Their temperature dependence is negligible, therefore, we present only the results for $T=600$~K.
At low frequencies, there are clearly two ranges of group velocities of the acoustic phonons. 
The larger values correspond to the longitudinal modes, while the lower values are obtained from the
transverse acoustic branches. Group velocities of acoustic phonons decrease for larger frequencies
and reach the average values typical for optic branches.

\begin{figure}[!t]
\centering
  \includegraphics[width=\linewidth]{GV.pdf}
\caption{Mode group velocities calculated as a function of phonon frequency. The colors correspond to the phonon branches.
}
  \label{thermvelocity}
\end{figure}

In Fig.~\ref{anizotropy}(a), we present the three diagonal elements of $\kappa_{\text{ph}}^{\mu\nu}$ corresponding to the main directions of the crystal structure. 
They were obtained from the force constants calculated at the base temperature $T=600$~K and the crystallite size $0.1$~$\mu$m to account for boundary-limited phonon transport. 
Due to the orthorhombic symmetry, we observe a small anisotropy in phonon transport in the whole temperature range. 
At low temperatures, the three components of the heat conductivity increase in a very similar way with the $\kappa_{\text{ph}}^{yy}$ element slightly larger than two other components. 
After reaching the maximum, we observe a change in the largest component from $\kappa_{\text{ph}}^{yy}$ to 
$\kappa_{\text{ph}}^{xx}$.    
In Fig.~\ref{anizotropy}(b), the thermal conductivity is shown for three base temperatures, at which the interatomic potential was obtained ($300$~K, $600$~K, and $1000$~K), using the energy expansion up to third- and fourth-order anharmonic terms, and the same structure size of $0.1$~$\mu$m. At lowest temperatures, the thermal conductivity strongly increases, reaching the maximum around $T=180$~K, then it shows a slower decrease with temperature.
The differences between the two levels of approximation are minimal, suggesting that third-order calculations already capture the dominant phonon scattering mechanisms. The dependence on the base temperature is also very weak, showing the changes in the heat conductivity within a few percent. 
\SP{}{Further verification of the reliability of our thermal conductivity results is given in Appendix~\ref{ThermalAB}, where the full BTE calculations show good agreement with RTA and higher-resolution RTA results, demonstrating that the $8\times8\times8$ $\bm{q}$-mesh already provides converged values.}

\begin{figure}[!t]
\centering
  \includegraphics[width=\linewidth]{Anizotropy.pdf}
\caption{(a) The anisotropic thermal conductivity of $\beta$-FeSi$_2$ calculated along the lattice directions at 600~K. (b) The average temperature-dependent thermal conductivity taken at $300$~K, $600$~K, and $1000$~K, including anharmonic corrections up to cubic (A3) and quartic (A4) terms. In both cases the crystallite size is 0.1~$\mu$m.}
  \label{anizotropy}
\end{figure}


In Fig.~\ref{therm}, we fix the base temperature at $600$~K and examine the effect of crystallite size on thermal conductivity, varying it from $0.01$ to $0.5$~$\mu$m.
With decreasing the crystallite size, we observe a shift of the position of the maximum to larger temperatures and a decrease of the thermal conductivity in the entire temperature range.
Theoretical results are compared with several experimental data obtained above the room temperature. 
The measured thermal conductivity depends to a large extent on the sample quality, its purity and the size of the crystalline grains which depends on 
the production processes.
Many measurements were performed using crystallites of micrometric or unknown size ~\cite{waldecker_1973,ito_2002,kim_2003,du_2020}, however, 
numerous attempts to minimize $\kappa$ by reducing grain sizes to $56$~nm~\cite{dabrowski_2019, dabrowski_microstructure_2021}, $30$-$400$~nm~\cite{le_tonquesse_2019}, $50$ and $200$~nm~\cite{abbassi_2021}, or introducing pores into the material~\cite{sam2023improved} 
are also carried out. 
Another way to change the thermal conductivity is to dope $\beta$-FeSi$_2$ with different elements~\cite{ito_2002,kim_2003,du_2020,cheng_2024}, however, this effect is beyond our investigation.

\begin{figure}[!t]
\centering
  \includegraphics[width=\linewidth]{Thermal_conductivity3.pdf}
\caption{
The phonon thermal conductivity of $\beta$-FeSi$_2$: theoretical results for the infinite crystalline size and with boundary conditions, compared with experimental data for different structure sizes.
}
  \label{therm}
\end{figure}
%Fig.~\ref{therm}\cite{abbassi_2021}\cite{waldecker_1973}\cite{dabrowski_2019}\cite{sam2023improved}\cite{kim_2003}\cite{du_2020}\cite{ito_2002}\cite{le_tonquesse_2019}.

We observe a decrease in the thermal conductivity with reducing crystalline grain sizes in all analyzed experimental data. 
For instance, by decreasing the crystallite size to $50$~nm, the thermal conductivity at room temperature was reduced by a factor of $1.7$, what can be  compared to the annealed sample with 200 nm grains~\cite{abbassi_2021}. 
It is worth noting that the rate of decrease in value with increasing temperature in both cases, for grain sizes of $50$~nm and $200$~nm, is significantly different, which is consistent with our calculations. 
The same trend can be observed by comparing the thermal conductivity measured for a sample with bulk crystallite sizes with the thermal conductivity of a sample with grains smaller than 400 nm~\cite{le_tonquesse_2019}.
The theoretical results obtained for the same crystallite size show higher values due to factors not captured in the idealized model, such as crystal imperfection or mechanical strain. Usually, a decrease in the crystallite size is related to an increased concentration of grain boundaries, point defects, and stacking faults that influence the phonon scattering~\cite{le_tonquesse_2019,abbassi_2021}.   

We should note that the total thermal conductivity is a combination
of the lattice and electronic contributions to the heat transport.
In semiconductors, the electronic thermal conductivity is negligible at low temperatures and significantly increases 
only much above the room temperatures~\cite{gu_2020}.
For $\beta$-FeSi$_2$, the electronic thermal conductivity was obtained from the electric conductivity using the Wiedemann-Franz law~\cite{ito_2002,kim_2003,le_tonquesse_2019}.
In the undoped material, its value does not exceed $0.1$~W/mK in the measurement up to $T=950$~K~\cite{kim_2003}.
By doping, the electronic thermal conductivity can be enhanced, and it has a direct impact on the thermoelectric properties of $\beta$-FeSi$_2$ at high temperatures~\cite{ito_2002,kim_2003}.
In the present study, we consider only the phonon contribution to the thermal conductivity,
therefore, agreement with experimental data may deteriorate with increasing temperature.

\section{Summary}
\label{sec.summary}

We performed {\it ab initio} studies on lattice dynamical and thermal transport properties of $\beta$-FeSi$_2$. The effect of anharmonicity was analyzed within two approaches -- the SCPH method and the perturbation theory.
The phonon dispersion curves obtained within SCPH show small renormalization of frequencies comparing to the harmonic approximation. 
The Raman spectra were calculated within the procedure which takes into account the peak intensities obtained from the Raman tensors and the line profiles obtained from the phonon self energy derived within the perturbation theory based on the large, temperature-independent, quartic model fitted to the data from the wide range of temperatures (300-1000~K). 
The anharmonic corrections strongly affect the frequencies and line profiles of some modes and results in overall better agreement with the experimental data. 
We analyzed the phonon lifetimes and group velocities obtained as functions of the phonon frequency.
Then the lattice thermal conductivity was calculated for a broad range of temperatures and grain sizes.
We found a small anisotropy in the phonon thermal transport resulting from the orthorhombic structure and a weak effect of the quartic anharmonic terms. 
The thermal conductivity calculated for various crystalline grain sizes show a good qualitative agreement with the available measurements.

\begin{acknowledgments}
Some figures in this work were rendered using {\sc Vesta}~\cite{momma.izumi.11} software.
This work was partially supported by the Ministry of Education, Youth and Sports of the Czech Republic through the e-INFRA CZ (ID:90254).
\end{acknowledgments}

\appendix

\section{Raman spectrum}
\label{RamanAA}

Based on the polarized Raman measurements reported in Ref.~\cite{maeda_2004}, two Raman peaks were identified as belonging to the A$_g$ symmetry class, and several additional peaks were observed with similar or different polarization dependence. Although the authors of Ref.~\cite{maeda_2004} provided estimates of the relative Raman tensor components, they did not specify which of the remaining modes correspond to the B$_{1g}$, B$_{2g}$, or B$_{3g}$ symmetries. Because of this missing experimental information, a direct symmetry-resolved comparison between the measurement and theory is not currently possible for the non-A$_g$ modes. 
To provide a complete theoretical picture of the B$_g$-type modes, we show here the calculated Raman-active frequencies and intensities for the B$_{1g}$, B$_{2g}$, and B$_{3g}$ symmetries only. 
%These results represent the predicted Raman modes for the B$_g$ symmetries in $\beta$-FeSi$_2$. 
\textcolor{red} {Fig.~\ref{raman1} shows the predicted Raman modes for the B$_g$ symmetries in $\beta$-FeSi$_2$. The results obtained within the harmonic approximation are compared with the anharmonic perturbation theory calculations which provides both, frequency shifts and predicted line profiles of the modes.}
Although the experimentally measured peaks cannot be directly assigned to these symmetries due to the lack of polarization-resolved data, the theoretical predictions provide a reference for comparison. Matching the measured frequencies to the closest theoretical B$_g$ modes (Table~\ref{T1Raman}) allows for a tentative assignment, which can guide future polarization-resolved Raman experiments aimed at determining the precise symmetry of the unresolved peaks.

\begin{figure*}[t]
\centering
  \includegraphics[width=\linewidth]{spectrum_RTA+raman_Bi_modes.pdf}
\caption{Raman spectrum of $\beta$-FeSi$_2$ calculated at three temperatures 
($300$, $600$, $900$~K -- blue, orange, and red lines, respectively). 
The calculated spectrum includes all Raman-active modes. The B$_{ig}$ modes
are indicated by purple vertical lines at peak positions corresponding to T=300~K. 
The frequencies derived from the harmonic approximation at T=300~K are indicated 
by green, yellow and pink lines.
Connecting arrows indicate the correspondence between harmonic and anharmonic 
frequencies, demonstrating the frequency shifts due to phonon interactions.}
\label{raman1}
\end{figure*}
\section{Thermal conductivity obtained from BTE and RTA}
\label{ThermalAB}

The thermal conductivity was computed by solving the full BTE on the largest feasible $\bm{q}$-point grid, $8\times8\times8$, and compared with the corresponding RTA results obtained on the same grid. As shown in Fig.~\ref{bte}, the difference between the components of the thermal conductivity tensor obtained within BTE and RTA at this resolution is very small, indicating a good agreement between these two approaches.
%THIS PART SHOULD GO RATHER TO THE RESPONSE
%Extending the full BTE calculation to larger grids is computationally prohibitive: the computational cost of BTE is roughly two orders of %magnitude higher than that of RTA, and the required memory and runtime exceed our available resources. 
Moreover, we performed an additional calculation using RTA on a denser $20\times20\times20$ grid. As seen in the Fig.~\ref{bte}, the higher-resolution data remain in a good agreement with both the BTE and RTA results for the $8\times8\times8$ grid.
It shows that the $8\times8\times8$ mesh already provides good results for this structure and confirms reliability of the calculations.

\begin{figure}[!h]
\centering
  \includegraphics[width=\linewidth]{BTEvsRTAvsANP.pdf}
\caption{Thermal conductivity of $\beta$-FeSi$_2$ obtained within the BTE and RTA methods using the Phono3py software on the $8\times8\times8$ q-point grid, compared with the RTA results computed with ALAMODE on a denser $20\times20\times20$ grid.}
\label{bte}
\end{figure}

%\section*{Data availability}
%The data that support the findings of this article are openly available~\footnote{give me DOI}.
% see https://journals.aps.org/authors/data-availability-statements#citation

\bibliography{refs.bib}
%\bibliographystyle{ieeetr}


\end{document}

\documentclass[%
%reprint,
superscriptaddress,
%groupedaddress,
longbibliography,
%unsortedaddress,
%runinaddress,
%frontmatterverbose, 
%preprint,
%preprintnumbers,
%nofootinbib,
nobibnotes,
%bibnotes,
amsmath,amssymb,
aps,
%pra,
prb,
%rmp,
%prstab,
%prstper,
%showkeys,
floatfix,
twocolumn
]{revtex4-2}

\usepackage{graphicx}% Include figure files
\usepackage{calc}% Calculate margins
\usepackage{dcolumn}% Align table columns on decimal point
\usepackage{bm}% bold math

\usepackage[urlcolor=blue,colorlinks=true,citecolor=blue,linkcolor=blue,pdfstartview={FitH},bookmarks=false]{hyperref} % add hypertext capabilities

%\usepackage[mathlines]{lineno} % Enable numbering of text and display math
% \linenumbers\relax % Commence numbering lines

% \usepackage[showframe,%Uncomment any one of the following lines to test 
% %scale=0.7, marginratio={1:1, 2:3}, ignoreall,% default settings
% %text={7in,10in},centering,
% %margin=1.5in,
% % total={6.5in,8.75in}, top=1.2in, left=0.9in, includefoot,
% % height=10in,a5paper,hmargin={3cm,0.8in},
% ]{geometry}

\usepackage{amsmath}
\usepackage{amssymb}
%\usepackage{orcidlink}
\usepackage{xcolor}
%\usepackage{datetime}
\usepackage[normalem]{ulem}

% Change tracking commands
\newcommand{\trackchange}[3]{\textcolor{#3}{\sout{#1}#2}}  % Full color strikeout, insert
%\renewcommand{\trackchange}[3]{\textcolor{#3}{#2}}        % Just color silent remove and insert
%\renewcommand{\trackchange}[3]{{#2}}                      % No indication, silent remove and insert

% Author marker definitions

\definecolor{myblue}{RGB}{0,127,85}
% \definecolor{violet}{RGB}{102,0,204}
% \definecolor{orange}{RGB}{255,128,0}
% \definecolor{green}{RGB}{0,128,0}
\newcommand{\DL}[1]{\trackchange{}{#1}{blue}}
\newcommand{\AP}[1]{\trackchange{}{#1}{red}}
\newcommand{\PJ}[2]{\trackchange{#1}{#2}{orange}}
\newcommand{\JL}[2]{\trackchange{#1}{#2}{myblue}}
\newcommand{\SP}[2]{\trackchange{#1}{#2}{blue}}
\newcommand{\PP}[2]{\trackchange{#1}{#2}{teal}}
\newcommand{\AI}[2]{\trackchange{#1}{#2}{olive}}
\newcommand{\MS}[1]{\trackchange{}{#1}{purple}}

\newcommand{\TODO}[1]{\textcolor{red}{TODO: #1}}

\sloppy

\begin{document}

\title{Ab initio study of the anharmonic properties and thermal conductivity in $\beta$-FeSi$_2$}

\author{Svitlana~Pastukh}
\email[e-mail: ]{svitlana.pastukh@ifj.edu.pl}
\affiliation{Institute of Nuclear Physics, Polish Academy of Sciences, ul. W. E. Radzikowskiego 152, 31-342 Krak\'{o}w, Poland}

\author{Ma\l{}gorzata~Sternik}
\affiliation{Institute of Nuclear Physics, Polish Academy of Sciences, ul. W. E. Radzikowskiego 152, 31-342 Krak\'{o}w, Poland}

\author{Pawe\l{}~T.~Jochym}
\affiliation{Institute of Nuclear Physics, Polish Academy of Sciences, ul. W. E. Radzikowskiego 152, 31-342 Krak\'{o}w, Poland}


\author{Jan~\L{}a\.{z}ewski}
\affiliation{Institute of Nuclear Physics, Polish Academy of Sciences, ul. W. E. Radzikowskiego 152, 31-342 Krak\'{o}w, Poland}

\author{Andrzej~Ptok}
\affiliation{Institute of Nuclear Physics, Polish Academy of Sciences, ul. W. E. Radzikowskiego 152, 31-342 Krak\'{o}w, Poland}

\author{Svetoslav~Stankov}
\affiliation{Institute for Photon Science and Synchrotron Radiation, Karlsruhe Institute of Technology, D-76131 Karlsruhe, Germany}
\affiliation{Laboratory for Applications of Synchrotron Radiation, Karlsruhe Institute of Technology, D-76131 Karlsruhe, Germany}

\author{Przemys\l{}aw~Piekarz}
\affiliation{Institute of Nuclear Physics, Polish Academy of Sciences, ul. W. E. Radzikowskiego 152, 31-342 Krak\'{o}w, Poland}

\date{\today}

\begin{abstract}

Iron silicides are good candidates for applications in  optoelectronic and thermoelectric devices.
Lattice dynamical properties and thermal conductivity in the $\beta$-FeSi$_2$ semiconductor
are investigated with the first-principles computational methods. 
Phonon dispersion relations are calculated via the
temperature-dependent effective potential method and self-consistent phonon theory. 
To properly model thermal transport, we explicitly consider
the impact of phonon-phonon interactions by analyzing
anharmonic contributions to the phonon self-energy. 
This yields temperature-dependent phonon frequencies and linewidths,
reflecting the finite lifetime of phonons due to scattering
processes. The calculated phonon frequencies and line profiles are used to obtain 
the Raman spectra, which shows good agreement with the experimental data. 
We revealed an enhanced anharmonic behaviour of the Raman modes with the highest frequencies.  
The lattice thermal conductivity is then obtained as a function of temperature and crystallite size within
the relaxation-time approximation.
Phonon transport shows a small anisotropy due to the orthorhombic structure and a very weak dependence
on the quartic anharmonic corrections. The results obtained for an infinite material and for several crystallite sizes
were analyzed and compared with the available experimental data.
\end{abstract}

\maketitle


\section{Introduction}

The comprehensive determination of important physical properties of
crystals, such as thermal expansion, lattice thermal
conductivity or structural phase transitions, requires a fundamental 
understanding of the anharmonic effects.
Although the investigation of anharmonic interactions in crystals has
attracted a considerable interest for decades~\cite{cowley_1968}, a substantial progress
has only recently been achieved thanks to advances in theoretical and
numerical methods and increased computational power.
Now, phonon frequencies, lifetimes, and heat transfer in a wide range of
materials
can be quantitatively predicted using the available computational resources
based on the density functional theory (DFT)~\cite{lindsay_2013,mcgaughey_2019,lindsay_2019}.
In the case of strongly anharmonic systems, the self-consistent phonon
(SCPH) theory~\cite{tadano_2015} as well as the perturbative approach~\cite{tadano_2018}, using higher
order interatomic force constants derived from the fitting to the displacement force data obtained
with DFT, have proven to be successful.

Transition-metal silicides are promising materials for fabrication of electronic components
designed for integration with silicon-based circuits~\cite{murarka_1995}.
At room temperature, iron disilicide ($\beta$-FeSi$_2$) is a
direct-bandgap semiconductor~\cite{bost_1985}, making this material a good
candidate for application in optoelectronic devices such as infrared
detectors or light emitters~\cite{bost_1988}. The development of light-emitting diodes utilizing FeSi$_2$/Si heterostructures has been successfully demonstrated~\cite{leong_1997,suemasu_2001}. Due to
a high thermal stability and strong light absorption, FeSi$_2$ is
also a suitable photovoltaic material~\cite{powalla_1993,liu_2006,okuhara_2017}.

$\beta$-FeSi$_2$ crystallizes in the base-centered orthorhombic
lattice~\cite{dusausoy_1971} transforming to the
tetragonal metallic $\alpha$-FeSi$_{2}$ phase around $1200$~K~\cite{starke_2002}.
Optical studies indicated a direct band gap of the values
$0.85$--$0.89$~eV~\cite{bost_1988,dimitriadis_1990,arushanov_1995,wan_2003},
however, the {\it ab initio} calculations predicted a smaller indirect gap close to 0.8 eV~\cite{christensen_1990}. 
The existence of such an indirect gap was then confirmed by the optical linear transmittance measurements at
low temperatures~\cite{giannini_1992}. As shown by first-principles studies,
the character of the band gap is very sensitive to the orientation 
of a crystal grown on silicon~\cite{clark_1998}.

$\beta$-FeSi$_2$ belongs also to good thermoelectric materials~\cite{ware_1964}, with potential applications resulting from its chemical stability up to high temperatures, nontoxicity, and low cost of preparation~\cite{yamada_2012,nozariasbmarz_2017}. 
It has already been implemented in cars~\cite{birkholz_1988} and portable power sources~\cite{uemura_1989}. 
Its thermoelectric performance can be improved by doping~\cite{ito_2001,tani_2001,kim_2003,chen_2005,pandey_2013,le_tonquesse_2019}, 
which enhances the electric transport and reduces the thermal conductivity~\cite{waldecker_1973,du_2019,du_2020}.  
The thermal conductivity can be also reduced by the modification of microstructure~\cite{ail_2015} or by
nanostructurization~\cite{watanabe_2017,taniguchi_2017,hsin_2017,abbassi_2021}.

The lattice thermal conductivity is directly connected with anharmonic effects and phonon scattering processes.
The vibrational properties of \mbox{$\beta$-FeSi$_2$} were studied by the infrared and Raman spectroscopy~\cite{lefki_1991,guizzetti_1997,maeda_2004,baleva_2008,liu_2011,maeda_2011}. The observed anisotropy in the phonon spectra results from the enhanced sensitivity of the infrared and Raman features to the local lattice distortions~\cite{guizzetti_1997}. The Fe phonon density of states was measured by nuclear inelastic scattering (NIS), showing a good agreement with the density functional theory (DFT) calculations~\cite{walterfang_2005}. Using the DFT approach, the phonon dispersion curves, phonon density of states, as well as various thermodynamic properties were obtained within the harmonic approximation~\cite{tani_2010,liang_2011}. The extended Klemens model was applied to study
the anharmonic effect on phonon frequencies and linewidths observed by the Raman spectroscopy~\cite{zhang_2023}.
The impact of nanostructurization on lattice dynamics was explored in the $\beta$-FeSi$_2$ nanorods grown on the Si(110) surface by the NIS and {\it ab initio} methods~\cite{kalt_2022}.

In this work, we investigate the lattice dynamical properties of $\beta$-FeSi$_2$ 
using the DFT calculations. We study the effect of anharmonic terms in the temperature-dependent potential on phonon frequencies and lifetimes. We focus on the Raman modes, comparing the theoretical results with the experimental data.
The thermal conductivity is derived in a broad temperature range and the effect of crystallite size is analyzed.



This study is structured as follows.
In Sec.~\ref{sec.com} we describe the details of computational methods.
Next, in Sec.~\ref{sec.result} we present and discuss our results.
In particular we present the crystal structure (Sec.~\ref{sec.crys}) and lattice dynamics (Sec.~\ref{sec.lattice}).
We investigate also the thermal conductivity comparing the obtained results with the available experimental data (Sec.~\ref{sec.thermal}).
Finally, Sec.~\ref{sec.summary} summarizes our key findings and conclusions.

\section{Calculation method}
\label{sec.com}


The calculations were performed using the projector augmented-wave potentials~\cite{blochl_1994} and the generalized gradient approximation~\cite{perdew_1996} implemented in the Vienna Ab initio Simulation Package (VASP)~\cite{kresse.hafner.94,kresse.furthmuller.96,kresse.joubert.99}. 
The lattice parameters and atomic positions were optimized in the ${\bm a} \times ({\bm b}-{\bm c}) \times ({\bm b}+{\bm c})$ supercell containing 32 formula units and four primitive cells.
The integration in the reciprocal space was conducted using the $2 \times 2 \times 2$ Monkhorst--Pack mesh~\cite{monkhorst_1976} and the cut-off energy was set to $500$~eV. For convergence conditions, we set the energy change below $10^{-5}$ and $10^{-8}$ for the ionic and electronic loops, respectively. 

The lattice dynamical properties were studied within the temperature-dependent effective potential (TDEP) approach~\cite{hellman_2013}. The atomic potential with the third and fourth order anharmonic terms was derived from interatomic forces induced by displacements of all atoms at finite temperatures.
The sets of atomic displacements were generated by the high efficiency configuration space sampling (HECSS)~\cite{jochym_2021} and forces were obtained by VASP. The interatomic force constants and phonon frequencies were calculated with the {\sc Alamode} software~\cite{tadano_2014}.

Furthermore, we have attempted to construct a {\emph{temperature independent}} 
anharmonic model. We have used combined data from all investigated temperatures 
(300, 600 and 1000~K) and fitted a large (over 15~000 free parameters), fourth-order 
interaction model to this dataset. 
Subsequently, we have used this model to calculate line profiles and positions of Raman-active modes
at multiple temperatures.

The changes in phonon frequencies induced by the anharmonic effects were investigated within two approaches.
First, the impact of the quartic anharmonic terms was included using the SCPH theory~\cite{tadano_2015}.
Second, the mode profiles (frequency shifts and line widths) were determined from the real and imaginary parts of the phonon self-energy resulting from the cubic and quartic anharmonic terms of the above mentioned large model~\cite{tadano_2018}. 
The longitudinal optic-transverse optic (LO-TO) splitting was also evaluated, using the static dielectric tensor and Born effective charges calculated within density functional perturbation theory~\cite{gajdos_2006}.

\begin{figure}[]
    \centering
    \includegraphics[width=\linewidth]{fig1_new.png}
\caption{%
(a) The conventional unit cell of $\beta$-FeSi$_2$ (with Cmca symmetry) and (b) the corresponding Brillouin zone with selected high-symmetry points.
}
  \label{fig.struct}
\end{figure}


To further characterize the vibrational properties, Raman scattering was investigated using the Phonopy-Spectroscopy package~\cite{skelton_2017}. This enabled the identification of Raman-active modes and the calculation of Raman tensors. Anharmonic force constants, derived from calculations using {\sc Alamode}, were then used to obtain theoretical line profiles for the Raman modes. The presented Raman scattering spectra combine these anharmonic line profiles with the Raman tensor amplitudes.
This analysis was also based on the large quartic model mentioned above.

Finally, the thermal conductivity was calculated as a function of temperature and crystallite size within the relaxation-time approximation (RTA) \SP{}{as implemented in {\sc Alamode}~\cite{tadano_2014}}. The phonon lifetimes were calculated from the phonon self-energy including the cubic and quartic anharmonic terms. \SP{}{The RTA provides a solution to the Boltzmann transport equation (BTE) under the assumption that scattering events are independent and can be treated through mode-resolved relaxation times.
To verify the validity of this approximation for $\beta$-FeSi$_2$, we additionally
solved the BTE iteratively.} \SP{}{These calculations were performed with
{\sc Phono3py}~\cite{togo_2023}. For cross-validation, the additional RTA calculations were performed on the same $\bm{q}$-grid as the BTE calculations, using the implementation provided by {\sc Phono3py}.}

\section{Results}
\label{sec.result}

\subsection{Crystal structure}
\label{sec.crys}

The $\beta$-FeSi$_2$ structure adopts a base-centered orthorhombic lattice with the space group Cmca (No.~64) as shown in Fig.~\ref{fig.struct}(a).
The unit cell consists of two primitive cells and contains 48 atoms.
Iron (silicon) atoms possess two nonequivalent positions: \mbox{Fe-I} and \mbox{Fe-II} (\mbox{Si-I} and \mbox{Si-II}), presented in Fig.~\ref{fig.struct}(a) as gray and purple (orange and yellow) spheres, respectively.
This crystal structure is derived from the fluorite-type lattice with strongly distorted Si cubes and Fe atoms occupying 
one-half of the central sites.
The Fe-I and Fe-II sites create different layers perpendicular to the $x$ direction, and they are separated by layers containing both Si sites.
Each Fe atom is coordinated by 8 Si atoms with slightly different Fe-Si distances.
The optimized lattice constants ($a = 9.874$~{\AA}, $b = 7.767$~{\AA}, and
$c = 7.811$~{\AA}) agree very well with the experimental data ($a = 9.863$~{\AA}, $b = 7.791$~{\AA}, and $c = 7.833$~{\AA})~\cite{dusausoy_1971}.

Iron atoms occupy the Wyckoff sites 8\textit{d} ($0.2166$, $0$, $0$) and  8\textit{f} ($0$, $0.3072$, $0.1879$), corresponding to \mbox{Fe-I} and \mbox{Fe-II}, respectively. 
Silicon atoms are located at two inequivalent 16\textit{g} positions: ($0.1282$, $0.2737$, $0.0495$) and \mbox{($0.3734$, $0.0445$, $0.2270$)}, assigned as \mbox{Si-I} and \mbox{Si-II}.
The optimized positions of atoms agree very well with the experimental data~\cite{dusausoy_1971} and the previous theoretical studies~\cite{tani_2010,liang_2011}.


\subsection{Lattice dynamics}
\label{sec.lattice}

\begin{figure}[]
    \centering
    \includegraphics[width=\linewidth]{Fig1c.pdf}
\caption{%
The phonon dispersion curves along high symmetry directions obtained within SCPH (for temperatures from $0$ to $1000$~K).
Dashed black lines indicate the phonon dispersions obtained from the harmonic approximation. The white dots indicate Raman-active modes with A$_g$ symmetry.
The vertical plot shows the phonon density of states (DOS) calculated at a reference temperature of $600$~K.
}
  \label{fig.ph_band}
\end{figure}


In Fig.~\ref{fig.ph_band} we present the phonon dispersion relations of $\beta$-FeSi$_2$ along high-symmetry directions in the Brillouin zone [Fig.~\ref{fig.struct}(b)].
Due to 24 atoms in the primitive cell, the phonon spectrum consists of 69 optical modes and three acoustic modes.
The phonon dispersions were calculated within the SCPH approach in the temperature range $0$--$1000$~K (presented by color lines in Fig.~\ref{fig.ph_band}), and they are compared with the results obtained from the harmonic part of the effective potential corresponding to temperature $T=300$~K (indicated by dashed black lines in Fig.~\ref{fig.ph_band}).
As we can see within the SCPH method, the anharmonic effects are rather weak and leads only to small renormalization of phonon frequencies.
Only the highest modes show more pronounced shifts of their frequencies to larger values.
The total and partial element-projected phonon density of states obtained within the harmonic approximation are presented in Fig.~\ref{fig.ph_band}.
Up to around $320$~cm$^{-1}$, the contributions from both elements are very similar, while for higher frequencies the spectrum is dominated by
the Si vibrations.


\begin{table}[!t]
\begin{ruledtabular}
\caption{Calculated and experimental Raman-active modes of $\beta$-FeSi$_2$ with their irreducible representations (IR). Present theoretical results are compared with the previous theoretical data from Ref.~\cite{tani_2010} and experimental results from \mbox{Refs.~\cite{lefki_1991,maeda_2004}.} The experimental frequencies with known symmetries (A$_g$) are shown in bold, while other experimental modes are assigned to the best fitting theoretical values.}
\begin{tabular}{c c c c c}
\textbf{IR} & \multicolumn{4}{c}{\textbf{Frequency (cm$^{-1}$)}} \\
 & Present  & Theor.~\cite{tani_2010} & Exp.~\cite{lefki_1991} & Exp.~\cite{maeda_2004} \\
\hline
B$_{2g}$ & 175.4 & 179 & 176 &  \\
B$_{1g}$ & 176.3 & 185 & 179  &  \\
B$_{1g}$ & 193.3 & 198 &  & 190.6 \\
A$_g$    & 196.6 & 208 & \textbf{195} & \textbf{194.0} \\
A$_g$    & 203.9 & 210 & \textbf{197} & 199.6 \\
B$_{3g}$ & 205.5 & 212 & 200 &  \\
B$_{3g}$ & 226.3 & 236 & 206 & 227.1 \\
B$_{1g}$ & 233.3 & 240 &  &  231.6 \\
B$_{2g}$ & 248.6 & 254 &  &  \\
A$_g$    & 250.1 & 257 & \textbf{247} & \textbf{247.3} \\
A$_g$    & 254.9 & 264 & \textbf{253} & 254.3 \\
B$_{1g}$ & 275.5 & 285 &  &  274.1 \\
B$_{2g}$ & 282.4 & 295 &  &  281.2 \\
B$_{3g}$ & 286.8 & 297 &  &  \\
B$_{1g}$ & 307.1 & 317 &  &  \\
B$_{2g}$ & 312.6 & 326 &  &  311.8 \\
B$_{1g}$ & 319.0 & 324 &  &  \\
B$_{3g}$ & 327.9 & 341 &  & 325.8 \\
B$_{2g}$ & 333.3 & 345 &  &  \\
A$_g$    & 339.3 & 352 & \textbf{346} & 339.5 \\
B$_{2g}$ & 343.0 & 350 &  &  \\
B$_{1g}$ & 353.5 & 366 &  &  \\
B$_{2g}$ & 372.8 & 383 &  & 370.7 \\
B$_{1g}$ & 375.1 & 385 &  &  \\
B$_{3g}$ & 375.2 & 386 &  &  \\
B$_{3g}$ & 385.3 & 401 &  &  \\
A$_g$    & 386.5 & 398 &  & 386.2 \\
B$_{2g}$ & 387.7 & 402 &  & 388.2 \\
A$_g$    & 404.6 & 415 &  & 400.4 \\
B$_{2g}$ & 405.1 & 420 &  &  \\
B$_{1g}$ & 412.7 & 428 &  &  \\
B$_{3g}$ & 418.4 & 431 &  &  \\
B$_{3g}$ & 441.4 & 458 &  & 442.6 \\
B$_{3g}$ & 447.3 & 466 &  & 446.3 \\
A$_g$    & 448.4 & 464 &  &  \\
A$_g$    & 499.1 & 517 &  &  \\
\end{tabular}
\label{T1Raman}
\end{ruledtabular}
\end{table}

%subsection{Raman}

\begin{figure*}
\centering
  \includegraphics[width=\textwidth]{spectrum_RTA+raman.pdf}
\caption{Raman spectrum of $\beta$-FeSi$_2$ calculated within the perturbative approach at three temperatures 
($300$, $600$, $900$~K -- blue, orange, and red lines, respectively). 
The calculated spectrum includes all Raman-active modes. The A$_g$ modes
are indicated by purple vertical lines at peak positions corresponding to T=300~K. 
The frequencies derived from the harmonic approximation at T=300~K are indicated by green lines.
Connecting arrows indicate the correspondence between harmonic and
anharmonic frequencies, demonstrating the frequency shifts due to 
phonon interactions. Experimental values for the A$_g$ modes
based on Ref.~\cite{lefki_1991} are marked with black dashed lines.
}
\label{raman}
\end{figure*}


The phonon spectrum at the $\Gamma$ point consists of 36 Raman modes classified according to the irreducible representations: $9A_\text{g}+9B_\text{1g}+9B_\text{2g}+9B_\text{3g}$. 
Fig.~\ref{raman} shows the Raman spectrum of A$_g$ symmetry calculated for $\beta$-FeSi$_2$ within the perturbative approach at three temperatures $300$, $600$, and $900$~K (solid blue, orange, and red curves, respectively), including third-order and fourth-order anharmonic corrections.
The calculated Raman spectrum includes all Raman-active modes.
The five experimentally observed A$_g$ modes are highlighted by black dashed lines, based on data from Ref.~\cite{lefki_1991}.
Additional peaks not marked with vertical lines correspond to phonon modes with symmetries other than A$_g$.
The frequencies of Raman modes obtained in anharmonic calculations are compared with the previous results calculated within the harmonic approximation and the experimental values in Tab.~\ref{T1Raman}. 
We have marked in bold the experimentally determined A$_g$ modes, which are compared with the calculations.
Since the experimental studies did not provide the accurate assignement of the Raman modes with the B$_{1g}$, B$_{2g}$, and B$_{3g}$ symmetry~\cite{lefki_1991,maeda_2004}, we cannot compare them directly with the theoretical results.
However, in Tab.~\ref{T1Raman} we have assigned the measured frequencies to the best fitting theoretical values without taking into account the symmetry of the modes, except for the known A$_{g}$ modes. 


The impact of anharmonicity on the phonon frequencies is well visible from the comparison of the results obtained within the harmonic approximation 
and from the anharmonic calculation (vertical green and purple lines in Fig.~\ref{raman}, respectively).
Here we show only the A$_g$ modes, which are compared with the experimental results (vertical black dashed lines).
Anharmonic frequencies calculated at $300$~K are indicated by purple lines, while frequencies derived from harmonic approximation are marked with green lines. The grey solid lines connect corresponding modes obtained in both approximations.
In most cases, the results obtained within the harmonic approximation do not agree with the experimental frequencies. 
%Only the modes close to $250$~cm$^{-1}$ and $400$~cm$^{-1}$ correspond well to the experimental values. MS
As we can see, the inclusion of the anharmonic correction leads to a significant modification of the phonon frequencies.
These anharmonic effects are stronger for higher-frequency modes mainly because of
the dominant contribution from Si atoms, which vibrate with larger amplitudes than heavier Fe atoms. 
When atoms move to larger distances the potential deviates more from the harmonic approximation,
and the anharmonic corrections become stronger.


The modification of phonon frequencies observed in Fig.~\ref{raman} is much larger than in the SCPH scheme presented in Fig.~\ref{fig.ph_band}. The SCPH approach includes only the leading-order contribution to 
the phonon self-energy obtained from the quartic anharmonic terms~\cite{tadano_2015}. Therefore, it does not describe fully the changes of phonon frequencies found within the perturbation theory (see Fig.~\ref{raman}).
Especially, it is well visible for two highest A$_g$ modes, which exhibit also the largest line broadening 
and the strongest dependence on temperature.
Therefore, a better agreement with experimentally observed frequencies is visible,
confirming the significant influence of the anharmonicity on the frequencies and line profiles of phonon modes.
In fact, the decrease of phonon frequencies should be even stronger due to thermal expansion, 
which is not included in our calculations.
Within SCPH the frequencies of the highest modes increase with increasing temperature as we see in Fig.~\ref{fig.ph_band}. The comparison of two different approaches applied to study anharmonic properties of $\beta$-FeSi$_2$ shows that the perturbation theory, which includes the cubic and quartic terms, better describes the changes of phonon frequencies with temperature than the SCPH method. \SP{}{This indicates that the leading-order contribution included in SCPH are not important in this material.}

Additionaly we should nottice that for other than A$_g$ modes we cannot make an unambiguous assignment of theoretical frequencies to experimental ones. Note that the spectrum in Fig.~\ref{fig.ph_band} contains all Raman-active modes. 
The limitation to A$_g$ modes concerns only the indicated positions of the peaks. 

\SP{}{The full Raman spectrum, with the frequencies of the B$_{1g}$, B$_{2g}$ and B$_{3g}$ modes marked, is shown in Appendix A, Fig.~\ref{raman1}. This represents our theoretical prediction of possible Raman-mode assignments, which can be verified in future experiments.}
%This is the theoretical prediction of possible assignment of Raman modes that can be verified in future experiments.  


\subsection{Thermal conductivity}
\label{sec.thermal}


\begin{figure*}[!t]
\centering
  \includegraphics[width=\linewidth]{time3.pdf}
\caption{Phonon lifetimes calculated for three temperatures as a function of phonon frequency. The colors correspond to the phonon branches.}
  \label{thermtime}
\end{figure*}


In this section, we analyze the thermal conductivity tensor of $\beta$-FeSi$_2$ obtained within the RTA approach~\cite{tadano_2018} as a function of temperature
%
\begin{equation}
\kappa_{\text{ph}}^{\mu\nu}(T) = \frac{1}{NV} \sum_{\bm{q},j} c_{\bm{q}j}(T) v_{\bm{q}j}^{\mu} v_{\bm{q}j}^{\nu}\tau_{\bm{q}j}(T),
\end{equation}
% 
where $c_{\bm{q}j}$ is the mode heat capacity and $v_{\bm{q}j}$ is the mode group velocity. 
The relaxation time is approximated by the phonon lifetime $\tau_{\bm{q}j}$
calculated for $j$-th branch at the wave vector $\bm{q}$.
$V$ is the unit cell volume and $N$ is the number of unit cells in the crystal.
The phonon lifetime is calculated using this formula
%
\begin{equation}
\tau_{\bm{q}j}(T)=\frac{1}{2\Gamma_{\bm{q}j}^{\text{anh}}(T)},
\end{equation}
%
where $\Gamma_{\bm{q}j}^{\text{anh}}$ is the anharmonic phonon linewidth obtained from 
the imaginary part of the phonon self-energy within the perturbation theory.

In Fig.~\ref{thermtime}, we present $\tau_{\bm{q}j}$ obtained for three temperatures 300, 600, and 1000~K as a function of frequency. As we see, the acoustic phonons close to the $\Gamma$ point have the longest lifetimes,
which are diminished with increasing frequency reaching local minima around $200$~cm$^{-1}$.
For higher frequencies, phonon lifetimes first increase to local maxima around $300$~cm$^{-1}$ and then decrease to
the lowest values in the range of highest optical modes. The shortest lifetimes correspond to the largest line broadening
observed for the Raman modes in Fig~\ref{raman}. The phonon group velocities 
$v_{\bm{q}j}=\partial\omega_{\bm{q}j}/\partial\bm{q}$, which are obtained by the central difference formula, are presented in Fig.~\ref{thermvelocity}. Their temperature dependence is negligible, therefore, we present only the results for $T=600$~K.
At low frequencies, there are clearly two ranges of group velocities of the acoustic phonons. 
The larger values correspond to the longitudinal modes, while the lower values are obtained from the
transverse acoustic branches. Group velocities of acoustic phonons decrease for larger frequencies
and reach the average values typical for optic branches.

\begin{figure}[!t]
\centering
  \includegraphics[width=\linewidth]{GV.pdf}
\caption{Mode group velocities calculated as a function of phonon frequency. The colors correspond to the phonon branches.
}
  \label{thermvelocity}
\end{figure}

In Fig.~\ref{anizotropy}(a), we present the three diagonal elements of $\kappa_{\text{ph}}^{\mu\nu}$ corresponding to the main directions of the crystal structure. 
They were obtained from the force constants calculated at the base temperature $T=600$~K and the crystallite size $0.1$~$\mu$m to account for boundary-limited phonon transport. 
Due to the orthorhombic symmetry, we observe a small anisotropy in phonon transport in the whole temperature range. 
At low temperatures, the three components of the heat conductivity increase in a very similar way with the $\kappa_{\text{ph}}^{yy}$ element slightly larger than two other components. 
After reaching the maximum, we observe a change in the largest component from $\kappa_{\text{ph}}^{yy}$ to 
$\kappa_{\text{ph}}^{xx}$.    
In Fig.~\ref{anizotropy}(b), the thermal conductivity is shown for three base temperatures, at which the interatomic potential was obtained ($300$~K, $600$~K, and $1000$~K), using the energy expansion up to third- and fourth-order anharmonic terms, and the same structure size of $0.1$~$\mu$m. At lowest temperatures, the thermal conductivity strongly increases, reaching the maximum around $T=180$~K, then it shows a slower decrease with temperature.
The differences between the two levels of approximation are minimal, suggesting that third-order calculations already capture the dominant phonon scattering mechanisms. The dependence on the base temperature is also very weak, showing the changes in the heat conductivity within a few percent. 
\SP{}{Further verification of the reliability of our thermal conductivity results is given in Appendix~\ref{ThermalAB}, where the full BTE calculations show good agreement with RTA and higher-resolution RTA results, demonstrating that the $8\times8\times8$ $\bm{q}$-mesh already provides converged values.}

\begin{figure}[!t]
\centering
  \includegraphics[width=\linewidth]{Anizotropy.pdf}
\caption{(a) The anisotropic thermal conductivity of $\beta$-FeSi$_2$ calculated along the lattice directions at 600~K. (b) The average temperature-dependent thermal conductivity taken at $300$~K, $600$~K, and $1000$~K, including anharmonic corrections up to cubic (A3) and quartic (A4) terms. In both cases the crystallite size is 0.1~$\mu$m.}
  \label{anizotropy}
\end{figure}


In Fig.~\ref{therm}, we fix the base temperature at $600$~K and examine the effect of crystallite size on thermal conductivity, varying it from $0.01$ to $0.5$~$\mu$m.
With decreasing the crystallite size, we observe a shift of the position of the maximum to larger temperatures and a decrease of the thermal conductivity in the entire temperature range.
Theoretical results are compared with several experimental data obtained above the room temperature. 
The measured thermal conductivity depends to a large extent on the sample quality, its purity and the size of the crystalline grains which depends on 
the production processes.
Many measurements were performed using crystallites of micrometric or unknown size ~\cite{waldecker_1973,ito_2002,kim_2003,du_2020}, however, 
numerous attempts to minimize $\kappa$ by reducing grain sizes to $56$~nm~\cite{dabrowski_2019, dabrowski_microstructure_2021}, $30$-$400$~nm~\cite{le_tonquesse_2019}, $50$ and $200$~nm~\cite{abbassi_2021}, or introducing pores into the material~\cite{sam2023improved} 
are also carried out. 
Another way to change the thermal conductivity is to dope $\beta$-FeSi$_2$ with different elements~\cite{ito_2002,kim_2003,du_2020,cheng_2024}, however, this effect is beyond our investigation.

\begin{figure}[!t]
\centering
  \includegraphics[width=\linewidth]{Thermal_conductivity3.pdf}
\caption{
The phonon thermal conductivity of $\beta$-FeSi$_2$: theoretical results for the infinite crystalline size and with boundary conditions, compared with experimental data for different structure sizes.
}
  \label{therm}
\end{figure}
%Fig.~\ref{therm}\cite{abbassi_2021}\cite{waldecker_1973}\cite{dabrowski_2019}\cite{sam2023improved}\cite{kim_2003}\cite{du_2020}\cite{ito_2002}\cite{le_tonquesse_2019}.

We observe a decrease in the thermal conductivity with reducing crystalline grain sizes in all analyzed experimental data. 
For instance, by decreasing the crystallite size to $50$~nm, the thermal conductivity at room temperature was reduced by a factor of $1.7$, what can be  compared to the annealed sample with 200 nm grains~\cite{abbassi_2021}. 
It is worth noting that the rate of decrease in value with increasing temperature in both cases, for grain sizes of $50$~nm and $200$~nm, is significantly different, which is consistent with our calculations. 
The same trend can be observed by comparing the thermal conductivity measured for a sample with bulk crystallite sizes with the thermal conductivity of a sample with grains smaller than 400 nm~\cite{le_tonquesse_2019}.
The theoretical results obtained for the same crystallite size show higher values due to factors not captured in the idealized model, such as crystal imperfection or mechanical strain. Usually, a decrease in the crystallite size is related to an increased concentration of grain boundaries, point defects, and stacking faults that influence the phonon scattering~\cite{le_tonquesse_2019,abbassi_2021}.   

We should note that the total thermal conductivity is a combination
of the lattice and electronic contributions to the heat transport.
In semiconductors, the electronic thermal conductivity is negligible at low temperatures and significantly increases 
only much above the room temperatures~\cite{gu_2020}.
For $\beta$-FeSi$_2$, the electronic thermal conductivity was obtained from the electric conductivity using the Wiedemann-Franz law~\cite{ito_2002,kim_2003,le_tonquesse_2019}.
In the undoped material, its value does not exceed $0.1$~W/mK in the measurement up to $T=950$~K~\cite{kim_2003}.
By doping, the electronic thermal conductivity can be enhanced, and it has a direct impact on the thermoelectric properties of $\beta$-FeSi$_2$ at high temperatures~\cite{ito_2002,kim_2003}.
In the present study, we consider only the phonon contribution to the thermal conductivity,
therefore, agreement with experimental data may deteriorate with increasing temperature.

\section{Summary}
\label{sec.summary}

We performed {\it ab initio} studies on lattice dynamical and thermal transport properties of $\beta$-FeSi$_2$. The effect of anharmonicity was analyzed within two approaches -- the SCPH method and the perturbation theory.
The phonon dispersion curves obtained within SCPH show small renormalization of frequencies comparing to the harmonic approximation. 
The Raman spectra were calculated within the procedure which takes into account the peak intensities obtained from the Raman tensors and the line profiles obtained from the phonon self energy derived within the perturbation theory based on the large, temperature-independent, quartic model fitted to the data from the wide range of temperatures (300-1000~K). 
The anharmonic corrections strongly affect the frequencies and line profiles of some modes and results in overall better agreement with the experimental data. 
We analyzed the phonon lifetimes and group velocities obtained as functions of the phonon frequency.
Then the lattice thermal conductivity was calculated for a broad range of temperatures and grain sizes.
We found a small anisotropy in the phonon thermal transport resulting from the orthorhombic structure and a weak effect of the quartic anharmonic terms. 
The thermal conductivity calculated for various crystalline grain sizes show a good qualitative agreement with the available measurements.

\begin{acknowledgments}
Some figures in this work were rendered using {\sc Vesta}~\cite{momma.izumi.11} software.
This work was partially supported by the Ministry of Education, Youth and Sports of the Czech Republic through the e-INFRA CZ (ID:90254).
\end{acknowledgments}

\appendix

\section{Raman spectrum}
\label{RamanAA}

Based on the polarized Raman measurements reported in Ref.~\cite{maeda_2004}, two Raman peaks were identified as belonging to the A$_g$ symmetry class, and several additional peaks were observed with similar or different polarization dependence. Although the authors of Ref.~\cite{maeda_2004} provided estimates of the relative Raman tensor components, they did not specify which of the remaining modes correspond to the B$_{1g}$, B$_{2g}$, or B$_{3g}$ symmetries. Because of this missing experimental information, a direct symmetry-resolved comparison between the measurement and theory is not currently possible for the non-A$_g$ modes. 
To provide a complete theoretical picture of the B$_g$-type modes, we show here the calculated Raman-active frequencies and intensities for the B$_{1g}$, B$_{2g}$, and B$_{3g}$ symmetries only. 
Fig.~\ref{raman1} shows the predicted Raman modes for the B$_g$ symmetries in $\beta$-FeSi$_2$. The results obtained within the harmonic approximation are compared with the anharmonic perturbation theory calculations which provides both, frequency shifts and predicted line profiles of the modes.
Although the experimentally measured peaks cannot be directly assigned to these symmetries due to the lack of polarization-resolved data, the theoretical predictions provide a reference for comparison. Matching the measured frequencies to the closest theoretical B$_g$ modes (Table~\ref{T1Raman}) allows for a tentative assignment, which can guide future polarization-resolved Raman experiments aimed at determining the precise symmetry of the unresolved peaks.

\begin{figure*}[t]
\centering
  \includegraphics[width=\linewidth]{spectrum_RTA+raman_Bi_modes.pdf}
\caption{Raman spectrum of $\beta$-FeSi$_2$ calculated at three temperatures 
($300$, $600$, $900$~K -- blue, orange, and red lines, respectively). 
The calculated spectrum includes all Raman-active modes. The B$_{ig}$ modes
are indicated by purple vertical lines at peak positions corresponding to T=300~K. 
The frequencies derived from the harmonic approximation at T=300~K are indicated 
by green, yellow and pink lines.
Connecting arrows indicate the correspondence between harmonic and anharmonic 
frequencies, demonstrating the frequency shifts due to phonon interactions.}
\label{raman1}
\end{figure*}
\section{Thermal conductivity obtained from BTE and RTA}
\label{ThermalAB}

The thermal conductivity was computed by solving the full BTE on the largest feasible $\bm{q}$-point grid, $8\times8\times8$, and compared with the corresponding RTA results obtained on the same grid. As shown in Fig.~\ref{bte}, the difference between the components of the thermal conductivity tensor obtained within BTE and RTA at this resolution is very small, indicating a good agreement between these two approaches.
Moreover, we performed an additional calculation using RTA on a denser $20\times20\times20$ grid. As seen in the Fig.~\ref{bte}, the higher-resolution data remain in a good agreement with both the BTE and RTA results for the $8\times8\times8$ grid.
It shows that the $8\times8\times8$ mesh already provides good results for this structure and confirms reliability of the calculations.

\begin{figure}[!h]
\centering
  \includegraphics[width=\linewidth]{BTEvsRTAvsANP.pdf}
\caption{Thermal conductivity of $\beta$-FeSi$_2$ obtained within the BTE and RTA methods using the Phono3py software on the $8\times8\times8$ q-point grid, compared with the RTA results computed with ALAMODE on a denser $20\times20\times20$ grid.}
\label{bte}
\end{figure}

%\section*{Data availability}
%The data that support the findings of this article are openly available~\footnote{give me DOI}.
% see https://journals.aps.org/authors/data-availability-statements#citation

\bibliography{refs.bib}
%\bibliographystyle{ieeetr}


\end{document}

\documentclass[%
%reprint,
superscriptaddress,
%groupedaddress,
longbibliography,
%unsortedaddress,
%runinaddress,
%frontmatterverbose, 
%preprint,
%preprintnumbers,
%nofootinbib,
nobibnotes,
%bibnotes,
amsmath,amssymb,
aps,
%pra,
prb,
%rmp,
%prstab,
%prstper,
%showkeys,
floatfix,
twocolumn
]{revtex4-2}

\usepackage{graphicx}% Include figure files
\usepackage{calc}% Calculate margins
\usepackage{dcolumn}% Align table columns on decimal point
\usepackage{bm}% bold math

\usepackage[urlcolor=blue,colorlinks=true,citecolor=blue,linkcolor=blue,pdfstartview={FitH},bookmarks=false]{hyperref} % add hypertext capabilities

%\usepackage[mathlines]{lineno} % Enable numbering of text and display math
% \linenumbers\relax % Commence numbering lines

% \usepackage[showframe,%Uncomment any one of the following lines to test 
% %scale=0.7, marginratio={1:1, 2:3}, ignoreall,% default settings
% %text={7in,10in},centering,
% %margin=1.5in,
% % total={6.5in,8.75in}, top=1.2in, left=0.9in, includefoot,
% % height=10in,a5paper,hmargin={3cm,0.8in},
% ]{geometry}

\usepackage{amsmath}
\usepackage{amssymb}
%\usepackage{orcidlink}
\usepackage{xcolor}
%\usepackage{datetime}
\usepackage[normalem]{ulem}

% Change tracking commands
\newcommand{\trackchange}[3]{\textcolor{#3}{\sout{#1}#2}}  % Full color strikeout, insert
%\renewcommand{\trackchange}[3]{\textcolor{#3}{#2}}        % Just color silent remove and insert
%\renewcommand{\trackchange}[3]{{#2}}                      % No indication, silent remove and insert

% Author marker definitions

\definecolor{myblue}{RGB}{0,127,85}
% \definecolor{violet}{RGB}{102,0,204}
% \definecolor{orange}{RGB}{255,128,0}
% \definecolor{green}{RGB}{0,128,0}
\newcommand{\DL}[1]{\trackchange{}{#1}{blue}}
\newcommand{\AP}[1]{\trackchange{}{#1}{red}}
\newcommand{\PJ}[2]{\trackchange{#1}{#2}{orange}}
\newcommand{\JL}[2]{\trackchange{#1}{#2}{myblue}}
\newcommand{\SP}[2]{\trackchange{#1}{#2}{blue}}
\newcommand{\PP}[2]{\trackchange{#1}{#2}{teal}}
\newcommand{\AI}[2]{\trackchange{#1}{#2}{olive}}
\newcommand{\MS}[1]{\trackchange{}{#1}{purple}}

\newcommand{\TODO}[1]{\textcolor{red}{TODO: #1}}

\sloppy

\begin{document}

\title{Ab initio study of the anharmonic properties and thermal conductivity in $\beta$-FeSi$_2$}

\author{Svitlana~Pastukh}
\email[e-mail: ]{svitlana.pastukh@ifj.edu.pl}
\affiliation{Institute of Nuclear Physics, Polish Academy of Sciences, ul. W. E. Radzikowskiego 152, 31-342 Krak\'{o}w, Poland}

\author{Ma\l{}gorzata~Sternik}
\affiliation{Institute of Nuclear Physics, Polish Academy of Sciences, ul. W. E. Radzikowskiego 152, 31-342 Krak\'{o}w, Poland}

\author{Pawe\l{}~T.~Jochym}
\affiliation{Institute of Nuclear Physics, Polish Academy of Sciences, ul. W. E. Radzikowskiego 152, 31-342 Krak\'{o}w, Poland}


\author{Jan~\L{}a\.{z}ewski}
\affiliation{Institute of Nuclear Physics, Polish Academy of Sciences, ul. W. E. Radzikowskiego 152, 31-342 Krak\'{o}w, Poland}

\author{Andrzej~Ptok}
\affiliation{Institute of Nuclear Physics, Polish Academy of Sciences, ul. W. E. Radzikowskiego 152, 31-342 Krak\'{o}w, Poland}

\author{Svetoslav~Stankov}
\affiliation{Institute for Photon Science and Synchrotron Radiation, Karlsruhe Institute of Technology, D-76131 Karlsruhe, Germany}
\affiliation{Laboratory for Applications of Synchrotron Radiation, Karlsruhe Institute of Technology, D-76131 Karlsruhe, Germany}

\author{Przemys\l{}aw~Piekarz}
\affiliation{Institute of Nuclear Physics, Polish Academy of Sciences, ul. W. E. Radzikowskiego 152, 31-342 Krak\'{o}w, Poland}

\date{\today}

\begin{abstract}

Iron silicides are good candidates for applications in  optoelectronic and thermoelectric devices.
Lattice dynamical properties and thermal conductivity in the $\beta$-FeSi$_2$ semiconductor
are investigated with the first-principles computational methods. 
Phonon dispersion relations are calculated via the
temperature-dependent effective potential method and self-consistent phonon theory. 
To properly model thermal transport, we explicitly consider
the impact of phonon-phonon interactions by analyzing
anharmonic contributions to the phonon self-energy. 
This yields temperature-dependent phonon frequencies and linewidths,
reflecting the finite lifetime of phonons due to scattering
processes. The calculated phonon frequencies and line profiles are used to obtain 
the Raman spectra, which shows good agreement with the experimental data. 
We revealed an enhanced anharmonic behaviour of the Raman modes with the highest frequencies.  
The lattice thermal conductivity is then obtained as a function of temperature and crystallite size within
the relaxation-time approximation.
Phonon transport shows a small anisotropy due to the orthorhombic structure and a very weak dependence
on the quartic anharmonic corrections. The results obtained for an infinite material and for several crystallite sizes
were analyzed and compared with the available experimental data.
\end{abstract}

\maketitle


\section{Introduction}

The comprehensive determination of important physical properties of
crystals, such as thermal expansion, lattice thermal
conductivity or structural phase transitions, requires a fundamental 
understanding of the anharmonic effects.
Although the investigation of anharmonic interactions in crystals has
attracted a considerable interest for decades~\cite{cowley_1968}, a substantial progress
has only recently been achieved thanks to advances in theoretical and
numerical methods and increased computational power.
Now, phonon frequencies, lifetimes, and heat transfer in a wide range of
materials
can be quantitatively predicted using the available computational resources
based on the density functional theory (DFT)~\cite{lindsay_2013,mcgaughey_2019,lindsay_2019}.
In the case of strongly anharmonic systems, the self-consistent phonon
(SCPH) theory~\cite{tadano_2015} as well as the perturbative approach~\cite{tadano_2018}, using higher
order interatomic force constants derived from the fitting to the displacement force data obtained
with DFT, have proven to be successful.

Transition-metal silicides are promising materials for fabrication of electronic components
designed for integration with silicon-based circuits~\cite{murarka_1995}.
At room temperature, iron disilicide ($\beta$-FeSi$_2$) is a
direct-bandgap semiconductor~\cite{bost_1985}, making this material a good
candidate for application in optoelectronic devices such as infrared
detectors or light emitters~\cite{bost_1988}. The development of light-emitting diodes utilizing FeSi$_2$/Si heterostructures has been successfully demonstrated~\cite{leong_1997,suemasu_2001}. Due to
a high thermal stability and strong light absorption, FeSi$_2$ is
also a suitable photovoltaic material~\cite{powalla_1993,liu_2006,okuhara_2017}.

$\beta$-FeSi$_2$ crystallizes in the base-centered orthorhombic
lattice~\cite{dusausoy_1971} transforming to the
tetragonal metallic $\alpha$-FeSi$_{2}$ phase around $1200$~K~\cite{starke_2002}.
Optical studies indicated a direct band gap of the values
$0.85$--$0.89$~eV~\cite{bost_1988,dimitriadis_1990,arushanov_1995,wan_2003},
however, the {\it ab initio} calculations predicted a smaller indirect gap close to 0.8 eV~\cite{christensen_1990}. 
The existence of such an indirect gap was then confirmed by the optical linear transmittance measurements at
low temperatures~\cite{giannini_1992}. As shown by first-principles studies,
the character of the band gap is very sensitive to the orientation 
of a crystal grown on silicon~\cite{clark_1998}.

$\beta$-FeSi$_2$ belongs also to good thermoelectric materials~\cite{ware_1964}, with potential applications resulting from its chemical stability up to high temperatures, nontoxicity, and low cost of preparation~\cite{yamada_2012,nozariasbmarz_2017}. 
It has already been implemented in cars~\cite{birkholz_1988} and portable power sources~\cite{uemura_1989}. 
Its thermoelectric performance can be improved by doping~\cite{ito_2001,tani_2001,kim_2003,chen_2005,pandey_2013,le_tonquesse_2019}, 
which enhances the electric transport and reduces the thermal conductivity~\cite{waldecker_1973,du_2019,du_2020}.  
The thermal conductivity can be also reduced by the modification of microstructure~\cite{ail_2015} or by
nanostructurization~\cite{watanabe_2017,taniguchi_2017,hsin_2017,abbassi_2021}.

The lattice thermal conductivity is directly connected with anharmonic effects and phonon scattering processes.
The vibrational properties of \mbox{$\beta$-FeSi$_2$} were studied by the infrared and Raman spectroscopy~\cite{lefki_1991,guizzetti_1997,maeda_2004,baleva_2008,liu_2011,maeda_2011}. The observed anisotropy in the phonon spectra results from the enhanced sensitivity of the infrared and Raman features to the local lattice distortions~\cite{guizzetti_1997}. The Fe phonon density of states was measured by nuclear inelastic scattering (NIS), showing a good agreement with the density functional theory (DFT) calculations~\cite{walterfang_2005}. Using the DFT approach, the phonon dispersion curves, phonon density of states, as well as various thermodynamic properties were obtained within the harmonic approximation~\cite{tani_2010,liang_2011}. The extended Klemens model was applied to study
the anharmonic effect on phonon frequencies and linewidths observed by the Raman spectroscopy~\cite{zhang_2023}.
The impact of nanostructurization on lattice dynamics was explored in the $\beta$-FeSi$_2$ nanorods grown on the Si(110) surface by the NIS and {\it ab initio} methods~\cite{kalt_2022}.

In this work, we investigate the lattice dynamical properties of $\beta$-FeSi$_2$ 
using the DFT calculations. We study the effect of anharmonic terms in the temperature-dependent potential on phonon frequencies and lifetimes. We focus on the Raman modes, comparing the theoretical results with the experimental data.
The thermal conductivity is derived in a broad temperature range and the effect of crystallite size is analyzed.



This study is structured as follows.
In Sec.~\ref{sec.com} we describe the details of computational methods.
Next, in Sec.~\ref{sec.result} we present and discuss our results.
In particular we present the crystal structure (Sec.~\ref{sec.crys}) and lattice dynamics (Sec.~\ref{sec.lattice}).
We investigate also the thermal conductivity comparing the obtained results with the available experimental data (Sec.~\ref{sec.thermal}).
Finally, Sec.~\ref{sec.summary} summarizes our key findings and conclusions.

\section{Calculation method}
\label{sec.com}


The calculations were performed using the projector augmented-wave potentials~\cite{blochl_1994} and the generalized gradient approximation~\cite{perdew_1996} implemented in the Vienna Ab initio Simulation Package (VASP)~\cite{kresse.hafner.94,kresse.furthmuller.96,kresse.joubert.99}. 
The lattice parameters and atomic positions were optimized in the ${\bm a} \times ({\bm b}-{\bm c}) \times ({\bm b}+{\bm c})$ supercell containing 32 formula units and four primitive cells.
The integration in the reciprocal space was conducted using the $2 \times 2 \times 2$ Monkhorst--Pack mesh~\cite{monkhorst_1976} and the cut-off energy was set to $500$~eV. For convergence conditions, we set the energy change below $10^{-5}$ and $10^{-8}$ for the ionic and electronic loops, respectively. 

The lattice dynamical properties were studied within the temperature-dependent effective potential (TDEP) approach~\cite{hellman_2013}. The atomic potential with the third and fourth order anharmonic terms was derived from interatomic forces induced by displacements of all atoms at finite temperatures.
The sets of atomic displacements were generated by the high efficiency configuration space sampling (HECSS)~\cite{jochym_2021} and forces were obtained by VASP. The interatomic force constants and phonon frequencies were calculated with the {\sc Alamode} software~\cite{tadano_2014}.

Furthermore, we have attempted to construct a {\emph{temperature independent}} 
anharmonic model. We have used combined data from all investigated temperatures 
(300, 600 and 1000~K) and fitted a large (over 15~000 free parameters), fourth-order 
interaction model to this dataset. 
Subsequently, we have used this model to calculate line profiles and positions of Raman-active modes
at multiple temperatures.

The changes in phonon frequencies induced by the anharmonic effects were investigated within two approaches.
First, the impact of the quartic anharmonic terms was included using the SCPH theory~\cite{tadano_2015}.
Second, the mode profiles (frequency shifts and line widths) were determined from the real and imaginary parts of the phonon self-energy resulting from the cubic and quartic anharmonic terms of the above mentioned large model~\cite{tadano_2018}. 
The longitudinal optic-transverse optic (LO-TO) splitting was also evaluated, using the static dielectric tensor and Born effective charges calculated within density functional perturbation theory~\cite{gajdos_2006}.

\begin{figure}[]
    \centering
    \includegraphics[width=\linewidth]{fig1_new.png}
\caption{%
(a) The conventional unit cell of $\beta$-FeSi$_2$ (with Cmca symmetry) and (b) the corresponding Brillouin zone with selected high-symmetry points.
}
  \label{fig.struct}
\end{figure}

% To further characterize the vibrational properties, Raman-active scattering was investigated using the Phonopy-Spectroscopy package~\cite{skelton_2017}. This enabled the identification of Raman-active modes and the calculation of Raman tensors. Anharmonic force constants, obtained from calculations using {\sc Alamode}, were then used to obtain theoretical line profiles for the Raman modes. The presented Raman scattering spectra combine these anharmonic line profiles with the Raman tensor amplitudes.
% This analysis was also based on the large quartic model mentioned above.

% Finally, the thermal conductivity was obtained as a function of temperature and crystallite size within the relaxation-time approximation (RTA) \PJ{}{as implemented in {\sc Alamode}\cite{tadano_2014}}. The phonon lifetimes were calculated from the phonon self-energy including the cubic and quartic anharmonic terms. \SP{}{The RTA provides a solution of the Boltzmann transport equation (BTE) under the assumption that scattering events are independent and can be treated through mode-resolved relaxation times.
% To verify the validity of this approximation for $\beta$-FeSi$_2$, we have additionally
% executed an iterative solution of the BTE.}\PJ{}{These calculations were performed with
% {\sc Phono3py}\cite{togo_2023}. For cross-validation, the additional RTA calculations were performed on the same $\bm{q}$-grid as BTE with implementation provided by {\sc Phono3py}.}.

To further characterize the vibrational properties, Raman scattering was investigated using the Phonopy-Spectroscopy package~\cite{skelton_2017}. This enabled the identification of Raman-active modes and the calculation of Raman tensors. Anharmonic force constants, derived from calculations using {\sc Alamode}, were then used to obtain theoretical line profiles for the Raman modes. The presented Raman scattering spectra combine these anharmonic line profiles with the Raman tensor amplitudes.
This analysis was also based on the large quartic model described above.

Finally, the thermal conductivity was calculated as a function of temperature and crystallite size within the relaxation-time approximation (RTA) \PJ{}{as implemented in {\sc Alamode}~\cite{tadano_2014}}. The phonon lifetimes were calculated from the phonon self-energy including the cubic and quartic anharmonic terms. \SP{}{The RTA provides a solution to the Boltzmann transport equation (BTE) under the assumption that scattering events are independent and can be treated through mode-resolved relaxation times.
To verify the validity of this approximation for $\beta$-FeSi$_2$, we additionally
solved the BTE iteratively.} \PJ{}{These calculations were performed with
{\sc Phono3py}~\cite{togo_2023}. For cross-validation, the additional RTA calculations were performed on the same $\bm{q}$-grid as the BTE calculations, using the implementation provided by {\sc Phono3py}.}

% (see. Appendix~\ref{ThermalAB} for the comparison)\cite{togo_2023}.}
%As shown in Appendix~\ref{ThermalAB}, the iterative BTE results exhibit good agreement with the RTA values, confirming that RTA is sufficiently accurate for this material and for the considered temperature range.}


\section{Results}
\label{sec.result}

\subsection{Crystal structure}
\label{sec.crys}

The $\beta$-FeSi$_2$ structure adopts a base-centered orthorhombic lattice with the space group Cmca (No.~64) as shown in Fig.~\ref{fig.struct}(a).
The unit cell consists of two primitive cells and contains 48 atoms.
Iron (silicon) atoms possess two nonequivalent positions: \mbox{Fe-I} and \mbox{Fe-II} (\mbox{Si-I} and \mbox{Si-II}), presented in Fig.~\ref{fig.struct}(a) as gray and purple (orange and yellow) spheres, respectively.
This crystal structure is derived from the fluorite-type lattice with strongly distorted Si cubes and Fe atoms occupying 
one-half of the central sites.
The Fe-I and Fe-II sites create different layers perpendicular to the $x$ direction, and they are separated by layers containing both Si sites.
Each Fe atom is coordinated by 8 Si atoms with slightly different Fe-Si distances.
The optimized lattice constants ($a = 9.874$~{\AA}, $b = 7.767$~{\AA}, and
$c = 7.811$~{\AA}) agree very well with the experimental data ($a = 9.863$~{\AA}, $b = 7.791$~{\AA}, and $c = 7.833$~{\AA})~\cite{dusausoy_1971}.

Iron atoms occupy the Wyckoff sites 8\textit{d} ($0.2166$, $0$, $0$) and  8\textit{f} ($0$, $0.3072$, $0.1879$), corresponding to \mbox{Fe-I} and \mbox{Fe-II}, respectively. 
Silicon atoms are located at two inequivalent 16\textit{g} positions: ($0.1282$, $0.2737$, $0.0495$) and \mbox{($0.3734$, $0.0445$, $0.2270$)}, assigned as \mbox{Si-I} and \mbox{Si-II}.
The optimized positions of atoms agree very well with the experimental data~\cite{dusausoy_1971} and the previous theoretical studies~\cite{tani_2010,liang_2011}.


\subsection{Lattice dynamics}
\label{sec.lattice}

\begin{figure}[]
    \centering
    \includegraphics[width=\linewidth]{Fig1c.pdf}
\caption{%
The phonon dispersion curves along high symmetry directions obtained within SCPH (for temperatures from $0$ to $1000$~K).
Dashed black lines indicate the phonon dispersions obtained from the harmonic approximation. The white dots indicate Raman-active modes with A$_g$ symmetry.
The vertical plot shows the phonon density of states (DOS) calculated at a reference temperature of $600$~K.
}
  \label{fig.ph_band}
\end{figure}


In Fig.~\ref{fig.ph_band} we present the phonon dispersion relations of $\beta$-FeSi$_2$ along high-symmetry directions in the Brillouin zone [Fig.~\ref{fig.struct}(b)].
Due to 24 atoms in the primitive cell, the phonon spectrum consists of 69 optical modes and three acoustic modes.
The phonon dispersions were calculated within the SCPH approach in the temperature range $0$--$1000$~K (presented by color lines in Fig.~\ref{fig.ph_band}), and they are compared with the results obtained from the harmonic part of the effective potential corresponding to temperature $T=300$~K (indicated by dashed black lines in Fig.~\ref{fig.ph_band}).
As we can see within the SCPH method, the anharmonic effects are rather weak and leads only to small renormalization of phonon frequencies.
Only the highest modes show more pronounced shifts of their frequencies to larger values.
The total and partial element-projected phonon density of states obtained within the harmonic approximation are presented in Fig.~\ref{fig.ph_band}.
Up to around $320$~cm$^{-1}$, the contributions from both elements are very similar, while for higher frequencies the spectrum is dominated by
the Si vibrations.


\begin{table}[!t]
\begin{ruledtabular}
\caption{Calculated and experimental Raman-active modes of $\beta$-FeSi$_2$ with their irreducible representations (IR). Present theoretical results are compared with the previous theoretical data from Ref.~\cite{tani_2010} and experimental results from \mbox{Refs.~\cite{lefki_1991,maeda_2004}.} The experimental frequencies with known symmetries (A$_g$) are shown in bold, while other experimental modes are assigned to the best fitting theoretical values.}
\begin{tabular}{c c c c c}
\textbf{IR} & \multicolumn{4}{c}{\textbf{Frequency (cm$^{-1}$)}} \\
 & Present  & Theor.~\cite{tani_2010} & Exp.~\cite{lefki_1991} & Exp.~\cite{maeda_2004} \\
\hline
B$_{2g}$ & 175.4 & 179 & 176 &  \\
B$_{1g}$ & 176.3 & 185 & 179  &  \\
B$_{1g}$ & 193.3 & 198 &  & 190.6 \\
A$_g$    & 196.6 & 208 & \textbf{195} & \textbf{194.0} \\
A$_g$    & 203.9 & 210 & \textbf{197} & 199.6 \\
B$_{3g}$ & 205.5 & 212 & 200 &  \\
B$_{3g}$ & 226.3 & 236 & 206 & 227.1 \\
B$_{1g}$ & 233.3 & 240 &  &  231.6 \\
B$_{2g}$ & 248.6 & 254 &  &  \\
A$_g$    & 250.1 & 257 & \textbf{247} & \textbf{247.3} \\
A$_g$    & 254.9 & 264 & \textbf{253} & 254.3 \\
B$_{1g}$ & 275.5 & 285 &  &  274.1 \\
B$_{2g}$ & 282.4 & 295 &  &  281.2 \\
B$_{3g}$ & 286.8 & 297 &  &  \\
B$_{1g}$ & 307.1 & 317 &  &  \\
B$_{2g}$ & 312.6 & 326 &  &  311.8 \\
B$_{1g}$ & 319.0 & 324 &  &  \\
B$_{3g}$ & 327.9 & 341 &  & 325.8 \\
B$_{2g}$ & 333.3 & 345 &  &  \\
A$_g$    & 339.3 & 352 & \textbf{346} & 339.5 \\
B$_{2g}$ & 343.0 & 350 &  &  \\
B$_{1g}$ & 353.5 & 366 &  &  \\
B$_{2g}$ & 372.8 & 383 &  & 370.7 \\
B$_{1g}$ & 375.1 & 385 &  &  \\
B$_{3g}$ & 375.2 & 386 &  &  \\
B$_{3g}$ & 385.3 & 401 &  &  \\
A$_g$    & 386.5 & 398 &  & 386.2 \\
B$_{2g}$ & 387.7 & 402 &  & 388.2 \\
A$_g$    & 404.6 & 415 &  & 400.4 \\
B$_{2g}$ & 405.1 & 420 &  &  \\
B$_{1g}$ & 412.7 & 428 &  &  \\
B$_{3g}$ & 418.4 & 431 &  &  \\
B$_{3g}$ & 441.4 & 458 &  & 442.6 \\
B$_{3g}$ & 447.3 & 466 &  & 446.3 \\
A$_g$    & 448.4 & 464 &  &  \\
A$_g$    & 499.1 & 517 &  &  \\
\end{tabular}
\label{T1Raman}
\end{ruledtabular}
\end{table}

%subsection{Raman}

\begin{figure*}
\centering
  \includegraphics[width=\textwidth]{spectrum_RTA+raman.pdf}
\caption{Raman spectrum of $\beta$-FeSi$_2$ calculated within the perturbative approach at three temperatures 
($300$, $600$, $900$~K -- blue, orange, and red lines, respectively). 
The calculated spectrum includes all Raman-active modes. The A$_g$ modes
are indicated by purple vertical lines at peak positions corresponding to T=300~K. 
The frequencies derived from the harmonic approximation at T=300~K are indicated by green lines.
Connecting arrows indicate the correspondence between harmonic and
anharmonic frequencies, demonstrating the frequency shifts due to 
phonon interactions. Experimental values for the A$_g$ modes
based on Ref.~\cite{lefki_1991} are marked with black dashed lines.
}
\label{raman}
\end{figure*}


The phonon spectrum at the $\Gamma$ point consists of 36 Raman modes classified according to the irreducible representations: $9A_\text{g}+9B_\text{1g}+9B_\text{2g}+9B_\text{3g}$. 
Fig.~\ref{raman} shows the Raman spectrum of A$_g$ symmetry calculated for $\beta$-FeSi$_2$ within the perturbative approach at three temperatures $300$, $600$, and $900$~K (solid blue, orange, and red curves, respectively), including third-order and fourth-order anharmonic corrections.
The calculated Raman spectrum includes all Raman-active modes.
The five experimentally observed A$_g$ modes are highlighted by black dashed lines, based on data from Ref.~\cite{lefki_1991}.
Additional peaks not marked with vertical lines correspond to phonon modes with symmetries other than A$_g$.
The frequencies of Raman modes obtained in anharmonic calculations are compared with the previous results calculated within the harmonic approximation and the experimental values in Tab.~\ref{T1Raman}. 
We have marked in bold the experimentally determined A$_g$ modes, which are compared with the calculations.
Since the experimental studies did not provide the accurate assignement of the Raman modes with the B$_{1g}$, B$_{2g}$, and B$_{3g}$ symmetry~\cite{lefki_1991,maeda_2004}, we cannot compare them directly with the theoretical results.
However, in Tab.~\ref{T1Raman} we have assigned the measured frequencies to the best fitting theoretical values without taking into account the symmetry of the modes, except for the known A$_{g}$ modes. 


The impact of anharmonicity on the phonon frequencies is well visible from the comparison of the results obtained within the harmonic approximation 
and from the anharmonic calculation (vertical green and purple lines in Fig.~\ref{raman}, respectively).
Here we show only the A$_g$ modes, which are compared with the experimental results (vertical black dashed lines).
Anharmonic frequencies calculated at $300$~K are indicated by purple lines, while frequencies derived from harmonic approximation are marked with green lines. The grey solid lines connect corresponding modes obtained in both approximations.
In most cases, the results obtained within the harmonic approximation do not agree with the experimental frequencies. 
%Only the modes close to $250$~cm$^{-1}$ and $400$~cm$^{-1}$ correspond well to the experimental values. MS
As we can see, the inclusion of the anharmonic correction leads to a significant modification of the phonon frequencies.
These anharmonic effects are stronger for higher-frequency modes mainly because of
the dominant contribution from Si atoms, which vibrate with larger amplitudes than heavier Fe atoms. 
When atoms move to larger distances the potential deviates more from the harmonic approximation,
and the anharmonic corrections become stronger.


The modification of phonon frequencies observed in Fig.~\ref{raman} is much larger than in the SCPH scheme presented in Fig.~\ref{fig.ph_band}. The SCPH approach includes only the leading-order contribution to 
the phonon self-energy obtained from the quartic anharmonic terms~\cite{tadano_2015}. Therefore, it does not describe fully the changes of phonon frequencies found within the perturbation theory (see Fig.~\ref{raman}).
Especially, it is well visible for two highest A$_g$ modes, which exhibit also the largest line broadening 
and the strongest dependence on temperature.
Therefore, a better agreement with experimentally observed frequencies is visible,
confirming the significant influence of the anharmonicity on the frequencies and line profiles of phonon modes.
In fact, the decrease of phonon frequencies should be even stronger due to thermal expansion, 
which is not included in our calculations.
Within SCPH the frequencies of the highest modes increase with increasing temperature as we see in Fig.~\ref{fig.ph_band}. The comparison of two different approaches applied to study anharmonic properties of $\beta$-FeSi$_2$ shows that the perturbation theory, which includes the cubic and quartic terms, better describes the changes of phonon frequencies with temperature than the SCPH method. \PJ{}{This indicates that the leading-order contribution included in SCPH are not important in this material.}

Additionaly we should nottice that for other than A$_g$ modes we cannot make an unambiguous assignment of theoretical frequencies to experimental ones. Note that the spectrum in Fig.~\ref{fig.ph_band} contains all Raman-active modes. 
The limitation to A$_g$ modes concerns only the indicated positions of the peaks. 
%\SP{}{Furthermore, we present the full spectrum for all Raman active modes with the comparison of the harmonic and anharmonic calculations in the Appendix~\ref{RamanAA}.}
\textcolor{red}{The full Raman spectrum, with the frequencies of the B$_{1g}$, B$_{2g}$ and B$_{3g}$ modes marked, is shown in Appendix A, Fig.~\ref{raman1}. This represents our theoretical prediction of possible Raman-mode assignments, which can be verified in future experiments.}
%This is the theoretical prediction of possible assignment of Raman modes that can be verified in future experiments.  


\subsection{Thermal conductivity}
\label{sec.thermal}


\begin{figure*}[!t]
\centering
  \includegraphics[width=\linewidth]{time3.pdf}
\caption{Phonon lifetimes calculated for three temperatures as a function of phonon frequency. The colors correspond to the phonon branches.}
  \label{thermtime}
\end{figure*}


In this section, we analyze the thermal conductivity tensor of $\beta$-FeSi$_2$ obtained within the RTA approach~\cite{tadano_2018} as a function of temperature
%
\begin{equation}
\kappa_{\text{ph}}^{\mu\nu}(T) = \frac{1}{NV} \sum_{\bm{q},j} c_{\bm{q}j}(T) v_{\bm{q}j}^{\mu} v_{\bm{q}j}^{\nu}\tau_{\bm{q}j}(T),
\end{equation}
% 
where $c_{\bm{q}j}$ is the mode heat capacity and $v_{\bm{q}j}$ is the mode group velocity. 
The relaxation time is approximated by the phonon lifetime $\tau_{\bm{q}j}$
calculated for $j$-th branch at the wave vector $\bm{q}$.
$V$ is the unit cell volume and $N$ is the number of unit cells in the crystal.
The phonon lifetime is calculated using this formula
%
\begin{equation}
\tau_{\bm{q}j}(T)=\frac{1}{2\Gamma_{\bm{q}j}^{\text{anh}}(T)},
\end{equation}
%
where $\Gamma_{\bm{q}j}^{\text{anh}}$ is the anharmonic phonon linewidth obtained from 
the imaginary part of the phonon self-energy within the perturbation theory.

In Fig.~\ref{thermtime}, we present $\tau_{\bm{q}j}$ obtained for three temperatures 300, 600, and 1000~K as a function of frequency. As we see, the acoustic phonons close to the $\Gamma$ point have the longest lifetimes,
which are diminished with increasing frequency reaching local minima around $200$~cm$^{-1}$.
For higher frequencies, phonon lifetimes first increase to local maxima around $300$~cm$^{-1}$ and then decrease to
the lowest values in the range of highest optical modes. The shortest lifetimes correspond to the largest line broadening
observed for the Raman modes in Fig~\ref{raman}. The phonon group velocities 
$v_{\bm{q}j}=\partial\omega_{\bm{q}j}/\partial\bm{q}$, which are obtained by the central difference formula, are presented in Fig.~\ref{thermvelocity}. Their temperature dependence is negligible, therefore, we present only the results for $T=600$~K.
At low frequencies, there are clearly two ranges of group velocities of the acoustic phonons. 
The larger values correspond to the longitudinal modes, while the lower values are obtained from the
transverse acoustic branches. Group velocities of acoustic phonons decrease for larger frequencies
and reach the average values typical for optic branches.

\begin{figure}[!t]
\centering
  \includegraphics[width=\linewidth]{GV.pdf}
\caption{Mode group velocities calculated as a function of phonon frequency. The colors correspond to the phonon branches.
}
  \label{thermvelocity}
\end{figure}

In Fig.~\ref{anizotropy}(a), we present the three diagonal elements of $\kappa_{\text{ph}}^{\mu\nu}$ corresponding to the main directions of the crystal structure. 
They were obtained from the force constants calculated at the base temperature $T=600$~K and the crystallite size $0.1$~$\mu$m to account for boundary-limited phonon transport. 
Due to the orthorhombic symmetry, we observe a small anisotropy in phonon transport in the whole temperature range. 
At low temperatures, the three components of the heat conductivity increase in a very similar way with the $\kappa_{\text{ph}}^{yy}$ element slightly larger than two other components. 
After reaching the maximum, we observe a change in the largest component from $\kappa_{\text{ph}}^{yy}$ to 
$\kappa_{\text{ph}}^{xx}$.    
In Fig.~\ref{anizotropy}(b), the thermal conductivity is shown for three base temperatures, at which the interatomic potential was obtained ($300$~K, $600$~K, and $1000$~K), using the energy expansion up to third- and fourth-order anharmonic terms, and the same structure size of $0.1$~$\mu$m. At lowest temperatures, the thermal conductivity strongly increases, reaching the maximum around $T=180$~K, then it shows a slower decrease with temperature.
The differences between the two levels of approximation are minimal, suggesting that third-order calculations already capture the dominant phonon scattering mechanisms. The dependence on the base temperature is also very weak, showing the changes in the heat conductivity within a few percent. 
\SP{}{Further verification of the reliability of our thermal conductivity results is given in Appendix~\ref{ThermalAB}, where the full BTE calculations show good agreement with RTA and higher-resolution RTA results, demonstrating that the $8\times8\times8$ $\bm{q}$-mesh already provides converged values.}

\begin{figure}[!t]
\centering
  \includegraphics[width=\linewidth]{Anizotropy.pdf}
\caption{(a) The anisotropic thermal conductivity of $\beta$-FeSi$_2$ calculated along the lattice directions at 600~K. (b) The average temperature-dependent thermal conductivity taken at $300$~K, $600$~K, and $1000$~K, including anharmonic corrections up to cubic (A3) and quartic (A4) terms. In both cases the crystallite size is 0.1~$\mu$m.}
  \label{anizotropy}
\end{figure}


In Fig.~\ref{therm}, we fix the base temperature at $600$~K and examine the effect of crystallite size on thermal conductivity, varying it from $0.01$ to $0.5$~$\mu$m.
With decreasing the crystallite size, we observe a shift of the position of the maximum to larger temperatures and a decrease of the thermal conductivity in the entire temperature range.
Theoretical results are compared with several experimental data obtained above the room temperature. 
The measured thermal conductivity depends to a large extent on the sample quality, its purity and the size of the crystalline grains which depends on 
the production processes.
Many measurements were performed using crystallites of micrometric or unknown size ~\cite{waldecker_1973,ito_2002,kim_2003,du_2020}, however, 
numerous attempts to minimize $\kappa$ by reducing grain sizes to $56$~nm~\cite{dabrowski_2019, dabrowski_microstructure_2021}, $30$-$400$~nm~\cite{le_tonquesse_2019}, $50$ and $200$~nm~\cite{abbassi_2021}, or introducing pores into the material~\cite{sam2023improved} 
are also carried out. 
Another way to change the thermal conductivity is to dope $\beta$-FeSi$_2$ with different elements~\cite{ito_2002,kim_2003,du_2020,cheng_2024}, however, this effect is beyond our investigation.

\begin{figure}[!t]
\centering
  \includegraphics[width=\linewidth]{Thermal_conductivity3.pdf}
\caption{
The phonon thermal conductivity of $\beta$-FeSi$_2$: theoretical results for the infinite crystalline size and with boundary conditions, compared with experimental data for different structure sizes.
}
  \label{therm}
\end{figure}
%Fig.~\ref{therm}\cite{abbassi_2021}\cite{waldecker_1973}\cite{dabrowski_2019}\cite{sam2023improved}\cite{kim_2003}\cite{du_2020}\cite{ito_2002}\cite{le_tonquesse_2019}.

We observe a decrease in the thermal conductivity with reducing crystalline grain sizes in all analyzed experimental data. 
For instance, by decreasing the crystallite size to $50$~nm, the thermal conductivity at room temperature was reduced by a factor of $1.7$, what can be  compared to the annealed sample with 200 nm grains~\cite{abbassi_2021}. 
It is worth noting that the rate of decrease in value with increasing temperature in both cases, for grain sizes of $50$~nm and $200$~nm, is significantly different, which is consistent with our calculations. 
The same trend can be observed by comparing the thermal conductivity measured for a sample with bulk crystallite sizes with the thermal conductivity of a sample with grains smaller than 400 nm~\cite{le_tonquesse_2019}.
The theoretical results obtained for the same crystallite size show higher values due to factors not captured in the idealized model, such as crystal imperfection or mechanical strain. Usually, a decrease in the crystallite size is related to an increased concentration of grain boundaries, point defects, and stacking faults that influence the phonon scattering~\cite{le_tonquesse_2019,abbassi_2021}.   

We should note that the total thermal conductivity is a combination
of the lattice and electronic contributions to the heat transport.
In semiconductors, the electronic thermal conductivity is negligible at low temperatures and significantly increases 
only much above the room temperatures~\cite{gu_2020}.
For $\beta$-FeSi$_2$, the electronic thermal conductivity was obtained from the electric conductivity using the Wiedemann-Franz law~\cite{ito_2002,kim_2003,le_tonquesse_2019}.
In the undoped material, its value does not exceed $0.1$~W/mK in the measurement up to $T=950$~K~\cite{kim_2003}.
By doping, the electronic thermal conductivity can be enhanced, and it has a direct impact on the thermoelectric properties of $\beta$-FeSi$_2$ at high temperatures~\cite{ito_2002,kim_2003}.
In the present study, we consider only the phonon contribution to the thermal conductivity,
therefore, agreement with experimental data may deteriorate with increasing temperature.

\section{Summary}
\label{sec.summary}

We performed {\it ab initio} studies on lattice dynamical and thermal transport properties of $\beta$-FeSi$_2$. The effect of anharmonicity was analyzed within two approaches -- the SCPH method and the perturbation theory.
The phonon dispersion curves obtained within SCPH show small renormalization of frequencies comparing to the harmonic approximation. 
The Raman spectra were calculated within the procedure which takes into account the peak intensities obtained from the Raman tensors and the line profiles obtained from the phonon self energy derived within the perturbation theory based on the large, temperature-independent, quartic model fitted to the data from the wide range of temperatures (300-1000~K). 
The anharmonic corrections strongly affect the frequencies and line profiles of some modes and results in overall better agreement with the experimental data. 
We analyzed the phonon lifetimes and group velocities obtained as functions of the phonon frequency.
Then the lattice thermal conductivity was calculated for a broad range of temperatures and grain sizes.
We found a small anisotropy in the phonon thermal transport resulting from the orthorhombic structure and a weak effect of the quartic anharmonic terms. 
The thermal conductivity calculated for various crystalline grain sizes show a good qualitative agreement with the available measurements.

\begin{acknowledgments}
Some figures in this work were rendered using {\sc Vesta}~\cite{momma.izumi.11} software.
This work was partially supported by the Ministry of Education, Youth and Sports of the Czech Republic through the e-INFRA CZ (ID:90254).
\end{acknowledgments}

\appendix

\section{Raman spectrum}
\label{RamanAA}

Based on the polarized Raman measurements reported in Ref.~\cite{maeda_2004}, two Raman peaks were identified as belonging to the A$_g$ symmetry class, and several additional peaks were observed with similar or different polarization dependence. Although the authors of Ref.~\cite{maeda_2004} provided estimates of the relative Raman tensor components, they did not specify which of the remaining modes correspond to the B$_{1g}$, B$_{2g}$, or B$_{3g}$ symmetries. Because of this missing experimental information, a direct symmetry-resolved comparison between the measurement and theory is not currently possible for the non-A$_g$ modes. 
To provide a complete theoretical picture of the B$_g$-type modes, we show here the calculated Raman-active frequencies and intensities for the B$_{1g}$, B$_{2g}$, and B$_{3g}$ symmetries only. 
%These results represent the predicted Raman modes for the B$_g$ symmetries in $\beta$-FeSi$_2$. 
\textcolor{red} {Fig.~\ref{raman1} shows the predicted Raman modes for the B$_g$ symmetries in $\beta$-FeSi$_2$. The results obtained within the harmonic approximation are compared with the anharmonic perturbation theory calculations which provides both, frequency shifts and predicted line profiles of the modes.}
Although the experimentally measured peaks cannot be directly assigned to these symmetries due to the lack of polarization-resolved data, the theoretical predictions provide a reference for comparison. Matching the measured frequencies to the closest theoretical B$_g$ modes (Table~\ref{T1Raman}) allows for a tentative assignment, which can guide future polarization-resolved Raman experiments aimed at determining the precise symmetry of the unresolved peaks.

\begin{figure*}[t]
\centering
  \includegraphics[width=\linewidth]{spectrum_RTA+raman_Bi_modes.pdf}
\caption{Raman spectrum of $\beta$-FeSi$_2$ calculated at three temperatures 
($300$, $600$, $900$~K -- blue, orange, and red lines, respectively). 
The calculated spectrum includes all Raman-active modes. The B$_{ig}$ modes
are indicated by purple vertical lines at peak positions corresponding to T=300~K. 
The frequencies derived from the harmonic approximation at T=300~K are indicated 
by green, yellow and pink lines.
Connecting arrows indicate the correspondence between harmonic and anharmonic 
frequencies, demonstrating the frequency shifts due to phonon interactions.}
\label{raman1}
\end{figure*}
\section{Thermal conductivity obtained from BTE and RTA}
\label{ThermalAB}

The thermal conductivity was computed by solving the full BTE on the largest feasible $\bm{q}$-point grid, $8\times8\times8$, and compared with the corresponding RTA results obtained on the same grid. As shown in Fig.~\ref{bte}, the difference between the components of the thermal conductivity tensor obtained within BTE and RTA at this resolution is very small, indicating a good agreement between these two approaches.
%THIS PART SHOULD GO RATHER TO THE RESPONSE
%Extending the full BTE calculation to larger grids is computationally prohibitive: the computational cost of BTE is roughly two orders of %magnitude higher than that of RTA, and the required memory and runtime exceed our available resources. 
Moreover, we performed an additional calculation using RTA on a denser $20\times20\times20$ grid. As seen in the Fig.~\ref{bte}, the higher-resolution data remain in a good agreement with both the BTE and RTA results for the $8\times8\times8$ grid.
It shows that the $8\times8\times8$ mesh already provides good results for this structure and confirms reliability of the calculations.

\begin{figure}[!h]
\centering
  \includegraphics[width=\linewidth]{BTEvsRTAvsANP.pdf}
\caption{Thermal conductivity of $\beta$-FeSi$_2$ obtained within the BTE and RTA methods using the Phono3py software on the $8\times8\times8$ q-point grid, compared with the RTA results computed with ALAMODE on a denser $20\times20\times20$ grid.}
\label{bte}
\end{figure}

%\section*{Data availability}
%The data that support the findings of this article are openly available~\footnote{give me DOI}.
% see https://journals.aps.org/authors/data-availability-statements#citation

\bibliography{refs.bib}
%\bibliographystyle{ieeetr}


\end{document}

\documentclass[%
%reprint,
superscriptaddress,
%groupedaddress,
longbibliography,
%unsortedaddress,
%runinaddress,
%frontmatterverbose, 
%preprint,
%preprintnumbers,
%nofootinbib,
nobibnotes,
%bibnotes,
amsmath,amssymb,
aps,
%pra,
prb,
%rmp,
%prstab,
%prstper,
%showkeys,
floatfix,
twocolumn
]{revtex4-2}

\usepackage{graphicx}% Include figure files
\usepackage{calc}% Calculate margins
\usepackage{dcolumn}% Align table columns on decimal point
\usepackage{bm}% bold math

\usepackage[urlcolor=blue,colorlinks=true,citecolor=blue,linkcolor=blue,pdfstartview={FitH},bookmarks=false]{hyperref} % add hypertext capabilities

%\usepackage[mathlines]{lineno} % Enable numbering of text and display math
% \linenumbers\relax % Commence numbering lines

% \usepackage[showframe,%Uncomment any one of the following lines to test 
% %scale=0.7, marginratio={1:1, 2:3}, ignoreall,% default settings
% %text={7in,10in},centering,
% %margin=1.5in,
% % total={6.5in,8.75in}, top=1.2in, left=0.9in, includefoot,
% % height=10in,a5paper,hmargin={3cm,0.8in},
% ]{geometry}

\usepackage{amsmath}
\usepackage{amssymb}
%\usepackage{orcidlink}
\usepackage{xcolor}
%\usepackage{datetime}
\usepackage[normalem]{ulem}

% Change tracking commands
\newcommand{\trackchange}[3]{\textcolor{#3}{\sout{#1}#2}}  % Full color strikeout, insert
%\renewcommand{\trackchange}[3]{\textcolor{#3}{#2}}        % Just color silent remove and insert
%\renewcommand{\trackchange}[3]{{#2}}                      % No indication, silent remove and insert

% Author marker definitions

\definecolor{myblue}{RGB}{0,127,85}
% \definecolor{violet}{RGB}{102,0,204}
% \definecolor{orange}{RGB}{255,128,0}
% \definecolor{green}{RGB}{0,128,0}
\newcommand{\DL}[1]{\trackchange{}{#1}{blue}}
\newcommand{\AP}[1]{\trackchange{}{#1}{red}}
\newcommand{\PJ}[2]{\trackchange{#1}{#2}{orange}}
\newcommand{\JL}[2]{\trackchange{#1}{#2}{myblue}}
\newcommand{\SP}[2]{\trackchange{#1}{#2}{blue}}
\newcommand{\PP}[2]{\trackchange{#1}{#2}{teal}}
\newcommand{\AI}[2]{\trackchange{#1}{#2}{olive}}
\newcommand{\MS}[1]{\trackchange{}{#1}{purple}}

\newcommand{\TODO}[1]{\textcolor{red}{TODO: #1}}

\sloppy

\begin{document}

\title{Ab initio study of the anharmonic properties and thermal conductivity in $\beta$-FeSi$_2$}

\author{Svitlana~Pastukh}
\email[e-mail: ]{svitlana.pastukh@ifj.edu.pl}
\affiliation{Institute of Nuclear Physics, Polish Academy of Sciences, ul. W. E. Radzikowskiego 152, 31-342 Krak\'{o}w, Poland}

\author{Ma\l{}gorzata~Sternik}
\affiliation{Institute of Nuclear Physics, Polish Academy of Sciences, ul. W. E. Radzikowskiego 152, 31-342 Krak\'{o}w, Poland}

\author{Pawe\l{}~T.~Jochym}
\affiliation{Institute of Nuclear Physics, Polish Academy of Sciences, ul. W. E. Radzikowskiego 152, 31-342 Krak\'{o}w, Poland}


\author{Jan~\L{}a\.{z}ewski}
\affiliation{Institute of Nuclear Physics, Polish Academy of Sciences, ul. W. E. Radzikowskiego 152, 31-342 Krak\'{o}w, Poland}

\author{Andrzej~Ptok}
\affiliation{Institute of Nuclear Physics, Polish Academy of Sciences, ul. W. E. Radzikowskiego 152, 31-342 Krak\'{o}w, Poland}

\author{Svetoslav~Stankov}
\affiliation{Institute for Photon Science and Synchrotron Radiation, Karlsruhe Institute of Technology, D-76131 Karlsruhe, Germany}
\affiliation{Laboratory for Applications of Synchrotron Radiation, Karlsruhe Institute of Technology, D-76131 Karlsruhe, Germany}

\author{Przemys\l{}aw~Piekarz}
\affiliation{Institute of Nuclear Physics, Polish Academy of Sciences, ul. W. E. Radzikowskiego 152, 31-342 Krak\'{o}w, Poland}

\date{\today}

\begin{abstract}

Iron silicides are good candidates for applications in  optoelectronic and thermoelectric devices.
Lattice dynamical properties and thermal conductivity in the $\beta$-FeSi$_2$ semiconductor
are investigated with the first-principles computational methods. 
Phonon dispersion relations are calculated via the
temperature-dependent effective potential method and self-consistent phonon theory. 
To properly model thermal transport, we explicitly consider
the impact of phonon-phonon interactions by analyzing
anharmonic contributions to the phonon self-energy. 
This yields temperature-dependent phonon frequencies and linewidths,
reflecting the finite lifetime of phonons due to scattering
processes. The calculated phonon frequencies and line profiles are used to obtain 
the Raman spectra, which shows good agreement with the experimental data. 
We revealed an enhanced anharmonic behaviour of the Raman modes with the highest frequencies.  
The lattice thermal conductivity is then obtained as a function of temperature and crystallite size within
the relaxation-time approximation.
Phonon transport shows a small anisotropy due to the orthorhombic structure and a very weak dependence
on the quartic anharmonic corrections. The results obtained for an infinite material and for several crystallite sizes
were analyzed and compared with the available experimental data.
\end{abstract}

\maketitle


\section{Introduction}

The comprehensive determination of important physical properties of
crystals, such as thermal expansion, lattice thermal
conductivity or structural phase transitions, requires a fundamental 
understanding of the anharmonic effects.
Although the investigation of anharmonic interactions in crystals has
attracted a considerable interest for decades~\cite{cowley_1968}, a substantial progress
has only recently been achieved thanks to advances in theoretical and
numerical methods and increased computational power.
Now, phonon frequencies, lifetimes, and heat transfer in a wide range of
materials
can be quantitatively predicted using the available computational resources
based on the density functional theory (DFT)~\cite{lindsay_2013,mcgaughey_2019,lindsay_2019}.
In the case of strongly anharmonic systems, the self-consistent phonon
(SCPH) theory~\cite{tadano_2015} as well as the perturbative approach~\cite{tadano_2018}, using higher
order interatomic force constants derived from the fitting to the displacement force data obtained
with DFT, have proven to be successful.

Transition-metal silicides are promising materials for fabrication of electronic components
designed for integration with silicon-based circuits~\cite{murarka_1995}.
At room temperature, iron disilicide ($\beta$-FeSi$_2$) is a
direct-bandgap semiconductor~\cite{bost_1985}, making this material a good
candidate for application in optoelectronic devices such as infrared
detectors or light emitters~\cite{bost_1988}. The development of light-emitting diodes utilizing FeSi$_2$/Si heterostructures has been successfully demonstrated~\cite{leong_1997,suemasu_2001}. Due to
a high thermal stability and strong light absorption, FeSi$_2$ is
also a suitable photovoltaic material~\cite{powalla_1993,liu_2006,okuhara_2017}.

$\beta$-FeSi$_2$ crystallizes in the base-centered orthorhombic
lattice~\cite{dusausoy_1971} transforming to the
tetragonal metallic $\alpha$-FeSi$_{2}$ phase around $1200$~K~\cite{starke_2002}.
Optical studies indicated a direct band gap of the values
$0.85$--$0.89$~eV~\cite{bost_1988,dimitriadis_1990,arushanov_1995,wan_2003},
however, the {\it ab initio} calculations predicted a smaller indirect gap close to 0.8 eV~\cite{christensen_1990}. 
The existence of such an indirect gap was then confirmed by the optical linear transmittance measurements at
low temperatures~\cite{giannini_1992}. As shown by first-principles studies,
the character of the band gap is very sensitive to the orientation 
of a crystal grown on silicon~\cite{clark_1998}.

$\beta$-FeSi$_2$ belongs also to good thermoelectric materials~\cite{ware_1964}, with potential applications resulting from its chemical stability up to high temperatures, nontoxicity, and low cost of preparation~\cite{yamada_2012,nozariasbmarz_2017}. 
It has already been implemented in cars~\cite{birkholz_1988} and portable power sources~\cite{uemura_1989}. 
Its thermoelectric performance can be improved by doping~\cite{ito_2001,tani_2001,kim_2003,chen_2005,pandey_2013,le_tonquesse_2019}, 
which enhances the electric transport and reduces the thermal conductivity~\cite{waldecker_1973,du_2019,du_2020}.  
The thermal conductivity can be also reduced by the modification of microstructure~\cite{ail_2015} or by
nanostructurization~\cite{watanabe_2017,taniguchi_2017,hsin_2017,abbassi_2021}.

The lattice thermal conductivity is directly connected with anharmonic effects and phonon scattering processes.
The vibrational properties of \mbox{$\beta$-FeSi$_2$} were studied by the infrared and Raman spectroscopy~\cite{lefki_1991,guizzetti_1997,maeda_2004,baleva_2008,liu_2011,maeda_2011}. The observed anisotropy in the phonon spectra results from the enhanced sensitivity of the infrared and Raman features to the local lattice distortions~\cite{guizzetti_1997}. The Fe phonon density of states was measured by nuclear inelastic scattering (NIS), showing a good agreement with the density functional theory (DFT) calculations~\cite{walterfang_2005}. Using the DFT approach, the phonon dispersion curves, phonon density of states, as well as various thermodynamic properties were obtained within the harmonic approximation~\cite{tani_2010,liang_2011}. The extended Klemens model was applied to study
the anharmonic effect on phonon frequencies and linewidths observed by the Raman spectroscopy~\cite{zhang_2023}.
The impact of nanostructurization on lattice dynamics was explored in the $\beta$-FeSi$_2$ nanorods grown on the Si(110) surface by the NIS and {\it ab initio} methods~\cite{kalt_2022}.

In this work, we investigate the lattice dynamical properties of $\beta$-FeSi$_2$ 
using the DFT calculations. We study the effect of anharmonic terms in the temperature-dependent potential on phonon frequencies and lifetimes. We focus on the Raman modes, comparing the theoretical results with the experimental data.
The thermal conductivity is derived in a broad temperature range and the effect of crystallite size is analyzed.



This study is structured as follows.
In Sec.~\ref{sec.com} we describe the details of computational methods.
Next, in Sec.~\ref{sec.result} we present and discuss our results.
In particular we present the crystal structure (Sec.~\ref{sec.crys}) and lattice dynamics (Sec.~\ref{sec.lattice}).
We investigate also the thermal conductivity comparing the obtained results with the available experimental data (Sec.~\ref{sec.thermal}).
Finally, Sec.~\ref{sec.summary} summarizes our key findings and conclusions.

\section{Calculation method}
\label{sec.com}


The calculations were performed using the projector augmented-wave potentials~\cite{blochl_1994} and the generalized gradient approximation~\cite{perdew_1996} implemented in the Vienna Ab initio Simulation Package (VASP)~\cite{kresse.hafner.94,kresse.furthmuller.96,kresse.joubert.99}. 
The lattice parameters and atomic positions were optimized in the ${\bm a} \times ({\bm b}-{\bm c}) \times ({\bm b}+{\bm c})$ supercell containing 32 formula units and four primitive cells.
The integration in the reciprocal space was conducted using the $2 \times 2 \times 2$ Monkhorst--Pack mesh~\cite{monkhorst_1976} and the cut-off energy was set to $500$~eV. For convergence conditions, we set the energy change below $10^{-5}$ and $10^{-8}$ for the ionic and electronic loops, respectively. 

The lattice dynamical properties were studied within the temperature-dependent effective potential (TDEP) approach~\cite{hellman_2013}. The atomic potential with the third and fourth order anharmonic terms was derived from interatomic forces induced by displacements of all atoms at finite temperatures.
The sets of atomic displacements were generated by the high efficiency configuration space sampling (HECSS)~\cite{jochym_2021} and forces were obtained by VASP. The interatomic force constants and phonon frequencies were calculated with the {\sc Alamode} software~\cite{tadano_2014}.

Furthermore, we have attempted to construct a {\emph{temperature independent}} 
anharmonic model. We have used combined data from all investigated temperatures 
(300, 600 and 1000~K) and fitted a large (over 15~000 free parameters), fourth-order 
interaction model to this dataset. 
Subsequently, we have used this model to calculate line profiles and positions of Raman-active modes
at multiple temperatures.

The changes in phonon frequencies induced by the anharmonic effects were investigated within two approaches.
First, the impact of the quartic anharmonic terms was included using the SCPH theory~\cite{tadano_2015}.
Second, the mode profiles (frequency shifts and line widths) were determined from the real and imaginary parts of the phonon self-energy resulting from the cubic and quartic anharmonic terms of the above mentioned large model~\cite{tadano_2018}. 
The longitudinal optic-transverse optic (LO-TO) splitting was also evaluated, using the static dielectric tensor and Born effective charges calculated within density functional perturbation theory~\cite{gajdos_2006}.

\begin{figure}[]
    \centering
    \includegraphics[width=\linewidth]{fig1_new.png}
\caption{%
(a) The conventional unit cell of $\beta$-FeSi$_2$ (with Cmca symmetry) and (b) the corresponding Brillouin zone with selected high-symmetry points.
}
  \label{fig.struct}
\end{figure}

% To further characterize the vibrational properties, Raman-active scattering was investigated using the Phonopy-Spectroscopy package~\cite{skelton_2017}. This enabled the identification of Raman-active modes and the calculation of Raman tensors. Anharmonic force constants, obtained from calculations using {\sc Alamode}, were then used to obtain theoretical line profiles for the Raman modes. The presented Raman scattering spectra combine these anharmonic line profiles with the Raman tensor amplitudes.
% This analysis was also based on the large quartic model mentioned above.

% Finally, the thermal conductivity was obtained as a function of temperature and crystallite size within the relaxation-time approximation (RTA) \PJ{}{as implemented in {\sc Alamode}\cite{tadano_2014}}. The phonon lifetimes were calculated from the phonon self-energy including the cubic and quartic anharmonic terms. \SP{}{The RTA provides a solution of the Boltzmann transport equation (BTE) under the assumption that scattering events are independent and can be treated through mode-resolved relaxation times.
% To verify the validity of this approximation for $\beta$-FeSi$_2$, we have additionally
% executed an iterative solution of the BTE.}\PJ{}{These calculations were performed with
% {\sc Phono3py}\cite{togo_2023}. For cross-validation, the additional RTA calculations were performed on the same $\bm{q}$-grid as BTE with implementation provided by {\sc Phono3py}.}.

To further characterize the vibrational properties, Raman scattering was investigated using the Phonopy-Spectroscopy package~\cite{skelton_2017}. This enabled the identification of Raman-active modes and the calculation of Raman tensors. Anharmonic force constants, derived from calculations using {\sc Alamode}, were then used to obtain theoretical line profiles for the Raman modes. The presented Raman scattering spectra combine these anharmonic line profiles with the Raman tensor amplitudes.
This analysis was also based on the large quartic model described above.

Finally, the thermal conductivity was calculated as a function of temperature and crystallite size within the relaxation-time approximation (RTA) \PJ{}{as implemented in {\sc Alamode}~\cite{tadano_2014}}. The phonon lifetimes were calculated from the phonon self-energy including the cubic and quartic anharmonic terms. \SP{}{The RTA provides a solution to the Boltzmann transport equation (BTE) under the assumption that scattering events are independent and can be treated through mode-resolved relaxation times.
To verify the validity of this approximation for $\beta$-FeSi$_2$, we additionally
solved the BTE iteratively.} \PJ{}{These calculations were performed with
{\sc Phono3py}~\cite{togo_2023}. For cross-validation, the additional RTA calculations were performed on the same $\bm{q}$-grid as the BTE calculations, using the implementation provided by {\sc Phono3py}.}

% (see. Appendix~\ref{ThermalAB} for the comparison)\cite{togo_2023}.}
%As shown in Appendix~\ref{ThermalAB}, the iterative BTE results exhibit good agreement with the RTA values, confirming that RTA is sufficiently accurate for this material and for the considered temperature range.}


\section{Results}
\label{sec.result}

\subsection{Crystal structure}
\label{sec.crys}

The $\beta$-FeSi$_2$ structure adopts a base-centered orthorhombic lattice with the space group Cmca (No.~64) as shown in Fig.~\ref{fig.struct}(a).
The unit cell consists of two primitive cells and contains 48 atoms.
Iron (silicon) atoms possess two nonequivalent positions: \mbox{Fe-I} and \mbox{Fe-II} (\mbox{Si-I} and \mbox{Si-II}), presented in Fig.~\ref{fig.struct}(a) as gray and purple (orange and yellow) spheres, respectively.
This crystal structure is derived from the fluorite-type lattice with strongly distorted Si cubes and Fe atoms occupying 
one-half of the central sites.
The Fe-I and Fe-II sites create different layers perpendicular to the $x$ direction, and they are separated by layers containing both Si sites.
Each Fe atom is coordinated by 8 Si atoms with slightly different Fe-Si distances.
The optimized lattice constants ($a = 9.874$~{\AA}, $b = 7.767$~{\AA}, and
$c = 7.811$~{\AA}) agree very well with the experimental data ($a = 9.863$~{\AA}, $b = 7.791$~{\AA}, and $c = 7.833$~{\AA})~\cite{dusausoy_1971}.

Iron atoms occupy the Wyckoff sites 8\textit{d} ($0.2166$, $0$, $0$) and  8\textit{f} ($0$, $0.3072$, $0.1879$), corresponding to \mbox{Fe-I} and \mbox{Fe-II}, respectively. 
Silicon atoms are located at two inequivalent 16\textit{g} positions: ($0.1282$, $0.2737$, $0.0495$) and \mbox{($0.3734$, $0.0445$, $0.2270$)}, assigned as \mbox{Si-I} and \mbox{Si-II}.
The optimized positions of atoms agree very well with the experimental data~\cite{dusausoy_1971} and the previous theoretical studies~\cite{tani_2010,liang_2011}.


\subsection{Lattice dynamics}
\label{sec.lattice}

\begin{figure}[]
    \centering
    \includegraphics[width=\linewidth]{Fig1c.pdf}
\caption{%
The phonon dispersion curves along high symmetry directions obtained within SCPH (for temperatures from $0$ to $1000$~K).
Dashed black lines indicate the phonon dispersions obtained from the harmonic approximation. The white dots indicate Raman-active modes with A$_g$ symmetry.
The vertical plot shows the phonon density of states (DOS) calculated at a reference temperature of $600$~K.
}
  \label{fig.ph_band}
\end{figure}


In Fig.~\ref{fig.ph_band} we present the phonon dispersion relations of $\beta$-FeSi$_2$ along high-symmetry directions in the Brillouin zone [Fig.~\ref{fig.struct}(b)].
Due to 24 atoms in the primitive cell, the phonon spectrum consists of 69 optical modes and three acoustic modes.
The phonon dispersions were calculated within the SCPH approach in the temperature range $0$--$1000$~K (presented by color lines in Fig.~\ref{fig.ph_band}), and they are compared with the results obtained from the harmonic part of the effective potential corresponding to temperature $T=300$~K (indicated by dashed black lines in Fig.~\ref{fig.ph_band}).
As we can see within the SCPH method, the anharmonic effects are rather weak and leads only to small renormalization of phonon frequencies.
Only the highest modes show more pronounced shifts of their frequencies to larger values.
The total and partial element-projected phonon density of states obtained within the harmonic approximation are presented in Fig.~\ref{fig.ph_band}.
Up to around $320$~cm$^{-1}$, the contributions from both elements are very similar, while for higher frequencies the spectrum is dominated by
the Si vibrations.


\begin{table}[!t]
\begin{ruledtabular}
\caption{Calculated and experimental Raman-active modes of $\beta$-FeSi$_2$ with their irreducible representations (IR). Present theoretical results are compared with the previous theoretical data from Ref.~\cite{tani_2010} and experimental results from \mbox{Refs.~\cite{lefki_1991,maeda_2004}.} The experimental frequencies with known symmetries (A$_g$) are shown in bold, while other experimental modes are assigned to the best fitting theoretical values.}
\begin{tabular}{c c c c c}
\textbf{IR} & \multicolumn{4}{c}{\textbf{Frequency (cm$^{-1}$)}} \\
 & Present  & Theor.~\cite{tani_2010} & Exp.~\cite{lefki_1991} & Exp.~\cite{maeda_2004} \\
\hline
B$_{2g}$ & 175.4 & 179 & 176 &  \\
B$_{1g}$ & 176.3 & 185 & 179  &  \\
B$_{1g}$ & 193.3 & 198 &  & 190.6 \\
A$_g$    & 196.6 & 208 & \textbf{195} & \textbf{194.0} \\
A$_g$    & 203.9 & 210 & \textbf{197} & 199.6 \\
B$_{3g}$ & 205.5 & 212 & 200 &  \\
B$_{3g}$ & 226.3 & 236 & 206 & 227.1 \\
B$_{1g}$ & 233.3 & 240 &  &  231.6 \\
B$_{2g}$ & 248.6 & 254 &  &  \\
A$_g$    & 250.1 & 257 & \textbf{247} & \textbf{247.3} \\
A$_g$    & 254.9 & 264 & \textbf{253} & 254.3 \\
B$_{1g}$ & 275.5 & 285 &  &  274.1 \\
B$_{2g}$ & 282.4 & 295 &  &  281.2 \\
B$_{3g}$ & 286.8 & 297 &  &  \\
B$_{1g}$ & 307.1 & 317 &  &  \\
B$_{2g}$ & 312.6 & 326 &  &  311.8 \\
B$_{1g}$ & 319.0 & 324 &  &  \\
B$_{3g}$ & 327.9 & 341 &  & 325.8 \\
B$_{2g}$ & 333.3 & 345 &  &  \\
A$_g$    & 339.3 & 352 & \textbf{346} & 339.5 \\
B$_{2g}$ & 343.0 & 350 &  &  \\
B$_{1g}$ & 353.5 & 366 &  &  \\
B$_{2g}$ & 372.8 & 383 &  & 370.7 \\
B$_{1g}$ & 375.1 & 385 &  &  \\
B$_{3g}$ & 375.2 & 386 &  &  \\
B$_{3g}$ & 385.3 & 401 &  &  \\
A$_g$    & 386.5 & 398 &  & 386.2 \\
B$_{2g}$ & 387.7 & 402 &  & 388.2 \\
A$_g$    & 404.6 & 415 &  & 400.4 \\
B$_{2g}$ & 405.1 & 420 &  &  \\
B$_{1g}$ & 412.7 & 428 &  &  \\
B$_{3g}$ & 418.4 & 431 &  &  \\
B$_{3g}$ & 441.4 & 458 &  & 442.6 \\
B$_{3g}$ & 447.3 & 466 &  & 446.3 \\
A$_g$    & 448.4 & 464 &  &  \\
A$_g$    & 499.1 & 517 &  &  \\
\end{tabular}
\label{T1Raman}
\end{ruledtabular}
\end{table}

%subsection{Raman}

\begin{figure*}
\centering
  \includegraphics[width=\textwidth]{spectrum_RTA+raman.pdf}
\caption{Raman spectrum of $\beta$-FeSi$_2$ calculated within the perturbative approach at three temperatures 
($300$, $600$, $900$~K -- blue, orange, and red lines, respectively). 
The calculated spectrum includes all Raman-active modes. The A$_g$ modes
are indicated by purple vertical lines at peak positions corresponding to T=300~K. 
The frequencies derived from the harmonic approximation at T=300~K are indicated by green lines.
Connecting arrows indicate the correspondence between harmonic and
anharmonic frequencies, demonstrating the frequency shifts due to 
phonon interactions. Experimental values for the A$_g$ modes
based on Ref.~\cite{lefki_1991} are marked with black dashed lines.
}
\label{raman}
\end{figure*}


The phonon spectrum at the $\Gamma$ point consists of 36 Raman modes classified according to the irreducible representations: $9A_\text{g}+9B_\text{1g}+9B_\text{2g}+9B_\text{3g}$. 
Fig.~\ref{raman} shows the Raman spectrum of A$_g$ symmetry calculated for $\beta$-FeSi$_2$ within the perturbative approach at three temperatures $300$, $600$, and $900$~K (solid blue, orange, and red curves, respectively), including third-order and fourth-order anharmonic corrections.
The calculated Raman spectrum includes all Raman-active modes.
The five experimentally observed A$_g$ modes are highlighted by black dashed lines, based on data from Ref.~\cite{lefki_1991}.
Additional peaks not marked with vertical lines correspond to phonon modes with symmetries other than A$_g$.
The frequencies of Raman modes obtained in anharmonic calculations are compared with the previous results calculated within the harmonic approximation and the experimental values in Tab.~\ref{T1Raman}. 
We have marked in bold the experimentally determined A$_g$ modes, which are compared with the calculations.
Since the experimental studies did not provide the accurate assignement of the Raman modes with the B$_{1g}$, B$_{2g}$, and B$_{3g}$ symmetry~\cite{lefki_1991,maeda_2004}, we cannot compare them directly with the theoretical results.
However, in Tab.~\ref{T1Raman} we have assigned the measured frequencies to the best fitting theoretical values without taking into account the symmetry of the modes, except for the known A$_{g}$ modes. 


The impact of anharmonicity on the phonon frequencies is well visible from the comparison of the results obtained within the harmonic approximation 
and from the anharmonic calculation (vertical green and purple lines in Fig.~\ref{raman}, respectively).
Here we show only the A$_g$ modes, which are compared with the experimental results (vertical black dashed lines).
Anharmonic frequencies calculated at $300$~K are indicated by purple lines, while frequencies derived from harmonic approximation are marked with green lines. The grey solid lines connect corresponding modes obtained in both approximations.
In most cases, the results obtained within the harmonic approximation do not agree with the experimental frequencies. 
%Only the modes close to $250$~cm$^{-1}$ and $400$~cm$^{-1}$ correspond well to the experimental values. MS
As we can see, the inclusion of the anharmonic correction leads to a significant modification of the phonon frequencies.
These anharmonic effects are stronger for higher-frequency modes mainly because of
the dominant contribution from Si atoms, which vibrate with larger amplitudes than heavier Fe atoms. 
When atoms move to larger distances the potential deviates more from the harmonic approximation,
and the anharmonic corrections become stronger.


The modification of phonon frequencies observed in Fig.~\ref{raman} is much larger than in the SCPH scheme presented in Fig.~\ref{fig.ph_band}. The SCPH approach includes only the leading-order contribution to 
the phonon self-energy obtained from the quartic anharmonic terms~\cite{tadano_2015}. Therefore, it does not describe fully the changes of phonon frequencies found within the perturbation theory (see Fig.~\ref{raman}).
Especially, it is well visible for two highest A$_g$ modes, which exhibit also the largest line broadening 
and the strongest dependence on temperature.
Therefore, a better agreement with experimentally observed frequencies is visible,
confirming the significant influence of the anharmonicity on the frequencies and line profiles of phonon modes.
In fact, the decrease of phonon frequencies should be even stronger due to thermal expansion, 
which is not included in our calculations.
Within SCPH the frequencies of the highest modes increase with increasing temperature as we see in Fig.~\ref{fig.ph_band}. The comparison of two different approaches applied to study anharmonic properties of $\beta$-FeSi$_2$ shows that the perturbation theory, which includes the cubic and quartic terms, better describes the changes of phonon frequencies with temperature than the SCPH method. \PJ{}{This indicates that the leading-order contribution included in SCPH are not important in this material.}

Additionaly we should nottice that for other than A$_g$ modes we cannot make an unambiguous assignment of theoretical frequencies to experimental ones. Note that the spectrum in Fig.~\ref{fig.ph_band} contains all Raman-active modes. 
The limitation to A$_g$ modes concerns only the indicated positions of the peaks. 
%\SP{}{Furthermore, we present the full spectrum for all Raman active modes with the comparison of the harmonic and anharmonic calculations in the Appendix~\ref{RamanAA}.}
\textcolor{red}{The full Raman spectrum, with the frequencies of the B$_{1g}$, B$_{2g}$ and B$_{3g}$ modes marked, is shown in Appendix A, Fig.~\ref{raman1}. This represents our theoretical prediction of possible Raman-mode assignments, which can be verified in future experiments.}
%This is the theoretical prediction of possible assignment of Raman modes that can be verified in future experiments.  


\subsection{Thermal conductivity}
\label{sec.thermal}


\begin{figure*}[!t]
\centering
  \includegraphics[width=\linewidth]{time3.pdf}
\caption{Phonon lifetimes calculated for three temperatures as a function of phonon frequency. The colors correspond to the phonon branches.}
  \label{thermtime}
\end{figure*}


In this section, we analyze the thermal conductivity tensor of $\beta$-FeSi$_2$ obtained within the RTA approach~\cite{tadano_2018} as a function of temperature
%
\begin{equation}
\kappa_{\text{ph}}^{\mu\nu}(T) = \frac{1}{NV} \sum_{\bm{q},j} c_{\bm{q}j}(T) v_{\bm{q}j}^{\mu} v_{\bm{q}j}^{\nu}\tau_{\bm{q}j}(T),
\end{equation}
% 
where $c_{\bm{q}j}$ is the mode heat capacity and $v_{\bm{q}j}$ is the mode group velocity. 
The relaxation time is approximated by the phonon lifetime $\tau_{\bm{q}j}$
calculated for $j$-th branch at the wave vector $\bm{q}$.
$V$ is the unit cell volume and $N$ is the number of unit cells in the crystal.
The phonon lifetime is calculated using this formula
%
\begin{equation}
\tau_{\bm{q}j}(T)=\frac{1}{2\Gamma_{\bm{q}j}^{\text{anh}}(T)},
\end{equation}
%
where $\Gamma_{\bm{q}j}^{\text{anh}}$ is the anharmonic phonon linewidth obtained from 
the imaginary part of the phonon self-energy within the perturbation theory.

In Fig.~\ref{thermtime}, we present $\tau_{\bm{q}j}$ obtained for three temperatures 300, 600, and 1000~K as a function of frequency. As we see, the acoustic phonons close to the $\Gamma$ point have the longest lifetimes,
which are diminished with increasing frequency reaching local minima around $200$~cm$^{-1}$.
For higher frequencies, phonon lifetimes first increase to local maxima around $300$~cm$^{-1}$ and then decrease to
the lowest values in the range of highest optical modes. The shortest lifetimes correspond to the largest line broadening
observed for the Raman modes in Fig~\ref{raman}. The phonon group velocities 
$v_{\bm{q}j}=\partial\omega_{\bm{q}j}/\partial\bm{q}$, which are obtained by the central difference formula, are presented in Fig.~\ref{thermvelocity}. Their temperature dependence is negligible, therefore, we present only the results for $T=600$~K.
At low frequencies, there are clearly two ranges of group velocities of the acoustic phonons. 
The larger values correspond to the longitudinal modes, while the lower values are obtained from the
transverse acoustic branches. Group velocities of acoustic phonons decrease for larger frequencies
and reach the average values typical for optic branches.

\begin{figure}[!t]
\centering
  \includegraphics[width=\linewidth]{GV.pdf}
\caption{Mode group velocities calculated as a function of phonon frequency. The colors correspond to the phonon branches.
}
  \label{thermvelocity}
\end{figure}

In Fig.~\ref{anizotropy}(a), we present the three diagonal elements of $\kappa_{\text{ph}}^{\mu\nu}$ corresponding to the main directions of the crystal structure. 
They were obtained from the force constants calculated at the base temperature $T=600$~K and the crystallite size $0.1$~$\mu$m to account for boundary-limited phonon transport. 
Due to the orthorhombic symmetry, we observe a small anisotropy in phonon transport in the whole temperature range. 
At low temperatures, the three components of the heat conductivity increase in a very similar way with the $\kappa_{\text{ph}}^{yy}$ element slightly larger than two other components. 
After reaching the maximum, we observe a change in the largest component from $\kappa_{\text{ph}}^{yy}$ to 
$\kappa_{\text{ph}}^{xx}$.    
In Fig.~\ref{anizotropy}(b), the thermal conductivity is shown for three base temperatures, at which the interatomic potential was obtained ($300$~K, $600$~K, and $1000$~K), using the energy expansion up to third- and fourth-order anharmonic terms, and the same structure size of $0.1$~$\mu$m. At lowest temperatures, the thermal conductivity strongly increases, reaching the maximum around $T=180$~K, then it shows a slower decrease with temperature.
The differences between the two levels of approximation are minimal, suggesting that third-order calculations already capture the dominant phonon scattering mechanisms. The dependence on the base temperature is also very weak, showing the changes in the heat conductivity within a few percent. 
\SP{}{Further verification of the reliability of our thermal conductivity results is given in Appendix~\ref{ThermalAB}, where the full BTE calculations show good agreement with RTA and higher-resolution RTA results, demonstrating that the $8\times8\times8$ $\bm{q}$-mesh already provides converged values.}

\begin{figure}[!t]
\centering
  \includegraphics[width=\linewidth]{Anizotropy.pdf}
\caption{(a) The anisotropic thermal conductivity of $\beta$-FeSi$_2$ calculated along the lattice directions at 600~K. (b) The average temperature-dependent thermal conductivity taken at $300$~K, $600$~K, and $1000$~K, including anharmonic corrections up to cubic (A3) and quartic (A4) terms. In both cases the crystallite size is 0.1~$\mu$m.}
  \label{anizotropy}
\end{figure}


In Fig.~\ref{therm}, we fix the base temperature at $600$~K and examine the effect of crystallite size on thermal conductivity, varying it from $0.01$ to $0.5$~$\mu$m.
With decreasing the crystallite size, we observe a shift of the position of the maximum to larger temperatures and a decrease of the thermal conductivity in the entire temperature range.
Theoretical results are compared with several experimental data obtained above the room temperature. 
The measured thermal conductivity depends to a large extent on the sample quality, its purity and the size of the crystalline grains which depends on 
the production processes.
Many measurements were performed using crystallites of micrometric or unknown size ~\cite{waldecker_1973,ito_2002,kim_2003,du_2020}, however, 
numerous attempts to minimize $\kappa$ by reducing grain sizes to $56$~nm~\cite{dabrowski_2019, dabrowski_microstructure_2021}, $30$-$400$~nm~\cite{le_tonquesse_2019}, $50$ and $200$~nm~\cite{abbassi_2021}, or introducing pores into the material~\cite{sam2023improved} 
are also carried out. 
Another way to change the thermal conductivity is to dope $\beta$-FeSi$_2$ with different elements~\cite{ito_2002,kim_2003,du_2020,cheng_2024}, however, this effect is beyond our investigation.

\begin{figure}[!t]
\centering
  \includegraphics[width=\linewidth]{Thermal_conductivity3.pdf}
\caption{
The phonon thermal conductivity of $\beta$-FeSi$_2$: theoretical results for the infinite crystalline size and with boundary conditions, compared with experimental data for different structure sizes.
}
  \label{therm}
\end{figure}
%Fig.~\ref{therm}\cite{abbassi_2021}\cite{waldecker_1973}\cite{dabrowski_2019}\cite{sam2023improved}\cite{kim_2003}\cite{du_2020}\cite{ito_2002}\cite{le_tonquesse_2019}.

We observe a decrease in the thermal conductivity with reducing crystalline grain sizes in all analyzed experimental data. 
For instance, by decreasing the crystallite size to $50$~nm, the thermal conductivity at room temperature was reduced by a factor of $1.7$, what can be  compared to the annealed sample with 200 nm grains~\cite{abbassi_2021}. 
It is worth noting that the rate of decrease in value with increasing temperature in both cases, for grain sizes of $50$~nm and $200$~nm, is significantly different, which is consistent with our calculations. 
The same trend can be observed by comparing the thermal conductivity measured for a sample with bulk crystallite sizes with the thermal conductivity of a sample with grains smaller than 400 nm~\cite{le_tonquesse_2019}.
The theoretical results obtained for the same crystallite size show higher values due to factors not captured in the idealized model, such as crystal imperfection or mechanical strain. Usually, a decrease in the crystallite size is related to an increased concentration of grain boundaries, point defects, and stacking faults that influence the phonon scattering~\cite{le_tonquesse_2019,abbassi_2021}.   

We should note that the total thermal conductivity is a combination
of the lattice and electronic contributions to the heat transport.
In semiconductors, the electronic thermal conductivity is negligible at low temperatures and significantly increases 
only much above the room temperatures~\cite{gu_2020}.
For $\beta$-FeSi$_2$, the electronic thermal conductivity was obtained from the electric conductivity using the Wiedemann-Franz law~\cite{ito_2002,kim_2003,le_tonquesse_2019}.
In the undoped material, its value does not exceed $0.1$~W/mK in the measurement up to $T=950$~K~\cite{kim_2003}.
By doping, the electronic thermal conductivity can be enhanced, and it has a direct impact on the thermoelectric properties of $\beta$-FeSi$_2$ at high temperatures~\cite{ito_2002,kim_2003}.
In the present study, we consider only the phonon contribution to the thermal conductivity,
therefore, agreement with experimental data may deteriorate with increasing temperature.

\section{Summary}
\label{sec.summary}

We performed {\it ab initio} studies on lattice dynamical and thermal transport properties of $\beta$-FeSi$_2$. The effect of anharmonicity was analyzed within two approaches -- the SCPH method and the perturbation theory.
The phonon dispersion curves obtained within SCPH show small renormalization of frequencies comparing to the harmonic approximation. 
The Raman spectra were calculated within the procedure which takes into account the peak intensities obtained from the Raman tensors and the line profiles obtained from the phonon self energy derived within the perturbation theory based on the large, temperature-independent, quartic model fitted to the data from the wide range of temperatures (300-1000~K). 
The anharmonic corrections strongly affect the frequencies and line profiles of some modes and results in overall better agreement with the experimental data. 
We analyzed the phonon lifetimes and group velocities obtained as functions of the phonon frequency.
Then the lattice thermal conductivity was calculated for a broad range of temperatures and grain sizes.
We found a small anisotropy in the phonon thermal transport resulting from the orthorhombic structure and a weak effect of the quartic anharmonic terms. 
The thermal conductivity calculated for various crystalline grain sizes show a good qualitative agreement with the available measurements.

\begin{acknowledgments}
Some figures in this work were rendered using {\sc Vesta}~\cite{momma.izumi.11} software.
This work was partially supported by the Ministry of Education, Youth and Sports of the Czech Republic through the e-INFRA CZ (ID:90254).
\end{acknowledgments}

\appendix

\section{Raman spectrum}
\label{RamanAA}

Based on the polarized Raman measurements reported in Ref.~\cite{maeda_2004}, two Raman peaks were identified as belonging to the A$_g$ symmetry class, and several additional peaks were observed with similar or different polarization dependence. Although the authors of Ref.~\cite{maeda_2004} provided estimates of the relative Raman tensor components, they did not specify which of the remaining modes correspond to the B$_{1g}$, B$_{2g}$, or B$_{3g}$ symmetries. Because of this missing experimental information, a direct symmetry-resolved comparison between the measurement and theory is not currently possible for the non-A$_g$ modes. 
To provide a complete theoretical picture of the B$_g$-type modes, we show here the calculated Raman-active frequencies and intensities for the B$_{1g}$, B$_{2g}$, and B$_{3g}$ symmetries only. 
%These results represent the predicted Raman modes for the B$_g$ symmetries in $\beta$-FeSi$_2$. 
\textcolor{red} {Fig.~\ref{raman1} shows the predicted Raman modes for the B$_g$ symmetries in $\beta$-FeSi$_2$. The results obtained within the harmonic approximation are compared with the anharmonic perturbation theory calculations which provides both, frequency shifts and predicted line profiles of the modes.}
Although the experimentally measured peaks cannot be directly assigned to these symmetries due to the lack of polarization-resolved data, the theoretical predictions provide a reference for comparison. Matching the measured frequencies to the closest theoretical B$_g$ modes (Table~\ref{T1Raman}) allows for a tentative assignment, which can guide future polarization-resolved Raman experiments aimed at determining the precise symmetry of the unresolved peaks.

\begin{figure*}[t]
\centering
  \includegraphics[width=\linewidth]{spectrum_RTA+raman_Bi_modes.pdf}
\caption{Raman spectrum of $\beta$-FeSi$_2$ calculated at three temperatures 
($300$, $600$, $900$~K -- blue, orange, and red lines, respectively). 
The calculated spectrum includes all Raman-active modes. The B$_{ig}$ modes
are indicated by purple vertical lines at peak positions corresponding to T=300~K. 
The frequencies derived from the harmonic approximation at T=300~K are indicated 
by green, yellow and pink lines.
Connecting arrows indicate the correspondence between harmonic and anharmonic 
frequencies, demonstrating the frequency shifts due to phonon interactions.}
\label{raman1}
\end{figure*}
\section{Thermal conductivity obtained from BTE and RTA}
\label{ThermalAB}

The thermal conductivity was computed by solving the full BTE on the largest feasible $\bm{q}$-point grid, $8\times8\times8$, and compared with the corresponding RTA results obtained on the same grid. As shown in Fig.~\ref{bte}, the difference between the components of the thermal conductivity tensor obtained within BTE and RTA at this resolution is very small, indicating a good agreement between these two approaches.
%THIS PART SHOULD GO RATHER TO THE RESPONSE
%Extending the full BTE calculation to larger grids is computationally prohibitive: the computational cost of BTE is roughly two orders of %magnitude higher than that of RTA, and the required memory and runtime exceed our available resources. 
Moreover, we performed an additional calculation using RTA on a denser $20\times20\times20$ grid. As seen in the Fig.~\ref{bte}, the higher-resolution data remain in a good agreement with both the BTE and RTA results for the $8\times8\times8$ grid.
It shows that the $8\times8\times8$ mesh already provides good results for this structure and confirms reliability of the calculations.

\begin{figure}[!h]
\centering
  \includegraphics[width=\linewidth]{BTEvsRTAvsANP.pdf}
\caption{Thermal conductivity of $\beta$-FeSi$_2$ obtained within the BTE and RTA methods using the Phono3py software on the $8\times8\times8$ q-point grid, compared with the RTA results computed with ALAMODE on a denser $20\times20\times20$ grid.}
\label{bte}
\end{figure}

%\section*{Data availability}
%The data that support the findings of this article are openly available~\footnote{give me DOI}.
% see https://journals.aps.org/authors/data-availability-statements#citation

\bibliography{refs.bib}
%\bibliographystyle{ieeetr}


\end{document}

\documentclass[%
%reprint,
superscriptaddress,
%groupedaddress,
longbibliography,
%unsortedaddress,
%runinaddress,
%frontmatterverbose, 
%preprint,
%preprintnumbers,
%nofootinbib,
nobibnotes,
%bibnotes,
amsmath,amssymb,
aps,
%pra,
prb,
%rmp,
%prstab,
%prstper,
%showkeys,
floatfix,
twocolumn
]{revtex4-2}

\usepackage{graphicx}% Include figure files
\usepackage{calc}% Calculate margins
\usepackage{dcolumn}% Align table columns on decimal point
\usepackage{bm}% bold math

\usepackage[urlcolor=blue,colorlinks=true,citecolor=blue,linkcolor=blue,pdfstartview={FitH},bookmarks=false]{hyperref} % add hypertext capabilities

%\usepackage[mathlines]{lineno} % Enable numbering of text and display math
% \linenumbers\relax % Commence numbering lines

% \usepackage[showframe,%Uncomment any one of the following lines to test 
% %scale=0.7, marginratio={1:1, 2:3}, ignoreall,% default settings
% %text={7in,10in},centering,
% %margin=1.5in,
% % total={6.5in,8.75in}, top=1.2in, left=0.9in, includefoot,
% % height=10in,a5paper,hmargin={3cm,0.8in},
% ]{geometry}

\usepackage{amsmath}
\usepackage{amssymb}
%\usepackage{orcidlink}
\usepackage{xcolor}
%\usepackage{datetime}
\usepackage[normalem]{ulem}

% Change tracking commands
\newcommand{\trackchange}[3]{\textcolor{#3}{\sout{#1}#2}}  % Full color strikeout, insert
%\renewcommand{\trackchange}[3]{\textcolor{#3}{#2}}        % Just color silent remove and insert
%\renewcommand{\trackchange}[3]{{#2}}                      % No indication, silent remove and insert

% Author marker definitions

\definecolor{myblue}{RGB}{0,127,85}
% \definecolor{violet}{RGB}{102,0,204}
% \definecolor{orange}{RGB}{255,128,0}
% \definecolor{green}{RGB}{0,128,0}
\newcommand{\DL}[1]{\trackchange{}{#1}{blue}}
\newcommand{\AP}[1]{\trackchange{}{#1}{red}}
\newcommand{\PJ}[2]{\trackchange{#1}{#2}{orange}}
\newcommand{\JL}[2]{\trackchange{#1}{#2}{myblue}}
\newcommand{\SP}[2]{\trackchange{#1}{#2}{blue}}
\newcommand{\PP}[2]{\trackchange{#1}{#2}{teal}}
\newcommand{\AI}[2]{\trackchange{#1}{#2}{olive}}
\newcommand{\MS}[1]{\trackchange{}{#1}{purple}}

\newcommand{\TODO}[1]{\textcolor{red}{TODO: #1}}

\sloppy

\begin{document}

\title{Ab initio study of the anharmonic properties and thermal conductivity in $\beta$-FeSi$_2$}

\author{Svitlana~Pastukh}
\email[e-mail: ]{svitlana.pastukh@ifj.edu.pl}
\affiliation{Institute of Nuclear Physics, Polish Academy of Sciences, ul. W. E. Radzikowskiego 152, 31-342 Krak\'{o}w, Poland}

\author{Ma\l{}gorzata~Sternik}
\affiliation{Institute of Nuclear Physics, Polish Academy of Sciences, ul. W. E. Radzikowskiego 152, 31-342 Krak\'{o}w, Poland}

\author{Pawe\l{}~T.~Jochym}
\affiliation{Institute of Nuclear Physics, Polish Academy of Sciences, ul. W. E. Radzikowskiego 152, 31-342 Krak\'{o}w, Poland}


\author{Jan~\L{}a\.{z}ewski}
\affiliation{Institute of Nuclear Physics, Polish Academy of Sciences, ul. W. E. Radzikowskiego 152, 31-342 Krak\'{o}w, Poland}

\author{Andrzej~Ptok}
\affiliation{Institute of Nuclear Physics, Polish Academy of Sciences, ul. W. E. Radzikowskiego 152, 31-342 Krak\'{o}w, Poland}

\author{Svetoslav~Stankov}
\affiliation{Institute for Photon Science and Synchrotron Radiation, Karlsruhe Institute of Technology, D-76131 Karlsruhe, Germany}
\affiliation{Laboratory for Applications of Synchrotron Radiation, Karlsruhe Institute of Technology, D-76131 Karlsruhe, Germany}

\author{Przemys\l{}aw~Piekarz}
\affiliation{Institute of Nuclear Physics, Polish Academy of Sciences, ul. W. E. Radzikowskiego 152, 31-342 Krak\'{o}w, Poland}

\date{\today}

\begin{abstract}

Iron silicides are good candidates for applications in  optoelectronic and thermoelectric devices.
Lattice dynamical properties and thermal conductivity in the $\beta$-FeSi$_2$ semiconductor
are investigated with the first-principles computational methods. 
Phonon dispersion relations are calculated via the
temperature-dependent effective potential method and self-consistent phonon theory. 
To properly model thermal transport, we explicitly consider
the impact of phonon-phonon interactions by analyzing
anharmonic contributions to the phonon self-energy. 
This yields temperature-dependent phonon frequencies and linewidths,
reflecting the finite lifetime of phonons due to scattering
processes. The calculated phonon frequencies and line profiles are used to obtain 
the Raman spectra, which shows good agreement with the experimental data. 
We revealed an enhanced anharmonic behaviour of the Raman modes with the highest frequencies.  
The lattice thermal conductivity is then obtained as a function of temperature and crystallite size within
the relaxation-time approximation.
Phonon transport shows a small anisotropy due to the orthorhombic structure and a very weak dependence
on the quartic anharmonic corrections. The results obtained for an infinite material and for several crystallite sizes
were analyzed and compared with the available experimental data.
\end{abstract}

\maketitle


\section{Introduction}

The comprehensive determination of important physical properties of
crystals, such as thermal expansion, lattice thermal
conductivity or structural phase transitions, requires a fundamental 
understanding of the anharmonic effects.
Although the investigation of anharmonic interactions in crystals has
attracted a considerable interest for decades~\cite{cowley_1968}, a substantial progress
has only recently been achieved thanks to advances in theoretical and
numerical methods and increased computational power.
Now, phonon frequencies, lifetimes, and heat transfer in a wide range of
materials
can be quantitatively predicted using the available computational resources
based on the density functional theory (DFT)~\cite{lindsay_2013,mcgaughey_2019,lindsay_2019}.
In the case of strongly anharmonic systems, the self-consistent phonon
(SCPH) theory~\cite{tadano_2015} as well as the perturbative approach~\cite{tadano_2018}, using higher
order interatomic force constants derived from the fitting to the displacement force data obtained
with DFT, have proven to be successful.

Transition-metal silicides are promising materials for fabrication of electronic components
designed for integration with silicon-based circuits~\cite{murarka_1995}.
At room temperature, iron disilicide ($\beta$-FeSi$_2$) is a
direct-bandgap semiconductor~\cite{bost_1985}, making this material a good
candidate for application in optoelectronic devices such as infrared
detectors or light emitters~\cite{bost_1988}. The development of light-emitting diodes utilizing FeSi$_2$/Si heterostructures has been successfully demonstrated~\cite{leong_1997,suemasu_2001}. Due to
a high thermal stability and strong light absorption, FeSi$_2$ is
also a suitable photovoltaic material~\cite{powalla_1993,liu_2006,okuhara_2017}.

$\beta$-FeSi$_2$ crystallizes in the base-centered orthorhombic
lattice~\cite{dusausoy_1971} transforming to the
tetragonal metallic $\alpha$-FeSi$_{2}$ phase around $1200$~K~\cite{starke_2002}.
Optical studies indicated a direct band gap of the values
$0.85$--$0.89$~eV~\cite{bost_1988,dimitriadis_1990,arushanov_1995,wan_2003},
however, the {\it ab initio} calculations predicted a smaller indirect gap close to 0.8 eV~\cite{christensen_1990}. 
The existence of such an indirect gap was then confirmed by the optical linear transmittance measurements at
low temperatures~\cite{giannini_1992}. As shown by first-principles studies,
the character of the band gap is very sensitive to the orientation 
of a crystal grown on silicon~\cite{clark_1998}.

$\beta$-FeSi$_2$ belongs also to good thermoelectric materials~\cite{ware_1964}, with potential applications resulting from its chemical stability up to high temperatures, nontoxicity, and low cost of preparation~\cite{yamada_2012,nozariasbmarz_2017}. 
It has already been implemented in cars~\cite{birkholz_1988} and portable power sources~\cite{uemura_1989}. 
Its thermoelectric performance can be improved by doping~\cite{ito_2001,tani_2001,kim_2003,chen_2005,pandey_2013,le_tonquesse_2019}, 
which enhances the electric transport and reduces the thermal conductivity~\cite{waldecker_1973,du_2019,du_2020}.  
The thermal conductivity can be also reduced by the modification of microstructure~\cite{ail_2015} or by
nanostructurization~\cite{watanabe_2017,taniguchi_2017,hsin_2017,abbassi_2021}.

The lattice thermal conductivity is directly connected with anharmonic effects and phonon scattering processes.
The vibrational properties of \mbox{$\beta$-FeSi$_2$} were studied by the infrared and Raman spectroscopy~\cite{lefki_1991,guizzetti_1997,maeda_2004,baleva_2008,liu_2011,maeda_2011}. The observed anisotropy in the phonon spectra results from the enhanced sensitivity of the infrared and Raman features to the local lattice distortions~\cite{guizzetti_1997}. The Fe phonon density of states was measured by nuclear inelastic scattering (NIS), showing a good agreement with the density functional theory (DFT) calculations~\cite{walterfang_2005}. Using the DFT approach, the phonon dispersion curves, phonon density of states, as well as various thermodynamic properties were obtained within the harmonic approximation~\cite{tani_2010,liang_2011}. The extended Klemens model was applied to study
the anharmonic effect on phonon frequencies and linewidths observed by the Raman spectroscopy~\cite{zhang_2023}.
The impact of nanostructurization on lattice dynamics was explored in the $\beta$-FeSi$_2$ nanorods grown on the Si(110) surface by the NIS and {\it ab initio} methods~\cite{kalt_2022}.

In this work, we investigate the lattice dynamical properties of $\beta$-FeSi$_2$ 
using the DFT calculations. We study the effect of anharmonic terms in the temperature-dependent potential on phonon frequencies and lifetimes. We focus on the Raman modes, comparing the theoretical results with the experimental data.
The thermal conductivity is derived in a broad temperature range and the effect of crystallite size is analyzed.



This study is structured as follows.
In Sec.~\ref{sec.com} we describe the details of computational methods.
Next, in Sec.~\ref{sec.result} we present and discuss our results.
In particular we present the crystal structure (Sec.~\ref{sec.crys}) and lattice dynamics (Sec.~\ref{sec.lattice}).
We investigate also the thermal conductivity comparing the obtained results with the available experimental data (Sec.~\ref{sec.thermal}).
Finally, Sec.~\ref{sec.summary} summarizes our key findings and conclusions.

\section{Calculation method}
\label{sec.com}


The calculations were performed using the projector augmented-wave potentials~\cite{blochl_1994} and the generalized gradient approximation~\cite{perdew_1996} implemented in the Vienna Ab initio Simulation Package (VASP)~\cite{kresse.hafner.94,kresse.furthmuller.96,kresse.joubert.99}. 
The lattice parameters and atomic positions were optimized in the ${\bm a} \times ({\bm b}-{\bm c}) \times ({\bm b}+{\bm c})$ supercell containing 32 formula units and four primitive cells.
The integration in the reciprocal space was conducted using the $2 \times 2 \times 2$ Monkhorst--Pack mesh~\cite{monkhorst_1976} and the cut-off energy was set to $500$~eV. For convergence conditions, we set the energy change below $10^{-5}$ and $10^{-8}$ for the ionic and electronic loops, respectively. 

The lattice dynamical properties were studied within the temperature-dependent effective potential (TDEP) approach~\cite{hellman_2013}. The atomic potential with the third and fourth order anharmonic terms was derived from interatomic forces induced by displacements of all atoms at finite temperatures.
The sets of atomic displacements were generated by the high efficiency configuration space sampling (HECSS)~\cite{jochym_2021} and forces were obtained by VASP. The interatomic force constants and phonon frequencies were calculated with the {\sc Alamode} software~\cite{tadano_2014}.

Furthermore, we have attempted to construct a {\emph{temperature independent}} 
anharmonic model. We have used combined data from all investigated temperatures 
(300, 600 and 1000~K) and fitted a large (over 15~000 free parameters), fourth-order 
interaction model to this dataset. 
Subsequently, we have used this model to calculate line profiles and positions of Raman-active modes
at multiple temperatures.

The changes in phonon frequencies induced by the anharmonic effects were investigated within two approaches.
First, the impact of the quartic anharmonic terms was included using the SCPH theory~\cite{tadano_2015}.
Second, the mode profiles (frequency shifts and line widths) were determined from the real and imaginary parts of the phonon self-energy resulting from the cubic and quartic anharmonic terms of the above mentioned large model~\cite{tadano_2018}. 
The longitudinal optic-transverse optic (LO-TO) splitting was also evaluated, using the static dielectric tensor and Born effective charges calculated within density functional perturbation theory~\cite{gajdos_2006}.

\begin{figure}[]
    \centering
    \includegraphics[width=\linewidth]{fig1_new.png}
\caption{%
(a) The conventional unit cell of $\beta$-FeSi$_2$ (with Cmca symmetry) and (b) the corresponding Brillouin zone with selected high-symmetry points.
}
  \label{fig.struct}
\end{figure}

% To further characterize the vibrational properties, Raman-active scattering was investigated using the Phonopy-Spectroscopy package~\cite{skelton_2017}. This enabled the identification of Raman-active modes and the calculation of Raman tensors. Anharmonic force constants, obtained from calculations using {\sc Alamode}, were then used to obtain theoretical line profiles for the Raman modes. The presented Raman scattering spectra combine these anharmonic line profiles with the Raman tensor amplitudes.
% This analysis was also based on the large quartic model mentioned above.

% Finally, the thermal conductivity was obtained as a function of temperature and crystallite size within the relaxation-time approximation (RTA) \PJ{}{as implemented in {\sc Alamode}\cite{tadano_2014}}. The phonon lifetimes were calculated from the phonon self-energy including the cubic and quartic anharmonic terms. \SP{}{The RTA provides a solution of the Boltzmann transport equation (BTE) under the assumption that scattering events are independent and can be treated through mode-resolved relaxation times.
% To verify the validity of this approximation for $\beta$-FeSi$_2$, we have additionally
% executed an iterative solution of the BTE.}\PJ{}{These calculations were performed with
% {\sc Phono3py}\cite{togo_2023}. For cross-validation, the additional RTA calculations were performed on the same $\bm{q}$-grid as BTE with implementation provided by {\sc Phono3py}.}.

To further characterize the vibrational properties, Raman scattering was investigated using the Phonopy-Spectroscopy package~\cite{skelton_2017}. This enabled the identification of Raman-active modes and the calculation of Raman tensors. Anharmonic force constants, derived from calculations using {\sc Alamode}, were then used to obtain theoretical line profiles for the Raman modes. The presented Raman scattering spectra combine these anharmonic line profiles with the Raman tensor amplitudes.
This analysis was also based on the large quartic model described above.

Finally, the thermal conductivity was calculated as a function of temperature and crystallite size within the relaxation-time approximation (RTA) \PJ{}{as implemented in {\sc Alamode}~\cite{tadano_2014}}. The phonon lifetimes were calculated from the phonon self-energy including the cubic and quartic anharmonic terms. \SP{}{The RTA provides a solution to the Boltzmann transport equation (BTE) under the assumption that scattering events are independent and can be treated through mode-resolved relaxation times.
To verify the validity of this approximation for $\beta$-FeSi$_2$, we additionally
solved the BTE iteratively.} \PJ{}{These calculations were performed with
{\sc Phono3py}~\cite{togo_2023}. For cross-validation, the additional RTA calculations were performed on the same $\bm{q}$-grid as the BTE calculations, using the implementation provided by {\sc Phono3py}.}

% (see. Appendix~\ref{ThermalAB} for the comparison)\cite{togo_2023}.}
%As shown in Appendix~\ref{ThermalAB}, the iterative BTE results exhibit good agreement with the RTA values, confirming that RTA is sufficiently accurate for this material and for the considered temperature range.}


\section{Results}
\label{sec.result}

\subsection{Crystal structure}
\label{sec.crys}

The $\beta$-FeSi$_2$ structure adopts a base-centered orthorhombic lattice with the space group Cmca (No.~64) as shown in Fig.~\ref{fig.struct}(a).
The unit cell consists of two primitive cells and contains 48 atoms.
Iron (silicon) atoms possess two nonequivalent positions: \mbox{Fe-I} and \mbox{Fe-II} (\mbox{Si-I} and \mbox{Si-II}), presented in Fig.~\ref{fig.struct}(a) as gray and purple (orange and yellow) spheres, respectively.
This crystal structure is derived from the fluorite-type lattice with strongly distorted Si cubes and Fe atoms occupying 
one-half of the central sites.
The Fe-I and Fe-II sites create different layers perpendicular to the $x$ direction, and they are separated by layers containing both Si sites.
Each Fe atom is coordinated by 8 Si atoms with slightly different Fe-Si distances.
The optimized lattice constants ($a = 9.874$~{\AA}, $b = 7.767$~{\AA}, and
$c = 7.811$~{\AA}) agree very well with the experimental data ($a = 9.863$~{\AA}, $b = 7.791$~{\AA}, and $c = 7.833$~{\AA})~\cite{dusausoy_1971}.

Iron atoms occupy the Wyckoff sites 8\textit{d} ($0.2166$, $0$, $0$) and  8\textit{f} ($0$, $0.3072$, $0.1879$), corresponding to \mbox{Fe-I} and \mbox{Fe-II}, respectively. 
Silicon atoms are located at two inequivalent 16\textit{g} positions: ($0.1282$, $0.2737$, $0.0495$) and \mbox{($0.3734$, $0.0445$, $0.2270$)}, assigned as \mbox{Si-I} and \mbox{Si-II}.
The optimized positions of atoms agree very well with the experimental data~\cite{dusausoy_1971} and the previous theoretical studies~\cite{tani_2010,liang_2011}.


\subsection{Lattice dynamics}
\label{sec.lattice}

\begin{figure}[]
    \centering
    \includegraphics[width=\linewidth]{Fig1c.pdf}
\caption{%
The phonon dispersion curves along high symmetry directions obtained within SCPH (for temperatures from $0$ to $1000$~K).
Dashed black lines indicate the phonon dispersions obtained from the harmonic approximation. The white dots indicate Raman-active modes with A$_g$ symmetry.
The vertical plot shows the phonon density of states (DOS) calculated at a reference temperature of $600$~K.
}
  \label{fig.ph_band}
\end{figure}


In Fig.~\ref{fig.ph_band} we present the phonon dispersion relations of $\beta$-FeSi$_2$ along high-symmetry directions in the Brillouin zone [Fig.~\ref{fig.struct}(b)].
Due to 24 atoms in the primitive cell, the phonon spectrum consists of 69 optical modes and three acoustic modes.
The phonon dispersions were calculated within the SCPH approach in the temperature range $0$--$1000$~K (presented by color lines in Fig.~\ref{fig.ph_band}), and they are compared with the results obtained from the harmonic part of the effective potential corresponding to temperature $T=300$~K (indicated by dashed black lines in Fig.~\ref{fig.ph_band}).
As we can see within the SCPH method, the anharmonic effects are rather weak and leads only to small renormalization of phonon frequencies.
Only the highest modes show more pronounced shifts of their frequencies to larger values.
The total and partial element-projected phonon density of states obtained within the harmonic approximation are presented in Fig.~\ref{fig.ph_band}.
Up to around $320$~cm$^{-1}$, the contributions from both elements are very similar, while for higher frequencies the spectrum is dominated by
the Si vibrations.


\begin{table}[!t]
\begin{ruledtabular}
\caption{Calculated and experimental Raman-active modes of $\beta$-FeSi$_2$ with their irreducible representations (IR). Present theoretical results are compared with the previous theoretical data from Ref.~\cite{tani_2010} and experimental results from \mbox{Refs.~\cite{lefki_1991,maeda_2004}.} The experimental frequencies with known symmetries (A$_g$) are shown in bold, while other experimental modes are assigned to the best fitting theoretical values.}
\begin{tabular}{c c c c c}
\textbf{IR} & \multicolumn{4}{c}{\textbf{Frequency (cm$^{-1}$)}} \\
 & Present  & Theor.~\cite{tani_2010} & Exp.~\cite{lefki_1991} & Exp.~\cite{maeda_2004} \\
\hline
B$_{2g}$ & 175.4 & 179 & 176 &  \\
B$_{1g}$ & 176.3 & 185 & 179  &  \\
B$_{1g}$ & 193.3 & 198 &  & 190.6 \\
A$_g$    & 196.6 & 208 & \textbf{195} & \textbf{194.0} \\
A$_g$    & 203.9 & 210 & \textbf{197} & 199.6 \\
B$_{3g}$ & 205.5 & 212 & 200 &  \\
B$_{3g}$ & 226.3 & 236 & 206 & 227.1 \\
B$_{1g}$ & 233.3 & 240 &  &  231.6 \\
B$_{2g}$ & 248.6 & 254 &  &  \\
A$_g$    & 250.1 & 257 & \textbf{247} & \textbf{247.3} \\
A$_g$    & 254.9 & 264 & \textbf{253} & 254.3 \\
B$_{1g}$ & 275.5 & 285 &  &  274.1 \\
B$_{2g}$ & 282.4 & 295 &  &  281.2 \\
B$_{3g}$ & 286.8 & 297 &  &  \\
B$_{1g}$ & 307.1 & 317 &  &  \\
B$_{2g}$ & 312.6 & 326 &  &  311.8 \\
B$_{1g}$ & 319.0 & 324 &  &  \\
B$_{3g}$ & 327.9 & 341 &  & 325.8 \\
B$_{2g}$ & 333.3 & 345 &  &  \\
A$_g$    & 339.3 & 352 & \textbf{346} & 339.5 \\
B$_{2g}$ & 343.0 & 350 &  &  \\
B$_{1g}$ & 353.5 & 366 &  &  \\
B$_{2g}$ & 372.8 & 383 &  & 370.7 \\
B$_{1g}$ & 375.1 & 385 &  &  \\
B$_{3g}$ & 375.2 & 386 &  &  \\
B$_{3g}$ & 385.3 & 401 &  &  \\
A$_g$    & 386.5 & 398 &  & 386.2 \\
B$_{2g}$ & 387.7 & 402 &  & 388.2 \\
A$_g$    & 404.6 & 415 &  & 400.4 \\
B$_{2g}$ & 405.1 & 420 &  &  \\
B$_{1g}$ & 412.7 & 428 &  &  \\
B$_{3g}$ & 418.4 & 431 &  &  \\
B$_{3g}$ & 441.4 & 458 &  & 442.6 \\
B$_{3g}$ & 447.3 & 466 &  & 446.3 \\
A$_g$    & 448.4 & 464 &  &  \\
A$_g$    & 499.1 & 517 &  &  \\
\end{tabular}
\label{T1Raman}
\end{ruledtabular}
\end{table}

%subsection{Raman}

\begin{figure*}
\centering
  \includegraphics[width=\textwidth]{spectrum_RTA+raman.pdf}
\caption{Raman spectrum of $\beta$-FeSi$_2$ calculated within the perturbative approach at three temperatures 
($300$, $600$, $900$~K -- blue, orange, and red lines, respectively). 
The calculated spectrum includes all Raman-active modes. The A$_g$ modes
are indicated by purple vertical lines at peak positions corresponding to T=300~K. 
The frequencies derived from the harmonic approximation at T=300~K are indicated by green lines.
Connecting arrows indicate the correspondence between harmonic and
anharmonic frequencies, demonstrating the frequency shifts due to 
phonon interactions. Experimental values for the A$_g$ modes
based on Ref.~\cite{lefki_1991} are marked with black dashed lines.
}
\label{raman}
\end{figure*}


The phonon spectrum at the $\Gamma$ point consists of 36 Raman modes classified according to the irreducible representations: $9A_\text{g}+9B_\text{1g}+9B_\text{2g}+9B_\text{3g}$. 
Fig.~\ref{raman} shows the Raman spectrum of A$_g$ symmetry calculated for $\beta$-FeSi$_2$ within the perturbative approach at three temperatures $300$, $600$, and $900$~K (solid blue, orange, and red curves, respectively), including third-order and fourth-order anharmonic corrections.
The calculated Raman spectrum includes all Raman-active modes.
The five experimentally observed A$_g$ modes are highlighted by black dashed lines, based on data from Ref.~\cite{lefki_1991}.
Additional peaks not marked with vertical lines correspond to phonon modes with symmetries other than A$_g$.
The frequencies of Raman modes obtained in anharmonic calculations are compared with the previous results calculated within the harmonic approximation and the experimental values in Tab.~\ref{T1Raman}. 
We have marked in bold the experimentally determined A$_g$ modes, which are compared with the calculations.
Since the experimental studies did not provide the accurate assignement of the Raman modes with the B$_{1g}$, B$_{2g}$, and B$_{3g}$ symmetry~\cite{lefki_1991,maeda_2004}, we cannot compare them directly with the theoretical results.
However, in Tab.~\ref{T1Raman} we have assigned the measured frequencies to the best fitting theoretical values without taking into account the symmetry of the modes, except for the known A$_{g}$ modes. 


The impact of anharmonicity on the phonon frequencies is well visible from the comparison of the results obtained within the harmonic approximation 
and from the anharmonic calculation (vertical green and purple lines in Fig.~\ref{raman}, respectively).
Here we show only the A$_g$ modes, which are compared with the experimental results (vertical black dashed lines).
Anharmonic frequencies calculated at $300$~K are indicated by purple lines, while frequencies derived from harmonic approximation are marked with green lines. The grey solid lines connect corresponding modes obtained in both approximations.
In most cases, the results obtained within the harmonic approximation do not agree with the experimental frequencies. 
%Only the modes close to $250$~cm$^{-1}$ and $400$~cm$^{-1}$ correspond well to the experimental values. MS
As we can see, the inclusion of the anharmonic correction leads to a significant modification of the phonon frequencies.
These anharmonic effects are stronger for higher-frequency modes mainly because of
the dominant contribution from Si atoms, which vibrate with larger amplitudes than heavier Fe atoms. 
When atoms move to larger distances the potential deviates more from the harmonic approximation,
and the anharmonic corrections become stronger.


The modification of phonon frequencies observed in Fig.~\ref{raman} is much larger than in the SCPH scheme presented in Fig.~\ref{fig.ph_band}. The SCPH approach includes only the leading-order contribution to 
the phonon self-energy obtained from the quartic anharmonic terms~\cite{tadano_2015}. Therefore, it does not describe fully the changes of phonon frequencies found within the perturbation theory (see Fig.~\ref{raman}).
Especially, it is well visible for two highest A$_g$ modes, which exhibit also the largest line broadening 
and the strongest dependence on temperature.
Therefore, a better agreement with experimentally observed frequencies is visible,
confirming the significant influence of the anharmonicity on the frequencies and line profiles of phonon modes.
In fact, the decrease of phonon frequencies should be even stronger due to thermal expansion, 
which is not included in our calculations.
Within SCPH the frequencies of the highest modes increase with increasing temperature as we see in Fig.~\ref{fig.ph_band}. The comparison of two different approaches applied to study anharmonic properties of $\beta$-FeSi$_2$ shows that the perturbation theory, which includes the cubic and quartic terms, better describes the changes of phonon frequencies with temperature than the SCPH method. \PJ{}{This indicates that the leading-order contribution included in SCPH are not important in this material.}

Additionaly we should nottice that for other than A$_g$ modes we cannot make an unambiguous assignment of theoretical frequencies to experimental ones. Note that the spectrum in Fig.~\ref{fig.ph_band} contains all Raman-active modes. 
The limitation to A$_g$ modes concerns only the indicated positions of the peaks. 
%\SP{}{Furthermore, we present the full spectrum for all Raman active modes with the comparison of the harmonic and anharmonic calculations in the Appendix~\ref{RamanAA}.}
\textcolor{red}{The full Raman spectrum, with the frequencies of the B$_{1g}$, B$_{2g}$ and B$_{3g}$ modes marked, is shown in Appendix A, Fig.~\ref{raman1}. This represents our theoretical prediction of possible Raman-mode assignments, which can be verified in future experiments.}
%This is the theoretical prediction of possible assignment of Raman modes that can be verified in future experiments.  


\subsection{Thermal conductivity}
\label{sec.thermal}


\begin{figure*}[!t]
\centering
  \includegraphics[width=\linewidth]{time3.pdf}
\caption{Phonon lifetimes calculated for three temperatures as a function of phonon frequency. The colors correspond to the phonon branches.}
  \label{thermtime}
\end{figure*}


In this section, we analyze the thermal conductivity tensor of $\beta$-FeSi$_2$ obtained within the RTA approach~\cite{tadano_2018} as a function of temperature
%
\begin{equation}
\kappa_{\text{ph}}^{\mu\nu}(T) = \frac{1}{NV} \sum_{\bm{q},j} c_{\bm{q}j}(T) v_{\bm{q}j}^{\mu} v_{\bm{q}j}^{\nu}\tau_{\bm{q}j}(T),
\end{equation}
% 
where $c_{\bm{q}j}$ is the mode heat capacity and $v_{\bm{q}j}$ is the mode group velocity. 
The relaxation time is approximated by the phonon lifetime $\tau_{\bm{q}j}$
calculated for $j$-th branch at the wave vector $\bm{q}$.
$V$ is the unit cell volume and $N$ is the number of unit cells in the crystal.
The phonon lifetime is calculated using this formula
%
\begin{equation}
\tau_{\bm{q}j}(T)=\frac{1}{2\Gamma_{\bm{q}j}^{\text{anh}}(T)},
\end{equation}
%
where $\Gamma_{\bm{q}j}^{\text{anh}}$ is the anharmonic phonon linewidth obtained from 
the imaginary part of the phonon self-energy within the perturbation theory.

In Fig.~\ref{thermtime}, we present $\tau_{\bm{q}j}$ obtained for three temperatures 300, 600, and 1000~K as a function of frequency. As we see, the acoustic phonons close to the $\Gamma$ point have the longest lifetimes,
which are diminished with increasing frequency reaching local minima around $200$~cm$^{-1}$.
For higher frequencies, phonon lifetimes first increase to local maxima around $300$~cm$^{-1}$ and then decrease to
the lowest values in the range of highest optical modes. The shortest lifetimes correspond to the largest line broadening
observed for the Raman modes in Fig~\ref{raman}. The phonon group velocities 
$v_{\bm{q}j}=\partial\omega_{\bm{q}j}/\partial\bm{q}$, which are obtained by the central difference formula, are presented in Fig.~\ref{thermvelocity}. Their temperature dependence is negligible, therefore, we present only the results for $T=600$~K.
At low frequencies, there are clearly two ranges of group velocities of the acoustic phonons. 
The larger values correspond to the longitudinal modes, while the lower values are obtained from the
transverse acoustic branches. Group velocities of acoustic phonons decrease for larger frequencies
and reach the average values typical for optic branches.

\begin{figure}[!t]
\centering
  \includegraphics[width=\linewidth]{GV.pdf}
\caption{Mode group velocities calculated as a function of phonon frequency. The colors correspond to the phonon branches.
}
  \label{thermvelocity}
\end{figure}

In Fig.~\ref{anizotropy}(a), we present the three diagonal elements of $\kappa_{\text{ph}}^{\mu\nu}$ corresponding to the main directions of the crystal structure. 
They were obtained from the force constants calculated at the base temperature $T=600$~K and the crystallite size $0.1$~$\mu$m to account for boundary-limited phonon transport. 
Due to the orthorhombic symmetry, we observe a small anisotropy in phonon transport in the whole temperature range. 
At low temperatures, the three components of the heat conductivity increase in a very similar way with the $\kappa_{\text{ph}}^{yy}$ element slightly larger than two other components. 
After reaching the maximum, we observe a change in the largest component from $\kappa_{\text{ph}}^{yy}$ to 
$\kappa_{\text{ph}}^{xx}$.    
In Fig.~\ref{anizotropy}(b), the thermal conductivity is shown for three base temperatures, at which the interatomic potential was obtained ($300$~K, $600$~K, and $1000$~K), using the energy expansion up to third- and fourth-order anharmonic terms, and the same structure size of $0.1$~$\mu$m. At lowest temperatures, the thermal conductivity strongly increases, reaching the maximum around $T=180$~K, then it shows a slower decrease with temperature.
The differences between the two levels of approximation are minimal, suggesting that third-order calculations already capture the dominant phonon scattering mechanisms. The dependence on the base temperature is also very weak, showing the changes in the heat conductivity within a few percent. 
\SP{}{Further verification of the reliability of our thermal conductivity results is given in Appendix~\ref{ThermalAB}, where the full BTE calculations show good agreement with RTA and higher-resolution RTA results, demonstrating that the $8\times8\times8$ $\bm{q}$-mesh already provides converged values.}

\begin{figure}[!t]
\centering
  \includegraphics[width=\linewidth]{Anizotropy.pdf}
\caption{(a) The anisotropic thermal conductivity of $\beta$-FeSi$_2$ calculated along the lattice directions at 600~K. (b) The average temperature-dependent thermal conductivity taken at $300$~K, $600$~K, and $1000$~K, including anharmonic corrections up to cubic (A3) and quartic (A4) terms. In both cases the crystallite size is 0.1~$\mu$m.}
  \label{anizotropy}
\end{figure}


In Fig.~\ref{therm}, we fix the base temperature at $600$~K and examine the effect of crystallite size on thermal conductivity, varying it from $0.01$ to $0.5$~$\mu$m.
With decreasing the crystallite size, we observe a shift of the position of the maximum to larger temperatures and a decrease of the thermal conductivity in the entire temperature range.
Theoretical results are compared with several experimental data obtained above the room temperature. 
The measured thermal conductivity depends to a large extent on the sample quality, its purity and the size of the crystalline grains which depends on 
the production processes.
Many measurements were performed using crystallites of micrometric or unknown size ~\cite{waldecker_1973,ito_2002,kim_2003,du_2020}, however, 
numerous attempts to minimize $\kappa$ by reducing grain sizes to $56$~nm~\cite{dabrowski_2019, dabrowski_microstructure_2021}, $30$-$400$~nm~\cite{le_tonquesse_2019}, $50$ and $200$~nm~\cite{abbassi_2021}, or introducing pores into the material~\cite{sam2023improved} 
are also carried out. 
Another way to change the thermal conductivity is to dope $\beta$-FeSi$_2$ with different elements~\cite{ito_2002,kim_2003,du_2020,cheng_2024}, however, this effect is beyond our investigation.

\begin{figure}[!t]
\centering
  \includegraphics[width=\linewidth]{Thermal_conductivity3.pdf}
\caption{
The phonon thermal conductivity of $\beta$-FeSi$_2$: theoretical results for the infinite crystalline size and with boundary conditions, compared with experimental data for different structure sizes.
}
  \label{therm}
\end{figure}
%Fig.~\ref{therm}\cite{abbassi_2021}\cite{waldecker_1973}\cite{dabrowski_2019}\cite{sam2023improved}\cite{kim_2003}\cite{du_2020}\cite{ito_2002}\cite{le_tonquesse_2019}.

We observe a decrease in the thermal conductivity with reducing crystalline grain sizes in all analyzed experimental data. 
For instance, by decreasing the crystallite size to $50$~nm, the thermal conductivity at room temperature was reduced by a factor of $1.7$, what can be  compared to the annealed sample with 200 nm grains~\cite{abbassi_2021}. 
It is worth noting that the rate of decrease in value with increasing temperature in both cases, for grain sizes of $50$~nm and $200$~nm, is significantly different, which is consistent with our calculations. 
The same trend can be observed by comparing the thermal conductivity measured for a sample with bulk crystallite sizes with the thermal conductivity of a sample with grains smaller than 400 nm~\cite{le_tonquesse_2019}.
The theoretical results obtained for the same crystallite size show higher values due to factors not captured in the idealized model, such as crystal imperfection or mechanical strain. Usually, a decrease in the crystallite size is related to an increased concentration of grain boundaries, point defects, and stacking faults that influence the phonon scattering~\cite{le_tonquesse_2019,abbassi_2021}.   

We should note that the total thermal conductivity is a combination
of the lattice and electronic contributions to the heat transport.
In semiconductors, the electronic thermal conductivity is negligible at low temperatures and significantly increases 
only much above the room temperatures~\cite{gu_2020}.
For $\beta$-FeSi$_2$, the electronic thermal conductivity was obtained from the electric conductivity using the Wiedemann-Franz law~\cite{ito_2002,kim_2003,le_tonquesse_2019}.
In the undoped material, its value does not exceed $0.1$~W/mK in the measurement up to $T=950$~K~\cite{kim_2003}.
By doping, the electronic thermal conductivity can be enhanced, and it has a direct impact on the thermoelectric properties of $\beta$-FeSi$_2$ at high temperatures~\cite{ito_2002,kim_2003}.
In the present study, we consider only the phonon contribution to the thermal conductivity,
therefore, agreement with experimental data may deteriorate with increasing temperature.

\section{Summary}
\label{sec.summary}

We performed {\it ab initio} studies on lattice dynamical and thermal transport properties of $\beta$-FeSi$_2$. The effect of anharmonicity was analyzed within two approaches -- the SCPH method and the perturbation theory.
The phonon dispersion curves obtained within SCPH show small renormalization of frequencies comparing to the harmonic approximation. 
The Raman spectra were calculated within the procedure which takes into account the peak intensities obtained from the Raman tensors and the line profiles obtained from the phonon self energy derived within the perturbation theory based on the large, temperature-independent, quartic model fitted to the data from the wide range of temperatures (300-1000~K). 
The anharmonic corrections strongly affect the frequencies and line profiles of some modes and results in overall better agreement with the experimental data. 
We analyzed the phonon lifetimes and group velocities obtained as functions of the phonon frequency.
Then the lattice thermal conductivity was calculated for a broad range of temperatures and grain sizes.
We found a small anisotropy in the phonon thermal transport resulting from the orthorhombic structure and a weak effect of the quartic anharmonic terms. 
The thermal conductivity calculated for various crystalline grain sizes show a good qualitative agreement with the available measurements.

\begin{acknowledgments}
Some figures in this work were rendered using {\sc Vesta}~\cite{momma.izumi.11} software.
This work was partially supported by the Ministry of Education, Youth and Sports of the Czech Republic through the e-INFRA CZ (ID:90254).
\end{acknowledgments}

\appendix

\section{Raman spectrum}
\label{RamanAA}

Based on the polarized Raman measurements reported in Ref.~\cite{maeda_2004}, two Raman peaks were identified as belonging to the A$_g$ symmetry class, and several additional peaks were observed with similar or different polarization dependence. Although the authors of Ref.~\cite{maeda_2004} provided estimates of the relative Raman tensor components, they did not specify which of the remaining modes correspond to the B$_{1g}$, B$_{2g}$, or B$_{3g}$ symmetries. Because of this missing experimental information, a direct symmetry-resolved comparison between the measurement and theory is not currently possible for the non-A$_g$ modes. 
To provide a complete theoretical picture of the B$_g$-type modes, we show here the calculated Raman-active frequencies and intensities for the B$_{1g}$, B$_{2g}$, and B$_{3g}$ symmetries only. 
%These results represent the predicted Raman modes for the B$_g$ symmetries in $\beta$-FeSi$_2$. 
\textcolor{red} {Fig.~\ref{raman1} shows the predicted Raman modes for the B$_g$ symmetries in $\beta$-FeSi$_2$. The results obtained within the harmonic approximation are compared with the anharmonic perturbation theory calculations which provides both, frequency shifts and predicted line profiles of the modes.}
Although the experimentally measured peaks cannot be directly assigned to these symmetries due to the lack of polarization-resolved data, the theoretical predictions provide a reference for comparison. Matching the measured frequencies to the closest theoretical B$_g$ modes (Table~\ref{T1Raman}) allows for a tentative assignment, which can guide future polarization-resolved Raman experiments aimed at determining the precise symmetry of the unresolved peaks.

\begin{figure*}[t]
\centering
  \includegraphics[width=\linewidth]{spectrum_RTA+raman_Bi_modes.pdf}
\caption{Raman spectrum of $\beta$-FeSi$_2$ calculated at three temperatures 
($300$, $600$, $900$~K -- blue, orange, and red lines, respectively). 
The calculated spectrum includes all Raman-active modes. The B$_{ig}$ modes
are indicated by purple vertical lines at peak positions corresponding to T=300~K. 
The frequencies derived from the harmonic approximation at T=300~K are indicated 
by green, yellow and pink lines.
Connecting arrows indicate the correspondence between harmonic and anharmonic 
frequencies, demonstrating the frequency shifts due to phonon interactions.}
\label{raman1}
\end{figure*}
\section{Thermal conductivity obtained from BTE and RTA}
\label{ThermalAB}

The thermal conductivity was computed by solving the full BTE on the largest feasible $\bm{q}$-point grid, $8\times8\times8$, and compared with the corresponding RTA results obtained on the same grid. As shown in Fig.~\ref{bte}, the difference between the components of the thermal conductivity tensor obtained within BTE and RTA at this resolution is very small, indicating a good agreement between these two approaches.
%THIS PART SHOULD GO RATHER TO THE RESPONSE
%Extending the full BTE calculation to larger grids is computationally prohibitive: the computational cost of BTE is roughly two orders of %magnitude higher than that of RTA, and the required memory and runtime exceed our available resources. 
Moreover, we performed an additional calculation using RTA on a denser $20\times20\times20$ grid. As seen in the Fig.~\ref{bte}, the higher-resolution data remain in a good agreement with both the BTE and RTA results for the $8\times8\times8$ grid.
It shows that the $8\times8\times8$ mesh already provides good results for this structure and confirms reliability of the calculations.

\begin{figure}[!h]
\centering
  \includegraphics[width=\linewidth]{BTEvsRTAvsANP.pdf}
\caption{Thermal conductivity of $\beta$-FeSi$_2$ obtained within the BTE and RTA methods using the Phono3py software on the $8\times8\times8$ q-point grid, compared with the RTA results computed with ALAMODE on a denser $20\times20\times20$ grid.}
\label{bte}
\end{figure}

%\section*{Data availability}
%The data that support the findings of this article are openly available~\footnote{give me DOI}.
% see https://journals.aps.org/authors/data-availability-statements#citation

\bibliography{refs.bib}
%\bibliographystyle{ieeetr}


\end{document}

\documentclass[%
%reprint,
superscriptaddress,
%groupedaddress,
longbibliography,
%unsortedaddress,
%runinaddress,
%frontmatterverbose, 
%preprint,
%preprintnumbers,
%nofootinbib,
nobibnotes,
%bibnotes,
amsmath,amssymb,
aps,
%pra,
prb,
%rmp,
%prstab,
%prstper,
%showkeys,
floatfix,
twocolumn
]{revtex4-2}

\usepackage{graphicx}% Include figure files
\usepackage{calc}% Calculate margins
\usepackage{dcolumn}% Align table columns on decimal point
\usepackage{bm}% bold math

\usepackage[urlcolor=blue,colorlinks=true,citecolor=blue,linkcolor=blue,pdfstartview={FitH},bookmarks=false]{hyperref} % add hypertext capabilities

%\usepackage[mathlines]{lineno} % Enable numbering of text and display math
% \linenumbers\relax % Commence numbering lines

% \usepackage[showframe,%Uncomment any one of the following lines to test 
% %scale=0.7, marginratio={1:1, 2:3}, ignoreall,% default settings
% %text={7in,10in},centering,
% %margin=1.5in,
% % total={6.5in,8.75in}, top=1.2in, left=0.9in, includefoot,
% % height=10in,a5paper,hmargin={3cm,0.8in},
% ]{geometry}

\usepackage{amsmath}
\usepackage{amssymb}
%\usepackage{orcidlink}
\usepackage{xcolor}
%\usepackage{datetime}
\usepackage[normalem]{ulem}

% Change tracking commands
\newcommand{\trackchange}[3]{\textcolor{#3}{\sout{#1}#2}}  % Full color strikeout, insert
%\renewcommand{\trackchange}[3]{\textcolor{#3}{#2}}        % Just color silent remove and insert
%\renewcommand{\trackchange}[3]{{#2}}                      % No indication, silent remove and insert

% Author marker definitions

\definecolor{myblue}{RGB}{0,127,85}
% \definecolor{violet}{RGB}{102,0,204}
% \definecolor{orange}{RGB}{255,128,0}
% \definecolor{green}{RGB}{0,128,0}
\newcommand{\DL}[1]{\trackchange{}{#1}{blue}}
\newcommand{\AP}[1]{\trackchange{}{#1}{red}}
\newcommand{\PJ}[2]{\trackchange{#1}{#2}{orange}}
\newcommand{\JL}[2]{\trackchange{#1}{#2}{myblue}}
\newcommand{\SP}[2]{\trackchange{#1}{#2}{blue}}
\newcommand{\PP}[2]{\trackchange{#1}{#2}{teal}}
\newcommand{\AI}[2]{\trackchange{#1}{#2}{olive}}
\newcommand{\MS}[1]{\trackchange{}{#1}{purple}}

\newcommand{\TODO}[1]{\textcolor{red}{TODO: #1}}

\sloppy

\begin{document}

\title{Ab initio study of the anharmonic properties and thermal conductivity in $\beta$-FeSi$_2$}

\author{Svitlana~Pastukh}
\email[e-mail: ]{svitlana.pastukh@ifj.edu.pl}
\affiliation{Institute of Nuclear Physics, Polish Academy of Sciences, ul. W. E. Radzikowskiego 152, 31-342 Krak\'{o}w, Poland}

\author{Ma\l{}gorzata~Sternik}
\affiliation{Institute of Nuclear Physics, Polish Academy of Sciences, ul. W. E. Radzikowskiego 152, 31-342 Krak\'{o}w, Poland}

\author{Pawe\l{}~T.~Jochym}
\affiliation{Institute of Nuclear Physics, Polish Academy of Sciences, ul. W. E. Radzikowskiego 152, 31-342 Krak\'{o}w, Poland}


\author{Jan~\L{}a\.{z}ewski}
\affiliation{Institute of Nuclear Physics, Polish Academy of Sciences, ul. W. E. Radzikowskiego 152, 31-342 Krak\'{o}w, Poland}

\author{Andrzej~Ptok}
\affiliation{Institute of Nuclear Physics, Polish Academy of Sciences, ul. W. E. Radzikowskiego 152, 31-342 Krak\'{o}w, Poland}

\author{Svetoslav~Stankov}
\affiliation{Institute for Photon Science and Synchrotron Radiation, Karlsruhe Institute of Technology, D-76131 Karlsruhe, Germany}
\affiliation{Laboratory for Applications of Synchrotron Radiation, Karlsruhe Institute of Technology, D-76131 Karlsruhe, Germany}

\author{Przemys\l{}aw~Piekarz}
\affiliation{Institute of Nuclear Physics, Polish Academy of Sciences, ul. W. E. Radzikowskiego 152, 31-342 Krak\'{o}w, Poland}

\date{\today}

\begin{abstract}

Iron silicides are good candidates for applications in  optoelectronic and thermoelectric devices.
Lattice dynamical properties and thermal conductivity in the $\beta$-FeSi$_2$ semiconductor
are investigated with the first-principles computational methods. 
Phonon dispersion relations are calculated via the
temperature-dependent effective potential method and self-consistent phonon theory. 
To properly model thermal transport, we explicitly consider
the impact of phonon-phonon interactions by analyzing
anharmonic contributions to the phonon self-energy. 
This yields temperature-dependent phonon frequencies and linewidths,
reflecting the finite lifetime of phonons due to scattering
processes. The calculated phonon frequencies and line profiles are used to obtain 
the Raman spectra, which shows good agreement with the experimental data. 
We revealed an enhanced anharmonic behaviour of the Raman modes with the highest frequencies.  
The lattice thermal conductivity is then obtained as a function of temperature and crystallite size within
the relaxation-time approximation.
Phonon transport shows a small anisotropy due to the orthorhombic structure and a very weak dependence
on the quartic anharmonic corrections. The results obtained for an infinite material and for several crystallite sizes
were analyzed and compared with the available experimental data.
\end{abstract}

\maketitle


\section{Introduction}

The comprehensive determination of important physical properties of
crystals, such as thermal expansion, lattice thermal
conductivity or structural phase transitions, requires a fundamental 
understanding of the anharmonic effects.
Although the investigation of anharmonic interactions in crystals has
attracted a considerable interest for decades~\cite{cowley_1968}, a substantial progress
has only recently been achieved thanks to advances in theoretical and
numerical methods and increased computational power.
Now, phonon frequencies, lifetimes, and heat transfer in a wide range of
materials
can be quantitatively predicted using the available computational resources
based on the density functional theory (DFT)~\cite{lindsay_2013,mcgaughey_2019,lindsay_2019}.
In the case of strongly anharmonic systems, the self-consistent phonon
(SCPH) theory~\cite{tadano_2015} as well as the perturbative approach~\cite{tadano_2018}, using higher
order interatomic force constants derived from the fitting to the displacement force data obtained
with DFT, have proven to be successful.

Transition-metal silicides are promising materials for fabrication of electronic components
designed for integration with silicon-based circuits~\cite{murarka_1995}.
At room temperature, iron disilicide ($\beta$-FeSi$_2$) is a
direct-bandgap semiconductor~\cite{bost_1985}, making this material a good
candidate for application in optoelectronic devices such as infrared
detectors or light emitters~\cite{bost_1988}. The development of light-emitting diodes utilizing FeSi$_2$/Si heterostructures has been successfully demonstrated~\cite{leong_1997,suemasu_2001}. Due to
a high thermal stability and strong light absorption, FeSi$_2$ is
also a suitable photovoltaic material~\cite{powalla_1993,liu_2006,okuhara_2017}.

$\beta$-FeSi$_2$ crystallizes in the base-centered orthorhombic
lattice~\cite{dusausoy_1971} transforming to the
tetragonal metallic $\alpha$-FeSi$_{2}$ phase around $1200$~K~\cite{starke_2002}.
Optical studies indicated a direct band gap of the values
$0.85$--$0.89$~eV~\cite{bost_1988,dimitriadis_1990,arushanov_1995,wan_2003},
however, the {\it ab initio} calculations predicted a smaller indirect gap close to 0.8 eV~\cite{christensen_1990}. 
The existence of such an indirect gap was then confirmed by the optical linear transmittance measurements at
low temperatures~\cite{giannini_1992}. As shown by first-principles studies,
the character of the band gap is very sensitive to the orientation 
of a crystal grown on silicon~\cite{clark_1998}.

$\beta$-FeSi$_2$ belongs also to good thermoelectric materials~\cite{ware_1964}, with potential applications resulting from its chemical stability up to high temperatures, nontoxicity, and low cost of preparation~\cite{yamada_2012,nozariasbmarz_2017}. 
It has already been implemented in cars~\cite{birkholz_1988} and portable power sources~\cite{uemura_1989}. 
Its thermoelectric performance can be improved by doping~\cite{ito_2001,tani_2001,kim_2003,chen_2005,pandey_2013,le_tonquesse_2019}, 
which enhances the electric transport and reduces the thermal conductivity~\cite{waldecker_1973,du_2019,du_2020}.  
The thermal conductivity can be also reduced by the modification of microstructure~\cite{ail_2015} or by
nanostructurization~\cite{watanabe_2017,taniguchi_2017,hsin_2017,abbassi_2021}.

The lattice thermal conductivity is directly connected with anharmonic effects and phonon scattering processes.
The vibrational properties of \mbox{$\beta$-FeSi$_2$} were studied by the infrared and Raman spectroscopy~\cite{lefki_1991,guizzetti_1997,maeda_2004,baleva_2008,liu_2011,maeda_2011}. The observed anisotropy in the phonon spectra results from the enhanced sensitivity of the infrared and Raman features to the local lattice distortions~\cite{guizzetti_1997}. The Fe phonon density of states was measured by nuclear inelastic scattering (NIS), showing a good agreement with the density functional theory (DFT) calculations~\cite{walterfang_2005}. Using the DFT approach, the phonon dispersion curves, phonon density of states, as well as various thermodynamic properties were obtained within the harmonic approximation~\cite{tani_2010,liang_2011}. The extended Klemens model was applied to study
the anharmonic effect on phonon frequencies and linewidths observed by the Raman spectroscopy~\cite{zhang_2023}.
The impact of nanostructurization on lattice dynamics was explored in the $\beta$-FeSi$_2$ nanorods grown on the Si(110) surface by the NIS and {\it ab initio} methods~\cite{kalt_2022}.

In this work, we investigate the lattice dynamical properties of $\beta$-FeSi$_2$ 
using the DFT calculations. We study the effect of anharmonic terms in the temperature-dependent potential on phonon frequencies and lifetimes. We focus on the Raman modes, comparing the theoretical results with the experimental data.
The thermal conductivity is derived in a broad temperature range and the effect of crystallite size is analyzed.



This study is structured as follows.
In Sec.~\ref{sec.com} we describe the details of computational methods.
Next, in Sec.~\ref{sec.result} we present and discuss our results.
In particular we present the crystal structure (Sec.~\ref{sec.crys}) and lattice dynamics (Sec.~\ref{sec.lattice}).
We investigate also the thermal conductivity comparing the obtained results with the available experimental data (Sec.~\ref{sec.thermal}).
Finally, Sec.~\ref{sec.summary} summarizes our key findings and conclusions.

\section{Calculation method}
\label{sec.com}


The calculations were performed using the projector augmented-wave potentials~\cite{blochl_1994} and the generalized gradient approximation~\cite{perdew_1996} implemented in the Vienna Ab initio Simulation Package (VASP)~\cite{kresse.hafner.94,kresse.furthmuller.96,kresse.joubert.99}. 
The lattice parameters and atomic positions were optimized in the ${\bm a} \times ({\bm b}-{\bm c}) \times ({\bm b}+{\bm c})$ supercell containing 32 formula units and four primitive cells.
The integration in the reciprocal space was conducted using the $2 \times 2 \times 2$ Monkhorst--Pack mesh~\cite{monkhorst_1976} and the cut-off energy was set to $500$~eV. For convergence conditions, we set the energy change below $10^{-5}$ and $10^{-8}$ for the ionic and electronic loops, respectively. 

The lattice dynamical properties were studied within the temperature-dependent effective potential (TDEP) approach~\cite{hellman_2013}. The atomic potential with the third and fourth order anharmonic terms was derived from interatomic forces induced by displacements of all atoms at finite temperatures.
The sets of atomic displacements were generated by the high efficiency configuration space sampling (HECSS)~\cite{jochym_2021} and forces were obtained by VASP. The interatomic force constants and phonon frequencies were calculated with the {\sc Alamode} software~\cite{tadano_2014}.

Furthermore, we have attempted to construct a {\emph{temperature independent}} 
anharmonic model. We have used combined data from all investigated temperatures 
(300, 600 and 1000~K) and fitted a large (over 15~000 free parameters), fourth-order 
interaction model to this dataset. 
Subsequently, we have used this model to calculate line profiles and positions of Raman-active modes
at multiple temperatures.

The changes in phonon frequencies induced by the anharmonic effects were investigated within two approaches.
First, the impact of the quartic anharmonic terms was included using the SCPH theory~\cite{tadano_2015}.
Second, the mode profiles (frequency shifts and line widths) were determined from the real and imaginary parts of the phonon self-energy resulting from the cubic and quartic anharmonic terms of the above mentioned large model~\cite{tadano_2018}. 
The longitudinal optic-transverse optic (LO-TO) splitting was also evaluated, using the static dielectric tensor and Born effective charges calculated within density functional perturbation theory~\cite{gajdos_2006}.

\begin{figure}[]
    \centering
    \includegraphics[width=\linewidth]{fig1_new.png}
\caption{%
(a) The conventional unit cell of $\beta$-FeSi$_2$ (with Cmca symmetry) and (b) the corresponding Brillouin zone with selected high-symmetry points.
}
  \label{fig.struct}
\end{figure}


To further characterize the vibrational properties, Raman scattering was investigated using the Phonopy-Spectroscopy package~\cite{skelton_2017}. This enabled the identification of Raman-active modes and the calculation of Raman tensors. Anharmonic force constants, derived from calculations using {\sc Alamode}, were then used to obtain theoretical line profiles for the Raman modes. The presented Raman scattering spectra combine these anharmonic line profiles with the Raman tensor amplitudes.
This analysis was also based on the large quartic model mentioned above.

Finally, the thermal conductivity was calculated as a function of temperature and crystallite size within the relaxation-time approximation (RTA) \SP{}{as implemented in {\sc Alamode}~\cite{tadano_2014}}. The phonon lifetimes were calculated from the phonon self-energy including the cubic and quartic anharmonic terms. \SP{}{The RTA provides a solution to the Boltzmann transport equation (BTE) under the assumption that scattering events are independent and can be treated through mode-resolved relaxation times.
To verify the validity of this approximation for $\beta$-FeSi$_2$, we additionally
solved the BTE iteratively.} \SP{}{These calculations were performed with
{\sc Phono3py}~\cite{togo_2023}. For cross-validation, the additional RTA calculations were performed on the same $\bm{q}$-grid as the BTE calculations, using the implementation provided by {\sc Phono3py}.}

\section{Results}
\label{sec.result}

\subsection{Crystal structure}
\label{sec.crys}

The $\beta$-FeSi$_2$ structure adopts a base-centered orthorhombic lattice with the space group Cmca (No.~64) as shown in Fig.~\ref{fig.struct}(a).
The unit cell consists of two primitive cells and contains 48 atoms.
Iron (silicon) atoms possess two nonequivalent positions: \mbox{Fe-I} and \mbox{Fe-II} (\mbox{Si-I} and \mbox{Si-II}), presented in Fig.~\ref{fig.struct}(a) as gray and purple (orange and yellow) spheres, respectively.
This crystal structure is derived from the fluorite-type lattice with strongly distorted Si cubes and Fe atoms occupying 
one-half of the central sites.
The Fe-I and Fe-II sites create different layers perpendicular to the $x$ direction, and they are separated by layers containing both Si sites.
Each Fe atom is coordinated by 8 Si atoms with slightly different Fe-Si distances.
The optimized lattice constants ($a = 9.874$~{\AA}, $b = 7.767$~{\AA}, and
$c = 7.811$~{\AA}) agree very well with the experimental data ($a = 9.863$~{\AA}, $b = 7.791$~{\AA}, and $c = 7.833$~{\AA})~\cite{dusausoy_1971}.

Iron atoms occupy the Wyckoff sites 8\textit{d} ($0.2166$, $0$, $0$) and  8\textit{f} ($0$, $0.3072$, $0.1879$), corresponding to \mbox{Fe-I} and \mbox{Fe-II}, respectively. 
Silicon atoms are located at two inequivalent 16\textit{g} positions: ($0.1282$, $0.2737$, $0.0495$) and \mbox{($0.3734$, $0.0445$, $0.2270$)}, assigned as \mbox{Si-I} and \mbox{Si-II}.
The optimized positions of atoms agree very well with the experimental data~\cite{dusausoy_1971} and the previous theoretical studies~\cite{tani_2010,liang_2011}.


\subsection{Lattice dynamics}
\label{sec.lattice}

\begin{figure}[]
    \centering
    \includegraphics[width=\linewidth]{Fig1c.pdf}
\caption{%
The phonon dispersion curves along high symmetry directions obtained within SCPH (for temperatures from $0$ to $1000$~K).
Dashed black lines indicate the phonon dispersions obtained from the harmonic approximation. The white dots indicate Raman-active modes with A$_g$ symmetry.
The vertical plot shows the phonon density of states (DOS) calculated at a reference temperature of $600$~K.
}
  \label{fig.ph_band}
\end{figure}


In Fig.~\ref{fig.ph_band} we present the phonon dispersion relations of $\beta$-FeSi$_2$ along high-symmetry directions in the Brillouin zone [Fig.~\ref{fig.struct}(b)].
Due to 24 atoms in the primitive cell, the phonon spectrum consists of 69 optical modes and three acoustic modes.
The phonon dispersions were calculated within the SCPH approach in the temperature range $0$--$1000$~K (presented by color lines in Fig.~\ref{fig.ph_band}), and they are compared with the results obtained from the harmonic part of the effective potential corresponding to temperature $T=300$~K (indicated by dashed black lines in Fig.~\ref{fig.ph_band}).
As we can see within the SCPH method, the anharmonic effects are rather weak and leads only to small renormalization of phonon frequencies.
Only the highest modes show more pronounced shifts of their frequencies to larger values.
The total and partial element-projected phonon density of states obtained within the harmonic approximation are presented in Fig.~\ref{fig.ph_band}.
Up to around $320$~cm$^{-1}$, the contributions from both elements are very similar, while for higher frequencies the spectrum is dominated by
the Si vibrations.


\begin{table}[!t]
\begin{ruledtabular}
\caption{Calculated and experimental Raman-active modes of $\beta$-FeSi$_2$ with their irreducible representations (IR). Present theoretical results are compared with the previous theoretical data from Ref.~\cite{tani_2010} and experimental results from \mbox{Refs.~\cite{lefki_1991,maeda_2004}.} The experimental frequencies with known symmetries (A$_g$) are shown in bold, while other experimental modes are assigned to the best fitting theoretical values.}
\begin{tabular}{c c c c c}
\textbf{IR} & \multicolumn{4}{c}{\textbf{Frequency (cm$^{-1}$)}} \\
 & Present  & Theor.~\cite{tani_2010} & Exp.~\cite{lefki_1991} & Exp.~\cite{maeda_2004} \\
\hline
B$_{2g}$ & 175.4 & 179 & 176 &  \\
B$_{1g}$ & 176.3 & 185 & 179  &  \\
B$_{1g}$ & 193.3 & 198 &  & 190.6 \\
A$_g$    & 196.6 & 208 & \textbf{195} & \textbf{194.0} \\
A$_g$    & 203.9 & 210 & \textbf{197} & 199.6 \\
B$_{3g}$ & 205.5 & 212 & 200 &  \\
B$_{3g}$ & 226.3 & 236 & 206 & 227.1 \\
B$_{1g}$ & 233.3 & 240 &  &  231.6 \\
B$_{2g}$ & 248.6 & 254 &  &  \\
A$_g$    & 250.1 & 257 & \textbf{247} & \textbf{247.3} \\
A$_g$    & 254.9 & 264 & \textbf{253} & 254.3 \\
B$_{1g}$ & 275.5 & 285 &  &  274.1 \\
B$_{2g}$ & 282.4 & 295 &  &  281.2 \\
B$_{3g}$ & 286.8 & 297 &  &  \\
B$_{1g}$ & 307.1 & 317 &  &  \\
B$_{2g}$ & 312.6 & 326 &  &  311.8 \\
B$_{1g}$ & 319.0 & 324 &  &  \\
B$_{3g}$ & 327.9 & 341 &  & 325.8 \\
B$_{2g}$ & 333.3 & 345 &  &  \\
A$_g$    & 339.3 & 352 & \textbf{346} & 339.5 \\
B$_{2g}$ & 343.0 & 350 &  &  \\
B$_{1g}$ & 353.5 & 366 &  &  \\
B$_{2g}$ & 372.8 & 383 &  & 370.7 \\
B$_{1g}$ & 375.1 & 385 &  &  \\
B$_{3g}$ & 375.2 & 386 &  &  \\
B$_{3g}$ & 385.3 & 401 &  &  \\
A$_g$    & 386.5 & 398 &  & 386.2 \\
B$_{2g}$ & 387.7 & 402 &  & 388.2 \\
A$_g$    & 404.6 & 415 &  & 400.4 \\
B$_{2g}$ & 405.1 & 420 &  &  \\
B$_{1g}$ & 412.7 & 428 &  &  \\
B$_{3g}$ & 418.4 & 431 &  &  \\
B$_{3g}$ & 441.4 & 458 &  & 442.6 \\
B$_{3g}$ & 447.3 & 466 &  & 446.3 \\
A$_g$    & 448.4 & 464 &  &  \\
A$_g$    & 499.1 & 517 &  &  \\
\end{tabular}
\label{T1Raman}
\end{ruledtabular}
\end{table}

%subsection{Raman}

\begin{figure*}
\centering
  \includegraphics[width=\textwidth]{spectrum_RTA+raman.pdf}
\caption{Raman spectrum of $\beta$-FeSi$_2$ calculated within the perturbative approach at three temperatures 
($300$, $600$, $900$~K -- blue, orange, and red lines, respectively). 
The calculated spectrum includes all Raman-active modes. The A$_g$ modes
are indicated by purple vertical lines at peak positions corresponding to T=300~K. 
The frequencies derived from the harmonic approximation at T=300~K are indicated by green lines.
Connecting arrows indicate the correspondence between harmonic and
anharmonic frequencies, demonstrating the frequency shifts due to 
phonon interactions. Experimental values for the A$_g$ modes
based on Ref.~\cite{lefki_1991} are marked with black dashed lines.
}
\label{raman}
\end{figure*}


The phonon spectrum at the $\Gamma$ point consists of 36 Raman modes classified according to the irreducible representations: $9A_\text{g}+9B_\text{1g}+9B_\text{2g}+9B_\text{3g}$. 
Fig.~\ref{raman} shows the Raman spectrum of A$_g$ symmetry calculated for $\beta$-FeSi$_2$ within the perturbative approach at three temperatures $300$, $600$, and $900$~K (solid blue, orange, and red curves, respectively), including third-order and fourth-order anharmonic corrections.
The calculated Raman spectrum includes all Raman-active modes.
The five experimentally observed A$_g$ modes are highlighted by black dashed lines, based on data from Ref.~\cite{lefki_1991}.
Additional peaks not marked with vertical lines correspond to phonon modes with symmetries other than A$_g$.
The frequencies of Raman modes obtained in anharmonic calculations are compared with the previous results calculated within the harmonic approximation and the experimental values in Tab.~\ref{T1Raman}. 
We have marked in bold the experimentally determined A$_g$ modes, which are compared with the calculations.
Since the experimental studies did not provide the accurate assignement of the Raman modes with the B$_{1g}$, B$_{2g}$, and B$_{3g}$ symmetry~\cite{lefki_1991,maeda_2004}, we cannot compare them directly with the theoretical results.
However, in Tab.~\ref{T1Raman} we have assigned the measured frequencies to the best fitting theoretical values without taking into account the symmetry of the modes, except for the known A$_{g}$ modes. 


The impact of anharmonicity on the phonon frequencies is well visible from the comparison of the results obtained within the harmonic approximation 
and from the anharmonic calculation (vertical green and purple lines in Fig.~\ref{raman}, respectively).
Here we show only the A$_g$ modes, which are compared with the experimental results (vertical black dashed lines).
Anharmonic frequencies calculated at $300$~K are indicated by purple lines, while frequencies derived from harmonic approximation are marked with green lines. The grey solid lines connect corresponding modes obtained in both approximations.
In most cases, the results obtained within the harmonic approximation do not agree with the experimental frequencies. 
%Only the modes close to $250$~cm$^{-1}$ and $400$~cm$^{-1}$ correspond well to the experimental values. MS
As we can see, the inclusion of the anharmonic correction leads to a significant modification of the phonon frequencies.
These anharmonic effects are stronger for higher-frequency modes mainly because of
the dominant contribution from Si atoms, which vibrate with larger amplitudes than heavier Fe atoms. 
When atoms move to larger distances the potential deviates more from the harmonic approximation,
and the anharmonic corrections become stronger.


The modification of phonon frequencies observed in Fig.~\ref{raman} is much larger than in the SCPH scheme presented in Fig.~\ref{fig.ph_band}. The SCPH approach includes only the leading-order contribution to 
the phonon self-energy obtained from the quartic anharmonic terms~\cite{tadano_2015}. Therefore, it does not describe fully the changes of phonon frequencies found within the perturbation theory (see Fig.~\ref{raman}).
Especially, it is well visible for two highest A$_g$ modes, which exhibit also the largest line broadening 
and the strongest dependence on temperature.
Therefore, a better agreement with experimentally observed frequencies is visible,
confirming the significant influence of the anharmonicity on the frequencies and line profiles of phonon modes.
In fact, the decrease of phonon frequencies should be even stronger due to thermal expansion, 
which is not included in our calculations.
Within SCPH the frequencies of the highest modes increase with increasing temperature as we see in Fig.~\ref{fig.ph_band}. The comparison of two different approaches applied to study anharmonic properties of $\beta$-FeSi$_2$ shows that the perturbation theory, which includes the cubic and quartic terms, better describes the changes of phonon frequencies with temperature than the SCPH method. \SP{}{This indicates that the leading-order contribution included in SCPH are not important in this material.}

Additionaly we should nottice that for other than A$_g$ modes we cannot make an unambiguous assignment of theoretical frequencies to experimental ones. Note that the spectrum in Fig.~\ref{fig.ph_band} contains all Raman-active modes. 
The limitation to A$_g$ modes concerns only the indicated positions of the peaks. 

\SP{}{The full Raman spectrum, with the frequencies of the B$_{1g}$, B$_{2g}$ and B$_{3g}$ modes marked, is shown in Appendix A, Fig.~\ref{raman1}. This represents our theoretical prediction of possible Raman-mode assignments, which can be verified in future experiments.}
%This is the theoretical prediction of possible assignment of Raman modes that can be verified in future experiments.  


\subsection{Thermal conductivity}
\label{sec.thermal}


\begin{figure*}[!t]
\centering
  \includegraphics[width=\linewidth]{time3.pdf}
\caption{Phonon lifetimes calculated for three temperatures as a function of phonon frequency. The colors correspond to the phonon branches.}
  \label{thermtime}
\end{figure*}


In this section, we analyze the thermal conductivity tensor of $\beta$-FeSi$_2$ obtained within the RTA approach~\cite{tadano_2018} as a function of temperature
%
\begin{equation}
\kappa_{\text{ph}}^{\mu\nu}(T) = \frac{1}{NV} \sum_{\bm{q},j} c_{\bm{q}j}(T) v_{\bm{q}j}^{\mu} v_{\bm{q}j}^{\nu}\tau_{\bm{q}j}(T),
\end{equation}
% 
where $c_{\bm{q}j}$ is the mode heat capacity and $v_{\bm{q}j}$ is the mode group velocity. 
The relaxation time is approximated by the phonon lifetime $\tau_{\bm{q}j}$
calculated for $j$-th branch at the wave vector $\bm{q}$.
$V$ is the unit cell volume and $N$ is the number of unit cells in the crystal.
The phonon lifetime is calculated using this formula
%
\begin{equation}
\tau_{\bm{q}j}(T)=\frac{1}{2\Gamma_{\bm{q}j}^{\text{anh}}(T)},
\end{equation}
%
where $\Gamma_{\bm{q}j}^{\text{anh}}$ is the anharmonic phonon linewidth obtained from 
the imaginary part of the phonon self-energy within the perturbation theory.

In Fig.~\ref{thermtime}, we present $\tau_{\bm{q}j}$ obtained for three temperatures 300, 600, and 1000~K as a function of frequency. As we see, the acoustic phonons close to the $\Gamma$ point have the longest lifetimes,
which are diminished with increasing frequency reaching local minima around $200$~cm$^{-1}$.
For higher frequencies, phonon lifetimes first increase to local maxima around $300$~cm$^{-1}$ and then decrease to
the lowest values in the range of highest optical modes. The shortest lifetimes correspond to the largest line broadening
observed for the Raman modes in Fig~\ref{raman}. The phonon group velocities 
$v_{\bm{q}j}=\partial\omega_{\bm{q}j}/\partial\bm{q}$, which are obtained by the central difference formula, are presented in Fig.~\ref{thermvelocity}. Their temperature dependence is negligible, therefore, we present only the results for $T=600$~K.
At low frequencies, there are clearly two ranges of group velocities of the acoustic phonons. 
The larger values correspond to the longitudinal modes, while the lower values are obtained from the
transverse acoustic branches. Group velocities of acoustic phonons decrease for larger frequencies
and reach the average values typical for optic branches.

\begin{figure}[!t]
\centering
  \includegraphics[width=\linewidth]{GV.pdf}
\caption{Mode group velocities calculated as a function of phonon frequency. The colors correspond to the phonon branches.
}
  \label{thermvelocity}
\end{figure}

In Fig.~\ref{anizotropy}(a), we present the three diagonal elements of $\kappa_{\text{ph}}^{\mu\nu}$ corresponding to the main directions of the crystal structure. 
They were obtained from the force constants calculated at the base temperature $T=600$~K and the crystallite size $0.1$~$\mu$m to account for boundary-limited phonon transport. 
Due to the orthorhombic symmetry, we observe a small anisotropy in phonon transport in the whole temperature range. 
At low temperatures, the three components of the heat conductivity increase in a very similar way with the $\kappa_{\text{ph}}^{yy}$ element slightly larger than two other components. 
After reaching the maximum, we observe a change in the largest component from $\kappa_{\text{ph}}^{yy}$ to 
$\kappa_{\text{ph}}^{xx}$.    
In Fig.~\ref{anizotropy}(b), the thermal conductivity is shown for three base temperatures, at which the interatomic potential was obtained ($300$~K, $600$~K, and $1000$~K), using the energy expansion up to third- and fourth-order anharmonic terms, and the same structure size of $0.1$~$\mu$m. At lowest temperatures, the thermal conductivity strongly increases, reaching the maximum around $T=180$~K, then it shows a slower decrease with temperature.
The differences between the two levels of approximation are minimal, suggesting that third-order calculations already capture the dominant phonon scattering mechanisms. The dependence on the base temperature is also very weak, showing the changes in the heat conductivity within a few percent. 
\SP{}{Further verification of the reliability of our thermal conductivity results is given in Appendix~\ref{ThermalAB}, where the full BTE calculations show good agreement with RTA and higher-resolution RTA results, demonstrating that the $8\times8\times8$ $\bm{q}$-mesh already provides converged values.}

\begin{figure}[!t]
\centering
  \includegraphics[width=\linewidth]{Anizotropy.pdf}
\caption{(a) The anisotropic thermal conductivity of $\beta$-FeSi$_2$ calculated along the lattice directions at 600~K. (b) The average temperature-dependent thermal conductivity taken at $300$~K, $600$~K, and $1000$~K, including anharmonic corrections up to cubic (A3) and quartic (A4) terms. In both cases the crystallite size is 0.1~$\mu$m.}
  \label{anizotropy}
\end{figure}


In Fig.~\ref{therm}, we fix the base temperature at $600$~K and examine the effect of crystallite size on thermal conductivity, varying it from $0.01$ to $0.5$~$\mu$m.
With decreasing the crystallite size, we observe a shift of the position of the maximum to larger temperatures and a decrease of the thermal conductivity in the entire temperature range.
Theoretical results are compared with several experimental data obtained above the room temperature. 
The measured thermal conductivity depends to a large extent on the sample quality, its purity and the size of the crystalline grains which depends on 
the production processes.
Many measurements were performed using crystallites of micrometric or unknown size ~\cite{waldecker_1973,ito_2002,kim_2003,du_2020}, however, 
numerous attempts to minimize $\kappa$ by reducing grain sizes to $56$~nm~\cite{dabrowski_2019, dabrowski_microstructure_2021}, $30$-$400$~nm~\cite{le_tonquesse_2019}, $50$ and $200$~nm~\cite{abbassi_2021}, or introducing pores into the material~\cite{sam2023improved} 
are also carried out. 
Another way to change the thermal conductivity is to dope $\beta$-FeSi$_2$ with different elements~\cite{ito_2002,kim_2003,du_2020,cheng_2024}, however, this effect is beyond our investigation.

\begin{figure}[!t]
\centering
  \includegraphics[width=\linewidth]{Thermal_conductivity3.pdf}
\caption{
The phonon thermal conductivity of $\beta$-FeSi$_2$: theoretical results for the infinite crystalline size and with boundary conditions, compared with experimental data for different structure sizes.
}
  \label{therm}
\end{figure}
%Fig.~\ref{therm}\cite{abbassi_2021}\cite{waldecker_1973}\cite{dabrowski_2019}\cite{sam2023improved}\cite{kim_2003}\cite{du_2020}\cite{ito_2002}\cite{le_tonquesse_2019}.

We observe a decrease in the thermal conductivity with reducing crystalline grain sizes in all analyzed experimental data. 
For instance, by decreasing the crystallite size to $50$~nm, the thermal conductivity at room temperature was reduced by a factor of $1.7$, what can be  compared to the annealed sample with 200 nm grains~\cite{abbassi_2021}. 
It is worth noting that the rate of decrease in value with increasing temperature in both cases, for grain sizes of $50$~nm and $200$~nm, is significantly different, which is consistent with our calculations. 
The same trend can be observed by comparing the thermal conductivity measured for a sample with bulk crystallite sizes with the thermal conductivity of a sample with grains smaller than 400 nm~\cite{le_tonquesse_2019}.
The theoretical results obtained for the same crystallite size show higher values due to factors not captured in the idealized model, such as crystal imperfection or mechanical strain. Usually, a decrease in the crystallite size is related to an increased concentration of grain boundaries, point defects, and stacking faults that influence the phonon scattering~\cite{le_tonquesse_2019,abbassi_2021}.   

We should note that the total thermal conductivity is a combination
of the lattice and electronic contributions to the heat transport.
In semiconductors, the electronic thermal conductivity is negligible at low temperatures and significantly increases 
only much above the room temperatures~\cite{gu_2020}.
For $\beta$-FeSi$_2$, the electronic thermal conductivity was obtained from the electric conductivity using the Wiedemann-Franz law~\cite{ito_2002,kim_2003,le_tonquesse_2019}.
In the undoped material, its value does not exceed $0.1$~W/mK in the measurement up to $T=950$~K~\cite{kim_2003}.
By doping, the electronic thermal conductivity can be enhanced, and it has a direct impact on the thermoelectric properties of $\beta$-FeSi$_2$ at high temperatures~\cite{ito_2002,kim_2003}.
In the present study, we consider only the phonon contribution to the thermal conductivity,
therefore, agreement with experimental data may deteriorate with increasing temperature.

\section{Summary}
\label{sec.summary}

We performed {\it ab initio} studies on lattice dynamical and thermal transport properties of $\beta$-FeSi$_2$. The effect of anharmonicity was analyzed within two approaches -- the SCPH method and the perturbation theory.
The phonon dispersion curves obtained within SCPH show small renormalization of frequencies comparing to the harmonic approximation. 
The Raman spectra were calculated within the procedure which takes into account the peak intensities obtained from the Raman tensors and the line profiles obtained from the phonon self energy derived within the perturbation theory based on the large, temperature-independent, quartic model fitted to the data from the wide range of temperatures (300-1000~K). 
The anharmonic corrections strongly affect the frequencies and line profiles of some modes and results in overall better agreement with the experimental data. 
We analyzed the phonon lifetimes and group velocities obtained as functions of the phonon frequency.
Then the lattice thermal conductivity was calculated for a broad range of temperatures and grain sizes.
We found a small anisotropy in the phonon thermal transport resulting from the orthorhombic structure and a weak effect of the quartic anharmonic terms. 
The thermal conductivity calculated for various crystalline grain sizes show a good qualitative agreement with the available measurements.

\begin{acknowledgments}
Some figures in this work were rendered using {\sc Vesta}~\cite{momma.izumi.11} software.
This work was partially supported by the Ministry of Education, Youth and Sports of the Czech Republic through the e-INFRA CZ (ID:90254).
\end{acknowledgments}

\appendix

\section{Raman spectrum}
\label{RamanAA}

Based on the polarized Raman measurements reported in Ref.~\cite{maeda_2004}, two Raman peaks were identified as belonging to the A$_g$ symmetry class, and several additional peaks were observed with similar or different polarization dependence. Although the authors of Ref.~\cite{maeda_2004} provided estimates of the relative Raman tensor components, they did not specify which of the remaining modes correspond to the B$_{1g}$, B$_{2g}$, or B$_{3g}$ symmetries. Because of this missing experimental information, a direct symmetry-resolved comparison between the measurement and theory is not currently possible for the non-A$_g$ modes. 
To provide a complete theoretical picture of the B$_g$-type modes, we show here the calculated Raman-active frequencies and intensities for the B$_{1g}$, B$_{2g}$, and B$_{3g}$ symmetries only. 
Fig.~\ref{raman1} shows the predicted Raman modes for the B$_g$ symmetries in $\beta$-FeSi$_2$. The results obtained within the harmonic approximation are compared with the anharmonic perturbation theory calculations which provides both, frequency shifts and predicted line profiles of the modes.
Although the experimentally measured peaks cannot be directly assigned to these symmetries due to the lack of polarization-resolved data, the theoretical predictions provide a reference for comparison. Matching the measured frequencies to the closest theoretical B$_g$ modes (Table~\ref{T1Raman}) allows for a tentative assignment, which can guide future polarization-resolved Raman experiments aimed at determining the precise symmetry of the unresolved peaks.

\begin{figure*}[t]
\centering
  \includegraphics[width=\linewidth]{spectrum_RTA+raman_Bi_modes.pdf}
\caption{Raman spectrum of $\beta$-FeSi$_2$ calculated at three temperatures 
($300$, $600$, $900$~K -- blue, orange, and red lines, respectively). 
The calculated spectrum includes all Raman-active modes. The B$_{ig}$ modes
are indicated by purple vertical lines at peak positions corresponding to T=300~K. 
The frequencies derived from the harmonic approximation at T=300~K are indicated 
by green, yellow and pink lines.
Connecting arrows indicate the correspondence between harmonic and anharmonic 
frequencies, demonstrating the frequency shifts due to phonon interactions.}
\label{raman1}
\end{figure*}
\section{Thermal conductivity obtained from BTE and RTA}
\label{ThermalAB}

The thermal conductivity was computed by solving the full BTE on the largest feasible $\bm{q}$-point grid, $8\times8\times8$, and compared with the corresponding RTA results obtained on the same grid. As shown in Fig.~\ref{bte}, the difference between the components of the thermal conductivity tensor obtained within BTE and RTA at this resolution is very small, indicating a good agreement between these two approaches.
Moreover, we performed an additional calculation using RTA on a denser $20\times20\times20$ grid. As seen in the Fig.~\ref{bte}, the higher-resolution data remain in a good agreement with both the BTE and RTA results for the $8\times8\times8$ grid.
It shows that the $8\times8\times8$ mesh already provides good results for this structure and confirms reliability of the calculations.

\begin{figure}[!h]
\centering
  \includegraphics[width=\linewidth]{BTEvsRTAvsANP.pdf}
\caption{Thermal conductivity of $\beta$-FeSi$_2$ obtained within the BTE and RTA methods using the Phono3py software on the $8\times8\times8$ q-point grid, compared with the RTA results computed with ALAMODE on a denser $20\times20\times20$ grid.}
\label{bte}
\end{figure}

%\section*{Data availability}
%The data that support the findings of this article are openly available~\footnote{give me DOI}.
% see https://journals.aps.org/authors/data-availability-statements#citation

\bibliography{refs.bib}
%\bibliographystyle{ieeetr}


\end{document}

\documentclass[%
%reprint,
superscriptaddress,
%groupedaddress,
longbibliography,
%unsortedaddress,
%runinaddress,
%frontmatterverbose, 
%preprint,
%preprintnumbers,
%nofootinbib,
nobibnotes,
%bibnotes,
amsmath,amssymb,
aps,
%pra,
prb,
%rmp,
%prstab,
%prstper,
%showkeys,
floatfix,
twocolumn
]{revtex4-2}

\usepackage{graphicx}% Include figure files
\usepackage{calc}% Calculate margins
\usepackage{dcolumn}% Align table columns on decimal point
\usepackage{bm}% bold math

\usepackage[urlcolor=blue,colorlinks=true,citecolor=blue,linkcolor=blue,pdfstartview={FitH},bookmarks=false]{hyperref} % add hypertext capabilities

%\usepackage[mathlines]{lineno} % Enable numbering of text and display math
% \linenumbers\relax % Commence numbering lines

% \usepackage[showframe,%Uncomment any one of the following lines to test 
% %scale=0.7, marginratio={1:1, 2:3}, ignoreall,% default settings
% %text={7in,10in},centering,
% %margin=1.5in,
% % total={6.5in,8.75in}, top=1.2in, left=0.9in, includefoot,
% % height=10in,a5paper,hmargin={3cm,0.8in},
% ]{geometry}

\usepackage{amsmath}
\usepackage{amssymb}
%\usepackage{orcidlink}
\usepackage{xcolor}
%\usepackage{datetime}
\usepackage[normalem]{ulem}

% Change tracking commands
\newcommand{\trackchange}[3]{\textcolor{#3}{\sout{#1}#2}}  % Full color strikeout, insert
%\renewcommand{\trackchange}[3]{\textcolor{#3}{#2}}        % Just color silent remove and insert
%\renewcommand{\trackchange}[3]{{#2}}                      % No indication, silent remove and insert

% Author marker definitions

\definecolor{myblue}{RGB}{0,127,85}
% \definecolor{violet}{RGB}{102,0,204}
% \definecolor{orange}{RGB}{255,128,0}
% \definecolor{green}{RGB}{0,128,0}
\newcommand{\DL}[1]{\trackchange{}{#1}{blue}}
\newcommand{\AP}[1]{\trackchange{}{#1}{red}}
\newcommand{\PJ}[2]{\trackchange{#1}{#2}{orange}}
\newcommand{\JL}[2]{\trackchange{#1}{#2}{myblue}}
\newcommand{\SP}[2]{\trackchange{#1}{#2}{blue}}
\newcommand{\PP}[2]{\trackchange{#1}{#2}{teal}}
\newcommand{\AI}[2]{\trackchange{#1}{#2}{olive}}
\newcommand{\MS}[1]{\trackchange{}{#1}{purple}}

\newcommand{\TODO}[1]{\textcolor{red}{TODO: #1}}

\sloppy

\begin{document}

\title{Ab initio study of the anharmonic properties and thermal conductivity in $\beta$-FeSi$_2$}

\author{Svitlana~Pastukh}
\email[e-mail: ]{svitlana.pastukh@ifj.edu.pl}
\affiliation{Institute of Nuclear Physics, Polish Academy of Sciences, ul. W. E. Radzikowskiego 152, 31-342 Krak\'{o}w, Poland}

\author{Ma\l{}gorzata~Sternik}
\affiliation{Institute of Nuclear Physics, Polish Academy of Sciences, ul. W. E. Radzikowskiego 152, 31-342 Krak\'{o}w, Poland}

\author{Pawe\l{}~T.~Jochym}
\affiliation{Institute of Nuclear Physics, Polish Academy of Sciences, ul. W. E. Radzikowskiego 152, 31-342 Krak\'{o}w, Poland}


\author{Jan~\L{}a\.{z}ewski}
\affiliation{Institute of Nuclear Physics, Polish Academy of Sciences, ul. W. E. Radzikowskiego 152, 31-342 Krak\'{o}w, Poland}

\author{Andrzej~Ptok}
\affiliation{Institute of Nuclear Physics, Polish Academy of Sciences, ul. W. E. Radzikowskiego 152, 31-342 Krak\'{o}w, Poland}

\author{Svetoslav~Stankov}
\affiliation{Institute for Photon Science and Synchrotron Radiation, Karlsruhe Institute of Technology, D-76131 Karlsruhe, Germany}
\affiliation{Laboratory for Applications of Synchrotron Radiation, Karlsruhe Institute of Technology, D-76131 Karlsruhe, Germany}

\author{Przemys\l{}aw~Piekarz}
\affiliation{Institute of Nuclear Physics, Polish Academy of Sciences, ul. W. E. Radzikowskiego 152, 31-342 Krak\'{o}w, Poland}

\date{\today}

\begin{abstract}

Iron silicides are good candidates for applications in  optoelectronic and thermoelectric devices.
Lattice dynamical properties and thermal conductivity in the $\beta$-FeSi$_2$ semiconductor
are investigated with the first-principles computational methods. 
Phonon dispersion relations are calculated via the
temperature-dependent effective potential method and self-consistent phonon theory. 
To properly model thermal transport, we explicitly consider
the impact of phonon-phonon interactions by analyzing
anharmonic contributions to the phonon self-energy. 
This yields temperature-dependent phonon frequencies and linewidths,
reflecting the finite lifetime of phonons due to scattering
processes. The calculated phonon frequencies and line profiles are used to obtain 
the Raman spectra, which shows good agreement with the experimental data. 
We revealed an enhanced anharmonic behaviour of the Raman modes with the highest frequencies.  
The lattice thermal conductivity is then obtained as a function of temperature and crystallite size within
the relaxation-time approximation.
Phonon transport shows a small anisotropy due to the orthorhombic structure and a very weak dependence
on the quartic anharmonic corrections. The results obtained for an infinite material and for several crystallite sizes
were analyzed and compared with the available experimental data.
\end{abstract}

\maketitle


\section{Introduction}

The comprehensive determination of important physical properties of
crystals, such as thermal expansion, lattice thermal
conductivity or structural phase transitions, requires a fundamental 
understanding of the anharmonic effects.
Although the investigation of anharmonic interactions in crystals has
attracted a considerable interest for decades~\cite{cowley_1968}, a substantial progress
has only recently been achieved thanks to advances in theoretical and
numerical methods and increased computational power.
Now, phonon frequencies, lifetimes, and heat transfer in a wide range of
materials
can be quantitatively predicted using the available computational resources
based on the density functional theory (DFT)~\cite{lindsay_2013,mcgaughey_2019,lindsay_2019}.
In the case of strongly anharmonic systems, the self-consistent phonon
(SCPH) theory~\cite{tadano_2015} as well as the perturbative approach~\cite{tadano_2018}, using higher
order interatomic force constants derived from the fitting to the displacement force data obtained
with DFT, have proven to be successful.

Transition-metal silicides are promising materials for fabrication of electronic components
designed for integration with silicon-based circuits~\cite{murarka_1995}.
At room temperature, iron disilicide ($\beta$-FeSi$_2$) is a
direct-bandgap semiconductor~\cite{bost_1985}, making this material a good
candidate for application in optoelectronic devices such as infrared
detectors or light emitters~\cite{bost_1988}. The development of light-emitting diodes utilizing FeSi$_2$/Si heterostructures has been successfully demonstrated~\cite{leong_1997,suemasu_2001}. Due to
a high thermal stability and strong light absorption, FeSi$_2$ is
also a suitable photovoltaic material~\cite{powalla_1993,liu_2006,okuhara_2017}.

$\beta$-FeSi$_2$ crystallizes in the base-centered orthorhombic
lattice~\cite{dusausoy_1971} transforming to the
tetragonal metallic $\alpha$-FeSi$_{2}$ phase around $1200$~K~\cite{starke_2002}.
Optical studies indicated a direct band gap of the values
$0.85$--$0.89$~eV~\cite{bost_1988,dimitriadis_1990,arushanov_1995,wan_2003},
however, the {\it ab initio} calculations predicted a smaller indirect gap close to 0.8 eV~\cite{christensen_1990}. 
The existence of such an indirect gap was then confirmed by the optical linear transmittance measurements at
low temperatures~\cite{giannini_1992}. As shown by first-principles studies,
the character of the band gap is very sensitive to the orientation 
of a crystal grown on silicon~\cite{clark_1998}.

$\beta$-FeSi$_2$ belongs also to good thermoelectric materials~\cite{ware_1964}, with potential applications resulting from its chemical stability up to high temperatures, nontoxicity, and low cost of preparation~\cite{yamada_2012,nozariasbmarz_2017}. 
It has already been implemented in cars~\cite{birkholz_1988} and portable power sources~\cite{uemura_1989}. 
Its thermoelectric performance can be improved by doping~\cite{ito_2001,tani_2001,kim_2003,chen_2005,pandey_2013,le_tonquesse_2019}, 
which enhances the electric transport and reduces the thermal conductivity~\cite{waldecker_1973,du_2019,du_2020}.  
The thermal conductivity can be also reduced by the modification of microstructure~\cite{ail_2015} or by
nanostructurization~\cite{watanabe_2017,taniguchi_2017,hsin_2017,abbassi_2021}.

The lattice thermal conductivity is directly connected with anharmonic effects and phonon scattering processes.
The vibrational properties of \mbox{$\beta$-FeSi$_2$} were studied by the infrared and Raman spectroscopy~\cite{lefki_1991,guizzetti_1997,maeda_2004,baleva_2008,liu_2011,maeda_2011}. The observed anisotropy in the phonon spectra results from the enhanced sensitivity of the infrared and Raman features to the local lattice distortions~\cite{guizzetti_1997}. The Fe phonon density of states was measured by nuclear inelastic scattering (NIS), showing a good agreement with the density functional theory (DFT) calculations~\cite{walterfang_2005}. Using the DFT approach, the phonon dispersion curves, phonon density of states, as well as various thermodynamic properties were obtained within the harmonic approximation~\cite{tani_2010,liang_2011}. The extended Klemens model was applied to study
the anharmonic effect on phonon frequencies and linewidths observed by the Raman spectroscopy~\cite{zhang_2023}.
The impact of nanostructurization on lattice dynamics was explored in the $\beta$-FeSi$_2$ nanorods grown on the Si(110) surface by the NIS and {\it ab initio} methods~\cite{kalt_2022}.

In this work, we investigate the lattice dynamical properties of $\beta$-FeSi$_2$ 
using the DFT calculations. We study the effect of anharmonic terms in the temperature-dependent potential on phonon frequencies and lifetimes. We focus on the Raman modes, comparing the theoretical results with the experimental data.
The thermal conductivity is derived in a broad temperature range and the effect of crystallite size is analyzed.



This study is structured as follows.
In Sec.~\ref{sec.com} we describe the details of computational methods.
Next, in Sec.~\ref{sec.result} we present and discuss our results.
In particular we present the crystal structure (Sec.~\ref{sec.crys}) and lattice dynamics (Sec.~\ref{sec.lattice}).
We investigate also the thermal conductivity comparing the obtained results with the available experimental data (Sec.~\ref{sec.thermal}).
Finally, Sec.~\ref{sec.summary} summarizes our key findings and conclusions.

\section{Calculation method}
\label{sec.com}


The calculations were performed using the projector augmented-wave potentials~\cite{blochl_1994} and the generalized gradient approximation~\cite{perdew_1996} implemented in the Vienna Ab initio Simulation Package (VASP)~\cite{kresse.hafner.94,kresse.furthmuller.96,kresse.joubert.99}. 
The lattice parameters and atomic positions were optimized in the ${\bm a} \times ({\bm b}-{\bm c}) \times ({\bm b}+{\bm c})$ supercell containing 32 formula units and four primitive cells.
The integration in the reciprocal space was conducted using the $2 \times 2 \times 2$ Monkhorst--Pack mesh~\cite{monkhorst_1976} and the cut-off energy was set to $500$~eV. For convergence conditions, we set the energy change below $10^{-5}$ and $10^{-8}$ for the ionic and electronic loops, respectively. 

The lattice dynamical properties were studied within the temperature-dependent effective potential (TDEP) approach~\cite{hellman_2013}. The atomic potential with the third and fourth order anharmonic terms was derived from interatomic forces induced by displacements of all atoms at finite temperatures.
The sets of atomic displacements were generated by the high efficiency configuration space sampling (HECSS)~\cite{jochym_2021} and forces were obtained by VASP. The interatomic force constants and phonon frequencies were calculated with the {\sc Alamode} software~\cite{tadano_2014}.

Furthermore, we have attempted to construct a {\emph{temperature independent}} 
anharmonic model. We have used combined data from all investigated temperatures 
(300, 600 and 1000~K) and fitted a large (over 15~000 free parameters), fourth-order 
interaction model to this dataset. 
Subsequently, we have used this model to calculate line profiles and positions of Raman-active modes
at multiple temperatures.

The changes in phonon frequencies induced by the anharmonic effects were investigated within two approaches.
First, the impact of the quartic anharmonic terms was included using the SCPH theory~\cite{tadano_2015}.
Second, the mode profiles (frequency shifts and line widths) were determined from the real and imaginary parts of the phonon self-energy resulting from the cubic and quartic anharmonic terms of the above mentioned large model~\cite{tadano_2018}. 
The longitudinal optic-transverse optic (LO-TO) splitting was also evaluated, using the static dielectric tensor and Born effective charges calculated within density functional perturbation theory~\cite{gajdos_2006}.

\begin{figure}[]
    \centering
    \includegraphics[width=\linewidth]{fig1_new.png}
\caption{%
(a) The conventional unit cell of $\beta$-FeSi$_2$ (with Cmca symmetry) and (b) the corresponding Brillouin zone with selected high-symmetry points.
}
  \label{fig.struct}
\end{figure}

% To further characterize the vibrational properties, Raman-active scattering was investigated using the Phonopy-Spectroscopy package~\cite{skelton_2017}. This enabled the identification of Raman-active modes and the calculation of Raman tensors. Anharmonic force constants, obtained from calculations using {\sc Alamode}, were then used to obtain theoretical line profiles for the Raman modes. The presented Raman scattering spectra combine these anharmonic line profiles with the Raman tensor amplitudes.
% This analysis was also based on the large quartic model mentioned above.

% Finally, the thermal conductivity was obtained as a function of temperature and crystallite size within the relaxation-time approximation (RTA) \PJ{}{as implemented in {\sc Alamode}\cite{tadano_2014}}. The phonon lifetimes were calculated from the phonon self-energy including the cubic and quartic anharmonic terms. \SP{}{The RTA provides a solution of the Boltzmann transport equation (BTE) under the assumption that scattering events are independent and can be treated through mode-resolved relaxation times.
% To verify the validity of this approximation for $\beta$-FeSi$_2$, we have additionally
% executed an iterative solution of the BTE.}\PJ{}{These calculations were performed with
% {\sc Phono3py}\cite{togo_2023}. For cross-validation, the additional RTA calculations were performed on the same $\bm{q}$-grid as BTE with implementation provided by {\sc Phono3py}.}.

To further characterize the vibrational properties, Raman scattering was investigated using the Phonopy-Spectroscopy package~\cite{skelton_2017}. This enabled the identification of Raman-active modes and the calculation of Raman tensors. Anharmonic force constants, derived from calculations using {\sc Alamode}, were then used to obtain theoretical line profiles for the Raman modes. The presented Raman scattering spectra combine these anharmonic line profiles with the Raman tensor amplitudes.
This analysis was also based on the large quartic model described above.

Finally, the thermal conductivity was calculated as a function of temperature and crystallite size within the relaxation-time approximation (RTA) \PJ{}{as implemented in {\sc Alamode}~\cite{tadano_2014}}. The phonon lifetimes were calculated from the phonon self-energy including the cubic and quartic anharmonic terms. \SP{}{The RTA provides a solution to the Boltzmann transport equation (BTE) under the assumption that scattering events are independent and can be treated through mode-resolved relaxation times.
To verify the validity of this approximation for $\beta$-FeSi$_2$, we additionally
solved the BTE iteratively.} \PJ{}{These calculations were performed with
{\sc Phono3py}~\cite{togo_2023}. For cross-validation, the additional RTA calculations were performed on the same $\bm{q}$-grid as the BTE calculations, using the implementation provided by {\sc Phono3py}.}

% (see. Appendix~\ref{ThermalAB} for the comparison)\cite{togo_2023}.}
%As shown in Appendix~\ref{ThermalAB}, the iterative BTE results exhibit good agreement with the RTA values, confirming that RTA is sufficiently accurate for this material and for the considered temperature range.}


\section{Results}
\label{sec.result}

\subsection{Crystal structure}
\label{sec.crys}

The $\beta$-FeSi$_2$ structure adopts a base-centered orthorhombic lattice with the space group Cmca (No.~64) as shown in Fig.~\ref{fig.struct}(a).
The unit cell consists of two primitive cells and contains 48 atoms.
Iron (silicon) atoms possess two nonequivalent positions: \mbox{Fe-I} and \mbox{Fe-II} (\mbox{Si-I} and \mbox{Si-II}), presented in Fig.~\ref{fig.struct}(a) as gray and purple (orange and yellow) spheres, respectively.
This crystal structure is derived from the fluorite-type lattice with strongly distorted Si cubes and Fe atoms occupying 
one-half of the central sites.
The Fe-I and Fe-II sites create different layers perpendicular to the $x$ direction, and they are separated by layers containing both Si sites.
Each Fe atom is coordinated by 8 Si atoms with slightly different Fe-Si distances.
The optimized lattice constants ($a = 9.874$~{\AA}, $b = 7.767$~{\AA}, and
$c = 7.811$~{\AA}) agree very well with the experimental data ($a = 9.863$~{\AA}, $b = 7.791$~{\AA}, and $c = 7.833$~{\AA})~\cite{dusausoy_1971}.

Iron atoms occupy the Wyckoff sites 8\textit{d} ($0.2166$, $0$, $0$) and  8\textit{f} ($0$, $0.3072$, $0.1879$), corresponding to \mbox{Fe-I} and \mbox{Fe-II}, respectively. 
Silicon atoms are located at two inequivalent 16\textit{g} positions: ($0.1282$, $0.2737$, $0.0495$) and \mbox{($0.3734$, $0.0445$, $0.2270$)}, assigned as \mbox{Si-I} and \mbox{Si-II}.
The optimized positions of atoms agree very well with the experimental data~\cite{dusausoy_1971} and the previous theoretical studies~\cite{tani_2010,liang_2011}.


\subsection{Lattice dynamics}
\label{sec.lattice}

\begin{figure}[]
    \centering
    \includegraphics[width=\linewidth]{Fig1c.pdf}
\caption{%
The phonon dispersion curves along high symmetry directions obtained within SCPH (for temperatures from $0$ to $1000$~K).
Dashed black lines indicate the phonon dispersions obtained from the harmonic approximation. The white dots indicate Raman-active modes with A$_g$ symmetry.
The vertical plot shows the phonon density of states (DOS) calculated at a reference temperature of $600$~K.
}
  \label{fig.ph_band}
\end{figure}


In Fig.~\ref{fig.ph_band} we present the phonon dispersion relations of $\beta$-FeSi$_2$ along high-symmetry directions in the Brillouin zone [Fig.~\ref{fig.struct}(b)].
Due to 24 atoms in the primitive cell, the phonon spectrum consists of 69 optical modes and three acoustic modes.
The phonon dispersions were calculated within the SCPH approach in the temperature range $0$--$1000$~K (presented by color lines in Fig.~\ref{fig.ph_band}), and they are compared with the results obtained from the harmonic part of the effective potential corresponding to temperature $T=300$~K (indicated by dashed black lines in Fig.~\ref{fig.ph_band}).
As we can see within the SCPH method, the anharmonic effects are rather weak and leads only to small renormalization of phonon frequencies.
Only the highest modes show more pronounced shifts of their frequencies to larger values.
The total and partial element-projected phonon density of states obtained within the harmonic approximation are presented in Fig.~\ref{fig.ph_band}.
Up to around $320$~cm$^{-1}$, the contributions from both elements are very similar, while for higher frequencies the spectrum is dominated by
the Si vibrations.


\begin{table}[!t]
\begin{ruledtabular}
\caption{Calculated and experimental Raman-active modes of $\beta$-FeSi$_2$ with their irreducible representations (IR). Present theoretical results are compared with the previous theoretical data from Ref.~\cite{tani_2010} and experimental results from \mbox{Refs.~\cite{lefki_1991,maeda_2004}.} The experimental frequencies with known symmetries (A$_g$) are shown in bold, while other experimental modes are assigned to the best fitting theoretical values.}
\begin{tabular}{c c c c c}
\textbf{IR} & \multicolumn{4}{c}{\textbf{Frequency (cm$^{-1}$)}} \\
 & Present  & Theor.~\cite{tani_2010} & Exp.~\cite{lefki_1991} & Exp.~\cite{maeda_2004} \\
\hline
B$_{2g}$ & 175.4 & 179 & 176 &  \\
B$_{1g}$ & 176.3 & 185 & 179  &  \\
B$_{1g}$ & 193.3 & 198 &  & 190.6 \\
A$_g$    & 196.6 & 208 & \textbf{195} & \textbf{194.0} \\
A$_g$    & 203.9 & 210 & \textbf{197} & 199.6 \\
B$_{3g}$ & 205.5 & 212 & 200 &  \\
B$_{3g}$ & 226.3 & 236 & 206 & 227.1 \\
B$_{1g}$ & 233.3 & 240 &  &  231.6 \\
B$_{2g}$ & 248.6 & 254 &  &  \\
A$_g$    & 250.1 & 257 & \textbf{247} & \textbf{247.3} \\
A$_g$    & 254.9 & 264 & \textbf{253} & 254.3 \\
B$_{1g}$ & 275.5 & 285 &  &  274.1 \\
B$_{2g}$ & 282.4 & 295 &  &  281.2 \\
B$_{3g}$ & 286.8 & 297 &  &  \\
B$_{1g}$ & 307.1 & 317 &  &  \\
B$_{2g}$ & 312.6 & 326 &  &  311.8 \\
B$_{1g}$ & 319.0 & 324 &  &  \\
B$_{3g}$ & 327.9 & 341 &  & 325.8 \\
B$_{2g}$ & 333.3 & 345 &  &  \\
A$_g$    & 339.3 & 352 & \textbf{346} & 339.5 \\
B$_{2g}$ & 343.0 & 350 &  &  \\
B$_{1g}$ & 353.5 & 366 &  &  \\
B$_{2g}$ & 372.8 & 383 &  & 370.7 \\
B$_{1g}$ & 375.1 & 385 &  &  \\
B$_{3g}$ & 375.2 & 386 &  &  \\
B$_{3g}$ & 385.3 & 401 &  &  \\
A$_g$    & 386.5 & 398 &  & 386.2 \\
B$_{2g}$ & 387.7 & 402 &  & 388.2 \\
A$_g$    & 404.6 & 415 &  & 400.4 \\
B$_{2g}$ & 405.1 & 420 &  &  \\
B$_{1g}$ & 412.7 & 428 &  &  \\
B$_{3g}$ & 418.4 & 431 &  &  \\
B$_{3g}$ & 441.4 & 458 &  & 442.6 \\
B$_{3g}$ & 447.3 & 466 &  & 446.3 \\
A$_g$    & 448.4 & 464 &  &  \\
A$_g$    & 499.1 & 517 &  &  \\
\end{tabular}
\label{T1Raman}
\end{ruledtabular}
\end{table}

%subsection{Raman}

\begin{figure*}
\centering
  \includegraphics[width=\textwidth]{spectrum_RTA+raman.pdf}
\caption{Raman spectrum of $\beta$-FeSi$_2$ calculated within the perturbative approach at three temperatures 
($300$, $600$, $900$~K -- blue, orange, and red lines, respectively). 
The calculated spectrum includes all Raman-active modes. The A$_g$ modes
are indicated by purple vertical lines at peak positions corresponding to T=300~K. 
The frequencies derived from the harmonic approximation at T=300~K are indicated by green lines.
Connecting arrows indicate the correspondence between harmonic and
anharmonic frequencies, demonstrating the frequency shifts due to 
phonon interactions. Experimental values for the A$_g$ modes
based on Ref.~\cite{lefki_1991} are marked with black dashed lines.
}
\label{raman}
\end{figure*}


The phonon spectrum at the $\Gamma$ point consists of 36 Raman modes classified according to the irreducible representations: $9A_\text{g}+9B_\text{1g}+9B_\text{2g}+9B_\text{3g}$. 
Fig.~\ref{raman} shows the Raman spectrum of A$_g$ symmetry calculated for $\beta$-FeSi$_2$ within the perturbative approach at three temperatures $300$, $600$, and $900$~K (solid blue, orange, and red curves, respectively), including third-order and fourth-order anharmonic corrections.
The calculated Raman spectrum includes all Raman-active modes.
The five experimentally observed A$_g$ modes are highlighted by black dashed lines, based on data from Ref.~\cite{lefki_1991}.
Additional peaks not marked with vertical lines correspond to phonon modes with symmetries other than A$_g$.
The frequencies of Raman modes obtained in anharmonic calculations are compared with the previous results calculated within the harmonic approximation and the experimental values in Tab.~\ref{T1Raman}. 
We have marked in bold the experimentally determined A$_g$ modes, which are compared with the calculations.
Since the experimental studies did not provide the accurate assignement of the Raman modes with the B$_{1g}$, B$_{2g}$, and B$_{3g}$ symmetry~\cite{lefki_1991,maeda_2004}, we cannot compare them directly with the theoretical results.
However, in Tab.~\ref{T1Raman} we have assigned the measured frequencies to the best fitting theoretical values without taking into account the symmetry of the modes, except for the known A$_{g}$ modes. 


The impact of anharmonicity on the phonon frequencies is well visible from the comparison of the results obtained within the harmonic approximation 
and from the anharmonic calculation (vertical green and purple lines in Fig.~\ref{raman}, respectively).
Here we show only the A$_g$ modes, which are compared with the experimental results (vertical black dashed lines).
Anharmonic frequencies calculated at $300$~K are indicated by purple lines, while frequencies derived from harmonic approximation are marked with green lines. The grey solid lines connect corresponding modes obtained in both approximations.
In most cases, the results obtained within the harmonic approximation do not agree with the experimental frequencies. 
%Only the modes close to $250$~cm$^{-1}$ and $400$~cm$^{-1}$ correspond well to the experimental values. MS
As we can see, the inclusion of the anharmonic correction leads to a significant modification of the phonon frequencies.
These anharmonic effects are stronger for higher-frequency modes mainly because of
the dominant contribution from Si atoms, which vibrate with larger amplitudes than heavier Fe atoms. 
When atoms move to larger distances the potential deviates more from the harmonic approximation,
and the anharmonic corrections become stronger.


The modification of phonon frequencies observed in Fig.~\ref{raman} is much larger than in the SCPH scheme presented in Fig.~\ref{fig.ph_band}. The SCPH approach includes only the leading-order contribution to 
the phonon self-energy obtained from the quartic anharmonic terms~\cite{tadano_2015}. Therefore, it does not describe fully the changes of phonon frequencies found within the perturbation theory (see Fig.~\ref{raman}).
Especially, it is well visible for two highest A$_g$ modes, which exhibit also the largest line broadening 
and the strongest dependence on temperature.
Therefore, a better agreement with experimentally observed frequencies is visible,
confirming the significant influence of the anharmonicity on the frequencies and line profiles of phonon modes.
In fact, the decrease of phonon frequencies should be even stronger due to thermal expansion, 
which is not included in our calculations.
Within SCPH the frequencies of the highest modes increase with increasing temperature as we see in Fig.~\ref{fig.ph_band}. The comparison of two different approaches applied to study anharmonic properties of $\beta$-FeSi$_2$ shows that the perturbation theory, which includes the cubic and quartic terms, better describes the changes of phonon frequencies with temperature than the SCPH method. \PJ{}{This indicates that the leading-order contribution included in SCPH are not important in this material.}

Additionaly we should nottice that for other than A$_g$ modes we cannot make an unambiguous assignment of theoretical frequencies to experimental ones. Note that the spectrum in Fig.~\ref{fig.ph_band} contains all Raman-active modes. 
The limitation to A$_g$ modes concerns only the indicated positions of the peaks. 
%\SP{}{Furthermore, we present the full spectrum for all Raman active modes with the comparison of the harmonic and anharmonic calculations in the Appendix~\ref{RamanAA}.}
\textcolor{red}{The full Raman spectrum, with the frequencies of the B$_{1g}$, B$_{2g}$ and B$_{3g}$ modes marked, is shown in Appendix A, Fig.~\ref{raman1}. This represents our theoretical prediction of possible Raman-mode assignments, which can be verified in future experiments.}
%This is the theoretical prediction of possible assignment of Raman modes that can be verified in future experiments.  


\subsection{Thermal conductivity}
\label{sec.thermal}


\begin{figure*}[!t]
\centering
  \includegraphics[width=\linewidth]{time3.pdf}
\caption{Phonon lifetimes calculated for three temperatures as a function of phonon frequency. The colors correspond to the phonon branches.}
  \label{thermtime}
\end{figure*}


In this section, we analyze the thermal conductivity tensor of $\beta$-FeSi$_2$ obtained within the RTA approach~\cite{tadano_2018} as a function of temperature
%
\begin{equation}
\kappa_{\text{ph}}^{\mu\nu}(T) = \frac{1}{NV} \sum_{\bm{q},j} c_{\bm{q}j}(T) v_{\bm{q}j}^{\mu} v_{\bm{q}j}^{\nu}\tau_{\bm{q}j}(T),
\end{equation}
% 
where $c_{\bm{q}j}$ is the mode heat capacity and $v_{\bm{q}j}$ is the mode group velocity. 
The relaxation time is approximated by the phonon lifetime $\tau_{\bm{q}j}$
calculated for $j$-th branch at the wave vector $\bm{q}$.
$V$ is the unit cell volume and $N$ is the number of unit cells in the crystal.
The phonon lifetime is calculated using this formula
%
\begin{equation}
\tau_{\bm{q}j}(T)=\frac{1}{2\Gamma_{\bm{q}j}^{\text{anh}}(T)},
\end{equation}
%
where $\Gamma_{\bm{q}j}^{\text{anh}}$ is the anharmonic phonon linewidth obtained from 
the imaginary part of the phonon self-energy within the perturbation theory.

In Fig.~\ref{thermtime}, we present $\tau_{\bm{q}j}$ obtained for three temperatures 300, 600, and 1000~K as a function of frequency. As we see, the acoustic phonons close to the $\Gamma$ point have the longest lifetimes,
which are diminished with increasing frequency reaching local minima around $200$~cm$^{-1}$.
For higher frequencies, phonon lifetimes first increase to local maxima around $300$~cm$^{-1}$ and then decrease to
the lowest values in the range of highest optical modes. The shortest lifetimes correspond to the largest line broadening
observed for the Raman modes in Fig~\ref{raman}. The phonon group velocities 
$v_{\bm{q}j}=\partial\omega_{\bm{q}j}/\partial\bm{q}$, which are obtained by the central difference formula, are presented in Fig.~\ref{thermvelocity}. Their temperature dependence is negligible, therefore, we present only the results for $T=600$~K.
At low frequencies, there are clearly two ranges of group velocities of the acoustic phonons. 
The larger values correspond to the longitudinal modes, while the lower values are obtained from the
transverse acoustic branches. Group velocities of acoustic phonons decrease for larger frequencies
and reach the average values typical for optic branches.

\begin{figure}[!t]
\centering
  \includegraphics[width=\linewidth]{GV.pdf}
\caption{Mode group velocities calculated as a function of phonon frequency. The colors correspond to the phonon branches.
}
  \label{thermvelocity}
\end{figure}

In Fig.~\ref{anizotropy}(a), we present the three diagonal elements of $\kappa_{\text{ph}}^{\mu\nu}$ corresponding to the main directions of the crystal structure. 
They were obtained from the force constants calculated at the base temperature $T=600$~K and the crystallite size $0.1$~$\mu$m to account for boundary-limited phonon transport. 
Due to the orthorhombic symmetry, we observe a small anisotropy in phonon transport in the whole temperature range. 
At low temperatures, the three components of the heat conductivity increase in a very similar way with the $\kappa_{\text{ph}}^{yy}$ element slightly larger than two other components. 
After reaching the maximum, we observe a change in the largest component from $\kappa_{\text{ph}}^{yy}$ to 
$\kappa_{\text{ph}}^{xx}$.    
In Fig.~\ref{anizotropy}(b), the thermal conductivity is shown for three base temperatures, at which the interatomic potential was obtained ($300$~K, $600$~K, and $1000$~K), using the energy expansion up to third- and fourth-order anharmonic terms, and the same structure size of $0.1$~$\mu$m. At lowest temperatures, the thermal conductivity strongly increases, reaching the maximum around $T=180$~K, then it shows a slower decrease with temperature.
The differences between the two levels of approximation are minimal, suggesting that third-order calculations already capture the dominant phonon scattering mechanisms. The dependence on the base temperature is also very weak, showing the changes in the heat conductivity within a few percent. 
\SP{}{Further verification of the reliability of our thermal conductivity results is given in Appendix~\ref{ThermalAB}, where the full BTE calculations show good agreement with RTA and higher-resolution RTA results, demonstrating that the $8\times8\times8$ $\bm{q}$-mesh already provides converged values.}

\begin{figure}[!t]
\centering
  \includegraphics[width=\linewidth]{Anizotropy.pdf}
\caption{(a) The anisotropic thermal conductivity of $\beta$-FeSi$_2$ calculated along the lattice directions at 600~K. (b) The average temperature-dependent thermal conductivity taken at $300$~K, $600$~K, and $1000$~K, including anharmonic corrections up to cubic (A3) and quartic (A4) terms. In both cases the crystallite size is 0.1~$\mu$m.}
  \label{anizotropy}
\end{figure}


In Fig.~\ref{therm}, we fix the base temperature at $600$~K and examine the effect of crystallite size on thermal conductivity, varying it from $0.01$ to $0.5$~$\mu$m.
With decreasing the crystallite size, we observe a shift of the position of the maximum to larger temperatures and a decrease of the thermal conductivity in the entire temperature range.
Theoretical results are compared with several experimental data obtained above the room temperature. 
The measured thermal conductivity depends to a large extent on the sample quality, its purity and the size of the crystalline grains which depends on 
the production processes.
Many measurements were performed using crystallites of micrometric or unknown size ~\cite{waldecker_1973,ito_2002,kim_2003,du_2020}, however, 
numerous attempts to minimize $\kappa$ by reducing grain sizes to $56$~nm~\cite{dabrowski_2019, dabrowski_microstructure_2021}, $30$-$400$~nm~\cite{le_tonquesse_2019}, $50$ and $200$~nm~\cite{abbassi_2021}, or introducing pores into the material~\cite{sam2023improved} 
are also carried out. 
Another way to change the thermal conductivity is to dope $\beta$-FeSi$_2$ with different elements~\cite{ito_2002,kim_2003,du_2020,cheng_2024}, however, this effect is beyond our investigation.

\begin{figure}[!t]
\centering
  \includegraphics[width=\linewidth]{Thermal_conductivity3.pdf}
\caption{
The phonon thermal conductivity of $\beta$-FeSi$_2$: theoretical results for the infinite crystalline size and with boundary conditions, compared with experimental data for different structure sizes.
}
  \label{therm}
\end{figure}
%Fig.~\ref{therm}\cite{abbassi_2021}\cite{waldecker_1973}\cite{dabrowski_2019}\cite{sam2023improved}\cite{kim_2003}\cite{du_2020}\cite{ito_2002}\cite{le_tonquesse_2019}.

We observe a decrease in the thermal conductivity with reducing crystalline grain sizes in all analyzed experimental data. 
For instance, by decreasing the crystallite size to $50$~nm, the thermal conductivity at room temperature was reduced by a factor of $1.7$, what can be  compared to the annealed sample with 200 nm grains~\cite{abbassi_2021}. 
It is worth noting that the rate of decrease in value with increasing temperature in both cases, for grain sizes of $50$~nm and $200$~nm, is significantly different, which is consistent with our calculations. 
The same trend can be observed by comparing the thermal conductivity measured for a sample with bulk crystallite sizes with the thermal conductivity of a sample with grains smaller than 400 nm~\cite{le_tonquesse_2019}.
The theoretical results obtained for the same crystallite size show higher values due to factors not captured in the idealized model, such as crystal imperfection or mechanical strain. Usually, a decrease in the crystallite size is related to an increased concentration of grain boundaries, point defects, and stacking faults that influence the phonon scattering~\cite{le_tonquesse_2019,abbassi_2021}.   

We should note that the total thermal conductivity is a combination
of the lattice and electronic contributions to the heat transport.
In semiconductors, the electronic thermal conductivity is negligible at low temperatures and significantly increases 
only much above the room temperatures~\cite{gu_2020}.
For $\beta$-FeSi$_2$, the electronic thermal conductivity was obtained from the electric conductivity using the Wiedemann-Franz law~\cite{ito_2002,kim_2003,le_tonquesse_2019}.
In the undoped material, its value does not exceed $0.1$~W/mK in the measurement up to $T=950$~K~\cite{kim_2003}.
By doping, the electronic thermal conductivity can be enhanced, and it has a direct impact on the thermoelectric properties of $\beta$-FeSi$_2$ at high temperatures~\cite{ito_2002,kim_2003}.
In the present study, we consider only the phonon contribution to the thermal conductivity,
therefore, agreement with experimental data may deteriorate with increasing temperature.

\section{Summary}
\label{sec.summary}

We performed {\it ab initio} studies on lattice dynamical and thermal transport properties of $\beta$-FeSi$_2$. The effect of anharmonicity was analyzed within two approaches -- the SCPH method and the perturbation theory.
The phonon dispersion curves obtained within SCPH show small renormalization of frequencies comparing to the harmonic approximation. 
The Raman spectra were calculated within the procedure which takes into account the peak intensities obtained from the Raman tensors and the line profiles obtained from the phonon self energy derived within the perturbation theory based on the large, temperature-independent, quartic model fitted to the data from the wide range of temperatures (300-1000~K). 
The anharmonic corrections strongly affect the frequencies and line profiles of some modes and results in overall better agreement with the experimental data. 
We analyzed the phonon lifetimes and group velocities obtained as functions of the phonon frequency.
Then the lattice thermal conductivity was calculated for a broad range of temperatures and grain sizes.
We found a small anisotropy in the phonon thermal transport resulting from the orthorhombic structure and a weak effect of the quartic anharmonic terms. 
The thermal conductivity calculated for various crystalline grain sizes show a good qualitative agreement with the available measurements.

\begin{acknowledgments}
Some figures in this work were rendered using {\sc Vesta}~\cite{momma.izumi.11} software.
This work was partially supported by the Ministry of Education, Youth and Sports of the Czech Republic through the e-INFRA CZ (ID:90254).
\end{acknowledgments}

\appendix

\section{Raman spectrum}
\label{RamanAA}

Based on the polarized Raman measurements reported in Ref.~\cite{maeda_2004}, two Raman peaks were identified as belonging to the A$_g$ symmetry class, and several additional peaks were observed with similar or different polarization dependence. Although the authors of Ref.~\cite{maeda_2004} provided estimates of the relative Raman tensor components, they did not specify which of the remaining modes correspond to the B$_{1g}$, B$_{2g}$, or B$_{3g}$ symmetries. Because of this missing experimental information, a direct symmetry-resolved comparison between the measurement and theory is not currently possible for the non-A$_g$ modes. 
To provide a complete theoretical picture of the B$_g$-type modes, we show here the calculated Raman-active frequencies and intensities for the B$_{1g}$, B$_{2g}$, and B$_{3g}$ symmetries only. 
%These results represent the predicted Raman modes for the B$_g$ symmetries in $\beta$-FeSi$_2$. 
\textcolor{red} {Fig.~\ref{raman1} shows the predicted Raman modes for the B$_g$ symmetries in $\beta$-FeSi$_2$. The results obtained within the harmonic approximation are compared with the anharmonic perturbation theory calculations which provides both, frequency shifts and predicted line profiles of the modes.}
Although the experimentally measured peaks cannot be directly assigned to these symmetries due to the lack of polarization-resolved data, the theoretical predictions provide a reference for comparison. Matching the measured frequencies to the closest theoretical B$_g$ modes (Table~\ref{T1Raman}) allows for a tentative assignment, which can guide future polarization-resolved Raman experiments aimed at determining the precise symmetry of the unresolved peaks.

\begin{figure*}[t]
\centering
  \includegraphics[width=\linewidth]{spectrum_RTA+raman_Bi_modes.pdf}
\caption{Raman spectrum of $\beta$-FeSi$_2$ calculated at three temperatures 
($300$, $600$, $900$~K -- blue, orange, and red lines, respectively). 
The calculated spectrum includes all Raman-active modes. The B$_{ig}$ modes
are indicated by purple vertical lines at peak positions corresponding to T=300~K. 
The frequencies derived from the harmonic approximation at T=300~K are indicated 
by green, yellow and pink lines.
Connecting arrows indicate the correspondence between harmonic and anharmonic 
frequencies, demonstrating the frequency shifts due to phonon interactions.}
\label{raman1}
\end{figure*}
\section{Thermal conductivity obtained from BTE and RTA}
\label{ThermalAB}

The thermal conductivity was computed by solving the full BTE on the largest feasible $\bm{q}$-point grid, $8\times8\times8$, and compared with the corresponding RTA results obtained on the same grid. As shown in Fig.~\ref{bte}, the difference between the components of the thermal conductivity tensor obtained within BTE and RTA at this resolution is very small, indicating a good agreement between these two approaches.
%THIS PART SHOULD GO RATHER TO THE RESPONSE
%Extending the full BTE calculation to larger grids is computationally prohibitive: the computational cost of BTE is roughly two orders of %magnitude higher than that of RTA, and the required memory and runtime exceed our available resources. 
Moreover, we performed an additional calculation using RTA on a denser $20\times20\times20$ grid. As seen in the Fig.~\ref{bte}, the higher-resolution data remain in a good agreement with both the BTE and RTA results for the $8\times8\times8$ grid.
It shows that the $8\times8\times8$ mesh already provides good results for this structure and confirms reliability of the calculations.

\begin{figure}[!h]
\centering
  \includegraphics[width=\linewidth]{BTEvsRTAvsANP.pdf}
\caption{Thermal conductivity of $\beta$-FeSi$_2$ obtained within the BTE and RTA methods using the Phono3py software on the $8\times8\times8$ q-point grid, compared with the RTA results computed with ALAMODE on a denser $20\times20\times20$ grid.}
\label{bte}
\end{figure}

%\section*{Data availability}
%The data that support the findings of this article are openly available~\footnote{give me DOI}.
% see https://journals.aps.org/authors/data-availability-statements#citation

\bibliography{refs.bib}
%\bibliographystyle{ieeetr}


\end{document}

\documentclass[%
%reprint,
superscriptaddress,
%groupedaddress,
longbibliography,
%unsortedaddress,
%runinaddress,
%frontmatterverbose, 
%preprint,
%preprintnumbers,
%nofootinbib,
nobibnotes,
%bibnotes,
amsmath,amssymb,
aps,
%pra,
prb,
%rmp,
%prstab,
%prstper,
%showkeys,
floatfix,
twocolumn
]{revtex4-2}

\usepackage{graphicx}% Include figure files
\usepackage{calc}% Calculate margins
\usepackage{dcolumn}% Align table columns on decimal point
\usepackage{bm}% bold math

\usepackage[urlcolor=blue,colorlinks=true,citecolor=blue,linkcolor=blue,pdfstartview={FitH},bookmarks=false]{hyperref} % add hypertext capabilities

%\usepackage[mathlines]{lineno} % Enable numbering of text and display math
% \linenumbers\relax % Commence numbering lines

% \usepackage[showframe,%Uncomment any one of the following lines to test 
% %scale=0.7, marginratio={1:1, 2:3}, ignoreall,% default settings
% %text={7in,10in},centering,
% %margin=1.5in,
% % total={6.5in,8.75in}, top=1.2in, left=0.9in, includefoot,
% % height=10in,a5paper,hmargin={3cm,0.8in},
% ]{geometry}

\usepackage{amsmath}
\usepackage{amssymb}
%\usepackage{orcidlink}
\usepackage{xcolor}
%\usepackage{datetime}
\usepackage[normalem]{ulem}

% Change tracking commands
\newcommand{\trackchange}[3]{\textcolor{#3}{\sout{#1}#2}}  % Full color strikeout, insert
%\renewcommand{\trackchange}[3]{\textcolor{#3}{#2}}        % Just color silent remove and insert
%\renewcommand{\trackchange}[3]{{#2}}                      % No indication, silent remove and insert

% Author marker definitions

\definecolor{myblue}{RGB}{0,127,85}
% \definecolor{violet}{RGB}{102,0,204}
% \definecolor{orange}{RGB}{255,128,0}
% \definecolor{green}{RGB}{0,128,0}
\newcommand{\DL}[1]{\trackchange{}{#1}{blue}}
\newcommand{\AP}[1]{\trackchange{}{#1}{red}}
\newcommand{\PJ}[2]{\trackchange{#1}{#2}{orange}}
\newcommand{\JL}[2]{\trackchange{#1}{#2}{myblue}}
\newcommand{\SP}[2]{\trackchange{#1}{#2}{blue}}
\newcommand{\PP}[2]{\trackchange{#1}{#2}{teal}}
\newcommand{\AI}[2]{\trackchange{#1}{#2}{olive}}
\newcommand{\MS}[1]{\trackchange{}{#1}{purple}}

\newcommand{\TODO}[1]{\textcolor{red}{TODO: #1}}

\sloppy

\begin{document}

\title{Ab initio study of the anharmonic properties and thermal conductivity in $\beta$-FeSi$_2$}

\author{Svitlana~Pastukh}
\email[e-mail: ]{svitlana.pastukh@ifj.edu.pl}
\affiliation{Institute of Nuclear Physics, Polish Academy of Sciences, ul. W. E. Radzikowskiego 152, 31-342 Krak\'{o}w, Poland}

\author{Ma\l{}gorzata~Sternik}
\affiliation{Institute of Nuclear Physics, Polish Academy of Sciences, ul. W. E. Radzikowskiego 152, 31-342 Krak\'{o}w, Poland}

\author{Pawe\l{}~T.~Jochym}
\affiliation{Institute of Nuclear Physics, Polish Academy of Sciences, ul. W. E. Radzikowskiego 152, 31-342 Krak\'{o}w, Poland}


\author{Jan~\L{}a\.{z}ewski}
\affiliation{Institute of Nuclear Physics, Polish Academy of Sciences, ul. W. E. Radzikowskiego 152, 31-342 Krak\'{o}w, Poland}

\author{Andrzej~Ptok}
\affiliation{Institute of Nuclear Physics, Polish Academy of Sciences, ul. W. E. Radzikowskiego 152, 31-342 Krak\'{o}w, Poland}

\author{Svetoslav~Stankov}
\affiliation{Institute for Photon Science and Synchrotron Radiation, Karlsruhe Institute of Technology, D-76131 Karlsruhe, Germany}
\affiliation{Laboratory for Applications of Synchrotron Radiation, Karlsruhe Institute of Technology, D-76131 Karlsruhe, Germany}

\author{Przemys\l{}aw~Piekarz}
\affiliation{Institute of Nuclear Physics, Polish Academy of Sciences, ul. W. E. Radzikowskiego 152, 31-342 Krak\'{o}w, Poland}

\date{\today}

\begin{abstract}

Iron silicides are good candidates for applications in  optoelectronic and thermoelectric devices.
Lattice dynamical properties and thermal conductivity in the $\beta$-FeSi$_2$ semiconductor
are investigated with the first-principles computational methods. 
Phonon dispersion relations are calculated via the
temperature-dependent effective potential method and self-consistent phonon theory. 
To properly model thermal transport, we explicitly consider
the impact of phonon-phonon interactions by analyzing
anharmonic contributions to the phonon self-energy. 
This yields temperature-dependent phonon frequencies and linewidths,
reflecting the finite lifetime of phonons due to scattering
processes. The calculated phonon frequencies and line profiles are used to obtain 
the Raman spectra, which shows good agreement with the experimental data. 
We revealed an enhanced anharmonic behaviour of the Raman modes with the highest frequencies.  
The lattice thermal conductivity is then obtained as a function of temperature and crystallite size within
the relaxation-time approximation.
Phonon transport shows a small anisotropy due to the orthorhombic structure and a very weak dependence
on the quartic anharmonic corrections. The results obtained for an infinite material and for several crystallite sizes
were analyzed and compared with the available experimental data.
\end{abstract}

\maketitle


\section{Introduction}

The comprehensive determination of important physical properties of
crystals, such as thermal expansion, lattice thermal
conductivity or structural phase transitions, requires a fundamental 
understanding of the anharmonic effects.
Although the investigation of anharmonic interactions in crystals has
attracted a considerable interest for decades~\cite{cowley_1968}, a substantial progress
has only recently been achieved thanks to advances in theoretical and
numerical methods and increased computational power.
Now, phonon frequencies, lifetimes, and heat transfer in a wide range of
materials
can be quantitatively predicted using the available computational resources
based on the density functional theory (DFT)~\cite{lindsay_2013,mcgaughey_2019,lindsay_2019}.
In the case of strongly anharmonic systems, the self-consistent phonon
(SCPH) theory~\cite{tadano_2015} as well as the perturbative approach~\cite{tadano_2018}, using higher
order interatomic force constants derived from the fitting to the displacement force data obtained
with DFT, have proven to be successful.

Transition-metal silicides are promising materials for fabrication of electronic components
designed for integration with silicon-based circuits~\cite{murarka_1995}.
At room temperature, iron disilicide ($\beta$-FeSi$_2$) is a
direct-bandgap semiconductor~\cite{bost_1985}, making this material a good
candidate for application in optoelectronic devices such as infrared
detectors or light emitters~\cite{bost_1988}. The development of light-emitting diodes utilizing FeSi$_2$/Si heterostructures has been successfully demonstrated~\cite{leong_1997,suemasu_2001}. Due to
a high thermal stability and strong light absorption, FeSi$_2$ is
also a suitable photovoltaic material~\cite{powalla_1993,liu_2006,okuhara_2017}.

$\beta$-FeSi$_2$ crystallizes in the base-centered orthorhombic
lattice~\cite{dusausoy_1971} transforming to the
tetragonal metallic $\alpha$-FeSi$_{2}$ phase around $1200$~K~\cite{starke_2002}.
Optical studies indicated a direct band gap of the values
$0.85$--$0.89$~eV~\cite{bost_1988,dimitriadis_1990,arushanov_1995,wan_2003},
however, the {\it ab initio} calculations predicted a smaller indirect gap close to 0.8 eV~\cite{christensen_1990}. 
The existence of such an indirect gap was then confirmed by the optical linear transmittance measurements at
low temperatures~\cite{giannini_1992}. As shown by first-principles studies,
the character of the band gap is very sensitive to the orientation 
of a crystal grown on silicon~\cite{clark_1998}.

$\beta$-FeSi$_2$ belongs also to good thermoelectric materials~\cite{ware_1964}, with potential applications resulting from its chemical stability up to high temperatures, nontoxicity, and low cost of preparation~\cite{yamada_2012,nozariasbmarz_2017}. 
It has already been implemented in cars~\cite{birkholz_1988} and portable power sources~\cite{uemura_1989}. 
Its thermoelectric performance can be improved by doping~\cite{ito_2001,tani_2001,kim_2003,chen_2005,pandey_2013,le_tonquesse_2019}, 
which enhances the electric transport and reduces the thermal conductivity~\cite{waldecker_1973,du_2019,du_2020}.  
The thermal conductivity can be also reduced by the modification of microstructure~\cite{ail_2015} or by
nanostructurization~\cite{watanabe_2017,taniguchi_2017,hsin_2017,abbassi_2021}.

The lattice thermal conductivity is directly connected with anharmonic effects and phonon scattering processes.
The vibrational properties of \mbox{$\beta$-FeSi$_2$} were studied by the infrared and Raman spectroscopy~\cite{lefki_1991,guizzetti_1997,maeda_2004,baleva_2008,liu_2011,maeda_2011}. The observed anisotropy in the phonon spectra results from the enhanced sensitivity of the infrared and Raman features to the local lattice distortions~\cite{guizzetti_1997}. The Fe phonon density of states was measured by nuclear inelastic scattering (NIS), showing a good agreement with the density functional theory (DFT) calculations~\cite{walterfang_2005}. Using the DFT approach, the phonon dispersion curves, phonon density of states, as well as various thermodynamic properties were obtained within the harmonic approximation~\cite{tani_2010,liang_2011}. The extended Klemens model was applied to study
the anharmonic effect on phonon frequencies and linewidths observed by the Raman spectroscopy~\cite{zhang_2023}.
The impact of nanostructurization on lattice dynamics was explored in the $\beta$-FeSi$_2$ nanorods grown on the Si(110) surface by the NIS and {\it ab initio} methods~\cite{kalt_2022}.

In this work, we investigate the lattice dynamical properties of $\beta$-FeSi$_2$ 
using the DFT calculations. We study the effect of anharmonic terms in the temperature-dependent potential on phonon frequencies and lifetimes. We focus on the Raman modes, comparing the theoretical results with the experimental data.
The thermal conductivity is derived in a broad temperature range and the effect of crystallite size is analyzed.



This study is structured as follows.
In Sec.~\ref{sec.com} we describe the details of computational methods.
Next, in Sec.~\ref{sec.result} we present and discuss our results.
In particular we present the crystal structure (Sec.~\ref{sec.crys}) and lattice dynamics (Sec.~\ref{sec.lattice}).
We investigate also the thermal conductivity comparing the obtained results with the available experimental data (Sec.~\ref{sec.thermal}).
Finally, Sec.~\ref{sec.summary} summarizes our key findings and conclusions.

\section{Calculation method}
\label{sec.com}


The calculations were performed using the projector augmented-wave potentials~\cite{blochl_1994} and the generalized gradient approximation~\cite{perdew_1996} implemented in the Vienna Ab initio Simulation Package (VASP)~\cite{kresse.hafner.94,kresse.furthmuller.96,kresse.joubert.99}. 
The lattice parameters and atomic positions were optimized in the ${\bm a} \times ({\bm b}-{\bm c}) \times ({\bm b}+{\bm c})$ supercell containing 32 formula units and four primitive cells.
The integration in the reciprocal space was conducted using the $2 \times 2 \times 2$ Monkhorst--Pack mesh~\cite{monkhorst_1976} and the cut-off energy was set to $500$~eV. For convergence conditions, we set the energy change below $10^{-5}$ and $10^{-8}$ for the ionic and electronic loops, respectively. 

The lattice dynamical properties were studied within the temperature-dependent effective potential (TDEP) approach~\cite{hellman_2013}. The atomic potential with the third and fourth order anharmonic terms was derived from interatomic forces induced by displacements of all atoms at finite temperatures.
The sets of atomic displacements were generated by the high efficiency configuration space sampling (HECSS)~\cite{jochym_2021} and forces were obtained by VASP. The interatomic force constants and phonon frequencies were calculated with the {\sc Alamode} software~\cite{tadano_2014}.

Furthermore, we have attempted to construct a {\emph{temperature independent}} 
anharmonic model. We have used combined data from all investigated temperatures 
(300, 600 and 1000~K) and fitted a large (over 15~000 free parameters), fourth-order 
interaction model to this dataset. 
Subsequently, we have used this model to calculate line profiles and positions of Raman-active modes
at multiple temperatures.

The changes in phonon frequencies induced by the anharmonic effects were investigated within two approaches.
First, the impact of the quartic anharmonic terms was included using the SCPH theory~\cite{tadano_2015}.
Second, the mode profiles (frequency shifts and line widths) were determined from the real and imaginary parts of the phonon self-energy resulting from the cubic and quartic anharmonic terms of the above mentioned large model~\cite{tadano_2018}. 
The longitudinal optic-transverse optic (LO-TO) splitting was also evaluated, using the static dielectric tensor and Born effective charges calculated within density functional perturbation theory~\cite{gajdos_2006}.

\begin{figure}[]
    \centering
    \includegraphics[width=\linewidth]{fig1_new.png}
\caption{%
(a) The conventional unit cell of $\beta$-FeSi$_2$ (with Cmca symmetry) and (b) the corresponding Brillouin zone with selected high-symmetry points.
}
  \label{fig.struct}
\end{figure}

% To further characterize the vibrational properties, Raman-active scattering was investigated using the Phonopy-Spectroscopy package~\cite{skelton_2017}. This enabled the identification of Raman-active modes and the calculation of Raman tensors. Anharmonic force constants, obtained from calculations using {\sc Alamode}, were then used to obtain theoretical line profiles for the Raman modes. The presented Raman scattering spectra combine these anharmonic line profiles with the Raman tensor amplitudes.
% This analysis was also based on the large quartic model mentioned above.

% Finally, the thermal conductivity was obtained as a function of temperature and crystallite size within the relaxation-time approximation (RTA) \PJ{}{as implemented in {\sc Alamode}\cite{tadano_2014}}. The phonon lifetimes were calculated from the phonon self-energy including the cubic and quartic anharmonic terms. \SP{}{The RTA provides a solution of the Boltzmann transport equation (BTE) under the assumption that scattering events are independent and can be treated through mode-resolved relaxation times.
% To verify the validity of this approximation for $\beta$-FeSi$_2$, we have additionally
% executed an iterative solution of the BTE.}\PJ{}{These calculations were performed with
% {\sc Phono3py}\cite{togo_2023}. For cross-validation, the additional RTA calculations were performed on the same $\bm{q}$-grid as BTE with implementation provided by {\sc Phono3py}.}.

To further characterize the vibrational properties, Raman scattering was investigated using the Phonopy-Spectroscopy package~\cite{skelton_2017}. This enabled the identification of Raman-active modes and the calculation of Raman tensors. Anharmonic force constants, derived from calculations using {\sc Alamode}, were then used to obtain theoretical line profiles for the Raman modes. The presented Raman scattering spectra combine these anharmonic line profiles with the Raman tensor amplitudes.
This analysis was also based on the large quartic model described above.

Finally, the thermal conductivity was calculated as a function of temperature and crystallite size within the relaxation-time approximation (RTA) \PJ{}{as implemented in {\sc Alamode}~\cite{tadano_2014}}. The phonon lifetimes were calculated from the phonon self-energy including the cubic and quartic anharmonic terms. \SP{}{The RTA provides a solution to the Boltzmann transport equation (BTE) under the assumption that scattering events are independent and can be treated through mode-resolved relaxation times.
To verify the validity of this approximation for $\beta$-FeSi$_2$, we additionally
solved the BTE iteratively.} \PJ{}{These calculations were performed with
{\sc Phono3py}~\cite{togo_2023}. For cross-validation, the additional RTA calculations were performed on the same $\bm{q}$-grid as the BTE calculations, using the implementation provided by {\sc Phono3py}.}

% (see. Appendix~\ref{ThermalAB} for the comparison)\cite{togo_2023}.}
%As shown in Appendix~\ref{ThermalAB}, the iterative BTE results exhibit good agreement with the RTA values, confirming that RTA is sufficiently accurate for this material and for the considered temperature range.}


\section{Results}
\label{sec.result}

\subsection{Crystal structure}
\label{sec.crys}

The $\beta$-FeSi$_2$ structure adopts a base-centered orthorhombic lattice with the space group Cmca (No.~64) as shown in Fig.~\ref{fig.struct}(a).
The unit cell consists of two primitive cells and contains 48 atoms.
Iron (silicon) atoms possess two nonequivalent positions: \mbox{Fe-I} and \mbox{Fe-II} (\mbox{Si-I} and \mbox{Si-II}), presented in Fig.~\ref{fig.struct}(a) as gray and purple (orange and yellow) spheres, respectively.
This crystal structure is derived from the fluorite-type lattice with strongly distorted Si cubes and Fe atoms occupying 
one-half of the central sites.
The Fe-I and Fe-II sites create different layers perpendicular to the $x$ direction, and they are separated by layers containing both Si sites.
Each Fe atom is coordinated by 8 Si atoms with slightly different Fe-Si distances.
The optimized lattice constants ($a = 9.874$~{\AA}, $b = 7.767$~{\AA}, and
$c = 7.811$~{\AA}) agree very well with the experimental data ($a = 9.863$~{\AA}, $b = 7.791$~{\AA}, and $c = 7.833$~{\AA})~\cite{dusausoy_1971}.

Iron atoms occupy the Wyckoff sites 8\textit{d} ($0.2166$, $0$, $0$) and  8\textit{f} ($0$, $0.3072$, $0.1879$), corresponding to \mbox{Fe-I} and \mbox{Fe-II}, respectively. 
Silicon atoms are located at two inequivalent 16\textit{g} positions: ($0.1282$, $0.2737$, $0.0495$) and \mbox{($0.3734$, $0.0445$, $0.2270$)}, assigned as \mbox{Si-I} and \mbox{Si-II}.
The optimized positions of atoms agree very well with the experimental data~\cite{dusausoy_1971} and the previous theoretical studies~\cite{tani_2010,liang_2011}.


\subsection{Lattice dynamics}
\label{sec.lattice}

\begin{figure}[]
    \centering
    \includegraphics[width=\linewidth]{Fig1c.pdf}
\caption{%
The phonon dispersion curves along high symmetry directions obtained within SCPH (for temperatures from $0$ to $1000$~K).
Dashed black lines indicate the phonon dispersions obtained from the harmonic approximation. The white dots indicate Raman-active modes with A$_g$ symmetry.
The vertical plot shows the phonon density of states (DOS) calculated at a reference temperature of $600$~K.
}
  \label{fig.ph_band}
\end{figure}


In Fig.~\ref{fig.ph_band} we present the phonon dispersion relations of $\beta$-FeSi$_2$ along high-symmetry directions in the Brillouin zone [Fig.~\ref{fig.struct}(b)].
Due to 24 atoms in the primitive cell, the phonon spectrum consists of 69 optical modes and three acoustic modes.
The phonon dispersions were calculated within the SCPH approach in the temperature range $0$--$1000$~K (presented by color lines in Fig.~\ref{fig.ph_band}), and they are compared with the results obtained from the harmonic part of the effective potential corresponding to temperature $T=300$~K (indicated by dashed black lines in Fig.~\ref{fig.ph_band}).
As we can see within the SCPH method, the anharmonic effects are rather weak and leads only to small renormalization of phonon frequencies.
Only the highest modes show more pronounced shifts of their frequencies to larger values.
The total and partial element-projected phonon density of states obtained within the harmonic approximation are presented in Fig.~\ref{fig.ph_band}.
Up to around $320$~cm$^{-1}$, the contributions from both elements are very similar, while for higher frequencies the spectrum is dominated by
the Si vibrations.


\begin{table}[!t]
\begin{ruledtabular}
\caption{Calculated and experimental Raman-active modes of $\beta$-FeSi$_2$ with their irreducible representations (IR). Present theoretical results are compared with the previous theoretical data from Ref.~\cite{tani_2010} and experimental results from \mbox{Refs.~\cite{lefki_1991,maeda_2004}.} The experimental frequencies with known symmetries (A$_g$) are shown in bold, while other experimental modes are assigned to the best fitting theoretical values.}
\begin{tabular}{c c c c c}
\textbf{IR} & \multicolumn{4}{c}{\textbf{Frequency (cm$^{-1}$)}} \\
 & Present  & Theor.~\cite{tani_2010} & Exp.~\cite{lefki_1991} & Exp.~\cite{maeda_2004} \\
\hline
B$_{2g}$ & 175.4 & 179 & 176 &  \\
B$_{1g}$ & 176.3 & 185 & 179  &  \\
B$_{1g}$ & 193.3 & 198 &  & 190.6 \\
A$_g$    & 196.6 & 208 & \textbf{195} & \textbf{194.0} \\
A$_g$    & 203.9 & 210 & \textbf{197} & 199.6 \\
B$_{3g}$ & 205.5 & 212 & 200 &  \\
B$_{3g}$ & 226.3 & 236 & 206 & 227.1 \\
B$_{1g}$ & 233.3 & 240 &  &  231.6 \\
B$_{2g}$ & 248.6 & 254 &  &  \\
A$_g$    & 250.1 & 257 & \textbf{247} & \textbf{247.3} \\
A$_g$    & 254.9 & 264 & \textbf{253} & 254.3 \\
B$_{1g}$ & 275.5 & 285 &  &  274.1 \\
B$_{2g}$ & 282.4 & 295 &  &  281.2 \\
B$_{3g}$ & 286.8 & 297 &  &  \\
B$_{1g}$ & 307.1 & 317 &  &  \\
B$_{2g}$ & 312.6 & 326 &  &  311.8 \\
B$_{1g}$ & 319.0 & 324 &  &  \\
B$_{3g}$ & 327.9 & 341 &  & 325.8 \\
B$_{2g}$ & 333.3 & 345 &  &  \\
A$_g$    & 339.3 & 352 & \textbf{346} & 339.5 \\
B$_{2g}$ & 343.0 & 350 &  &  \\
B$_{1g}$ & 353.5 & 366 &  &  \\
B$_{2g}$ & 372.8 & 383 &  & 370.7 \\
B$_{1g}$ & 375.1 & 385 &  &  \\
B$_{3g}$ & 375.2 & 386 &  &  \\
B$_{3g}$ & 385.3 & 401 &  &  \\
A$_g$    & 386.5 & 398 &  & 386.2 \\
B$_{2g}$ & 387.7 & 402 &  & 388.2 \\
A$_g$    & 404.6 & 415 &  & 400.4 \\
B$_{2g}$ & 405.1 & 420 &  &  \\
B$_{1g}$ & 412.7 & 428 &  &  \\
B$_{3g}$ & 418.4 & 431 &  &  \\
B$_{3g}$ & 441.4 & 458 &  & 442.6 \\
B$_{3g}$ & 447.3 & 466 &  & 446.3 \\
A$_g$    & 448.4 & 464 &  &  \\
A$_g$    & 499.1 & 517 &  &  \\
\end{tabular}
\label{T1Raman}
\end{ruledtabular}
\end{table}

%subsection{Raman}

\begin{figure*}
\centering
  \includegraphics[width=\textwidth]{spectrum_RTA+raman.pdf}
\caption{Raman spectrum of $\beta$-FeSi$_2$ calculated within the perturbative approach at three temperatures 
($300$, $600$, $900$~K -- blue, orange, and red lines, respectively). 
The calculated spectrum includes all Raman-active modes. The A$_g$ modes
are indicated by purple vertical lines at peak positions corresponding to T=300~K. 
The frequencies derived from the harmonic approximation at T=300~K are indicated by green lines.
Connecting arrows indicate the correspondence between harmonic and
anharmonic frequencies, demonstrating the frequency shifts due to 
phonon interactions. Experimental values for the A$_g$ modes
based on Ref.~\cite{lefki_1991} are marked with black dashed lines.
}
\label{raman}
\end{figure*}


The phonon spectrum at the $\Gamma$ point consists of 36 Raman modes classified according to the irreducible representations: $9A_\text{g}+9B_\text{1g}+9B_\text{2g}+9B_\text{3g}$. 
Fig.~\ref{raman} shows the Raman spectrum of A$_g$ symmetry calculated for $\beta$-FeSi$_2$ within the perturbative approach at three temperatures $300$, $600$, and $900$~K (solid blue, orange, and red curves, respectively), including third-order and fourth-order anharmonic corrections.
The calculated Raman spectrum includes all Raman-active modes.
The five experimentally observed A$_g$ modes are highlighted by black dashed lines, based on data from Ref.~\cite{lefki_1991}.
Additional peaks not marked with vertical lines correspond to phonon modes with symmetries other than A$_g$.
The frequencies of Raman modes obtained in anharmonic calculations are compared with the previous results calculated within the harmonic approximation and the experimental values in Tab.~\ref{T1Raman}. 
We have marked in bold the experimentally determined A$_g$ modes, which are compared with the calculations.
Since the experimental studies did not provide the accurate assignement of the Raman modes with the B$_{1g}$, B$_{2g}$, and B$_{3g}$ symmetry~\cite{lefki_1991,maeda_2004}, we cannot compare them directly with the theoretical results.
However, in Tab.~\ref{T1Raman} we have assigned the measured frequencies to the best fitting theoretical values without taking into account the symmetry of the modes, except for the known A$_{g}$ modes. 


The impact of anharmonicity on the phonon frequencies is well visible from the comparison of the results obtained within the harmonic approximation 
and from the anharmonic calculation (vertical green and purple lines in Fig.~\ref{raman}, respectively).
Here we show only the A$_g$ modes, which are compared with the experimental results (vertical black dashed lines).
Anharmonic frequencies calculated at $300$~K are indicated by purple lines, while frequencies derived from harmonic approximation are marked with green lines. The grey solid lines connect corresponding modes obtained in both approximations.
In most cases, the results obtained within the harmonic approximation do not agree with the experimental frequencies. 
%Only the modes close to $250$~cm$^{-1}$ and $400$~cm$^{-1}$ correspond well to the experimental values. MS
As we can see, the inclusion of the anharmonic correction leads to a significant modification of the phonon frequencies.
These anharmonic effects are stronger for higher-frequency modes mainly because of
the dominant contribution from Si atoms, which vibrate with larger amplitudes than heavier Fe atoms. 
When atoms move to larger distances the potential deviates more from the harmonic approximation,
and the anharmonic corrections become stronger.


The modification of phonon frequencies observed in Fig.~\ref{raman} is much larger than in the SCPH scheme presented in Fig.~\ref{fig.ph_band}. The SCPH approach includes only the leading-order contribution to 
the phonon self-energy obtained from the quartic anharmonic terms~\cite{tadano_2015}. Therefore, it does not describe fully the changes of phonon frequencies found within the perturbation theory (see Fig.~\ref{raman}).
Especially, it is well visible for two highest A$_g$ modes, which exhibit also the largest line broadening 
and the strongest dependence on temperature.
Therefore, a better agreement with experimentally observed frequencies is visible,
confirming the significant influence of the anharmonicity on the frequencies and line profiles of phonon modes.
In fact, the decrease of phonon frequencies should be even stronger due to thermal expansion, 
which is not included in our calculations.
Within SCPH the frequencies of the highest modes increase with increasing temperature as we see in Fig.~\ref{fig.ph_band}. The comparison of two different approaches applied to study anharmonic properties of $\beta$-FeSi$_2$ shows that the perturbation theory, which includes the cubic and quartic terms, better describes the changes of phonon frequencies with temperature than the SCPH method. \PJ{}{This indicates that the leading-order contribution included in SCPH are not important in this material.}

Additionaly we should nottice that for other than A$_g$ modes we cannot make an unambiguous assignment of theoretical frequencies to experimental ones. Note that the spectrum in Fig.~\ref{fig.ph_band} contains all Raman-active modes. 
The limitation to A$_g$ modes concerns only the indicated positions of the peaks. 
%\SP{}{Furthermore, we present the full spectrum for all Raman active modes with the comparison of the harmonic and anharmonic calculations in the Appendix~\ref{RamanAA}.}
\textcolor{red}{The full Raman spectrum, with the frequencies of the B$_{1g}$, B$_{2g}$ and B$_{3g}$ modes marked, is shown in Appendix A, Fig.~\ref{raman1}. This represents our theoretical prediction of possible Raman-mode assignments, which can be verified in future experiments.}
%This is the theoretical prediction of possible assignment of Raman modes that can be verified in future experiments.  


\subsection{Thermal conductivity}
\label{sec.thermal}


\begin{figure*}[!t]
\centering
  \includegraphics[width=\linewidth]{time3.pdf}
\caption{Phonon lifetimes calculated for three temperatures as a function of phonon frequency. The colors correspond to the phonon branches.}
  \label{thermtime}
\end{figure*}


In this section, we analyze the thermal conductivity tensor of $\beta$-FeSi$_2$ obtained within the RTA approach~\cite{tadano_2018} as a function of temperature
%
\begin{equation}
\kappa_{\text{ph}}^{\mu\nu}(T) = \frac{1}{NV} \sum_{\bm{q},j} c_{\bm{q}j}(T) v_{\bm{q}j}^{\mu} v_{\bm{q}j}^{\nu}\tau_{\bm{q}j}(T),
\end{equation}
% 
where $c_{\bm{q}j}$ is the mode heat capacity and $v_{\bm{q}j}$ is the mode group velocity. 
The relaxation time is approximated by the phonon lifetime $\tau_{\bm{q}j}$
calculated for $j$-th branch at the wave vector $\bm{q}$.
$V$ is the unit cell volume and $N$ is the number of unit cells in the crystal.
The phonon lifetime is calculated using this formula
%
\begin{equation}
\tau_{\bm{q}j}(T)=\frac{1}{2\Gamma_{\bm{q}j}^{\text{anh}}(T)},
\end{equation}
%
where $\Gamma_{\bm{q}j}^{\text{anh}}$ is the anharmonic phonon linewidth obtained from 
the imaginary part of the phonon self-energy within the perturbation theory.

In Fig.~\ref{thermtime}, we present $\tau_{\bm{q}j}$ obtained for three temperatures 300, 600, and 1000~K as a function of frequency. As we see, the acoustic phonons close to the $\Gamma$ point have the longest lifetimes,
which are diminished with increasing frequency reaching local minima around $200$~cm$^{-1}$.
For higher frequencies, phonon lifetimes first increase to local maxima around $300$~cm$^{-1}$ and then decrease to
the lowest values in the range of highest optical modes. The shortest lifetimes correspond to the largest line broadening
observed for the Raman modes in Fig~\ref{raman}. The phonon group velocities 
$v_{\bm{q}j}=\partial\omega_{\bm{q}j}/\partial\bm{q}$, which are obtained by the central difference formula, are presented in Fig.~\ref{thermvelocity}. Their temperature dependence is negligible, therefore, we present only the results for $T=600$~K.
At low frequencies, there are clearly two ranges of group velocities of the acoustic phonons. 
The larger values correspond to the longitudinal modes, while the lower values are obtained from the
transverse acoustic branches. Group velocities of acoustic phonons decrease for larger frequencies
and reach the average values typical for optic branches.

\begin{figure}[!t]
\centering
  \includegraphics[width=\linewidth]{GV.pdf}
\caption{Mode group velocities calculated as a function of phonon frequency. The colors correspond to the phonon branches.
}
  \label{thermvelocity}
\end{figure}

In Fig.~\ref{anizotropy}(a), we present the three diagonal elements of $\kappa_{\text{ph}}^{\mu\nu}$ corresponding to the main directions of the crystal structure. 
They were obtained from the force constants calculated at the base temperature $T=600$~K and the crystallite size $0.1$~$\mu$m to account for boundary-limited phonon transport. 
Due to the orthorhombic symmetry, we observe a small anisotropy in phonon transport in the whole temperature range. 
At low temperatures, the three components of the heat conductivity increase in a very similar way with the $\kappa_{\text{ph}}^{yy}$ element slightly larger than two other components. 
After reaching the maximum, we observe a change in the largest component from $\kappa_{\text{ph}}^{yy}$ to 
$\kappa_{\text{ph}}^{xx}$.    
In Fig.~\ref{anizotropy}(b), the thermal conductivity is shown for three base temperatures, at which the interatomic potential was obtained ($300$~K, $600$~K, and $1000$~K), using the energy expansion up to third- and fourth-order anharmonic terms, and the same structure size of $0.1$~$\mu$m. At lowest temperatures, the thermal conductivity strongly increases, reaching the maximum around $T=180$~K, then it shows a slower decrease with temperature.
The differences between the two levels of approximation are minimal, suggesting that third-order calculations already capture the dominant phonon scattering mechanisms. The dependence on the base temperature is also very weak, showing the changes in the heat conductivity within a few percent. 
\SP{}{Further verification of the reliability of our thermal conductivity results is given in Appendix~\ref{ThermalAB}, where the full BTE calculations show good agreement with RTA and higher-resolution RTA results, demonstrating that the $8\times8\times8$ $\bm{q}$-mesh already provides converged values.}

\begin{figure}[!t]
\centering
  \includegraphics[width=\linewidth]{Anizotropy.pdf}
\caption{(a) The anisotropic thermal conductivity of $\beta$-FeSi$_2$ calculated along the lattice directions at 600~K. (b) The average temperature-dependent thermal conductivity taken at $300$~K, $600$~K, and $1000$~K, including anharmonic corrections up to cubic (A3) and quartic (A4) terms. In both cases the crystallite size is 0.1~$\mu$m.}
  \label{anizotropy}
\end{figure}


In Fig.~\ref{therm}, we fix the base temperature at $600$~K and examine the effect of crystallite size on thermal conductivity, varying it from $0.01$ to $0.5$~$\mu$m.
With decreasing the crystallite size, we observe a shift of the position of the maximum to larger temperatures and a decrease of the thermal conductivity in the entire temperature range.
Theoretical results are compared with several experimental data obtained above the room temperature. 
The measured thermal conductivity depends to a large extent on the sample quality, its purity and the size of the crystalline grains which depends on 
the production processes.
Many measurements were performed using crystallites of micrometric or unknown size ~\cite{waldecker_1973,ito_2002,kim_2003,du_2020}, however, 
numerous attempts to minimize $\kappa$ by reducing grain sizes to $56$~nm~\cite{dabrowski_2019, dabrowski_microstructure_2021}, $30$-$400$~nm~\cite{le_tonquesse_2019}, $50$ and $200$~nm~\cite{abbassi_2021}, or introducing pores into the material~\cite{sam2023improved} 
are also carried out. 
Another way to change the thermal conductivity is to dope $\beta$-FeSi$_2$ with different elements~\cite{ito_2002,kim_2003,du_2020,cheng_2024}, however, this effect is beyond our investigation.

\begin{figure}[!t]
\centering
  \includegraphics[width=\linewidth]{Thermal_conductivity3.pdf}
\caption{
The phonon thermal conductivity of $\beta$-FeSi$_2$: theoretical results for the infinite crystalline size and with boundary conditions, compared with experimental data for different structure sizes.
}
  \label{therm}
\end{figure}
%Fig.~\ref{therm}\cite{abbassi_2021}\cite{waldecker_1973}\cite{dabrowski_2019}\cite{sam2023improved}\cite{kim_2003}\cite{du_2020}\cite{ito_2002}\cite{le_tonquesse_2019}.

We observe a decrease in the thermal conductivity with reducing crystalline grain sizes in all analyzed experimental data. 
For instance, by decreasing the crystallite size to $50$~nm, the thermal conductivity at room temperature was reduced by a factor of $1.7$, what can be  compared to the annealed sample with 200 nm grains~\cite{abbassi_2021}. 
It is worth noting that the rate of decrease in value with increasing temperature in both cases, for grain sizes of $50$~nm and $200$~nm, is significantly different, which is consistent with our calculations. 
The same trend can be observed by comparing the thermal conductivity measured for a sample with bulk crystallite sizes with the thermal conductivity of a sample with grains smaller than 400 nm~\cite{le_tonquesse_2019}.
The theoretical results obtained for the same crystallite size show higher values due to factors not captured in the idealized model, such as crystal imperfection or mechanical strain. Usually, a decrease in the crystallite size is related to an increased concentration of grain boundaries, point defects, and stacking faults that influence the phonon scattering~\cite{le_tonquesse_2019,abbassi_2021}.   

We should note that the total thermal conductivity is a combination
of the lattice and electronic contributions to the heat transport.
In semiconductors, the electronic thermal conductivity is negligible at low temperatures and significantly increases 
only much above the room temperatures~\cite{gu_2020}.
For $\beta$-FeSi$_2$, the electronic thermal conductivity was obtained from the electric conductivity using the Wiedemann-Franz law~\cite{ito_2002,kim_2003,le_tonquesse_2019}.
In the undoped material, its value does not exceed $0.1$~W/mK in the measurement up to $T=950$~K~\cite{kim_2003}.
By doping, the electronic thermal conductivity can be enhanced, and it has a direct impact on the thermoelectric properties of $\beta$-FeSi$_2$ at high temperatures~\cite{ito_2002,kim_2003}.
In the present study, we consider only the phonon contribution to the thermal conductivity,
therefore, agreement with experimental data may deteriorate with increasing temperature.

\section{Summary}
\label{sec.summary}

We performed {\it ab initio} studies on lattice dynamical and thermal transport properties of $\beta$-FeSi$_2$. The effect of anharmonicity was analyzed within two approaches -- the SCPH method and the perturbation theory.
The phonon dispersion curves obtained within SCPH show small renormalization of frequencies comparing to the harmonic approximation. 
The Raman spectra were calculated within the procedure which takes into account the peak intensities obtained from the Raman tensors and the line profiles obtained from the phonon self energy derived within the perturbation theory based on the large, temperature-independent, quartic model fitted to the data from the wide range of temperatures (300-1000~K). 
The anharmonic corrections strongly affect the frequencies and line profiles of some modes and results in overall better agreement with the experimental data. 
We analyzed the phonon lifetimes and group velocities obtained as functions of the phonon frequency.
Then the lattice thermal conductivity was calculated for a broad range of temperatures and grain sizes.
We found a small anisotropy in the phonon thermal transport resulting from the orthorhombic structure and a weak effect of the quartic anharmonic terms. 
The thermal conductivity calculated for various crystalline grain sizes show a good qualitative agreement with the available measurements.

\begin{acknowledgments}
Some figures in this work were rendered using {\sc Vesta}~\cite{momma.izumi.11} software.
This work was partially supported by the Ministry of Education, Youth and Sports of the Czech Republic through the e-INFRA CZ (ID:90254).
\end{acknowledgments}

\appendix

\section{Raman spectrum}
\label{RamanAA}

Based on the polarized Raman measurements reported in Ref.~\cite{maeda_2004}, two Raman peaks were identified as belonging to the A$_g$ symmetry class, and several additional peaks were observed with similar or different polarization dependence. Although the authors of Ref.~\cite{maeda_2004} provided estimates of the relative Raman tensor components, they did not specify which of the remaining modes correspond to the B$_{1g}$, B$_{2g}$, or B$_{3g}$ symmetries. Because of this missing experimental information, a direct symmetry-resolved comparison between the measurement and theory is not currently possible for the non-A$_g$ modes. 
To provide a complete theoretical picture of the B$_g$-type modes, we show here the calculated Raman-active frequencies and intensities for the B$_{1g}$, B$_{2g}$, and B$_{3g}$ symmetries only. 
%These results represent the predicted Raman modes for the B$_g$ symmetries in $\beta$-FeSi$_2$. 
\textcolor{red} {Fig.~\ref{raman1} shows the predicted Raman modes for the B$_g$ symmetries in $\beta$-FeSi$_2$. The results obtained within the harmonic approximation are compared with the anharmonic perturbation theory calculations which provides both, frequency shifts and predicted line profiles of the modes.}
Although the experimentally measured peaks cannot be directly assigned to these symmetries due to the lack of polarization-resolved data, the theoretical predictions provide a reference for comparison. Matching the measured frequencies to the closest theoretical B$_g$ modes (Table~\ref{T1Raman}) allows for a tentative assignment, which can guide future polarization-resolved Raman experiments aimed at determining the precise symmetry of the unresolved peaks.

\begin{figure*}[t]
\centering
  \includegraphics[width=\linewidth]{spectrum_RTA+raman_Bi_modes.pdf}
\caption{Raman spectrum of $\beta$-FeSi$_2$ calculated at three temperatures 
($300$, $600$, $900$~K -- blue, orange, and red lines, respectively). 
The calculated spectrum includes all Raman-active modes. The B$_{ig}$ modes
are indicated by purple vertical lines at peak positions corresponding to T=300~K. 
The frequencies derived from the harmonic approximation at T=300~K are indicated 
by green, yellow and pink lines.
Connecting arrows indicate the correspondence between harmonic and anharmonic 
frequencies, demonstrating the frequency shifts due to phonon interactions.}
\label{raman1}
\end{figure*}
\section{Thermal conductivity obtained from BTE and RTA}
\label{ThermalAB}

The thermal conductivity was computed by solving the full BTE on the largest feasible $\bm{q}$-point grid, $8\times8\times8$, and compared with the corresponding RTA results obtained on the same grid. As shown in Fig.~\ref{bte}, the difference between the components of the thermal conductivity tensor obtained within BTE and RTA at this resolution is very small, indicating a good agreement between these two approaches.
%THIS PART SHOULD GO RATHER TO THE RESPONSE
%Extending the full BTE calculation to larger grids is computationally prohibitive: the computational cost of BTE is roughly two orders of %magnitude higher than that of RTA, and the required memory and runtime exceed our available resources. 
Moreover, we performed an additional calculation using RTA on a denser $20\times20\times20$ grid. As seen in the Fig.~\ref{bte}, the higher-resolution data remain in a good agreement with both the BTE and RTA results for the $8\times8\times8$ grid.
It shows that the $8\times8\times8$ mesh already provides good results for this structure and confirms reliability of the calculations.

\begin{figure}[!h]
\centering
  \includegraphics[width=\linewidth]{BTEvsRTAvsANP.pdf}
\caption{Thermal conductivity of $\beta$-FeSi$_2$ obtained within the BTE and RTA methods using the Phono3py software on the $8\times8\times8$ q-point grid, compared with the RTA results computed with ALAMODE on a denser $20\times20\times20$ grid.}
\label{bte}
\end{figure}

%\section*{Data availability}
%The data that support the findings of this article are openly available~\footnote{give me DOI}.
% see https://journals.aps.org/authors/data-availability-statements#citation

\bibliography{refs.bib}
%\bibliographystyle{ieeetr}


\end{document}

\documentclass[%
%reprint,
superscriptaddress,
%groupedaddress,
longbibliography,
%unsortedaddress,
%runinaddress,
%frontmatterverbose, 
%preprint,
%preprintnumbers,
%nofootinbib,
nobibnotes,
%bibnotes,
amsmath,amssymb,
aps,
%pra,
prb,
%rmp,
%prstab,
%prstper,
%showkeys,
floatfix,
twocolumn
]{revtex4-2}

\usepackage{graphicx}% Include figure files
\usepackage{calc}% Calculate margins
\usepackage{dcolumn}% Align table columns on decimal point
\usepackage{bm}% bold math

\usepackage[urlcolor=blue,colorlinks=true,citecolor=blue,linkcolor=blue,pdfstartview={FitH},bookmarks=false]{hyperref} % add hypertext capabilities

%\usepackage[mathlines]{lineno} % Enable numbering of text and display math
% \linenumbers\relax % Commence numbering lines

% \usepackage[showframe,%Uncomment any one of the following lines to test 
% %scale=0.7, marginratio={1:1, 2:3}, ignoreall,% default settings
% %text={7in,10in},centering,
% %margin=1.5in,
% % total={6.5in,8.75in}, top=1.2in, left=0.9in, includefoot,
% % height=10in,a5paper,hmargin={3cm,0.8in},
% ]{geometry}

\usepackage{amsmath}
\usepackage{amssymb}
%\usepackage{orcidlink}
\usepackage{xcolor}
%\usepackage{datetime}
\usepackage[normalem]{ulem}

% Change tracking commands
\newcommand{\trackchange}[3]{\textcolor{#3}{\sout{#1}#2}}  % Full color strikeout, insert
%\renewcommand{\trackchange}[3]{\textcolor{#3}{#2}}        % Just color silent remove and insert
%\renewcommand{\trackchange}[3]{{#2}}                      % No indication, silent remove and insert

% Author marker definitions

\definecolor{myblue}{RGB}{0,127,85}
% \definecolor{violet}{RGB}{102,0,204}
% \definecolor{orange}{RGB}{255,128,0}
% \definecolor{green}{RGB}{0,128,0}
\newcommand{\DL}[1]{\trackchange{}{#1}{blue}}
\newcommand{\AP}[1]{\trackchange{}{#1}{red}}
\newcommand{\PJ}[2]{\trackchange{#1}{#2}{orange}}
\newcommand{\JL}[2]{\trackchange{#1}{#2}{myblue}}
\newcommand{\SP}[2]{\trackchange{#1}{#2}{blue}}
\newcommand{\PP}[2]{\trackchange{#1}{#2}{teal}}
\newcommand{\AI}[2]{\trackchange{#1}{#2}{olive}}
\newcommand{\MS}[1]{\trackchange{}{#1}{purple}}

\newcommand{\TODO}[1]{\textcolor{red}{TODO: #1}}

\sloppy

\begin{document}

\title{Ab initio study of the anharmonic properties and thermal conductivity in $\beta$-FeSi$_2$}

\author{Svitlana~Pastukh}
\email[e-mail: ]{svitlana.pastukh@ifj.edu.pl}
\affiliation{Institute of Nuclear Physics, Polish Academy of Sciences, ul. W. E. Radzikowskiego 152, 31-342 Krak\'{o}w, Poland}

\author{Ma\l{}gorzata~Sternik}
\affiliation{Institute of Nuclear Physics, Polish Academy of Sciences, ul. W. E. Radzikowskiego 152, 31-342 Krak\'{o}w, Poland}

\author{Pawe\l{}~T.~Jochym}
\affiliation{Institute of Nuclear Physics, Polish Academy of Sciences, ul. W. E. Radzikowskiego 152, 31-342 Krak\'{o}w, Poland}


\author{Jan~\L{}a\.{z}ewski}
\affiliation{Institute of Nuclear Physics, Polish Academy of Sciences, ul. W. E. Radzikowskiego 152, 31-342 Krak\'{o}w, Poland}

\author{Andrzej~Ptok}
\affiliation{Institute of Nuclear Physics, Polish Academy of Sciences, ul. W. E. Radzikowskiego 152, 31-342 Krak\'{o}w, Poland}

\author{Svetoslav~Stankov}
\affiliation{Institute for Photon Science and Synchrotron Radiation, Karlsruhe Institute of Technology, D-76131 Karlsruhe, Germany}
\affiliation{Laboratory for Applications of Synchrotron Radiation, Karlsruhe Institute of Technology, D-76131 Karlsruhe, Germany}

\author{Przemys\l{}aw~Piekarz}
\affiliation{Institute of Nuclear Physics, Polish Academy of Sciences, ul. W. E. Radzikowskiego 152, 31-342 Krak\'{o}w, Poland}

\date{\today}

\begin{abstract}

Iron silicides are good candidates for applications in  optoelectronic and thermoelectric devices.
Lattice dynamical properties and thermal conductivity in the $\beta$-FeSi$_2$ semiconductor
are investigated with the first-principles computational methods. 
Phonon dispersion relations are calculated via the
temperature-dependent effective potential method and self-consistent phonon theory. 
To properly model thermal transport, we explicitly consider
the impact of phonon-phonon interactions by analyzing
anharmonic contributions to the phonon self-energy. 
This yields temperature-dependent phonon frequencies and linewidths,
reflecting the finite lifetime of phonons due to scattering
processes. The calculated phonon frequencies and line profiles are used to obtain 
the Raman spectra, which shows good agreement with the experimental data. 
We revealed an enhanced anharmonic behaviour of the Raman modes with the highest frequencies.  
The lattice thermal conductivity is then obtained as a function of temperature and crystallite size within
the relaxation-time approximation.
Phonon transport shows a small anisotropy due to the orthorhombic structure and a very weak dependence
on the quartic anharmonic corrections. The results obtained for an infinite material and for several crystallite sizes
were analyzed and compared with the available experimental data.
\end{abstract}

\maketitle


\section{Introduction}

The comprehensive determination of important physical properties of
crystals, such as thermal expansion, lattice thermal
conductivity or structural phase transitions, requires a fundamental 
understanding of the anharmonic effects.
Although the investigation of anharmonic interactions in crystals has
attracted a considerable interest for decades~\cite{cowley_1968}, a substantial progress
has only recently been achieved thanks to advances in theoretical and
numerical methods and increased computational power.
Now, phonon frequencies, lifetimes, and heat transfer in a wide range of
materials
can be quantitatively predicted using the available computational resources
based on the density functional theory (DFT)~\cite{lindsay_2013,mcgaughey_2019,lindsay_2019}.
In the case of strongly anharmonic systems, the self-consistent phonon
(SCPH) theory~\cite{tadano_2015} as well as the perturbative approach~\cite{tadano_2018}, using higher
order interatomic force constants derived from the fitting to the displacement force data obtained
with DFT, have proven to be successful.

Transition-metal silicides are promising materials for fabrication of electronic components
designed for integration with silicon-based circuits~\cite{murarka_1995}.
At room temperature, iron disilicide ($\beta$-FeSi$_2$) is a
direct-bandgap semiconductor~\cite{bost_1985}, making this material a good
candidate for application in optoelectronic devices such as infrared
detectors or light emitters~\cite{bost_1988}. The development of light-emitting diodes utilizing FeSi$_2$/Si heterostructures has been successfully demonstrated~\cite{leong_1997,suemasu_2001}. Due to
a high thermal stability and strong light absorption, FeSi$_2$ is
also a suitable photovoltaic material~\cite{powalla_1993,liu_2006,okuhara_2017}.

$\beta$-FeSi$_2$ crystallizes in the base-centered orthorhombic
lattice~\cite{dusausoy_1971} transforming to the
tetragonal metallic $\alpha$-FeSi$_{2}$ phase around $1200$~K~\cite{starke_2002}.
Optical studies indicated a direct band gap of the values
$0.85$--$0.89$~eV~\cite{bost_1988,dimitriadis_1990,arushanov_1995,wan_2003},
however, the {\it ab initio} calculations predicted a smaller indirect gap close to 0.8 eV~\cite{christensen_1990}. 
The existence of such an indirect gap was then confirmed by the optical linear transmittance measurements at
low temperatures~\cite{giannini_1992}. As shown by first-principles studies,
the character of the band gap is very sensitive to the orientation 
of a crystal grown on silicon~\cite{clark_1998}.

$\beta$-FeSi$_2$ belongs also to good thermoelectric materials~\cite{ware_1964}, with potential applications resulting from its chemical stability up to high temperatures, nontoxicity, and low cost of preparation~\cite{yamada_2012,nozariasbmarz_2017}. 
It has already been implemented in cars~\cite{birkholz_1988} and portable power sources~\cite{uemura_1989}. 
Its thermoelectric performance can be improved by doping~\cite{ito_2001,tani_2001,kim_2003,chen_2005,pandey_2013,le_tonquesse_2019}, 
which enhances the electric transport and reduces the thermal conductivity~\cite{waldecker_1973,du_2019,du_2020}.  
The thermal conductivity can be also reduced by the modification of microstructure~\cite{ail_2015} or by
nanostructurization~\cite{watanabe_2017,taniguchi_2017,hsin_2017,abbassi_2021}.

The lattice thermal conductivity is directly connected with anharmonic effects and phonon scattering processes.
The vibrational properties of \mbox{$\beta$-FeSi$_2$} were studied by the infrared and Raman spectroscopy~\cite{lefki_1991,guizzetti_1997,maeda_2004,baleva_2008,liu_2011,maeda_2011}. The observed anisotropy in the phonon spectra results from the enhanced sensitivity of the infrared and Raman features to the local lattice distortions~\cite{guizzetti_1997}. The Fe phonon density of states was measured by nuclear inelastic scattering (NIS), showing a good agreement with the density functional theory (DFT) calculations~\cite{walterfang_2005}. Using the DFT approach, the phonon dispersion curves, phonon density of states, as well as various thermodynamic properties were obtained within the harmonic approximation~\cite{tani_2010,liang_2011}. The extended Klemens model was applied to study
the anharmonic effect on phonon frequencies and linewidths observed by the Raman spectroscopy~\cite{zhang_2023}.
The impact of nanostructurization on lattice dynamics was explored in the $\beta$-FeSi$_2$ nanorods grown on the Si(110) surface by the NIS and {\it ab initio} methods~\cite{kalt_2022}.

In this work, we investigate the lattice dynamical properties of $\beta$-FeSi$_2$ 
using the DFT calculations. We study the effect of anharmonic terms in the temperature-dependent potential on phonon frequencies and lifetimes. We focus on the Raman modes, comparing the theoretical results with the experimental data.
The thermal conductivity is derived in a broad temperature range and the effect of crystallite size is analyzed.



This study is structured as follows.
In Sec.~\ref{sec.com} we describe the details of computational methods.
Next, in Sec.~\ref{sec.result} we present and discuss our results.
In particular we present the crystal structure (Sec.~\ref{sec.crys}) and lattice dynamics (Sec.~\ref{sec.lattice}).
We investigate also the thermal conductivity comparing the obtained results with the available experimental data (Sec.~\ref{sec.thermal}).
Finally, Sec.~\ref{sec.summary} summarizes our key findings and conclusions.

\section{Calculation method}
\label{sec.com}


The calculations were performed using the projector augmented-wave potentials~\cite{blochl_1994} and the generalized gradient approximation~\cite{perdew_1996} implemented in the Vienna Ab initio Simulation Package (VASP)~\cite{kresse.hafner.94,kresse.furthmuller.96,kresse.joubert.99}. 
The lattice parameters and atomic positions were optimized in the ${\bm a} \times ({\bm b}-{\bm c}) \times ({\bm b}+{\bm c})$ supercell containing 32 formula units and four primitive cells.
The integration in the reciprocal space was conducted using the $2 \times 2 \times 2$ Monkhorst--Pack mesh~\cite{monkhorst_1976} and the cut-off energy was set to $500$~eV. For convergence conditions, we set the energy change below $10^{-5}$ and $10^{-8}$ for the ionic and electronic loops, respectively. 

The lattice dynamical properties were studied within the temperature-dependent effective potential (TDEP) approach~\cite{hellman_2013}. The atomic potential with the third and fourth order anharmonic terms was derived from interatomic forces induced by displacements of all atoms at finite temperatures.
The sets of atomic displacements were generated by the high efficiency configuration space sampling (HECSS)~\cite{jochym_2021} and forces were obtained by VASP. The interatomic force constants and phonon frequencies were calculated with the {\sc Alamode} software~\cite{tadano_2014}.

Furthermore, we have attempted to construct a {\emph{temperature independent}} 
anharmonic model. We have used combined data from all investigated temperatures 
(300, 600 and 1000~K) and fitted a large (over 15~000 free parameters), fourth-order 
interaction model to this dataset. 
Subsequently, we have used this model to calculate line profiles and positions of Raman-active modes
at multiple temperatures.

The changes in phonon frequencies induced by the anharmonic effects were investigated within two approaches.
First, the impact of the quartic anharmonic terms was included using the SCPH theory~\cite{tadano_2015}.
Second, the mode profiles (frequency shifts and line widths) were determined from the real and imaginary parts of the phonon self-energy resulting from the cubic and quartic anharmonic terms of the above mentioned large model~\cite{tadano_2018}. 
The longitudinal optic-transverse optic (LO-TO) splitting was also evaluated, using the static dielectric tensor and Born effective charges calculated within density functional perturbation theory~\cite{gajdos_2006}.

\begin{figure}[]
    \centering
    \includegraphics[width=\linewidth]{fig1_new.png}
\caption{%
(a) The conventional unit cell of $\beta$-FeSi$_2$ (with Cmca symmetry) and (b) the corresponding Brillouin zone with selected high-symmetry points.
}
  \label{fig.struct}
\end{figure}

% To further characterize the vibrational properties, Raman-active scattering was investigated using the Phonopy-Spectroscopy package~\cite{skelton_2017}. This enabled the identification of Raman-active modes and the calculation of Raman tensors. Anharmonic force constants, obtained from calculations using {\sc Alamode}, were then used to obtain theoretical line profiles for the Raman modes. The presented Raman scattering spectra combine these anharmonic line profiles with the Raman tensor amplitudes.
% This analysis was also based on the large quartic model mentioned above.

% Finally, the thermal conductivity was obtained as a function of temperature and crystallite size within the relaxation-time approximation (RTA) \PJ{}{as implemented in {\sc Alamode}\cite{tadano_2014}}. The phonon lifetimes were calculated from the phonon self-energy including the cubic and quartic anharmonic terms. \SP{}{The RTA provides a solution of the Boltzmann transport equation (BTE) under the assumption that scattering events are independent and can be treated through mode-resolved relaxation times.
% To verify the validity of this approximation for $\beta$-FeSi$_2$, we have additionally
% executed an iterative solution of the BTE.}\PJ{}{These calculations were performed with
% {\sc Phono3py}\cite{togo_2023}. For cross-validation, the additional RTA calculations were performed on the same $\bm{q}$-grid as BTE with implementation provided by {\sc Phono3py}.}.

To further characterize the vibrational properties, Raman scattering was investigated using the Phonopy-Spectroscopy package~\cite{skelton_2017}. This enabled the identification of Raman-active modes and the calculation of Raman tensors. Anharmonic force constants, derived from calculations using {\sc Alamode}, were then used to obtain theoretical line profiles for the Raman modes. The presented Raman scattering spectra combine these anharmonic line profiles with the Raman tensor amplitudes.
This analysis was also based on the large quartic model described above.

Finally, the thermal conductivity was calculated as a function of temperature and crystallite size within the relaxation-time approximation (RTA) \PJ{}{as implemented in {\sc Alamode}~\cite{tadano_2014}}. The phonon lifetimes were calculated from the phonon self-energy including the cubic and quartic anharmonic terms. \SP{}{The RTA provides a solution to the Boltzmann transport equation (BTE) under the assumption that scattering events are independent and can be treated through mode-resolved relaxation times.
To verify the validity of this approximation for $\beta$-FeSi$_2$, we additionally
solved the BTE iteratively.} \PJ{}{These calculations were performed with
{\sc Phono3py}~\cite{togo_2023}. For cross-validation, the additional RTA calculations were performed on the same $\bm{q}$-grid as the BTE calculations, using the implementation provided by {\sc Phono3py}.}

% (see. Appendix~\ref{ThermalAB} for the comparison)\cite{togo_2023}.}
%As shown in Appendix~\ref{ThermalAB}, the iterative BTE results exhibit good agreement with the RTA values, confirming that RTA is sufficiently accurate for this material and for the considered temperature range.}


\section{Results}
\label{sec.result}

\subsection{Crystal structure}
\label{sec.crys}

The $\beta$-FeSi$_2$ structure adopts a base-centered orthorhombic lattice with the space group Cmca (No.~64) as shown in Fig.~\ref{fig.struct}(a).
The unit cell consists of two primitive cells and contains 48 atoms.
Iron (silicon) atoms possess two nonequivalent positions: \mbox{Fe-I} and \mbox{Fe-II} (\mbox{Si-I} and \mbox{Si-II}), presented in Fig.~\ref{fig.struct}(a) as gray and purple (orange and yellow) spheres, respectively.
This crystal structure is derived from the fluorite-type lattice with strongly distorted Si cubes and Fe atoms occupying 
one-half of the central sites.
The Fe-I and Fe-II sites create different layers perpendicular to the $x$ direction, and they are separated by layers containing both Si sites.
Each Fe atom is coordinated by 8 Si atoms with slightly different Fe-Si distances.
The optimized lattice constants ($a = 9.874$~{\AA}, $b = 7.767$~{\AA}, and
$c = 7.811$~{\AA}) agree very well with the experimental data ($a = 9.863$~{\AA}, $b = 7.791$~{\AA}, and $c = 7.833$~{\AA})~\cite{dusausoy_1971}.

Iron atoms occupy the Wyckoff sites 8\textit{d} ($0.2166$, $0$, $0$) and  8\textit{f} ($0$, $0.3072$, $0.1879$), corresponding to \mbox{Fe-I} and \mbox{Fe-II}, respectively. 
Silicon atoms are located at two inequivalent 16\textit{g} positions: ($0.1282$, $0.2737$, $0.0495$) and \mbox{($0.3734$, $0.0445$, $0.2270$)}, assigned as \mbox{Si-I} and \mbox{Si-II}.
The optimized positions of atoms agree very well with the experimental data~\cite{dusausoy_1971} and the previous theoretical studies~\cite{tani_2010,liang_2011}.


\subsection{Lattice dynamics}
\label{sec.lattice}

\begin{figure}[]
    \centering
    \includegraphics[width=\linewidth]{Fig1c.pdf}
\caption{%
The phonon dispersion curves along high symmetry directions obtained within SCPH (for temperatures from $0$ to $1000$~K).
Dashed black lines indicate the phonon dispersions obtained from the harmonic approximation. The white dots indicate Raman-active modes with A$_g$ symmetry.
The vertical plot shows the phonon density of states (DOS) calculated at a reference temperature of $600$~K.
}
  \label{fig.ph_band}
\end{figure}


In Fig.~\ref{fig.ph_band} we present the phonon dispersion relations of $\beta$-FeSi$_2$ along high-symmetry directions in the Brillouin zone [Fig.~\ref{fig.struct}(b)].
Due to 24 atoms in the primitive cell, the phonon spectrum consists of 69 optical modes and three acoustic modes.
The phonon dispersions were calculated within the SCPH approach in the temperature range $0$--$1000$~K (presented by color lines in Fig.~\ref{fig.ph_band}), and they are compared with the results obtained from the harmonic part of the effective potential corresponding to temperature $T=300$~K (indicated by dashed black lines in Fig.~\ref{fig.ph_band}).
As we can see within the SCPH method, the anharmonic effects are rather weak and leads only to small renormalization of phonon frequencies.
Only the highest modes show more pronounced shifts of their frequencies to larger values.
The total and partial element-projected phonon density of states obtained within the harmonic approximation are presented in Fig.~\ref{fig.ph_band}.
Up to around $320$~cm$^{-1}$, the contributions from both elements are very similar, while for higher frequencies the spectrum is dominated by
the Si vibrations.


\begin{table}[!t]
\begin{ruledtabular}
\caption{Calculated and experimental Raman-active modes of $\beta$-FeSi$_2$ with their irreducible representations (IR). Present theoretical results are compared with the previous theoretical data from Ref.~\cite{tani_2010} and experimental results from \mbox{Refs.~\cite{lefki_1991,maeda_2004}.} The experimental frequencies with known symmetries (A$_g$) are shown in bold, while other experimental modes are assigned to the best fitting theoretical values.}
\begin{tabular}{c c c c c}
\textbf{IR} & \multicolumn{4}{c}{\textbf{Frequency (cm$^{-1}$)}} \\
 & Present  & Theor.~\cite{tani_2010} & Exp.~\cite{lefki_1991} & Exp.~\cite{maeda_2004} \\
\hline
B$_{2g}$ & 175.4 & 179 & 176 &  \\
B$_{1g}$ & 176.3 & 185 & 179  &  \\
B$_{1g}$ & 193.3 & 198 &  & 190.6 \\
A$_g$    & 196.6 & 208 & \textbf{195} & \textbf{194.0} \\
A$_g$    & 203.9 & 210 & \textbf{197} & 199.6 \\
B$_{3g}$ & 205.5 & 212 & 200 &  \\
B$_{3g}$ & 226.3 & 236 & 206 & 227.1 \\
B$_{1g}$ & 233.3 & 240 &  &  231.6 \\
B$_{2g}$ & 248.6 & 254 &  &  \\
A$_g$    & 250.1 & 257 & \textbf{247} & \textbf{247.3} \\
A$_g$    & 254.9 & 264 & \textbf{253} & 254.3 \\
B$_{1g}$ & 275.5 & 285 &  &  274.1 \\
B$_{2g}$ & 282.4 & 295 &  &  281.2 \\
B$_{3g}$ & 286.8 & 297 &  &  \\
B$_{1g}$ & 307.1 & 317 &  &  \\
B$_{2g}$ & 312.6 & 326 &  &  311.8 \\
B$_{1g}$ & 319.0 & 324 &  &  \\
B$_{3g}$ & 327.9 & 341 &  & 325.8 \\
B$_{2g}$ & 333.3 & 345 &  &  \\
A$_g$    & 339.3 & 352 & \textbf{346} & 339.5 \\
B$_{2g}$ & 343.0 & 350 &  &  \\
B$_{1g}$ & 353.5 & 366 &  &  \\
B$_{2g}$ & 372.8 & 383 &  & 370.7 \\
B$_{1g}$ & 375.1 & 385 &  &  \\
B$_{3g}$ & 375.2 & 386 &  &  \\
B$_{3g}$ & 385.3 & 401 &  &  \\
A$_g$    & 386.5 & 398 &  & 386.2 \\
B$_{2g}$ & 387.7 & 402 &  & 388.2 \\
A$_g$    & 404.6 & 415 &  & 400.4 \\
B$_{2g}$ & 405.1 & 420 &  &  \\
B$_{1g}$ & 412.7 & 428 &  &  \\
B$_{3g}$ & 418.4 & 431 &  &  \\
B$_{3g}$ & 441.4 & 458 &  & 442.6 \\
B$_{3g}$ & 447.3 & 466 &  & 446.3 \\
A$_g$    & 448.4 & 464 &  &  \\
A$_g$    & 499.1 & 517 &  &  \\
\end{tabular}
\label{T1Raman}
\end{ruledtabular}
\end{table}

%subsection{Raman}

\begin{figure*}
\centering
  \includegraphics[width=\textwidth]{spectrum_RTA+raman.pdf}
\caption{Raman spectrum of $\beta$-FeSi$_2$ calculated within the perturbative approach at three temperatures 
($300$, $600$, $900$~K -- blue, orange, and red lines, respectively). 
The calculated spectrum includes all Raman-active modes. The A$_g$ modes
are indicated by purple vertical lines at peak positions corresponding to T=300~K. 
The frequencies derived from the harmonic approximation at T=300~K are indicated by green lines.
Connecting arrows indicate the correspondence between harmonic and
anharmonic frequencies, demonstrating the frequency shifts due to 
phonon interactions. Experimental values for the A$_g$ modes
based on Ref.~\cite{lefki_1991} are marked with black dashed lines.
}
\label{raman}
\end{figure*}


The phonon spectrum at the $\Gamma$ point consists of 36 Raman modes classified according to the irreducible representations: $9A_\text{g}+9B_\text{1g}+9B_\text{2g}+9B_\text{3g}$. 
Fig.~\ref{raman} shows the Raman spectrum of A$_g$ symmetry calculated for $\beta$-FeSi$_2$ within the perturbative approach at three temperatures $300$, $600$, and $900$~K (solid blue, orange, and red curves, respectively), including third-order and fourth-order anharmonic corrections.
The calculated Raman spectrum includes all Raman-active modes.
The five experimentally observed A$_g$ modes are highlighted by black dashed lines, based on data from Ref.~\cite{lefki_1991}.
Additional peaks not marked with vertical lines correspond to phonon modes with symmetries other than A$_g$.
The frequencies of Raman modes obtained in anharmonic calculations are compared with the previous results calculated within the harmonic approximation and the experimental values in Tab.~\ref{T1Raman}. 
We have marked in bold the experimentally determined A$_g$ modes, which are compared with the calculations.
Since the experimental studies did not provide the accurate assignement of the Raman modes with the B$_{1g}$, B$_{2g}$, and B$_{3g}$ symmetry~\cite{lefki_1991,maeda_2004}, we cannot compare them directly with the theoretical results.
However, in Tab.~\ref{T1Raman} we have assigned the measured frequencies to the best fitting theoretical values without taking into account the symmetry of the modes, except for the known A$_{g}$ modes. 


The impact of anharmonicity on the phonon frequencies is well visible from the comparison of the results obtained within the harmonic approximation 
and from the anharmonic calculation (vertical green and purple lines in Fig.~\ref{raman}, respectively).
Here we show only the A$_g$ modes, which are compared with the experimental results (vertical black dashed lines).
Anharmonic frequencies calculated at $300$~K are indicated by purple lines, while frequencies derived from harmonic approximation are marked with green lines. The grey solid lines connect corresponding modes obtained in both approximations.
In most cases, the results obtained within the harmonic approximation do not agree with the experimental frequencies. 
%Only the modes close to $250$~cm$^{-1}$ and $400$~cm$^{-1}$ correspond well to the experimental values. MS
As we can see, the inclusion of the anharmonic correction leads to a significant modification of the phonon frequencies.
These anharmonic effects are stronger for higher-frequency modes mainly because of
the dominant contribution from Si atoms, which vibrate with larger amplitudes than heavier Fe atoms. 
When atoms move to larger distances the potential deviates more from the harmonic approximation,
and the anharmonic corrections become stronger.


The modification of phonon frequencies observed in Fig.~\ref{raman} is much larger than in the SCPH scheme presented in Fig.~\ref{fig.ph_band}. The SCPH approach includes only the leading-order contribution to 
the phonon self-energy obtained from the quartic anharmonic terms~\cite{tadano_2015}. Therefore, it does not describe fully the changes of phonon frequencies found within the perturbation theory (see Fig.~\ref{raman}).
Especially, it is well visible for two highest A$_g$ modes, which exhibit also the largest line broadening 
and the strongest dependence on temperature.
Therefore, a better agreement with experimentally observed frequencies is visible,
confirming the significant influence of the anharmonicity on the frequencies and line profiles of phonon modes.
In fact, the decrease of phonon frequencies should be even stronger due to thermal expansion, 
which is not included in our calculations.
Within SCPH the frequencies of the highest modes increase with increasing temperature as we see in Fig.~\ref{fig.ph_band}. The comparison of two different approaches applied to study anharmonic properties of $\beta$-FeSi$_2$ shows that the perturbation theory, which includes the cubic and quartic terms, better describes the changes of phonon frequencies with temperature than the SCPH method. \PJ{}{This indicates that the leading-order contribution included in SCPH are not important in this material.}

Additionaly we should nottice that for other than A$_g$ modes we cannot make an unambiguous assignment of theoretical frequencies to experimental ones. Note that the spectrum in Fig.~\ref{fig.ph_band} contains all Raman-active modes. 
The limitation to A$_g$ modes concerns only the indicated positions of the peaks. 
%\SP{}{Furthermore, we present the full spectrum for all Raman active modes with the comparison of the harmonic and anharmonic calculations in the Appendix~\ref{RamanAA}.}
\textcolor{red}{The full Raman spectrum, with the frequencies of the B$_{1g}$, B$_{2g}$ and B$_{3g}$ modes marked, is shown in Appendix A, Fig.~\ref{raman1}. This represents our theoretical prediction of possible Raman-mode assignments, which can be verified in future experiments.}
%This is the theoretical prediction of possible assignment of Raman modes that can be verified in future experiments.  


\subsection{Thermal conductivity}
\label{sec.thermal}


\begin{figure*}[!t]
\centering
  \includegraphics[width=\linewidth]{time3.pdf}
\caption{Phonon lifetimes calculated for three temperatures as a function of phonon frequency. The colors correspond to the phonon branches.}
  \label{thermtime}
\end{figure*}


In this section, we analyze the thermal conductivity tensor of $\beta$-FeSi$_2$ obtained within the RTA approach~\cite{tadano_2018} as a function of temperature
%
\begin{equation}
\kappa_{\text{ph}}^{\mu\nu}(T) = \frac{1}{NV} \sum_{\bm{q},j} c_{\bm{q}j}(T) v_{\bm{q}j}^{\mu} v_{\bm{q}j}^{\nu}\tau_{\bm{q}j}(T),
\end{equation}
% 
where $c_{\bm{q}j}$ is the mode heat capacity and $v_{\bm{q}j}$ is the mode group velocity. 
The relaxation time is approximated by the phonon lifetime $\tau_{\bm{q}j}$
calculated for $j$-th branch at the wave vector $\bm{q}$.
$V$ is the unit cell volume and $N$ is the number of unit cells in the crystal.
The phonon lifetime is calculated using this formula
%
\begin{equation}
\tau_{\bm{q}j}(T)=\frac{1}{2\Gamma_{\bm{q}j}^{\text{anh}}(T)},
\end{equation}
%
where $\Gamma_{\bm{q}j}^{\text{anh}}$ is the anharmonic phonon linewidth obtained from 
the imaginary part of the phonon self-energy within the perturbation theory.

In Fig.~\ref{thermtime}, we present $\tau_{\bm{q}j}$ obtained for three temperatures 300, 600, and 1000~K as a function of frequency. As we see, the acoustic phonons close to the $\Gamma$ point have the longest lifetimes,
which are diminished with increasing frequency reaching local minima around $200$~cm$^{-1}$.
For higher frequencies, phonon lifetimes first increase to local maxima around $300$~cm$^{-1}$ and then decrease to
the lowest values in the range of highest optical modes. The shortest lifetimes correspond to the largest line broadening
observed for the Raman modes in Fig~\ref{raman}. The phonon group velocities 
$v_{\bm{q}j}=\partial\omega_{\bm{q}j}/\partial\bm{q}$, which are obtained by the central difference formula, are presented in Fig.~\ref{thermvelocity}. Their temperature dependence is negligible, therefore, we present only the results for $T=600$~K.
At low frequencies, there are clearly two ranges of group velocities of the acoustic phonons. 
The larger values correspond to the longitudinal modes, while the lower values are obtained from the
transverse acoustic branches. Group velocities of acoustic phonons decrease for larger frequencies
and reach the average values typical for optic branches.

\begin{figure}[!t]
\centering
  \includegraphics[width=\linewidth]{GV.pdf}
\caption{Mode group velocities calculated as a function of phonon frequency. The colors correspond to the phonon branches.
}
  \label{thermvelocity}
\end{figure}

In Fig.~\ref{anizotropy}(a), we present the three diagonal elements of $\kappa_{\text{ph}}^{\mu\nu}$ corresponding to the main directions of the crystal structure. 
They were obtained from the force constants calculated at the base temperature $T=600$~K and the crystallite size $0.1$~$\mu$m to account for boundary-limited phonon transport. 
Due to the orthorhombic symmetry, we observe a small anisotropy in phonon transport in the whole temperature range. 
At low temperatures, the three components of the heat conductivity increase in a very similar way with the $\kappa_{\text{ph}}^{yy}$ element slightly larger than two other components. 
After reaching the maximum, we observe a change in the largest component from $\kappa_{\text{ph}}^{yy}$ to 
$\kappa_{\text{ph}}^{xx}$.    
In Fig.~\ref{anizotropy}(b), the thermal conductivity is shown for three base temperatures, at which the interatomic potential was obtained ($300$~K, $600$~K, and $1000$~K), using the energy expansion up to third- and fourth-order anharmonic terms, and the same structure size of $0.1$~$\mu$m. At lowest temperatures, the thermal conductivity strongly increases, reaching the maximum around $T=180$~K, then it shows a slower decrease with temperature.
The differences between the two levels of approximation are minimal, suggesting that third-order calculations already capture the dominant phonon scattering mechanisms. The dependence on the base temperature is also very weak, showing the changes in the heat conductivity within a few percent. 
\SP{}{Further verification of the reliability of our thermal conductivity results is given in Appendix~\ref{ThermalAB}, where the full BTE calculations show good agreement with RTA and higher-resolution RTA results, demonstrating that the $8\times8\times8$ $\bm{q}$-mesh already provides converged values.}

\begin{figure}[!t]
\centering
  \includegraphics[width=\linewidth]{Anizotropy.pdf}
\caption{(a) The anisotropic thermal conductivity of $\beta$-FeSi$_2$ calculated along the lattice directions at 600~K. (b) The average temperature-dependent thermal conductivity taken at $300$~K, $600$~K, and $1000$~K, including anharmonic corrections up to cubic (A3) and quartic (A4) terms. In both cases the crystallite size is 0.1~$\mu$m.}
  \label{anizotropy}
\end{figure}


In Fig.~\ref{therm}, we fix the base temperature at $600$~K and examine the effect of crystallite size on thermal conductivity, varying it from $0.01$ to $0.5$~$\mu$m.
With decreasing the crystallite size, we observe a shift of the position of the maximum to larger temperatures and a decrease of the thermal conductivity in the entire temperature range.
Theoretical results are compared with several experimental data obtained above the room temperature. 
The measured thermal conductivity depends to a large extent on the sample quality, its purity and the size of the crystalline grains which depends on 
the production processes.
Many measurements were performed using crystallites of micrometric or unknown size ~\cite{waldecker_1973,ito_2002,kim_2003,du_2020}, however, 
numerous attempts to minimize $\kappa$ by reducing grain sizes to $56$~nm~\cite{dabrowski_2019, dabrowski_microstructure_2021}, $30$-$400$~nm~\cite{le_tonquesse_2019}, $50$ and $200$~nm~\cite{abbassi_2021}, or introducing pores into the material~\cite{sam2023improved} 
are also carried out. 
Another way to change the thermal conductivity is to dope $\beta$-FeSi$_2$ with different elements~\cite{ito_2002,kim_2003,du_2020,cheng_2024}, however, this effect is beyond our investigation.

\begin{figure}[!t]
\centering
  \includegraphics[width=\linewidth]{Thermal_conductivity3.pdf}
\caption{
The phonon thermal conductivity of $\beta$-FeSi$_2$: theoretical results for the infinite crystalline size and with boundary conditions, compared with experimental data for different structure sizes.
}
  \label{therm}
\end{figure}
%Fig.~\ref{therm}\cite{abbassi_2021}\cite{waldecker_1973}\cite{dabrowski_2019}\cite{sam2023improved}\cite{kim_2003}\cite{du_2020}\cite{ito_2002}\cite{le_tonquesse_2019}.

We observe a decrease in the thermal conductivity with reducing crystalline grain sizes in all analyzed experimental data. 
For instance, by decreasing the crystallite size to $50$~nm, the thermal conductivity at room temperature was reduced by a factor of $1.7$, what can be  compared to the annealed sample with 200 nm grains~\cite{abbassi_2021}. 
It is worth noting that the rate of decrease in value with increasing temperature in both cases, for grain sizes of $50$~nm and $200$~nm, is significantly different, which is consistent with our calculations. 
The same trend can be observed by comparing the thermal conductivity measured for a sample with bulk crystallite sizes with the thermal conductivity of a sample with grains smaller than 400 nm~\cite{le_tonquesse_2019}.
The theoretical results obtained for the same crystallite size show higher values due to factors not captured in the idealized model, such as crystal imperfection or mechanical strain. Usually, a decrease in the crystallite size is related to an increased concentration of grain boundaries, point defects, and stacking faults that influence the phonon scattering~\cite{le_tonquesse_2019,abbassi_2021}.   

We should note that the total thermal conductivity is a combination
of the lattice and electronic contributions to the heat transport.
In semiconductors, the electronic thermal conductivity is negligible at low temperatures and significantly increases 
only much above the room temperatures~\cite{gu_2020}.
For $\beta$-FeSi$_2$, the electronic thermal conductivity was obtained from the electric conductivity using the Wiedemann-Franz law~\cite{ito_2002,kim_2003,le_tonquesse_2019}.
In the undoped material, its value does not exceed $0.1$~W/mK in the measurement up to $T=950$~K~\cite{kim_2003}.
By doping, the electronic thermal conductivity can be enhanced, and it has a direct impact on the thermoelectric properties of $\beta$-FeSi$_2$ at high temperatures~\cite{ito_2002,kim_2003}.
In the present study, we consider only the phonon contribution to the thermal conductivity,
therefore, agreement with experimental data may deteriorate with increasing temperature.

\section{Summary}
\label{sec.summary}

We performed {\it ab initio} studies on lattice dynamical and thermal transport properties of $\beta$-FeSi$_2$. The effect of anharmonicity was analyzed within two approaches -- the SCPH method and the perturbation theory.
The phonon dispersion curves obtained within SCPH show small renormalization of frequencies comparing to the harmonic approximation. 
The Raman spectra were calculated within the procedure which takes into account the peak intensities obtained from the Raman tensors and the line profiles obtained from the phonon self energy derived within the perturbation theory based on the large, temperature-independent, quartic model fitted to the data from the wide range of temperatures (300-1000~K). 
The anharmonic corrections strongly affect the frequencies and line profiles of some modes and results in overall better agreement with the experimental data. 
We analyzed the phonon lifetimes and group velocities obtained as functions of the phonon frequency.
Then the lattice thermal conductivity was calculated for a broad range of temperatures and grain sizes.
We found a small anisotropy in the phonon thermal transport resulting from the orthorhombic structure and a weak effect of the quartic anharmonic terms. 
The thermal conductivity calculated for various crystalline grain sizes show a good qualitative agreement with the available measurements.

\begin{acknowledgments}
Some figures in this work were rendered using {\sc Vesta}~\cite{momma.izumi.11} software.
This work was partially supported by the Ministry of Education, Youth and Sports of the Czech Republic through the e-INFRA CZ (ID:90254).
\end{acknowledgments}

\appendix

\section{Raman spectrum}
\label{RamanAA}

Based on the polarized Raman measurements reported in Ref.~\cite{maeda_2004}, two Raman peaks were identified as belonging to the A$_g$ symmetry class, and several additional peaks were observed with similar or different polarization dependence. Although the authors of Ref.~\cite{maeda_2004} provided estimates of the relative Raman tensor components, they did not specify which of the remaining modes correspond to the B$_{1g}$, B$_{2g}$, or B$_{3g}$ symmetries. Because of this missing experimental information, a direct symmetry-resolved comparison between the measurement and theory is not currently possible for the non-A$_g$ modes. 
To provide a complete theoretical picture of the B$_g$-type modes, we show here the calculated Raman-active frequencies and intensities for the B$_{1g}$, B$_{2g}$, and B$_{3g}$ symmetries only. 
%These results represent the predicted Raman modes for the B$_g$ symmetries in $\beta$-FeSi$_2$. 
\textcolor{red} {Fig.~\ref{raman1} shows the predicted Raman modes for the B$_g$ symmetries in $\beta$-FeSi$_2$. The results obtained within the harmonic approximation are compared with the anharmonic perturbation theory calculations which provides both, frequency shifts and predicted line profiles of the modes.}
Although the experimentally measured peaks cannot be directly assigned to these symmetries due to the lack of polarization-resolved data, the theoretical predictions provide a reference for comparison. Matching the measured frequencies to the closest theoretical B$_g$ modes (Table~\ref{T1Raman}) allows for a tentative assignment, which can guide future polarization-resolved Raman experiments aimed at determining the precise symmetry of the unresolved peaks.

\begin{figure*}[t]
\centering
  \includegraphics[width=\linewidth]{spectrum_RTA+raman_Bi_modes.pdf}
\caption{Raman spectrum of $\beta$-FeSi$_2$ calculated at three temperatures 
($300$, $600$, $900$~K -- blue, orange, and red lines, respectively). 
The calculated spectrum includes all Raman-active modes. The B$_{ig}$ modes
are indicated by purple vertical lines at peak positions corresponding to T=300~K. 
The frequencies derived from the harmonic approximation at T=300~K are indicated 
by green, yellow and pink lines.
Connecting arrows indicate the correspondence between harmonic and anharmonic 
frequencies, demonstrating the frequency shifts due to phonon interactions.}
\label{raman1}
\end{figure*}
\section{Thermal conductivity obtained from BTE and RTA}
\label{ThermalAB}

The thermal conductivity was computed by solving the full BTE on the largest feasible $\bm{q}$-point grid, $8\times8\times8$, and compared with the corresponding RTA results obtained on the same grid. As shown in Fig.~\ref{bte}, the difference between the components of the thermal conductivity tensor obtained within BTE and RTA at this resolution is very small, indicating a good agreement between these two approaches.
%THIS PART SHOULD GO RATHER TO THE RESPONSE
%Extending the full BTE calculation to larger grids is computationally prohibitive: the computational cost of BTE is roughly two orders of %magnitude higher than that of RTA, and the required memory and runtime exceed our available resources. 
Moreover, we performed an additional calculation using RTA on a denser $20\times20\times20$ grid. As seen in the Fig.~\ref{bte}, the higher-resolution data remain in a good agreement with both the BTE and RTA results for the $8\times8\times8$ grid.
It shows that the $8\times8\times8$ mesh already provides good results for this structure and confirms reliability of the calculations.

\begin{figure}[!h]
\centering
  \includegraphics[width=\linewidth]{BTEvsRTAvsANP.pdf}
\caption{Thermal conductivity of $\beta$-FeSi$_2$ obtained within the BTE and RTA methods using the Phono3py software on the $8\times8\times8$ q-point grid, compared with the RTA results computed with ALAMODE on a denser $20\times20\times20$ grid.}
\label{bte}
\end{figure}

%\section*{Data availability}
%The data that support the findings of this article are openly available~\footnote{give me DOI}.
% see https://journals.aps.org/authors/data-availability-statements#citation

\bibliography{refs.bib}
%\bibliographystyle{ieeetr}


\end{document}

